\chapter{Dynamic Programming}
\section{Introduction}
\lecture{(07/04)}

Recall the \term{Fibonacci numbers} $F_n$ defined by
$F_0 = 0$, $F_1 = 1$, $F_n = F_{n-1} + F_{n-2}$ with the naive algorithm
\begin{algorithm}[H]
  \caption{\Call{Fib}{$n$}}
  \begin{algorithmic}[1]
    \If{$n=0$} \Return{0}
    \ElsIf{$n=1$} \Return{1}
    \Else \Return{$\Call{Fib}{n-1} + \Call{Fib}{n-2}$}
    \EndIf
  \end{algorithmic}
\end{algorithm}
Assuming we count additions as unit cost,
the runtime is $T(n) = F_{n+1} + 1 \in \Theta(\varphi^n)$ which is bad.

Notice that we are recomputing small $F_n$ a bunch of times,
but we actually only need each one once.
We can instead cache:
\begin{algorithm}[H]
  \caption{\Call{FibCached}{$n$}}
  \begin{algorithmic}[1]
    \Require $T \gets [0,1,\bot,\dotsc,\bot]$ global array of size $n$
    \If{$T[n] = \bot$} \State $T[n] \gets \Call{FibCached}{n-1} + \Call{FibCached}{n-2}$ \EndIf
    \State \Return{$T[n]$}
  \end{algorithmic}
\end{algorithm}
Also, notice that the dependency graph of which subproblems require each other is a DAG.
Therefore, we can take an order on the dependencies and iterate:
\begin{algorithm}
  \caption{\Call{FibIterative}{$n$}}
  \begin{algorithmic}[1]
    \State $T \gets [0,1,\bot,\dotsc,\bot]$ 0-indexed array of size $n$
    \For{$i = 2,\dotsc,n$}
    \State $T[i] \gets T[i-1] + T[i-2]$
    \EndFor
    \State \Return{$T[n]$}
  \end{algorithmic}
\end{algorithm}
This is our dynamic programming algorithm.
In fact, we can optimize even more by noticing that we can discard
all but the last two elements of the array,
giving a constant-space algorithm:
\begin{algorithm}[H]
  \caption{\Call{FibOptimal}{$n$}}
  \begin{algorithmic}[1]
    \State $(u,v) \gets (0,1)$
    \For{$i = 2,\dotsc,n$}
    \State $(u,v) \gets (v,u+v)$
    \EndFor
    \Return{$v$}
  \end{algorithmic}
\end{algorithm}
All these improved algorithms run in $O(n)$ time, a significant improvement.

We can give a general recipe for dynamic programming algorithms:
\begin{enumerate}[1.]
  \item \textbf{Identify the subproblem:}
        We are retaining solutions in an array.
        What are the dimensions of the array?
        What does each entry represent?
        Where will the final answer be in the array?
  \item \textbf{Establish DP-recurrence:}
        How does a subproblem contribute to a larger subproblem?
        What is the dependency between cells in the array?
  \item \textbf{Set base cases:}
        Initialize the array with some non-recursively defined base cases.
  \item \textbf{Specify the order of computation:}
        Clarify the DAG of subproblem dependencies.
        How does the algorithm maintain this order?
  \item \textbf{Recover the solution (if needed):}
        What subproblem answers provide the problem solution?
        How, if necessary, do we traceback the solution from the subproblems?
\end{enumerate}

We can often convert a DP algorithm into iterative loop(s).
Distinguish divide and conquer algorithms which do not always solve subproblems
and are not easily rewritten iteratively.

\section{Interval Scheduling}

\begin{problem}[weighted interval scheduling]
  Recall \Cref{prob:g:intsched}.
  Now, add a weight $w_i$ to each interval.
  We choose a subset $T \subseteq [n]$ which maximizes $W = \sum_{i \in T} w_i$.
\end{problem}

\begin{example}
  Let $I = [[2,8], [2,4], [5,6], [7,9]]$ with weights $[6,2,1,2]$.
\end{example}
\begin{sol}
  By inspection, since the weight $w_1 = 6 > 5 = w_2 + w_3 + w_4$,
  the solution is $T = [1]$ with $W = 6$.
\end{sol}

Notice that we can split on whether we accept the last interval $I_n$
and write for example that the optimal weight
\[
  W(I_1,\dotsc,I_n) = \begin{cases*}
    w_n + W(I_{m_1},\dotsc,I_{m_s}) & if we choose $I_n$ \\
    W(I_1,\dotsc,I_{n-1})           & if we do not
  \end{cases*}
\]
where $I_{m_1},\dotsc,I_{m_s}$ are the $s < n$ intervals not intersecting $I_n$.

Suppose we sort the intervals by finish time, i.e., $f_i \leq f_{i+1}$ for all $i$.
Then, we have $m_1,\dotsc,m_s = 1,\dotsc,j$ where $j = \max\{ i : f_i < s_n \}$
because $I_n$ is the last interval with the latest finish time,
so we only need to compare its start time with earlier intervals' finish times.
(If $j$ does not exist just return $w_n + 0$.)

We need to calculate the $j$-values for every $i$:
\begin{algorithm}[H]
  \caption{\Call{FindJs}{$A,s_1,\dotsc,s_n,f_1,\dotsc,f_n$}}
  \begin{algorithmic}[1]
    \State $j \gets$ array of size $n$
    \State $f_0 \gets \infty$
    \State $i \gets 1$
    \For{$k=0,\dotsc,n$}
    \While{$i \leq n$ and $f_k \leq s_{A[i]} < f_{k+1}$}
    \State $j[i] \gets k$
    \State $i\pp$
    \EndWhile
    \EndFor
    \State \Return{$j$}
  \end{algorithmic}
\end{algorithm}
where $A$ is a sorting permutation such that $(s_{A[i]})$ is non-decreasing.
This runs in $O(n\log n) + O(n) = O(n\log n)$ time.

Now, for the main procedure, we define $W[i]$ as the maximal weight
possible with the intervals $I_1,\dotsc,I_i$.

Then, for $W[0] = 0$ and $i \geq 1$, $W[i] = \max\{W[i-1], w_i + W[j[i]]\}$.

Since $W[i]$ depends only on entries in $W$ before it,
we can just iterate on $i = 1,\dotsc,n$ in $O(n)$ time.

\begin{algorithm}[H]
  \caption{\Call{IntervalScheduling}{$s_1,\dotsc,s_n,f_1,\dotsc,f_n,w_1,\dotsc,w_n$}}
  \begin{algorithmic}[1]
    \Require intervals are sorted by finish time
    \State $A \gets$ sorting permutation of $s_1,\dotsc,s_n$
    \State $j \gets \Call{FindJs}{A,s_1,\dotsc,s_n,f_1,\dotsc,f_n}$
    \State $W \gets$ 0-indexed array of size $n$
    \State $W[0] \gets 0$
    \For{$i = 1,\dotsc,n$}
    \State $W[i] \gets \max\{W[i-1], w_i + W[j[i]]\}$
    \EndFor
    \State \Return{$W[n]$}
  \end{algorithmic}
\end{algorithm}

This gives a total time for the algorithm of $O(n\log n) + O(n) = O(n\log n)$.

\section{Knapsack Problem}

\begin{problem}[0/1 knapsack]\label{prob:dp:k}
  Suppose we have items with weights $w_1,\dotsc,w_n$
  and values $v_1,\dotsc,v_n$ but our knapsack has capacity $W$.
  We want to select items $S \subseteq \{1,\dotsc,n\}$
  satisfying $\sum_{i \in S} w_i \leq W$
  and maximizes $\sum_{i \in S} v_i$.
\end{problem}

\begin{example}
  $\vb w = [3,4,6,5]$, $\vb v = [2,3,1,5]$, $W = 8$.
\end{example}
\begin{sol}
  The optimal $S = \{1,4\}$ with weight $3+5 = 8$ and value $2+5 = 7$.
\end{sol}

\lecture{(07/06)}
For each item $n$, we can either choose it or we can not.
Let $O[W,n]$ be best value for a knapsack of capacity $W$
and considering only the items $1,\dotsc,n$.
Then, $O[W,n]$ is either $v_n + O[W - w_n, n - 1]$ or $O[W, n - 1]$.

We can initialize $O[0, i] = 0$ for all $i$ and $O[w, 0] = 0$ for all $w$.
To be able to calculate $O[W,n]$, we must have already calculated
$O[W - w_n .. W, n-1]$.
In particular, if we iterate on $n$ first, we can guarantee that
the entire row $O[,n-1]$ exists before considering $O[W,n]$:

\begin{algorithm}[H]
  \caption{\Call{01KnapSack}{$v_1,\dotsc,v_n,w_1,\dotsc,w_n,W$}}
  \begin{algorithmic}[1]
    \State $O \gets$ 0-indexed array of size $(n+1) \times (W+1)$
    \State $O[0,] \gets \vb{0}$; $O[,0] \gets \vb{0}\trans$
    \For{$i = 1,\dotsc,n$}
    \For{$w = 1,\dotsc,W$}
    \If{$w_i > w$}
    \State $O[w,i] \gets O[w,i-1]$
    \Else
    \State $O[w,i] \gets \max\{v_n + O[W - w_n, n - 1], O[W, n - 1]\}$
    \EndIf\EndFor\EndFor
    \State \Return{$O[W,n]$}
  \end{algorithmic}
\end{algorithm}

The runtime here is obviously $\Theta(nW)$.
We call this \term{pseudo-polynomial} because it is polynomial in
$n$ (the size of the input) but also in $W$ (the \emph{value} of an input).
It is not polynomial because the size parameters are $n$ and $\lg W$,
but we have $n2^{\lg W}$.

\section{Subsequence Problems}

\begin{problem}[longest increasing subsequence]
  Find the longest (potentially discontinuous)
  increasing subsequence of an array $A[1..n]$ of integers.
\end{problem}
\begin{example}
  Given $A = [7,1,3,10,11,5,19]$, the longest increasing subsequence is
  $[1,3,10,11,19]$.
\end{example}

Notice that there are $\Theta(2^n)$ subsequences, so brute force is very bad here.

Suppose we try doing DP and storing $\ell[i]$ as the longest increasing subsequence of $A[1..i]$.
This doesn't work, since we can't immediately deduce $\ell[i+1]$ from just $\ell[i]$ and $A$.

We could instead store into $L[i]$ a pair of the length and the last entry $(\ell, c)$.
Then, we can add on the next element $L[i] \gets (\ell + 1, A[i])$,
but what is $L[i]$ if we do not select $A[i]$?

Alternatively, let $L[i]$ be the length of the longest increasing subsequence
of $A[1..i]$ that ends with $A[i]$. Then, $L[1] = 1$.
The longest increasing subsequence $S_i$ ending at $A[i]$
either looks like $[\dotsc,A[j],A[i]] = [\dots S_j, A[i]]$ for some $j$
or just $[A[i]]$.

\begin{algorithm}[H]
  \caption{\Call{LongestIncreasingSubsequence}{$A[1..n]$}}
  \begin{algorithmic}[1]
    \State $L \gets$ array of size $n$
    \State $L[1] \gets 1$
    \For{$i = 2,\dotsc,n$}
    \State $L[i] \gets 1$ \Comment{$S_i = [A[i]]$}
    \For{$j = 1,\dotsc,i-1$} \Comment{$S_i = [\dots S_j, A[i]]$}
    \If{$A[j] < A[i]$} 
    \State $L[i] \gets \max\{L[i], L[j] + 1\}$
    \EndIf
    \EndFor
    \EndFor
    \State \Return{$\max L$}
  \end{algorithmic}
\end{algorithm}

This algorithm runs in $\Theta(n^2)$ time which is much faster than $\Theta(2^n)$.
Note that we don't return the actual sequence here, only its length,
but it is trivial to find the sequence from the array $L$.

\begin{problem}[longest common subsequence]
  Given two arrays of characters (strings) $A[1..n]$ and $B[1..m]$,
  find the maximum length of a (potentially discontinuous) subsequence
  common to both $A$ and $B$.
\end{problem}
\begin{example}
  For $A = \vv{blurry}$ and $B = \vv{burger}$, we should return $\vv{burr}$ for $k=4$.
\end{example}

As with \Cref{prob:dp:k}, we have to work in a 2D problem space.
Let $M[i,j]$ be the longest subsequence length between $A[1..i]$ and $B[1..j]$.
Zero out $M[0,]$ and $M[,0]$.
Then, $M[i,j]$ will be the greatest of either 
(1) ignoring $B[j]$, (2) ignoring $A[i]$, or (3) adding $A[i] = B[j]$:

\begin{algorithm}
  \caption{\Call{LongestCommonSubsequence}{$A[1..n],B[1..m]$}}
  \begin{algorithmic}[1]
    \State $M \gets$ 0-indexed array of size $n+1 \times m+1$
    \State $M[0,] \gets \vb{0}$; $M[,0] \gets \vb{0}\trans$
    \For{$i = 1,\dotsc,n$}\For{$j = 1,\dotsc,m$}
    \State $M[i,j] \gets \max\{M[i,j-1], M[i-1,j]\}$
    \If{$A[i] = B[j]$} \State $M[i,j] \gets \max\{M[i,j], 1 + M[i-1,j-1]\}$ \EndIf
    \EndFor\EndFor
    \State \Return{$M[n,m]$}
  \end{algorithmic}
\end{algorithm}

Notice that because we iterate by $i$ first, $M[i-1,0..m]$ will have values.
Also, since we are iterating by increasing $j$, $M[i,1..j-1]$ will be calculated.
Therefore, this algorithm works and runs in $\Theta(nm)$ time.

\lecture{(07/11)}
\lecture{(07/13)}
\lecture{(07/18)}
