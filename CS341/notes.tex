\documentclass[notes,tikz,class=cs341]{agony}
% \includeonly{}
\graphicspath{{figures/}}
\usetikzlibrary{graphs,graphs.standard,graphdrawing,quotes,matrix}
\usegdlibrary{circular,force,trees,layered}
\makeatletter
\let\c@algorithm\relax
\@counteralias{algorithm}{theorem}
\makeatother

\title{CS 341 Spring 2023: Lecture Notes}
\begin{document}
\renewcommand{\contentsname}{CS 341 Spring 2023:\\{\huge Lecture Notes}}
\thispagestyle{firstpage}
\tableofcontents

Lecture notes taken, unless otherwise specified,
by myself during section 001 of the Spring 2023 offering of CS 350,
taught by Armin Jamshidpey.

\begin{multicols*}{2}
  \listoflecture
\end{multicols*}

\chapter{Operating Systems Introduction}

\lecture{(05/09)}
Generally, an \term{operating system} acts partially as a cop (e.g., watching for unplugged USB drives)
and as a facilitator (e.g., allowing you to interface with any storage device with \verb|fopen|).
It is responsible for:
\begin{itemize}[nosep]
  \item managing resources;
  \item creating execution environments;
  \item loading programs; and
  \item providing common services and utilities
\end{itemize}
We will consider an OS from three views:
\begin{enumerate}[1.,nosep]
  \item Application: what does an OS provide?
        Provides an \term{execution environment}
        which provides resources, interfaces, and isolation.
  \item System: what problems does an OS solve?
        Manages hardware resources, allocates them to programs,
        and controls access to shared resources between programs.
  \item Implementation: how is an OS built?
        It must be \term[concurrency]{concurrent} (allow multiple things to run at once)
        and \term{real-time} (respond to events in a set time).
\end{enumerate}

\begin{defn}[kernel]
  The part of the operating system that responds to system calls,
  interrupts, and exceptions.
\end{defn}
\begin{defn}[operating system]
  Includes the kernel, but also other related programs that provide services
  for applications. For example, utility programs, command interpreters, and programming libraries.
\end{defn}

The kernel protects from bad \term{user programs} by isolating
them in \term{user space} (resp. \term{kernel space})
and allowing them only to interact with hardware using system calls.

\begin{defn}[system call]
  A user interaction with the OS.
  For example, the C function \verb|fopen| makes the Linux syscall
  \verb|sys_open|.
  They are much slower than calling a user function.
\end{defn}

\begin{defn*}[types of kernels]
  \begin{itemize}[nosep]
    \item \term[kernel!monolithic kernel]{Monolithic}:
          when the entire OS is the kernel (e.g. Linux)
    \item \term[kernel!microkernel]{Microkernel}:
          when only absolutely necessary parts are in the kernel
    \item \term[kernel!hybrid kernel]{Hybrid}:
          somewhere between monolithic kernels and microkernels (e.g. Windows, macOS)
    \item \term[real-time operating system]{Real-time}:
          with stringent event response time, guarantees, and scheduling
  \end{itemize}
\end{defn*}

A monolithic kernel is faster, since we avoid slower system calls.
However, they are less secure since third-party drivers must be trusted and included in the kernel.

\lecture{(05/11)}
Provided by the execution environment are abstract entities
that a program is able to manipulate:
\begin{itemize}[nosep]
  \item files and file systems (secondary storage; e.g. HDDs)
  \item address spaces (primary memory; RAM)
  \item processes, threads (program execution)
  \item sockets, pipes (message channels)
\end{itemize}

\chapter{Threads}

\begin{defn}[thread]
  Sequence of instructions
\end{defn}

An ordinary \term{sequential program} has only a single thread.
Analogous to how DFAs have a single state and NFAs can follow multiple paths,
a program can have multiple threads of execution.
The threads can be for the same role (e.g. one per server visitor)
or for different roles (e.g. Chrome's JavaScript and graphics engines).

Threads allow for:
\begin{itemize}[nosep]
  \item \term{concurrency}: allow multiple tasks to occur at once
  \item \term{parallelism}: different threads on different processors to increase throughput
  \item \term{responsiveness}: allow blocking tasks to not block the whole system
  \item \term{priority}: do things that are more important first
  \item \term{modularity}: separate out tasks into threads that can't crash each other
\end{itemize}
A thread will pause execution when it is \term[blocking]{blocked}.

Consider for example the traffic simulation for Assignment 1.
Each thread represents a vehicle passing through an intersection,
and we are trying to prevent collisions.

A thread can create a new thread using \verb|thread_fork|.
The original and new threads share global data and the heap.
However, the new thread has a separate, private stack.

For example, in \texttt{/kern/synchprobs/traffic.c}:
\begin{minted}{c}
  for (i = 0; i < NumThreads; i++) {
    error = thread_fork("vehicle_simulation thread", NULL, vehicle_simulation,
                        NULL, i);
    if (error) {
      panic("traffic_simulation: thread_fork failed: %s\n", strerror(error));
    }
  }
\end{minted}
we start a thread running \mintinline{c}{vehicle_simulation(NULL, i)}.

In OS/161, we create a thread with
\begin{minted}{c}
  int thread_fork(
    const char *name,
    struct proc *proc,
    void (*func)(void *, unsigned long),
    void *data1,
    unsigned long data2
  );
\end{minted}
and can terminate with \mintinline{c}{void thread_exit(void)}
and yield with \mintinline{c}{void thread_yield(void)}.
However, we cannot control the order that threads run in.

Recall from CS 241 how to execute a single thread:
fetch--execute cycle.
In CS 241, we called all the registers $\$0,\dotsc,\$31$.
In real life, they have names like a0 and s8.

CS 241 passed all arguments via the stack.
We will pass the first four as a0 to a3 and the rest on the stack.

\lecture{(05/16)}
Recall: functions push arguments (not a0-a3),
return address, local variables, and temporary-use registers onto the stack.

With multiple threads, we need multiple stacks.
When swapping threads, save the value of registers to the stack
and then load from the other stack.

The threads share the same code, global read-only data, global data, and heap.
They have their own stacks and program counters.
Since we might have lots of threads, each stack has a fixed size (e.g. 2 MB).

We can \term{multithread} a core by having multiple sets of registers
but share an ALU, control unit, etc. by using the hardware when waiting for LW and SW instructions.
Therefore, given $P$ processors, each with $C$ cores and $M$ multithreads per core (almost always 2),
we can execute $PCM$ threads (truly) simultaneously.

\begin{defn}[timeshare]
  Switching rapidly from one thread to another.
  During a \term{context switch}, we schedule which thread runs next,
  save the register contents of the current thread, 
  and load the register contents of the next thread.
  The saved/restored contents are also called the \term{thread context}.
\end{defn}

The C function \mintinline{c}{thread_switch} saves and restores caller-save registers,
and calls the assembly language subroutine \mintinline{asm}{switchframe_switch}
which saves and restores callee-save registers.

\lecture{(05/18)}
There are four ways a context switch can be triggered:
\begin{enumerate}[1.,nosep]
  \item voluntarily, via \mintinline{c}{thread_switch};
  \item by termination, via \mintinline{c}{thread_exit};
  \item when a thread is blocked, via \mintinline{c}{wchan_sleep};\footnote{where \texttt{wchan} stands for ``wait channel''} or
  \item by \term{preemption}, when the thread schedule involuntarily stops it.
\end{enumerate}

A thread can be either:
\begin{itemize}[nosep]
  \item \term[thread!running]{running}, currently executing on the processor;
  \item \term[thread!ready]{ready}, waiting in a ready pool; or
  \item \term[thread!blocked]{blocked}, waiting for something to happen and not ready to execute
\end{itemize}
Running to blocked via \mintinline{c}{wchan_sleep},
blocked to ready via \mintinline{c}{wake_one} or \mintinline{c}{wake_all},
and ready to running via \mintinline{c}{dispatch}.
A running thread can become ready by preemption or via \mintinline{c}{thread_yield}.

\chapter{Divide and Conquer}
In general, we want to:
\begin{itemize}[nosep]
  \item divide: split a problem into subproblems;
  \item conquer: solve the subproblems recursively; and
  \item combine: use subproblem results to derive problem result
\end{itemize}
This is possible when:
\begin{itemize}[nosep]
  \item the original problem is easily decomposable into subproblems;
  \item combining solutions is not costly; and
  \item subproblems are not overly unbalanced
\end{itemize}

\section{Examples}
\begin{problem}[counting inversions]
  Given an unsorted array $A[1..n]$, find the number of \term[inversion]{inversions} in it,
  i.e., pairs $(i,j)$ such that $A[i] > A[j]$.
\end{problem}
\begin{example}
  Given $A = [1,5,2,6,3,8,7,4]$,
  we get (2,3), (2,5), (2,8), (4,5), (4,8), (6,7), (6,8), and (7,8).
\end{example}
The naive algorithm checks all pairs and takes $\Theta(n^2)$ time.
We can do better.

Let $c_\ell$ be the number of inversions in $A[1..n/2]$,
$c_r$ be the number of inversions in $A[n/2+1..n]$,
and $c_t$ be the number of transverse inversions, i.e.,
inversions where $i$ is on the left and $j$ is on the right.

We can find $c_\ell$ and $c_r$ by recursion.

To find $c_t$, we must count the number of left indices
greater than each right index.
This can be done by sorting and then binary searching,
since the binary search result index gives exactly what we want.
The sort takes $O(n\log n)$ and each of the $n$ binary searches takes $O(\log n)$.

This gives us $T(n) \leq 2T(n/2) + O(n\log n) = O(n\log^2 n)$.

We can instead modify \Call{MergeSort}{} and find $c_t$ using a modified \Call{Merge}{}:
\begin{algorithm}[H]
  \caption{Modified \Call{Merge}{$A[1..n]$} (additions in \textcolor{Green}{green})}
  \begin{algorithmic}[1]
    \Require{both halves of $A$ are sorted}
    \State \textcolor{Green}{copy $A$ into a new array $S$; $c = 0$}
    \State $i \gets 1$; $j \gets n/2+1$
    \For{$k \gets 1,\dotsc,n$}
      \If{$i > n/2$} $A[k] \gets S[j\pp]$
      \ElsIf{$j > n$}
        \State $A[k] \gets S[i\pp]$
        \State \textcolor{Green}{$c \gets c + \frac{n}{2}$}
      \ElsIf{$S[i] < S[j]$}
        \State $A[k] \gets S[i\pp]$
        \State \textcolor{Green}{$c \gets c + j - (\frac{n}{2}+1)$}
      \Else{} $A[k] \gets S[j\pp]$
      \EndIf
    \EndFor
  \end{algorithmic}
\end{algorithm}
Here, every time we merge in an element from the left,
we add to $c$ the number of elements on the right which are greater than it.
This will run in $\Theta(n\log n)$ time because the modified \Call{Merge}{} is still $\Theta(n)$.

\begin{problem}[polynomial multiplication]
  Given $F = f_0 + \dotsb + f_{n-1}x^{n-1}$
  and $G = g_0 + \dotsb + g_{n-1}x^{n-1}$,
  calculate $H = FG$.
\end{problem}

The naive algorithm takes $\Theta(n^2)$ time to expand.

Notice that we can split $F = F_0 + F_1x^{n/2}$ and $G = G_0 + G_1x^{n/2}$.
Then, we have $H = F_0G_0 + (F_0G_1+F_1G_0)x^{n/2} + F_1G_1x^n$.
If we divide and conquer, we make 4 recursive calls with size $n/2$
and $\Theta(n)$ extra work for the additions.

However, $T(n) = 4T(n/2)+\Theta(n) \in \Theta(n^2)$ which is not an improvement.

\begin{lemma}[Karatsuba's identity]
  $(x+y)(a+b) - xa - yb = xb + ya$
\end{lemma}

\lecture{(05/18)}
But if we already have $F_0G_0$ and $F_1G_1$,
we can use Karatsuba's identity to instead calculate
$(F_0 + F_1)(G_0 + G_1) - F_0G_0 - F_1G_1 = F_0G_1 + F_1G_0$.
That is, we will calculate:
\begin{align*}
  H & = (F_0 + F_1x^{n/2})(G_0 + G_1x^{n/2})                           \\
    & = F_0G_0 + ((F_0+F_1)(G_0+G_1)-F_0G_0-F_1G_1)x^{n/2} + F_1G_1x^n
\end{align*}
This means we only need to make 3 recursive calls instead of 4.

Then, $T(n) = 3T(n/2) + \Theta(n) \in \Theta(n^{\lg 3})$ which is an improvement.

Based on this observation, Toom--Cook created a family of algorithms
that for $k \geq 2$ make $2k-1$ recursive calls in size $n/k$,
i.e., $T(n) \in \Theta(n^{\log_k(2k-1)})$ which gets arbitrarily close to linear
(but with increasingly massive constants).

If $F,G \in \C[x]$, then we can use FFT to get $T(n) = 2T(n/2)+\Theta(n) \in \Theta(n\log n)$ time.

\begin{problem}[matrix multiplication]
  Given $A = [a_{i,j}] \in M_{n\times n}$ and $B = [b_{j,k}] \in M_{n \times n}$,
  calculate $C = AB$.
\end{problem}

The naive algorithm takes inputs of size $\Theta(n^2)$ in $\Theta(n^3)$ time.

Consider instead breaking into block matrices:
$A = \pmqty[c|c]{A_{1,1}&A_{2,2}\\\hline A_{2,1}&A_{2,2}}$ and
$B = \pmqty[c|c]{B_{1,1}&B_{2,2}\\\hline B_{2,1}&B_{2,2}}$.

Then, $C = \begin{pmatrix}[c|c]
    A_{1,1}B_{1,1} + A_{1,2}B_{2,1} & A_{1,1}B_{1,2} + A_{1,2}B_{2,2} \\ \hline
    A_{2,1}B_{1,1} + A_{2,2}B_{2,1} & A_{2,1}B_{1,2} + A_{2,2}B_{2,2}
  \end{pmatrix}$

This makes 8 recursive calls of size $n/2$ and $\Theta(n^2)$ additions,
which resolves to $\Theta(n^3)$ (no improvement).
However, due to Strassen, we can reduce this to 7, giving $\Theta(n^{\lg 7})$ time.

We can generalize to do $k$ multiplications of $\ell\times\ell$ matrices in $\Theta(n^{\log_\ell k})$ time
and $k$ multiplications of $\ell\times m$ by $m\times p$ in $\Theta(n^{3\log_{\ell m p}k})$ time.

\begin{problem}[closest pairs]
  Given $n$ distinct points $(x_i, y_i)$,
  find a pair $(i,j)$ that minimizes the distance
  \[ d_{i,j} = \sqrt{(x_i-x_j)^2+(y_i-y_j)^2} \]
  Equivalently, minimize $d^2_{i,j} = (x_i-x_j)^2+(y_i-y_j)^2$.
\end{problem}

Separate the space of points into $L$ and $R$ halfspaces based on the median $x$ value.
The closest pair is either entirely in $L$, entirely in $R$, or transverse.

We can recursively find minimum distances $\delta_L$ and $\delta_R$.
Then, if we let $\delta = \min\{\delta_L,\delta_R\}$,
any closer transverse points must be within $\delta$ units of the median $x$ value.

Now, if we start from the bottom point $P \in L$ by $y$-value in that band,
we only have to compare $P$ with points $Q \in R$ with $y_P \leq y_Q < y_P + \delta$.

We can only have a maximum of 8 points inside the $2\delta \times \delta$
rectangle of possible $Q$ points, because the points must be at least $\delta$ apart.

Therefore, we are doing $\Theta(1)$ work for each $P$, so we do $\Theta(n)$
work to find transverse pairs.

For this to work, we must first sort the points by $x$ and by $y$ (in $O(n\log n)$ time).
We can find the median in $O(1)$ time.
We split the sorted points in $O(n)$ time for the two recursive calls
and find the $\delta$ band in $O(n)$ time.
Again, it takes $O(n)$ time to find transverse pairs.
Therefore, $T(n) = 2T(n/2) + O(n) = O(n\log n)$.

\begin{problem}[selection]
  Given $A[0..n-1]$, find the entry that would be at index $k$ if $A$ were sorted.
\end{problem}

Recall from CS 240 that selection by sorting takes $O(n\log n)$ time
or $O(n)$ randomized expected time using \Call{QuickSelect}{$A$, $k$}:

\begin{algorithm}[H]
  \caption{\Call{QuickSelect}{$A$, $k$}}
  \begin{algorithmic}[1]
    \State $p \gets \Call{ChoosePivot}{A}$
    \State $i \gets \Call{Partition}{A, p}$ \Comment{$i$ is the correct index of $p$}
    \If{$i = k$} \Return $A[i]$
    \ElsIf{$i > k$} \Return \Call{QuickSelect}{$A[0..i-1]$, $k$}
    \Else{} \Return \Call{QuickSelect}{$A[i+1..n-1]$, $k-i-1$}
    \EndIf
  \end{algorithmic}
\end{algorithm}

Consider splitting $A$ into groups $G_i$ of size 5.
Then, find the medians $m_i$ of each group.
We can choose the pivot $p$ as the median of medians:
\begin{center}
  \begin{tikzpicture}
    \newcommand{\Geq}{\rotatebox{270}{$\geq$}}
    \matrix(M)[matrix of math nodes]{
           &      &      &      &        &      & n/2  &      &        &      &           &      & n       \\
      *    &      & *    &      & \dotsb &      & *    &      & \dotsb &      & *         &      & *       \\
      \Geq &      & \Geq &      & \Geq   &      & \Geq &      & \Geq   &      & \Geq      &      & \Geq    \\
      *    &      & *    &      & \dotsb &      & *    &      & \dotsb &      & *         &      & *       \\
      \Geq &      & \Geq &      & \Geq   &      & \Geq &      & \Geq   &      & \Geq      &      & \Geq    \\
      m_1  & \geq & m_2  & \geq & \dotsb & \geq & p    & \geq & \dotsb & \geq & m_{n/5-1} & \geq & m_{n/5} \\
      \Geq &      & \Geq &      & \Geq   &      & \Geq &      & \Geq   &      & \Geq      &      & \Geq    \\
      *    &      & *    &      & \dotsb &      & *    &      & \dotsb &      & *         &      & *       \\
      \Geq &      & \Geq &      & \Geq   &      & \Geq &      & \Geq   &      & \Geq      &      & \Geq    \\
      *    &      & *    &      & \dotsb &      & *    &      & \dotsb &      & *         &      & *       \\
    };
    \draw[red,ultra thick,rounded corners] (M-2-1.north west) -- (M-2-7.north east) -- (M-5-7.south east) -- (M-5-7.south west) -- (M-6-6.north east) -- (M-6-6.south east) -- (M-6-1.south west) -- cycle;
    \draw[Green,ultra thick,rounded corners] (M-10-13.south east) -- (M-10-7.south west) -- (M-7-7.north west) -- (M-7-7.north east) -- (M-6-8.south west) -- (M-6-8.north west) -- (M-6-13.north east) -- cycle;
  \end{tikzpicture}
\end{center}
Then, we are guaranteed to have $3n/10$ elements
\textcolor{Red}{above} and \textcolor{Green}{below} $p = A[i]$,
so the recursive calls to $A[0..i-1]$ and $A[i+1 .. n-1]$ have size at most $7n/10$
(with equality when $i$ is exactly $3n/10$ or $7n/10$).

Therefore, $T(n) \leq T(n/5) + T(7n/10) + O(n)$.

\lecture{(05/25)}

\begin{claim}
  $T(n/5) + T(7n/10) + O(n) \in O(n)$
\end{claim}
\begin{prf}
  Proceed by induction. Note that $T(n) \leq \begin{cases}
      O(1)                                       & n < 120    \\
      T(\frac{n}{5}) + T(\frac{7}{10}n+6) + O(n) & n \geq 120
    \end{cases}$

  We will show that $T(n) \leq cn$ for large enough $c$ and all $n > 0$.
  We know that there exists a sufficiently large $c$ such that
  $T(n) \leq cn$ for $n < 120$ because $T(n)$ is just $O(1) \subsetneq O(n)$.

  Choose a constant $a$ to write $O(n)$ as $an$.

  Suppose $T(m) \in O(m)$ for all $0 < m < n$. Then,
  \begin{align*}
    T(n) & \leq \frac{cn}{5} + c\qty(\frac{7n}{10} + 6) + an \\
         & \leq c\frac{n}{5} + c\frac{7n}{10} + 6c + an      \\
         & = 9c\frac{n}{10} + 6c + an                        \\
         & = cn + \qty(-c\frac{n}{10} + 6c + an)
  \end{align*}
  We can show that the latter term is non-positive:
  \begin{align*}
    -c\frac{n}{10} + 6c + an \leq 0
     & \iff c\qty(6-\frac{n}{10}) + an \leq 0 \\
     & \iff c\qty(6-\frac{n}{10}) \leq -an    \\
     & \iff c\qty(\frac{n}{10}-6) \geq an     \\
     & \iff c \geq 10a\frac{n}{n-60}
  \end{align*}
  Now, if $\frac{n}{n-60} \leq 2$, i.e., $n \geq 120$,
  then we can say that $c \geq 20a$.

  Therefore, we can say that $T(n) \leq cn$, i.e., $T(n) \in O(n)$.
\end{prf}

\begin{example}
  What does $T(n) = T(\frac23n) + T(\frac{n}{3}) + n$ resolve to?
\end{example}
\begin{sol}
  Notice that if we draw a tree, each layer sums to $n$
  (this makes sense inductively since we pass $\frac23$ of $n$ to the left and $\frac13$ of $n$ to the right).
  There will be $O(\log_{3/2}n)$ layers in the tree, so it should resolve to $O(n\log n)$.
\end{sol}

\chapter{Graph Algorithms}

\section{Graph Theory Review}
Recall graph theory from MATH 239, specifically:
ms
\begin{defn}[graph]
  A graph $G$ is a pair $(V,E)$
  where $V$ is a finite set of \term{vertices}
  and $E$ is a set of unordered pairs of distinct vertices, called \term{edges}.
  By convention, we write $n = \abs{V}$ and $m = \abs{E}$.
\end{defn}

Now, we can define some structures on a graph:

\begin{defn}[adjacency list]
  An array $A[1..n]$ such that $A[v]$ is a linked list
  containing all edges connected to $v$.
  This contains $2m$ list cells with total size $\Theta(n+m)$
  but takes more than $O(1)$ time to test if an edge exists.
\end{defn}

\begin{defn}[adjacency matrix]
  A matrix $M \in M_{n\times n}(\{0,1\})$
  such that $M[v,w] = 1$ if and only if $\{v,w\} \in E$.
  Size is $\Theta(n^2)$ but testing if an edge exists is $O(1)$.
\end{defn}

\begin{example}
  Given the graph
  \tikz[baseline=-17pt]\graph{{1,5}--{2,4}--3[y=-0.5];1--5,2--4,2--5};,
  the adjacency list is:
  \begin{enumerate}[1,nosep]
    \item $\to 2 \to 5$
    \item $\to 1 \to 3 \to 4 \to 5$
    \item $\to 2 \to 4$
    \item $\to 2 \to 3 \to 5$
    \item $\to 1 \to 2 \to 4$
  \end{enumerate}
  and the adjacency matrix is
  \[
    M = \begin{bmatrix}
      0 & 1 & 0 & 0 & 1 \\
      1 & 0 & 1 & 1 & 1 \\
      0 & 1 & 0 & 1 & 0 \\
      0 & 1 & 1 & 0 & 1 \\
      1 & 1 & 0 & 1 & 0
    \end{bmatrix}
  \]
\end{example}

\begin{defn*}[graph terminology]
  We also recall some terms from MATH 239:
  \begin{itemize}[nosep]
    \item A \term{path} is a sequence of vertices $v_1,\dotsc,v_k$
          such that $\{v_i,v_{i+1}\} \in E$ for all $i$.
          If a path from $v$ to $w$ exists, we write $v \leadsto w$.
    \item A \term{connected graph} has $v \leadsto w$ for all $v,w \in V$.
    \item A \term{cycle} is a path $v \leadsto v$ of length at least 3
          with all elements pairwise distinct.
    \item A \term{tree} is a graph with no cycles.
    \item A \term{rooted tree} is a tree with a vertex chosen to be the \term{root}.
    \item A \term{subgraph} of $G = (V,E)$ is a graph $G' = (V',E')$
          where $V' \subseteq V$, $E' \subseteq E$, and $u,v \in V'$ for all $uv \in E'$.
    \item A \term{connected component} of $G$ is a connected subgraph of $G$
          that is not a subset of any other connected subgraph.
  \end{itemize}
\end{defn*}

\section{Breadth-First Search}

\begin{problem}
  Search a graph $G$ starting from a vertex $s$
  in order of distance from $s$.
\end{problem}

\begin{algorithm}
  \caption{\Call{BFS}{$G$, $s$}}
  \begin{algorithmic}[1]
    \State let $Q$ be an empty queue
    \State let $\vv{visited}$ be a boolean array of size $n$ with all entries set to $\bot$
    \State \Call{enqueue}{$s$, $Q$}
    \State $\vv{visited}[s] \gets \top$
    \While{$Q$ is not empty}
    \State $v \gets \Call{dequeue}{Q}$
    \For{$w$ neighbours of $v$} \label{line:bfs:forloop}
    \If{$\vv{visited}[w] = \bot$}
    \State \Call{enqueue}{$w$, $Q$}
    \State $\vv{visited}[w] \gets \top$
    \EndIf
    \EndFor
    \EndWhile
  \end{algorithmic}
\end{algorithm}

Each vertex is enqueued at most once and dequeued at most once,
which has cost $O(n)$.
Therefore, each adjacency list is read at most once.
The cost for the for loop is $O(\sum \deg v) = O(m)$ by the Handshaking Lemma.

Therefore, the total cost of \Call{BFS}{} is $O(n+m)$.

\begin{lemma}\label{lemma:bfs:visited}
  $\vv{visited}[v]$ is true for some vertex $v$
  if and only if $s \leadsto v$ in $G$.
\end{lemma}
\begin{prf}
  Let $s = v_0, \dotsc, v_K$ be the vertices with $\vv{visited}{v_i} = \top$,
  in order of discovery.
  By induction, we show that $s \leadsto v_i$.

  For $i = 0$, $v_0 = s$, so trivially $s \leadsto s$.

  Otherwise, suppose $s \leadsto v_j$ for all $j < i$.
  We are currently in the for loop for some vertex $w$ already visited.
  Therefore, by assumption, $s \leadsto w$.
  But since $v_i$ is a neighbour of $w$, $s \leadsto v_i$.
\end{prf}

\lecture{(05/30)}

\begin{xca}
  For a connected graph, $m \geq n - 1$.
\end{xca}
\begin{prf}
  Recall from MATH 239 that if a graph $G$ is connected, then it has a spanning tree $T$.
  The spanning tree of $n$ vertices has exactly $n-1$ edges.
  Then, since the spanning tree is a subgraph of $G$,
  $m \geq \abs{E(T)} = n-1$, as desired.
\end{prf}

\section{Shortest Path by BFS}

\begin{problem}
  What is the shortest path from $s$ to $v$ in $G$?
\end{problem}

Consider now how we can keep track of parents (predecessors)
and levels (depths):
\begin{algorithm}[H]
  \caption{\Call{BFS}{$G$, $s$} with parents and levels}
  \begin{algorithmic}[1]
    \State let $Q$ be an empty queue
    \State \textcolor{Green}{let $\vv{parent}$ be an array of size $n$ with all entries set to $\bot$}
    \State \textcolor{Green}{let $\vv{level}$ be an array of size $n$ with all entries set to $\infty$}
    \State \Call{enqueue}{$s$, $Q$}
    \State \textcolor{Green}{$\vv{parent}[s] \gets s$}
    \State \textcolor{Green}{$\vv{level}[s] \gets0$}
    \While{$Q$ is not empty}
    \State $v \gets \Call{dequeue}{Q}$
    \For{$w$ neighbours of $v$}
    \If{$\vv{parent}[w] = \bot$}
    \State \Call{enqueue}{$w$, $Q$}
    \State \textcolor{Green}{$\vv{parent}[w] \gets v$}
    \State \textcolor{Green}{$\vv{level}[w] \gets \vv{level}[v] + 1$}
    \EndIf
    \EndFor
    \EndWhile
  \end{algorithmic}
\end{algorithm}
We can define a \term{BFS tree} $T$ as the subgraph of $G$
made of all $w$ such that $\vv{parent}[w] \neq \bot$
and all edges $\{w,\vv{parent}[w]\}$ between those vertices.

\begin{claim}
  The BFS tree $T$ is in fact a tree.
\end{claim}
\begin{prf}
  Proceed by induction on the vertices for which $\vv{parent}[v]$ is not $\bot$.

  When we set $\vv{parent}[s] \gets s$, we have one vertex and no edges.

  Suppose $T$ is a tree and we are adding $\vv{parent}[w] \gets v$.
  Then, $v$ must have already been in $T$ because it came from $Q$,
  so we are extending $T$ by adding (1) the vertex $w$ and (2) the edge $\{v,w\}$.
  This does not create a cycle because $\vv{parent}[w] = \bot$,
  so $T$ remains a tree.

  Therefore, by induction, at the end of \Call{BFS}{}, $T$ is a tree.
\end{prf}

\begin{claim}\label{claim:bfs:level-non-decreasing}
  The levels in the queue $Q$ are non-decreasing.
\end{claim}
\begin{prf}
  Exercise (TODO).
\end{prf}

\begin{claim}\label{claim:bfs:edge-level}
  For all vertices $u$ and $v$, if there is an edge $\{u,v\}$,
  then $\vv{level}[v] \leq \vv{level}[u] + 1$.
\end{claim}
\begin{prf}
  Suppose that $u$ and $v$ are adjacent and visited.

  If we dequeue $v$ before $u$, then $\vv{level}[v] \leq \vv{level}[u] + 1$
  by \Cref{claim:bfs:level-non-decreasing}.

  If $u$ is dequeued before $v$, then the parent of $v$ is
  either $u$ or something else before $u$.
  This is because while visiting $u$, we must either have enqueued $v$
  or already visited $v$.
  Therefore, $v$'s parent must be at or before $u$.
  Then, by \Cref{claim:bfs:level-non-decreasing},
  $\vv{level}[\vv{parent}[v]] \leq \vv{level}[u]$.

  That is, $\vv{level}[v] = \vv{level}[\vv{parent}[v]] + 1 \leq \vv{level}[u] + 1$.
\end{prf}

\begin{lemma}
  For all $v$ in $G$,
  there is a path $s \leadsto v$ in $G$ if and only if
  there is a path $s \leadsto v$ in $T$.
  If so, the path in $T$ is a shortest path and $\vv{level}[v]$ is the distance from $s$ to $t$.
\end{lemma}
\begin{prf}
  By \Cref{lemma:bfs:visited}, $s \leadsto_G v$ if and only if $v$ is visited.
  That is, all such $v$ are in $T$.
  But $T$ is connected as a tree, therefore $s \leadsto_G v \iff s \leadsto_T v$.

  Let $\delta$ be the distance from $s$ to $v$.
  We must show $\vv{level}[v] \leq \delta$ and $\delta \leq \vv{level}[v]$.

  Trivially, $\delta \leq \vv{level}[v]$ because $\vv{level}[v]$
  is the length of the path $s \leadsto_T v$.

  We will prove by induction that for all $i$,
  if there is a path of length $i$ from $s$ to $v$,
  then $\vv{level}[v] \leq i$.
  For the base case $i=0$, there are no such paths.

  Suppose this is true for $i-1$, and consider a path $P = s\dotsb uv$ with length $i$.
  Then, we can decompose $P$ as $P' = s\dotsb u$ and $uv$.
  But $P'$ has length $i-1$, so $\vv{level}[u] \leq i-1$.
  Then, by \Cref{claim:bfs:edge-level}, $\vv{level}[v] \leq \vv{level}[u] + 1 \leq i$.

  Therefore, since this is true for all $i$, it is true for $i=\delta$.

  Finally, we have that $\delta = \vv{level}[v]$ and $s \leadsto_T v$ is a shortest path.
\end{prf}

\section{Bipartiteness by BFS}

\begin{defn}[bipartite]
  A graph $G = (V, E)$ is bipartite
  if there exists a partition $U_1 \sqcup U_2 = V$
  such that for every $uv \in E$, $u \in U_1$ and $v \in U_2$ (or vice versa).
\end{defn}

\begin{problem}
  Is $G$ bipartite?
\end{problem}

\begin{lemma}
  Suppose $G$ is connected and we run \Call{BFS}{$G$, $s$} for some $s$.
  Let $V_1$ and $V_2$ be vertex sets with odd and even level respectively.
  Then, $G$ is bipartite if and only if all edges have one end in $V_1$ and one end in $V_2$.
\end{lemma}
\begin{prf}
  Suppose all edges have one end in $V_1$ and one end in $V_2$.
  Then, $G$ is bipartite by definition.
  \lecture{(06/01)}

  Suppose $G$ has bipartition $(W_1, W_2)$.
  Then, \Wlog say that $s \in W_2$.
  Since $s \leadsto v$ for all $v \in V$ and all paths alternate between $W_1$ and $W_2$,
  odd depth vertices will fall in $W_1 = V_1$ and even ones in $W_2 = V_2$.
\end{prf}

This is nice because we can test in $O(m)$ time.

\section{Depth-First Search}

Analogous to BFS, but we use a stack (implicitly with recursion,
or explicitly with a stack data structure) to follow neighbours until we cannot.

\begin{algorithm}[H]
  \caption{\Call{DFS}{$G$}}
  \begin{algorithmic}[1]
    \Require $G$ is a graph on $n$ vertices given by adjacency lists
    \State $\vv{visited} \gets$ array of size $n$ initialized to $\bot$
    \Procedure{explore}{$v$}
    \State $\vv{visited}[v] \gets \top$
    \For{$w$ neighbour of $v$}
    \If{$\vv{visited}[v] = \bot$}
    \State \Call{explore}{$v$}
    \EndIf
    \EndFor
    \EndProcedure
    \For{$v \in G$}
    \If{$\vv{visited}[v] = \bot$}
    \State \Call{explore}{$v$}
    \EndIf
    \EndFor
  \end{algorithmic}
\end{algorithm}

\begin{lemma}[white path lemma]\label{lem:dfs:white-path}
  When we start exploring $v$,
  any $w$ connected to $v$ by an unvisited path
  will be visited during \Call{explore}{$v$}.
\end{lemma}
\begin{prf}
  Let $v_0 = v\dotsb v_k = w$ be a path $v \leadsto w$ with $v_1,\dotsc,v_k$ all not visited.
  We prove all $v_i$ are visited before \Call{explore}{$v$} is finished.

  Obviously holds for $i=0$.
  Suppose it holds for $i < k$.
  When we visit $v_i$, \Call{explore}{$v$} is not finished and $v_{i+1}$
  is one of the neighbours.

  If $\vv{visited}[v_{i+1}]$ is already true (because $v \leadsto v_{i+1}$ by some other path), we are done.
  Otherwise, we are going to visit it now, which is before \Call{explore}{$v$} is finished.

  Therefore, $v_{i+1}$ is visited during \Call{explore}{$v$}, as desired.
\end{prf}

\begin{corollary}
  After we call \Call{explore}{} at $v_1,\dotsc,v_k$,
  we have visited exactly the connected components containing $v_1,\dotsc,v_k$.
\end{corollary}

Note: we cannot find shortest paths using a DFS tree without customization.
For example, the DFS tree for a cycle will be a path even though the root and leaf are adjacent.

The runtime is still $O(n+m)$.

\begin{defn*}
  Let $T_1,\dotsc,T_k$ be a DFS forest with vertices $u$ and $v$.
  Then, $u$ is an \term{ancestor} of $v$ if $u,v \in T_i$ for some $i$
  and $u$ is on the path from the root of $T_i$ to $v$.
  Equivalently, we write that $v$ is a \term{descendant} of $u$.
\end{defn*}

\begin{lemma}[key property]\label{lem:dfs:key-prop}
  All edges in $G$ connect a vertex to one of its descendants or ancestors.
\end{lemma}
\begin{prf}
  Let $\{v,w\}$ be an edge and suppose \Wlog we visit $v$ first.

  Then, when we visit $v$, $(v,w)$ is an unvisited path $v \leadsto w$,
  so by the \nameref{lem:dfs:white-path}, $w$ must become a descendant of $v$.
\end{prf}

\begin{defn}[back edge]
  An edge in $G$ connecting an ancestor to a descendant which is not in the BFS forest.
\end{defn}

\begin{corollary}
  All edges are either tree edges or back edges.
\end{corollary}
\begin{prf}
  Equivalent statement of the \ref{lem:dfs:key-prop}.
\end{prf}

We can extend DFS with $\vv{start}$ and $\vv{finish}$ arrays:

\begin{algorithm}[H]
  \caption{\Call{DFS}{$G$} with timing}
  \begin{algorithmic}[1]
    \Require $G$ is a graph on $n$ vertices given by adjacency lists
    \State $\vv{visited} \gets$ array of size $n$ initialized to $\bot$
    \State \textcolor{Green}{$\vv{start}, \vv{finish} \gets$ array of size $n$}
    \State \textcolor{Green}{$t \gets 1$}
    \Procedure{explore}{$v$}
    \State $\vv{visited}[v] \gets \top$
    \State \textcolor{Green}{$\vv{start}[v] \gets t$; $t\pp$}
    \For{$w$ neighbour of $v$}
    \If{$\vv{visited}[v] = \bot$}
    \State \Call{explore}{$v$}
    \EndIf
    \EndFor
    \State \textcolor{Green}{$\vv{finish}[v] \gets t$; $t\pp$}
    \EndProcedure
    \For{$v \in G$}
    \If{$\vv{visited}[v] = \bot$}
    \State \Call{explore}{$v$}
    \EndIf
    \EndFor
  \end{algorithmic}
\end{algorithm}

For example, we can draw a graph with $[\vv{start}[v],\vv{finish}[v]]$ labelled:
\begin{center}
  \tikz\graph[simple, math nodes, nodes={circle, draw}]{
  u[y=-0.5,"[2,7]" {left,orange}]--{s["[1,8]" {red}],v["[3,4]" {blue, below}]}--w["[5,6]" {color=Green, right},y=-0.5]; u--w; v-!-w;};
  \begin{tikzpicture}
    \draw[ultra thick,red] (1,1) --node[above]{$s$ [1,8]} (8,1);
    \draw[ultra thick,orange] (2,2) --node[above]{$u$ [2,7]} (7,2);
    \draw[ultra thick,color=Green] (5,3) --node[above]{$w$ [5,6]} (6,3);
    \draw[ultra thick,blue] (3,3) --node[above]{$v$ [3,4]} (4,3);
  \end{tikzpicture}
\end{center}

Notice that the intervals shrink with depth and follow a structure
similar to the well-formed parenthesis problem.
We can in fact prove:

\begin{lemma}[parentheses theorem]\label{lem:dfs:paren}
  If $\vv{start}[u] < \vv{start}[v]$, then either $\vv{finish}[u] < \vv{start}[v]$
  or $\vv{finish}[u] < \vv{finish}[v]$.
\end{lemma}
\begin{prf}
  If $\vv{start}[u] < \vv{start}[v]$, we push $v$ on the stack
  while $u$ is still there, so we pop $v$ before we pop $u$ since stacks are FIFO.
\end{prf}

\section{Cut Vertices by DFS}
\lecture{(06/06)}

We define a cut vertex analogous to a bridge edge from MATH 239.

\begin{defn*}[cut vertex]
  Given a connected graph $G$,
  a vertex $v \in V(G)$ is a \term*{cut vertex} (or \term{articulation point})
  if removing $v$ and its edges makes $G$ disconnected.
\end{defn*}

\begin{example}
  \tikz[baseline=-17pt]\graph[empty nodes, nodes={circle, draw, fill, inner sep=2pt}]{
  1--{2,3[red]}-!-{4,5}--{6,7};
  2--3--{4,5}; 4--5--6--7;
  };
  has a cut vertex in \textcolor{red}{red}.
\end{example}

\begin{problem}
  Which of the vertices in $G$ are cut vertices?
\end{problem}

Consider a rooted DFS tree $T$ with known $\vv{parent}$ and $\vv{level}$.

\begin{prop}
  The root $s$ is a cut vertex if and only if it has more than one child.
\end{prop}
\begin{prf}
  Suppose $s$ has one child $v$.
  Then, $T-s$ is a rooted DFS tree with root $v$ (i.e., it remains connected).

  Suppose $s$ has subtrees $S_1,\dotsc,S_k$.
  Let $u \in S_i$ and $v \in S_j$ for $i \neq j$.
  Then, there does not exist a path $u \leadsto v$ in $T - s$
  by the \nameref{lem:dfs:key-prop} since it would involve a non-tree, non-back cross edge.
  Therefore, the subtrees are disconnected in $T - s$.
\end{prf}

\begin{prop}
  Let $a(v) = \min\{\vv{level}[w] : vw \in E(G)\}$
  and $m(v) = \min\{a(w) : \text{$w$ descendant of $v$}\}$.
  Any non-root vertex $v$ is a cut vertex if and only if
  it has a child $w$ with $m(w) \geq \vv{level}[v]$.
\end{prop}
\begin{prf}
  Let $w$ be a child of $v$ with subtrees $T_w$ and $T_v$, respectively.

  Suppose $m(w) < \vv{level}[v]$ and we have removed $v$.
  Then, there is a vertex $w'$ in $T_w$ to some vertex $v'$ above $v$.
  That is, for any vertex $u \in V(T_w)$, we have that
  $u \leadsto w \leadsto w' \leadsto v' \leadsto s$
  and $T_w$ is still connected.

  Therefore, for $v$ to be a cut vertex, we must have $m(w) \geq \vv{level}[v]$.

  Suppose $m(w) \geq \vv{level}[v]$.
  Then, by the \nameref{lem:dfs:key-prop},
  all edges from $T_w$ end in $T_v$.
  They are either the tree edge $vw$
  or a back edge going to an ancestor at or below $v$.
  Therefore, removing $v$ will cause $T_w$ to be disconnected and $v$ is a cut vertex.
\end{prf}

Therefore, we can solve the cut vertex problem by calculating $m(v)$ for every vertex.

We can compute $a(v)$ in $O(\deg v)$.
Notice that if $v$ has children $w_1,\dotsc,w_k$,
then $m(v) = \min\{a(v), m(w_1), \dotsc, m(w_k)\}$.
Then, if we have the $m$ of the children, we get $m(v)$ in $O(\deg v)$.

By traversing the DFS tree, we get every $m(v)$ in $O(n+m) = O(m)$ (since $G$ connected).
Then, we can test the cut vertex condition for each vertex $v$
and each of its children in $O(\deg v)$.

Therefore, we can test all vertices in $O(m)$ time.

\begin{algorithm}[H]
  \caption{\Call{FindCutVertices}{$G$, $s$}}
  \begin{algorithmic}[1]
    \State $T \gets \Call{DFS}{G, s}$
    \Comment{DFS tree for $G$ with $\vv{root}$ and $\vv{level}$}
    \State $a, m \gets$ arrays of size $\abs{V(G)}$ initialized to $\infty$
    \State $\vv{cut} \gets$ array of size $\abs{V(G)}$ initialized to $\bot$
    \Procedure{Explore}{$v$}
    \For{$w$ child of $v$}
    \State $a[v] \gets \min\{a[v], \vv{level}[w]\}$
    \State \Call{Explore}{w}
    \State $m[v] \gets \min\{m[v], a[w]\}$
    \EndFor
    \State $m[v] \gets \min\{a[v], m[v]\}$
    \For{$w$ child of $v$}
    \If{$m[w] \geq T.\vv{level}[v]$} $\vv{cut}[v] \gets \top$ \EndIf
    \EndFor
    \EndProcedure
    \State \Call{Explore}{$T.\vv{root}$}
  \end{algorithmic}
\end{algorithm}

\section{Directed Graphs}

We can define a directed graph similar to an ordinary graph:

\begin{defn}[directed graph]
  A graph $G = (V, E)$ where edges are \emph{ordered} pairs $(u, v)$.

  If $G$ has no cycles, it is a \term{directed acyclic graph} (DAG).
\end{defn}

Note that we allow loops $(v, v)$.
Paths and cycles have the ordinary meaning.

\begin{defn}[topological ordering]
  An ordering $<$ of $V$ in a DAG such that $(a,b) \in E$ implies $a < b$.
\end{defn}

\lecture{(06/08)}
\begin{prop}
  A directed graph is acyclic if and only if there is a topological ordering on it.
\end{prop}
\begin{prf}
  The backwards direction is clear.

  Assume we have a DAG.
  There exists at least one vertex with in-degree 0,
  because otherwise there would be a cycle.
  We can inductively remove the vertex with in-degree 0 to get a topological ordering.

  In fact, if run DFS and we order $V$ with the ordering
  $v < w \iff \vv{finish}[w] < \vv{finish}[v]$,
  then we can show that $<$ is a topological order.

  Suppose that $(v, w) \in E$.

  If we discover $v$ before $w$, then $w$ is a descendant of $v$
  by the \nameref{lem:dfs:white-path} so we must finish exploring it before we finish $v$.

  Otherwise, if we discover $w$ before $v$,
  then there cannot exist a path $w \leadsto v$
  because otherwise $w \leadsto vw$ is a cycle.
  Therefore, $\vv{finish}[w] < \vv{start}[v] < \vv{finish}[v]$.

  Therefore, $<$ is a topological order whose existence
  is necessary and sufficient for a DAG.
\end{prf}

\begin{defn*}[strong connectivity]
  A directed graph $G$ is \term{strongly connected} if for all $v$ and $w$ in $G$,
  there is a path $v \leadsto w$ (and $w \leadsto v$)
\end{defn*}

\begin{corollary}
  $G$ is strongly connected if and only if there exists $s$
  such that for all $w$ there exist paths $s \leadsto w$ and $w \leadsto s$.
\end{corollary}
\begin{prf}
  The forwards direction is trivial.
  In the backwards direction, notice that for any two vertices $v$ and $w$,
  we have $v \leadsto s \leadsto w$ and $w \leadsto s \leadsto v$.
\end{prf}

\begin{problem}
  How can we test if a graph is strongly connected?
\end{problem}
\begin{sol}
  Call \Call{explore}{} twice, starting from the same vertex $s$.
  On the second run, reverse all the edges.
  Then, if every vertex $v$ is explored in both runs,
  we know that $s \leadsto v$ and $v \leadsto s$,
  i.e., the graph is strongly connected.

  We can reverse the edges using an adjacency list in $O(n+m)$ time,
  so this algorithm runs in $O(n+m)$ time.
\end{sol}

\begin{prop}
  Contracting the strongly connected components of a directed graph
  forms a DAG.
\end{prop}
\begin{prf}
  Suppose not. Then there exists a cycle of strongly connected components.
  However, this means that any vertex from any of these can be reached from any other.
  Therefore, the strongly connected component is not maximal.
\end{prf}

\lecture{(06/13)}
\begin{problem}
  What are the strongly connected components and their respective DAG?
\end{problem}
\begin{algorithm}
  \caption{Kosaraju's algorithm for strongly connected components}
  \begin{algorithmic}[1]
    \Procedure{SCC}{$G$}
    \State run \Call{DFS}{$G$} augmented with finish times
    \State sort the vertices by decreasing finish time
    \State run \Call{DFS}{$G\trans$}
    \State \Return{the trees in the DFS forest of $G\trans$}
    \EndProcedure
  \end{algorithmic}
\end{algorithm}

This has time complexity $O(n+m)$.

\begin{prop}
  For any vertices $v$ and $w$, \Tfae:
  $v$ and $w$ are in the same SCC; and
  $v$ and $w$ are in the same DFS tree of $G\trans$ (sorted by decreasing finish time).
\end{prop}
\begin{prf}
  Suppose $v, w \in C \in SCC(G)$ and let $s$ be the first vertex visited in $C$.
  Then, $s \leadsto v$ within $C$ and the path is white when visiting $s$ by supposition.
  By the \nameref{lem:dfs:white-path}, $v$ will be in the DFS tree.
  Likewise for $w$.

  Suppose $v$ and $w$ are in a DFS tree $T$ for $G\trans$ rooted at $s$.
  That is, among the vertices in $T$, $s$ has the highest finish time.
  Let $t \in T$.
  As a descendent, $s \leadsto_{G\trans} t$, so $t \leadsto_G s$.

  Claim that $t$ descends from $s$ in $G$, so we get a path $s \leadsto_G t$.

  Proceed by structural induction on $t$ and its children.
  Let $u$ be a child of $t$ in $T$.
  Suppose $\vv{start}[s] \leq \vv{start}[t] < \vv{finish}[t] \leq \vv{finish}[s]$.
  Since $\vv{finish}[u] < \vv{finish}[s]$, we have by the \nameref{lem:dfs:paren}
  that either $[s\ (u)]$ or $(u)\ [s]$.
  But the second option is impossible because if $tu \in E(T) \subseteq E(G\trans)$,
  then $ut \in E(G)$, which means that $u \leadsto t$ and by the \nameref{lem:dfs:white-path},
  $t$ should be a descendant of $u$, not $s$.
  Therefore, $u$ is a descendant of $s$, as desired.

  Finally, because $s \leadsto_G t$ and $t \leadsto_G s$,
  $t$ is in the strongly connected component of $s$.
\end{prf}

\begin{problem}
  Does a graph $G$ contain a Hamiltonian path (i.e., a path $P$ with $P(V) = V$)?
\end{problem}

For an undirected graph $G$, this is one of the canonical NP-complete problems.

For a DAG $G$, we can do this in linear time with a topological ordering.

\begin{prop}
  A DAG $G$ has a Hamiltonian path if and only if it has a topological ordering
  $v_1 < \dotsb < v_k$ such that $v_iv_{i+1} \in E(G)$ for all $i$.
\end{prop}
\begin{prf}
  Let $G$ have a Hamiltonian path $P = v_1\dotsb v_k$.
  Define an ordering $v_1 < \dotsb < v_k$.
  Suppose $v_i v_j \in E(G)$.
  If $i > j$, then $v_i v_j v_{j+1} \dotsb v_i$ is a cycle.
  However, $G$ is a DAG, so we must have $i < j$.
  Therefore, $<$ is a topological ordering as desired.

  Suppose $G$ has a topological ordering $v_1 < \dotsb < v_k$
  with $v_iv_{i+1} \in E(G)$ for all $i$.
  Then, we immediately get a Hamiltonian path given by $v_1 \dotsb v_k$.
\end{prf}

\chapter{Greedy Algorithms}

\section{Introduction}
\lecture{(06/15)}

Suppose we are solving a \term{combinatorial optimization} problem,
i.e., a problem with a large (but finite) domain $\mathcal D$
such that we are trying to find an optimal solution $E \in \mathcal D$
that maximizes/minimizes some sort of cost function.

We will build $E$ step-by-step by taking the locally best solution.
Usually, it is very hard to prove correctness/optimality
but easy to find a counterexample.

For example, recall the Huffman encoding from CS 240.
We build the binary code tree by joining trees with the least frequencies.
This actually minimizes the length of the encoding.

\section{Interval Scheduling}

\begin{problem}
  Suppose we have $n$ intervals $[s_i, f_i]$.
  What is the subset of disjoint intervals with maximum length?
\end{problem}

We can show that a few naive greedy algorithms are wrong by drawing counterexamples:
\begin{itemize}
  \item Choose $\min_i s_i$: \tikz[yscale=0.25]{\draw (1,1) -- (10,1); \draw (2,2) -- (3,2); \draw (4,2) -- (5,2); \draw (6,2) -- (7,2); \draw (8,2)--(9,2);}
  \item Choose $\min_i\{f_i - s_i\}$: \tikz[yscale=0.25]{\draw (1,1) -- (4,1); \draw (3.5,2) -- (5.5,2); \draw (5,1) -- (8,1);}
  \item Choose minimum conflicts:
        \begin{tikzpicture}[yscale=-0.25]
          \draw (2.75,1) -- (4.25,1); \draw (2.75,0) -- (4.25,0); \draw (2.75,-1) -- (4.25,-1);
          \draw (6.75,1) -- (8.25,1); \draw (6.75,0) -- (8.25,0); \draw (6.75,-1) -- (8.25,-1);
          \draw (4.75,1) -- (6.25,1);
          \draw (2,2) -- (3,2); \draw (4,2) -- (5,2); \draw (6,2) -- (7,2); \draw (8,2)--(9,2);
        \end{tikzpicture}
\end{itemize}

However, we can prove that the greedy algorithm taking the earliest finish time is optimal.

\begin{algorithm}[H]
  \caption{\Call{IntervalScheduling}{$I = [[s_1,f_1],\dotsc,[s_n,f_n]]$}}
  \begin{algorithmic}[1]
    \State $S \gets \varnothing$
    \State $I \gets$ sort $I$ by finish time
    \For{$[s_i,f_i] \in I$}
    \If{$[s_i,f_i]$ has no conflicts in $S$}
    \State $S \gets S \cup \{[s_i,f_i]\}$
    \EndIf
    \EndFor
  \end{algorithmic}
\end{algorithm}

\begin{prf}
  Suppose $O$ is optimal. We must show $\abs{S} = \abs{O}$.

  Let $i_1,\dotsc,i_k$ be the intervals in $S$ ordered by their addition
  and likewise $j_1,\dotsc,j_m$ be the intervals in $O$ ordered by increasing finish time.

  We prove the claim that for all $r \leq k$, $f_{i_r} \leq f_{j_r}$.
  Proceed by induction on $r$.

  For $r=1$ this is true since $i_1$ is the interval with the earliest finish time.

  Suppose $r > 1$ and it is true for $r-1$.
  Then, $f_{i_{r-1}} \leq f_{j_{r-1}}$ by assumption and $f_{j_{r-1}} < s_{j_r}$
  by the order we set on $O$.
  Therefore, $f_{i_{r-1}} < s_{j_r}$.

  That is, at the time the greedy algorithm chose $i_{r-1}$, $j_r$ was an option.
  Since the greedy algorithm picks the earliest finish time, $f_{i_r} \leq f_{j_r}$.

  Now, suppose for a contradiction that $S$ is not optimal, i.e., $\abs{S} < \abs{O}$.
  Then, there must be a $j_{k+1}$.
  But by the above claim, $f_{i_k} \leq f_{j_k} < s_{j_{k+1}}$.
  This means $j_{k+1}$ was an option for the greedy algorithm,
  so it would not have stopped at $i_k$ and instead added $j_{k+1}$.

  Therefore, $S$ must be optimal.
\end{prf}

We call proofs of this kind, i.e., contradicting that greedy could not have
chosen an optimal solution, \term{greedy stays ahead}.

\section{Interval Colouring}\lecture{(06/20)}
\begin{problem}
  Suppose we have $n$ intervals $[s_i, f_i]$.
  Use the minimum number of colours to colour the intervals,
  so that each interval gets one colour
  and any overlapping intervals get different colours.
\end{problem}


\chapter{Dynamic Programming}
\section{Introduction}
\lecture{(07/04)}

Recall the \term{Fibonacci numbers} $F_n$ defined by
$F_0 = 0$, $F_1 = 1$, $F_n = F_{n-1} + F_{n-2}$ with the naive algorithm
\begin{algorithm}[H]
  \caption{\Call{Fib}{$n$}}
  \begin{algorithmic}[1]
    \If{$n=0$} \Return{0}
    \ElsIf{$n=1$} \Return{1}
    \Else \Return{$\Call{Fib}{n-1} + \Call{Fib}{n-2}$}
    \EndIf
  \end{algorithmic}
\end{algorithm}
Assuming we count additions as unit cost,
the runtime is $T(n) = F_{n+1} + 1 \in \Theta(\varphi^n)$ which is bad.

Notice that we are recomputing small $F_n$ a bunch of times,
but we actually only need each one once.
We can instead cache:
\begin{algorithm}[H]
  \caption{\Call{FibCached}{$n$}}
  \begin{algorithmic}[1]
    \Require $T \gets [0,1,\bot,\dotsc,\bot]$ global array of size $n$
    \If{$T[n] = \bot$} \State $T[n] \gets \Call{FibCached}{n-1} + \Call{FibCached}{n-2}$ \EndIf
    \State \Return{$T[n]$}
  \end{algorithmic}
\end{algorithm}
Also, notice that the dependency graph of which subproblems require each other is a DAG.
Therefore, we can take an order on the dependencies and iterate:
\begin{algorithm}
  \caption{\Call{FibIterative}{$n$}}
  \begin{algorithmic}[1]
    \State $T \gets [0,1,\bot,\dotsc,\bot]$ 0-indexed array of size $n$
    \For{$i = 2,\dotsc,n$}
    \State $T[i] \gets T[i-1] + T[i-2]$
    \EndFor
    \State \Return{$T[n]$}
  \end{algorithmic}
\end{algorithm}
This is our dynamic programming algorithm.
In fact, we can optimize even more by noticing that we can discard
all but the last two elements of the array,
giving a constant-space algorithm:
\begin{algorithm}[H]
  \caption{\Call{FibOptimal}{$n$}}
  \begin{algorithmic}[1]
    \State $(u,v) \gets (0,1)$
    \For{$i = 2,\dotsc,n$}
    \State $(u,v) \gets (v,u+v)$
    \EndFor
    \Return{$v$}
  \end{algorithmic}
\end{algorithm}
All these improved algorithms run in $O(n)$ time, a significant improvement.

We can give a general recipe for dynamic programming algorithms:
\begin{enumerate}[1.]
  \item \textbf{Identify the subproblem:}
        We are retaining solutions in an array.
        What are the dimensions of the array?
        What does each entry represent?
        Where will the final answer be in the array?
  \item \textbf{Establish DP-recurrence:}
        How does a subproblem contribute to a larger subproblem?
        What is the dependency between cells in the array?
  \item \textbf{Set base cases:}
        Initialize the array with some non-recursively defined base cases.
  \item \textbf{Specify the order of computation:}
        Clarify the DAG of subproblem dependencies.
        How does the algorithm maintain this order?
  \item \textbf{Recover the solution (if needed):}
        What subproblem answers provide the problem solution?
        How, if necessary, do we traceback the solution from the subproblems?
\end{enumerate}

We can often convert a DP algorithm into iterative loop(s).
Distinguish divide and conquer algorithms which do not always solve subproblems
and are not easily rewritten iteratively.

\section{Interval Scheduling}

\begin{problem}[weighted interval scheduling]
  Recall \Cref{prob:g:intsched}.
  Now, add a weight $w_i$ to each interval.
  We choose a subset $T \subseteq [n]$ which maximizes $W = \sum_{i \in T} w_i$.
\end{problem}

\begin{example}
  Let $I = [[2,8], [2,4], [5,6], [7,9]]$ with weights $[6,2,1,2]$.
\end{example}
\begin{sol}
  By inspection, since the weight $w_1 = 6 > 5 = w_2 + w_3 + w_4$,
  the solution is $T = [1]$ with $W = 6$.
\end{sol}

Notice that we can split on whether we accept the last interval $I_n$
and write for example that the optimal weight
\[
  W(I_1,\dotsc,I_n) = \begin{cases*}
    w_n + W(I_{m_1},\dotsc,I_{m_s}) & if we choose $I_n$ \\
    W(I_1,\dotsc,I_{n-1})           & if we do not
  \end{cases*}
\]
where $I_{m_1},\dotsc,I_{m_s}$ are the $s < n$ intervals not intersecting $I_n$.

Suppose we sort the intervals by finish time, i.e., $f_i \leq f_{i+1}$ for all $i$.
Then, we have $m_1,\dotsc,m_s = 1,\dotsc,j$ where $j = \max\{ i : f_i < s_n \}$
because $I_n$ is the last interval with the latest finish time,
so we only need to compare its start time with earlier intervals' finish times.
(If $j$ does not exist just return $w_n + 0$.)

We need to calculate the $j$-values for every $i$:
\begin{algorithm}[H]
  \caption{\Call{FindJs}{$A,s_1,\dotsc,s_n,f_1,\dotsc,f_n$}}
  \begin{algorithmic}[1]
    \State $j \gets$ array of size $n$
    \State $f_0 \gets \infty$
    \State $i \gets 1$
    \For{$k=0,\dotsc,n$}
    \While{$i \leq n$ and $f_k \leq s_{A[i]} < f_{k+1}$}
    \State $j[i] \gets k$
    \State $i\pp$
    \EndWhile
    \EndFor
    \State \Return{$j$}
  \end{algorithmic}
\end{algorithm}
where $A$ is a sorting permutation such that $(s_{A[i]})$ is non-decreasing.
This runs in $O(n\log n) + O(n) = O(n\log n)$ time.

Now, for the main procedure, we define $W[i]$ as the maximal weight
possible with the intervals $I_1,\dotsc,I_i$.

Then, for $W[0] = 0$ and $i \geq 1$, $W[i] = \max\{W[i-1], w_i + W[j[i]]\}$.

Since $W[i]$ depends only on entries in $W$ before it,
we can just iterate on $i = 1,\dotsc,n$ in $O(n)$ time.

\begin{algorithm}[H]
  \caption{\Call{IntervalScheduling}{$s_1,\dotsc,s_n,f_1,\dotsc,f_n,w_1,\dotsc,w_n$}}
  \begin{algorithmic}[1]
    \Require intervals are sorted by finish time
    \State $A \gets$ sorting permutation of $s_1,\dotsc,s_n$
    \State $j \gets \Call{FindJs}{A,s_1,\dotsc,s_n,f_1,\dotsc,f_n}$
    \State $W \gets$ 0-indexed array of size $n$
    \State $W[0] \gets 0$
    \For{$i = 1,\dotsc,n$}
    \State $W[i] \gets \max\{W[i-1], w_i + W[j[i]]\}$
    \EndFor
    \State \Return{$W[n]$}
  \end{algorithmic}
\end{algorithm}

This gives a total time for the algorithm of $O(n\log n) + O(n) = O(n\log n)$.

\section{Knapsack Problem}

\begin{problem}[0/1 knapsack]\label{prob:dp:k}
  Suppose we have items with weights $w_1,\dotsc,w_n$
  and values $v_1,\dotsc,v_n$ but our knapsack has capacity $W$.
  We want to select items $S \subseteq \{1,\dotsc,n\}$
  satisfying $\sum_{i \in S} w_i \leq W$
  and maximizes $\sum_{i \in S} v_i$.
\end{problem}

\begin{example}
  $\vb w = [3,4,6,5]$, $\vb v = [2,3,1,5]$, $W = 8$.
\end{example}
\begin{sol}
  The optimal $S = \{1,4\}$ with weight $3+5 = 8$ and value $2+5 = 7$.
\end{sol}

\lecture{(07/06)}
For each item $n$, we can either choose it or we can not.
Let $O[W,n]$ be best value for a knapsack of capacity $W$
and considering only the items $1,\dotsc,n$.
Then, $O[W,n]$ is either $v_n + O[W - w_n, n - 1]$ or $O[W, n - 1]$.

We can initialize $O[0, i] = 0$ for all $i$ and $O[w, 0] = 0$ for all $w$.
To be able to calculate $O[W,n]$, we must have already calculated
$O[W - w_n .. W, n-1]$.
In particular, if we iterate on $n$ first, we can guarantee that
the entire row $O[,n-1]$ exists before considering $O[W,n]$:

\begin{algorithm}[H]
  \caption{\Call{01KnapSack}{$v_1,\dotsc,v_n,w_1,\dotsc,w_n,W$}}
  \begin{algorithmic}[1]
    \State $O \gets$ 0-indexed array of size $(n+1) \times (W+1)$
    \State $O[0,] \gets \vb{0}$; $O[,0] \gets \vb{0}\trans$
    \For{$i = 1,\dotsc,n$}
    \For{$w = 1,\dotsc,W$}
    \If{$w_i > w$}
    \State $O[w,i] \gets O[w,i-1]$
    \Else
    \State $O[w,i] \gets \max\{v_n + O[W - w_n, n - 1], O[W, n - 1]\}$
    \EndIf\EndFor\EndFor
    \State \Return{$O[W,n]$}
  \end{algorithmic}
\end{algorithm}

The runtime here is obviously $\Theta(nW)$.
We call this \term{pseudo-polynomial} because it is polynomial in
$n$ (the size of the input) but also in $W$ (the \emph{value} of an input).
It is not polynomial because the size parameters are $n$ and $\lg W$,
but we have $n2^{\lg W}$.

\section{Subsequence Problems}

\begin{problem}[longest increasing subsequence]
  Find the longest (potentially discontinuous)
  increasing subsequence of an array $A[1..n]$ of integers.
\end{problem}
\begin{example}
  Given $A = [7,1,3,10,11,5,19]$, the longest increasing subsequence is
  $[1,3,10,11,19]$.
\end{example}

Notice that there are $\Theta(2^n)$ subsequences, so brute force is very bad here.

Suppose we try doing DP and storing $\ell[i]$ as the longest increasing subsequence of $A[1..i]$.
This doesn't work, since we can't immediately deduce $\ell[i+1]$ from just $\ell[i]$ and $A$.

We could instead store into $L[i]$ a pair of the length and the last entry $(\ell, c)$.
Then, we can add on the next element $L[i] \gets (\ell + 1, A[i])$,
but what is $L[i]$ if we do not select $A[i]$?

Alternatively, let $L[i]$ be the length of the longest increasing subsequence
of $A[1..i]$ that ends with $A[i]$. Then, $L[1] = 1$.
The longest increasing subsequence $S_i$ ending at $A[i]$
either looks like $[\dotsc,A[j],A[i]] = [\dots S_j, A[i]]$ for some $j$
or just $[A[i]]$.

\begin{algorithm}[H]
  \caption{\Call{LongestIncreasingSubsequence}{$A[1..n]$}}
  \begin{algorithmic}[1]
    \State $L \gets$ array of size $n$
    \State $L[1] \gets 1$
    \For{$i = 2,\dotsc,n$}
    \State $L[i] \gets 1$ \Comment{$S_i = [A[i]]$}
    \For{$j = 1,\dotsc,i-1$} \Comment{$S_i = [\dots S_j, A[i]]$}
    \If{$A[j] < A[i]$} 
    \State $L[i] \gets \max\{L[i], L[j] + 1\}$
    \EndIf
    \EndFor
    \EndFor
    \State \Return{$\max L$}
  \end{algorithmic}
\end{algorithm}

This algorithm runs in $\Theta(n^2)$ time which is much faster than $\Theta(2^n)$.
Note that we don't return the actual sequence here, only its length,
but it is trivial to find the sequence from the array $L$.

\begin{problem}[longest common subsequence]
  Given two arrays of characters (strings) $A[1..n]$ and $B[1..m]$,
  find the maximum length of a (potentially discontinuous) subsequence
  common to both $A$ and $B$.
\end{problem}
\begin{example}
  For $A = \vv{blurry}$ and $B = \vv{burger}$, we should return $\vv{burr}$ for $k=4$.
\end{example}

As with \Cref{prob:dp:k}, we have to work in a 2D problem space.
Let $M[i,j]$ be the longest subsequence length between $A[1..i]$ and $B[1..j]$.
Zero out $M[0,]$ and $M[,0]$.
Then, $M[i,j]$ will be the greatest of either 
(1) ignoring $B[j]$, (2) ignoring $A[i]$, or (3) adding $A[i] = B[j]$:

\begin{algorithm}
  \caption{\Call{LongestCommonSubsequence}{$A[1..n],B[1..m]$}}
  \begin{algorithmic}[1]
    \State $M \gets$ 0-indexed array of size $n+1 \times m+1$
    \State $M[0,] \gets \vb{0}$; $M[,0] \gets \vb{0}\trans$
    \For{$i = 1,\dotsc,n$}\For{$j = 1,\dotsc,m$}
    \State $M[i,j] \gets \max\{M[i,j-1], M[i-1,j]\}$
    \If{$A[i] = B[j]$} \State $M[i,j] \gets \max\{M[i,j], 1 + M[i-1,j-1]\}$ \EndIf
    \EndFor\EndFor
    \State \Return{$M[n,m]$}
  \end{algorithmic}
\end{algorithm}

Notice that because we iterate by $i$ first, $M[i-1,0..m]$ will have values.
Also, since we are iterating by increasing $j$, $M[i,1..j-1]$ will be calculated.
Therefore, this algorithm works and runs in $\Theta(nm)$ time.

\lecture{(07/11)}
\lecture{(07/13)}
\lecture{(07/18)}

\chapter{Complexity Theory}

\section{Introduction}
\lecture{(07/20)}

We consider in general decision problems.

\begin{defn}[decision problem]
  A map that takes a \term{problem instance}
  and returns a truth value.
  In general, a map $\vv{A} : \I(\vv{A}) \to \{0,1\}$.
  We call an instance where $\vv{A}(I) = 1$ a \term{yes-instance}
  (otherwise a \term{no-instance}).
\end{defn}

Note that we can consider a problem instance $I \in \I(\vv{A})$
as a natural number (formally) since we typically
represent them as some sort of binary string.

\begin{prop}
  Almost all decision problems are undecidable.
\end{prop}
\begin{prf}
  Notice that we can represent the map $\vv{A} : \N \to \{0,1\}$
  as a binary string $\vv{A}(0)\vv{A}(1)\cdots$.
  Then, assuming there is no structure here,
  we can perform a diagonal argument to show that
  the problem is uncountably infinite.

  However, solutions are a finite bit string,
  which are only countably infinite.
\end{prf}

\begin{defn}[\P]
  The class of decision problems that can be solved in polynomial time,
  i.e., in $O(n^p)$ time for $p > 0$.
\end{defn}

\begin{defn}[polynomial reduction]
  A problem $\vv A$ is reducible to a problem $\vv B$
  if there exists a function $f : \I(\vv A) \to \I(\vv B)$
  such that yes-instances are mapped to yes-instances
  (no-instances to no-instances)
  and $f$ is computable in $O(n^p)$ for some $p$.
  We write $\vv A \pleq \vv B$.
\end{defn}

As with normal orders, we say that
$\vv A \pequiv \vv B \iff \vv A \pleq \vv B \land \vv B \pleq \vv A$.

Suppose that $\vv A \pleq \vv B$ and $\vv B \in \vv P$.
Then, given an instance $I \in \mathcal I(\vv A)$,
we can solve it by solving $f(I)$ (which is computed in polynomial time)
as an instance of $\vv B$ (which is polynomial).

\begin{lemma}[transitivity of polynomial reduction]
  If $\vv A \pleq \vv B$ and $\vv B \pleq \vv C$,
  then $\vv A \pleq \vv C$.
\end{lemma}
\begin{prf}
  By definition, there exists $f : \I(\vv A) \to \I(\vv B)$
  and $g : \I(\vv B) \to \I(\vv C)$.
  Then, $g \circ f : \I(\vv A) \to \I(\vv C)$ and preserves yes-no.
\end{prf}

\begin{lemma}[proving hardness]
  Suppose $\vv A \pleq \vv B$ and we know that $\vv A \not\in \vv P$.
  Then, $\vv B \not\in \vv P$.
\end{lemma}
\begin{prf}
  By contradiction.
  If $\vv B \in \vv P$, then we could solve $\vv A$ in polynomial time.
\end{prf}

\section{Sample Reductions}

Recall from graph theory that a \term{clique} is a set of vertices $S \subseteq V(G)$
such that for all $u, v \in S$, $uv \in E(G)$.
Similarly, an \term{independent set} is a set $S$ of vertices such that
for all $u, v \in S$, $uv \not\in E(G)$.

\begin{prop}
  Consider the problems $\vv{Clique}$ (does a graph $G$ have a clique of size at least $k$?)\@
  and $\vv{IS}$ (does a graph $G$ have an independent set of size at least $k$?).
  Then, $\vv{Clique} \pequiv \vv{IS}$.
\end{prop}
\begin{prf}
  Notice that a clique is the ``inverse'' of an independent set
  (i.e., a clique is a set $S \subseteq V$ such that $S^2 \subseteq E$
  while an independent set is a set $S \subseteq V$ such that $S^2 \sqcup E$).

  Define the inverse graph $\bar G = (V, \bar E)$ where $\bar E = V^2 - E$.
  Then, given an instance of one problem $(G,k)$,
  we can make a call to the other problem $(\bar G,k)$.
  The inversion takes $O(m)$ time, which is polynomial.

  Therefore, $\vv{Clique} \pequiv \vv{IS}$.
\end{prf}

Recall again that a \term{vertex cover} is a set $S \subseteq V$
such that $\{uv\} \cap S \neq \varnothing$ for all $uv \in E$
(i.e., at least one endpoint of every edge is in $S$).

\begin{prop}
  $\vv{IS} \pequiv \vv{VC}$ (does $G$ contain a vertex cover of at most $k$ vertices?)
\end{prop}
\begin{prf}
  Notice that $S$ is a vertex cover if and only if $V \setminus S$ is an independent set.

  Suppose $S$ is a vertex cover but $V \setminus S$ is not independent.
  That is, there exists $x,y \in V \setminus S$ such that $xy \in E$.
  But at least one of $x$ or $y$ must be in $S$ since it is a vertex cover.
  Contradiction.

  Conversely, suppose $V \setminus S$ is independent but $S$ is not a vertex cover.
  Then, there is an edge $xy \in E$ with $x, y \not\in S$.
  But that means $x$ and $y$ are adjacent in the independent set $V \setminus S$.
  Contradiction.

  Therefore, $G$ has a vertex cover of size at most $k$ if and only if
  $G$ has an independent set of size at least $n-k$.
  We map $\vv{VC}$ instances $(G,k) \leftrightarrow (G,n-k)$ instances of $\vv{IS}$.
  This map obviously takes polynomial time, so $\vv{IS} \pequiv \vv{VC}$.
\end{prf}

Therefore, finding cliques, independent sets, and vertex covers are all equivalent
up to polynomial-time reducibility.

\lecture{(07/25)}

\begin{prop}\label{prop:c:hchp}
  Consider the problems $\vv{HC}$ (does a graph $G$ have a Hamiltonian cycle?)\@
  and $\vv{HP}$ (does a graph $G$ contain a Hamiltonian path?).
  Then, $\vv{HC} \pequiv \vv{HP}$.
\end{prop}
\begin{prf}
  Consider first the Hamiltonian $s,t$-path problem
  (does $G$ contain a Hamiltonian path $s \leadsto t$?)
  and that we are given an instance $(G, s, t)$.
  Then, $(G, s, t)$ is a yes-instance if and only if
  $G+x+sx+tx$ is a yes-instance of $\vv{HC}$.

  Likewise, a Hamiltonian cycle $s \leadsto ts$
  exists if and only if $(G - st)$ is a yes-instance
  of the Hamiltonian $s,t$-path problem.

  Since both of those operations are polynomial time
  removals/additions of vertices/edges,
  we have that the Hamiltonian $s,t$-path problem $\pequiv \vv{HC}$.

  Now, consider the actual problem $\vv{HP}$.
  We can show $\vv{HP} \pleq \vv{HC}$ in the same way.
  Consider a graph $G$ and add a vertex $x$ adjacent to all vertices.
  Then, this new graph $G'$ has a Hamiltonian cycle
  if and only if $G$ has a Hamiltonian path.

  Finally, to show $\vv{HC} \pleq \vv{HP}$,
  consider a graph $G$ with a Hamiltonian cycle
  $s \leadsto t \leadsto s$.
  Create $G'$ by splitting an arbitrary vertex $t$
  into two vertices $t$ and $t'$ such that $s \leadsto t$ and $t' \leadsto s$.
  \marginnote{how do we divide the edges of $t$?}
  Then, because of the way we divided the edges of $t$,
  a Hamiltonian path in $G'$ must start at $t$ and end at $t'$.
  It follows that $G$ has a Hamiltonian cycle if and only if
  $G'$ has a Hamiltonian path.

  Therefore, $\vv{HC} = \vv{HP}$.
\end{prf}

Recall from CS 245 that a binary function of $n$ variables is \term{satisfiable}
if there exists an assignment of truth values to variables that makes the expression true.
Recall also that we may write any binary function
in \term{conjunctive normal form}, i.e., as a conjunction
of a finite set of $m$ disjunctions of \term{literals} (either variables or their negations).

\begin{prop}
  Consider the problem $\vv{3SAT}$ (is a CNF formula of at most 3 literals per clause satisfiable?).
  Then, $\vv{3SAT} \pleq \vv{IS}$.
\end{prop}
\begin{prf}
  The reduction will take advantage of the fact that to make the whole formula true,
  we must select at least one literal from each disjunction to make true.

  Construct a graph of all the literals.
  Attach each literal in the same clause.
  Attach any two complementary literals.

  Then, ask if there is an independent set of size at least $m$.

  For correctness, suppose that $G$ has an independent set $S$ of size at least $m$.
  Assign each variable $x = \top$ if $\exists x \in S$ and $x = \bot$ if $\exists \bar x \in S$.
  Since literals are adjacent to their complements, $x$ will either be set to true or false
  (or neither, in which case we just assign $\top$ arbitrarily).
  Also, since literals are adjacent in $G$ if they are from the same clause,
  the $m$ elements of the independent set must come from each of the $m$ clauses
  by the pigeonhole principle.
  Therefore, this assignment satisfies $G$.

  Conversely, suppose there exists a satisfying assignment.
  Then, simply construct the according independent set.
  Because edges exist only between contradictory literals
  or between clauses, the set will indeed be independent.
  \marginnote{this feels wrong}

  Therefore, $\vv{3SAT} \pleq \vv{IS}$.
\end{prf}

\section{\NP-completeness}

Consider the \textsf{SubsetSum} problem,
where we are given a set of integers $S$
and must find a subset $T \subseteq S$ such that $\sum_{x \in T}x = 0$.
This problem is hard.

However, suppose an oracle gives us $T$ and claims it has sum 0.
We can write a helper function \Call{Verify}{$I,C$}
which returns $\vv{yes}$ if $I$ is a yes-instance
and $C$ is a valid \term{certificate} that proves $I$ is a yes-instance.
In our example, \Call{VerifySubsetSum}{$S, T$} checks that indeed
every element in $T$ is also in $S$ and also that they sum to 0.

\begin{defn}[\NP]
  The class of decision problems with yes-instances
  that can be verified in polynomial time.
  Equivalently, the class of decision problems that can be solved
  by a non-deterministic algorithm in polynomial time.
\end{defn}

\lecture{(07/27)}
For example, $\vv{3SAT}$ is in \NP because evaluating the clauses
given a valid variable truth-value assignment can be done in polynomial time.

Not all decision problems are in \NP:
for example, consider whether a graph is non-Hamiltonian.
The no-instances are easily verifiable, but the yes-instances are hard.

\begin{defn}[\coNP]
  The class of decision problems with no-instances
  that can be verified in polynomial time.
\end{defn}

Clearly, all problems in \P are also in \NP and \coNP:
simply solve the problem in polynomial time.
Therefore, $\P \subseteq \NP$ and $\P \subseteq \coNP$.
This leads to the most famous problem in computer science.

\begin{conjecture}
  $\P \stackrel{?}{=} \NP$
\end{conjecture}

To make deciding whether $\NP \subseteq \P$ easier,
we create a notion of the ``hardest'' problems in \NP.

\begin{defn}[\NP-complete]
  A problem $\vv{X} \in \NP$ is \NP-complete if for all $\vv{Y} \in \NP$,
  we have $\vv{Y} \pleq \vv{X}$.
  Then, we write $\vv{X} \in \NPC$.
\end{defn}

Then, it immediately follows that $\P = \NP \iff \exists\vv{X} \in \NPC, \vv{X} \in \P$.

\begin{theorem}[Cook--Levin]\label{thm:c:3sat}
  $\vv{3SAT} \in \NPC$
\end{theorem}

This is a useful theorem, since once we have on \NP-complete problem,
we can just show that any other problem, for example $\vv{IS}$,
is \NP-complete because $\vv{3SAT} \pleq \vv{IS}$.

\begin{prf}[sketch]
  Consider the $\vv{CircuitSAT}$ problem.
  We are given a DAG with labelled vertices.
  Inputs are marked with $x_1,\dotsc,x_n$.
  Internal vertices are marked by Boolean operators $\vv{and}$, $\vv{or}$, and $\vv{not}$.
  For example,
  \begin{center}
    \tikz{\graph[layered layout,math nodes,nodes={circle,draw}]{A/{\vv{and}}["$v$" below]<-{B/{\vv{and}}<-{x_1,x_2},C/{\vv{or}}<-{x_2,x_3}}};}
  \end{center}
  For a vertex $v$, is there some truth-value assignment to the $x_i$'s that makes $v$ true?

  We will show that $\vv{CircuitSAT}$ is \NP-complete,
  and then that $\vv{CircuitSAT} \pleq\vv{3SAT}$.

  Let $\vv{A} \in \NP$ and $S$ be a yes-instance of $\vv{A}$.
  We want to find an algorithm that checks of a certificate $t$
  can prove that $S$ is indeed a yes-instance.

  This verification algorithm is a Boolean function (i.e., it takes in $t$
  as input and outputs a Boolean value), so we can write it as a circuit
  (recall from CS 245 that $\land$, $\lor$, and $\lnot$ are sufficient
  to write any Boolean function).
  Therefore, we can just call $\vv{CircuitSAT}$ to find $t$,
  which is as good as solving $S$.

  Therefore, $\vv{A} \pleq \vv{CircuitSAT}$, and $\vv{CircuitSAT} \in \NPC$.

  Now, notice that we can transform the $\vv{CircuitSAT}$ DAG into
  a set of conjunctive clauses with at most three literals:
  \begin{center}
    \begin{tabular}{c|c|c}
      \tikz{\graph[layered layout,math nodes,nodes={circle,draw}]{B/{\vv{and}}["$v_i$" below]<-{v_j,v_k}};}
       & \tikz{\graph[layered layout,math nodes,nodes={circle,draw}]{B/{\vv{or}}["$v_i$" below]<-{v_j,v_k}};}
       & \tikz{\graph[layered layout,math nodes,nodes={circle,draw}]{B/{\vv{not}}["$v_i$" below]<-{v_j}};}    \\
      $(\bar v_i \lor v_j)$, $(\bar v_i \lor v_k)$, $(v_i \lor \bar v_j \lor \bar v_k)$
       & $(v_i \lor \bar v_j)$, $(v_i \lor \bar v_k)$, $(\bar v_i \lor v_j \lor v_k)$
       & $(v_i \lor v_j), (\bar v_i \lor \bar v_j)$
    \end{tabular}
  \end{center}
  Therefore, $\vv{CircuitSAT} \pleq \vv{3SAT}$,
  which means that $\vv{3SAT} \in \NPC$.
\end{prf}

For reference, a list of \NP-complete problems:
\begin{itemize}[nosep]
  \item 3SAT, SAT
  \item independent set, vertex cover, clique
  \item (directed) Hamiltonian cycle, Hamiltonian path
  \item travelling salesman
  \item subset sum
  \item 0/1 knapsack
\end{itemize}

We will show \NP-completeness for Hamiltonian cycles and paths.

\begin{theorem}
  $\vv{3SAT} \pleq \vv{DirectedHamiltonianCycle} \pleq \vv{HamiltonianCycle}$
\end{theorem}
\begin{prf}
  We begin by showing $\vv{3SAT} \pleq \vv{DirectedHamiltonianCycle}$.
  That is, given a formula, we must create a directed graph
  such that the formula is satisfiable if and only if the graph is Hamiltonian.

  To do this, we will first create a graph with $2^n$ Hamiltonian cycles
  corresponding to each possible truth value assignment.
  Then, we will add vertices to constrain the valid cycles to those
  consistent with the given clauses.

  TODO

  To convert a directed graph $G$ to an undirected graph $G'$,
  replace each vertex $v$ by $v_i$, $v_m$, and $v_o$:
  \begin{center}
    \begin{tikzpicture}[vertalign]
      \graph[empty nodes]{{1,2}->v[math nodes,y=-0.5]->{3,4}};
    \end{tikzpicture}
    becomes
    \begin{tikzpicture}[vertalign]
      \graph[empty nodes] {
        {1,2}--{[nodes={math nodes,y=-0.5}] v_i--v_m--v_o}--{3,4}};
    \end{tikzpicture}
  \end{center}
  Suppose there is a directed Hamiltonian cycle in $G$.
  Then, we just have to follow it in $G'$ to hit every vertex.
  If there is a Hamiltonian cycle in $G'$, it must go
  \begin{center}
    \begin{tikzpicture}[vertalign]
      \graph[empty nodes] {
        {1,2}--{[nodes={math nodes,y=-0.5}] v_i--v_m--v_o}--{3,4};
        {[edges={red}] 1--v_i--v_m--v_o--4}
        };
    \end{tikzpicture}
    and not
    \begin{tikzpicture}[vertalign]
      \graph[empty nodes] {
        {1,2}--{[nodes={math nodes,y=-0.5}] v_i--v_m--v_o}--{3,4};
        {[edges={red}] 1--v_i--2; v_m--v_o--4;}
        };
    \end{tikzpicture}
    because it is a cycle,
  \end{center}
  so we can construct the directed Hamiltonian cycle in $G$.

  This takes polynomial time, so we have $\vv{DirectedHamiltonianCycle} \pleq \vv{HamiltonianCycle}$.
\end{prf}

Then, by combining with \cref{thm:c:3sat,prop:c:hchp},
we have proved \NP-completeness for $\vv{HamiltonianCycle}$ and $\vv{HamiltonianPath}$.


\chapter{Final Review}
\lecture{(08/01)}

\begin{example}
  Suppose we want to schedule people be on call between time $S$ and $T$.
  If each of the $n$ people are available from $s_i$ to $t_i$,
  give a greedy algorithm to assign the minimum number of people.

  Assume that the input is already sorted by start time $S \leq s_1 \leq \dotsb \leq s_n \leq T$.
\end{example}
\begin{sol}
  Since we know we must start at $S$,
  consider all the intervals that start at $S$.
  Then, pick the one with the latest end time.

  For each subsequent selection, consider all the intervals with start times
  before the last chosen end time and then select the one with the latest end time.

  \begin{algorithm}[H]
    \caption{\Call{GreedyScheduleAssign}{$S,T,[s_1,t_1],\dotsc,[s_n,t_n]$}}
    \begin{algorithmic}[1]
      \State $O \gets \varnothing$
      \State $s \gets S$, $o \gets 1$
      \For{$i = 1,\dotsc,n$}
      \If{$s_i \leq s$}
      \If{$t_i \geq t_o$}
      \State $o \gets i$
      \EndIf
      \Else
      \State $O \gets O \cup \{o\}$
      \State $s \gets t_o$
      \State $o \gets i$
      \EndIf
      \EndFor
    \end{algorithmic}
  \end{algorithm}

  This runs in $O(n)$ time.
\end{sol}

\begin{example}
  An $i \times j$ rectangle is worth $P[i,j]$.
  Given an $n \times m$ rectangle, give a dynamic programming algorithm
  to find the optimal way to cut the rectangle into smaller rectangles.
\end{example}
\begin{sol}
  Let $M[i,j]$ be the optimal value of a rectangle after considering subdivisions.
  Then, $M[i,j]$ is either:
  \begin{itemize}
    \item $P[i,j]$, the value without cutting;
    \item $\displaystyle M_V(i,j) = \max\bigcup_{\mathclap{1 \leq k \leq i}}\{ M[k,j], M[i-k,j] \}$, the maximum value of a vertical cut; or
    \item $\displaystyle M_H(i,j) = \max\bigcup_{\mathclap{1 \leq k \leq j}}\{ M[i,k], M[i,j-k] \}$, the maximum value of a horizontal cut.
  \end{itemize}
  Finally, we just iterate.
  Each of the $O(nm)$ iterations takes $O(n+m)$ time, 
  so we have $O(n^2m + nm^2)$.
\end{sol}

\begin{example}
  Consider the problem $\vv{ModifiedSTPath}$:
  given an edge-weighted directed graph, is there a simple $s,t$-path
  (i.e., with no repeated vertices) with total weight at most $k$.
  What complexity class is $\vv{ModifiedSTPath}$ in?
\end{example}
\begin{sol}
  We can find shortest paths in polynomial time,
  so find a shortest path via Dijkstra and check if its weight is at most $k$.
  This means $\vv{ModifiedSTPath} \in \P$.
  Therefore, it is also in \NP and \coNP.
\end{sol}

\begin{example}
  Show that $\vv{ModifiedSTPath}$ is \NP-complete.
\end{example}
\begin{prf}
  We will show that $\vv{DirectedHamiltonianPath} \pleq \vv{ModifiedSTPath}$.

  Suppose we have a directed graph $G = (V,E)$ as input to $\vv{DirectedHamiltonianPath}$.
  Notice that a Hamiltonian path is just a simple path with $\abs{V}-1$ edges.

  If we give every edge in $G$ a weight of $-1$,
  then we can just count the edges that a path traverses.
  Let $G' = (V \cup \{s,t\}, E \cup \{sv, tv : v \in V\})$ and $w(e) = -1$.
  Then, we can call \Call{ModifiedSTPath}{$G', w, s, t, -(\abs{V}+1)$}
  which will respond ``yes'' exactly when there is a path of maximum length
  $\abs{V}+1$ and ``no'' otherwise.

  Therefore, $\vv{DirectedHamiltonianPath} \pleq \vv{ModifiedSTPath}$,
  which means $\vv{ModifiedSTPath}$ is \NP-complete, as desired.
\end{prf}

\phantomsection\addcontentsline{toc}{chapter}{Back Matter}

\renewcommand{\listtheoremname}{List of Problems}
\phantomsection\addcontentsline{toc}{section}{\listtheoremname}
\listoftheorems[ignoreall,onlynamed={problem}]

\renewcommand{\listtheoremname}{List of Named Results}
\phantomsection\addcontentsline{toc}{section}{\listtheoremname}
\listoftheorems[ignoreall,onlynamed={theorem,lemma,corollary}]

\printindex

\end{document}