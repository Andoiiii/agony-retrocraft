\chapter{Synchronization}
\lecture{(05/25)}
todo: slides 90 to 99

\lecture{(05/30)}
\begin{defn}[race condition]
  A program where the order of execution affects the program result.
  The pieces of code that can create a race condition
  by accessing a shared variable are called \term[critical section]{critical sections}.
\end{defn}

We can use a \term{lock} to provide \term{mutual exclusion}
(when exactly one of the code chunks runs in its entirety):

\begin{minted}{c}
  int volatile total = 0;
  bool volatile total_lock = false; // false means unlocked
                                    // true means locked
  void add() {                void sub() {
    int i;                      int i;
    for (i=0; i<N; i++) {       for (i=0; i<N; i++) {
      Acquire(&total_lock);       Acquire(&total_lock);
      total++;                    total--;
      Release(&total_lock);       Release(&total_lock);
    }                           }
  }                           }
\end{minted}

We can imagine implementing \mintinline{c}{Acquire} and \mintinline{c}{Release}
by spinning until we can proceed (this is a \term{spinlock}):

\begin{minted}{c}
  Acquire(bool *lock) {
    while (*lock == true) {} // spin until the lock is free
    *lock = true;            // grab the lock
  }

  Release(bool *lock) {
    *lock = false;           // give up the lock
  }
\end{minted}

This does not actually work because \mintinline{c}{Acquire}
could be preempted before grabbing the lock.
There are special assembly language instructions
which allow us to make this work and create a spinlock.

Since spinlocks use the processor while they wait,
they should not be used for long waiting times.
While a spinlock is spinning, we must also disable interrupts.

In OS/161, we can create and manipulate \texttt{spinlock} structs
using the methods \mintinline{c}{spinlock_init},
\mintinline{c}{spinlock_acquire}, and \mintinline{c}{spinlock_release}.

We can instead block the thread instead of eating up CPU time.
The OS/161 methods \mintinline{c}{lock_create},
\mintinline{c}{lock_acquire}, and \mintinline{c}{lock_release}
are analogous to the \mintinline{c}{spinlock} methods.
A lock is owned by a thread (since it only blocks a single thread)
and not a CPU (since a spinlock takes up an entire CPU while spinning).

While a thread is blocked by a lock,
it goes on a \term{wait channel} for that lock, i.e.,
a queue of threads that will be awoken when the lock is released.
In OS/161, we can use \mintinline{c}{wchan_lock}, \mintinline{c}{wchan_sleep},
\mintinline{c}{wchan_wakeall}, and \mintinline{c}{wchan_wakeone}.

If we need something more complex than a boolean lock,
we can use a \term{semaphore}, which holds an integer.
We can either call \texttt{P}(roberen) (lit.\ ``try'', waits until able to decrement)
or \texttt{V}(erhogen) (lit.\ ``increase'', increments).

There are three kinds of semaphore:
\begin{itemize}[nosep]
  \item \term[semaphore!binary]{binary}: 0 or 1, behaves like a lock
  \item \term[semaphore!counting]{counting}: an arbitrary number of resources
  \item \term[semaphore!barrier]{barrier}: force one thread to wait for others to complete, start count at 0
\end{itemize}
We do not need to call \texttt{V} after \texttt{P}.
We can also start at whatever initial value we want.
Semaphores also do not have owners.

\lecture{(06/01)}
In summary:
\begin{center}
  \begin{tabular}{c|c|c}
    Spinlocks              & Locks             & Sempahores           \\ \hline
    Consumes a CPU         & Owned by a thread & No ownership         \\
    Spins (no interrupts)  & Blocks            & Blocks               \\
    Binary (held/not held) & Binary            & Non-negative integer
  \end{tabular}
\end{center}
We implement semaphores and locks using spinlocks and wait channels
because we do not want a race condition on \mintinline{c}{sem->sem_lock}.
Simplified:
\begin{minted}{c}
  void P(struct semaphore *sem) {
    spinlock_acquire(&sem->sem_lock);     // lock the semaphore
    while (sem->sem_count == 0) {         // check if resources available
      wchan_lock(sem->sem_wchan);         // lock wait channel's queue
      spinlock_release(&sem->sem_lock);   // unlock the semaphore
      wchan_sleep(sem->sem_wchan);        // context switch (unlocks queue)
      spinlock_acquire(&sem->sem_lock);   // relock the semaphore
    }
    sem->sem_count--;                     // use a resource
    spinlock_release(&sem->sem_lock);     // unlock the semaphore
  }
\end{minted}
and
\begin{minted}{c}
  void V(struct semaphore *sem) {
    spinlock_acquire(&sem->sem_lock);     // lock the semaphore
    sem->sem_count++;                     // add a resource
    wchan_wakeone(sem->sem_wchan);        // unblock a thread on the queue
    spinlock_release(&sem->sem_lock);     // unlock the semaphore
  }
\end{minted}
We can abstract away from integers entirely and have a \term{condition variable}.
When a condition is true, the thread can run;
when it is not true, the thread waits until it becomes true.
If a thread sets the condition to true, it can \texttt{signal} one
or \texttt{broadcast} all blocked threads.

Two threads can \term{deadlock} if they are trying to acquire locks held by each other.
To avoid this, either have a standard order of acquisition or do retries (No Hold and Wait):
\begin{minted}{c}
  lock_acquire(lock1);
  while (!try_acquire(lock2)) {
    lock_release(lock1);
    lock_acquire(lock1);
  }
\end{minted}
but OS/161 does not have \mintinline{c}{bool try_acquire(struct lock *)}.