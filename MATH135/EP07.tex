\documentclass[11pt]{article}

\usepackage{physics}
\usepackage{amsfonts,amsmath,amssymb,amsthm}
\usepackage{enumerate}
\usepackage{titlesec}
\usepackage{fancyhdr}
\usepackage{multicol}

\headheight 13.6pt
\setlength{\headsep}{10pt}
\textwidth 15cm
\textheight 24.3cm
\evensidemargin 6mm
\oddsidemargin 6mm
\topmargin -1.1cm
\setlength{\parskip}{1.5ex}
\parindent=0pt

\author{James Ah Yong}

\pagestyle{fancy}
\fancyhf{}
\fancyfoot[c]{\thepage}
\makeatletter
\lhead{\@title}
\rhead{\@author}

\fancypagestyle{firstpage}{
  \fancyhf{}
  \rhead{\@author}
  \fancyfoot[c]{\thepage}
}

% Sets
\newcommand{\N}{\mathbb{N}}
\newcommand{\Z}{\mathbb{Z}}
\newcommand{\Q}{\mathbb{Q}}
\newcommand{\R}{\mathbb{R}}
\newcommand{\C}{\mathbb{C}}
\newcommand{\U}{\mathcal{U}}
\newcommand{\sym}{\mathbin{\triangle}}

% Functions
\DeclareMathOperator{\sgn}{sgn}
\DeclareMathOperator{\im}{im}

% Operators
\newcommand{\Rarr}{\Rightarrow}
\newcommand{\Larr}{\Leftarrow}
\usepackage{mathtools} % for \DeclarePairedDelimiter macro
\DeclarePairedDelimiter\ceil{\lceil}{\rceil}
\DeclarePairedDelimiter\floor{\lfloor}{\rfloor}

% Macros
% properly typeset ε-δ (epsilon en dash delta)
\newcommand{\epsdel}[1][\delta]{\ensuremath{\epsilon\mathit{\textnormal{--}}#1}}
\newcommand{\by}[1]{& \text{by #1}}
\newcommand{\IH}{\by{inductive hypothesis}}
% multiple choice (remove spacing between items)
\newenvironment{choices}
{\begin{enumerate}[(a)]
    \setlength{\parskip}{0ex}
    }{
  \end{enumerate}}

% Typesetting
\usepackage{array}   % for \newcolumntype macro
\newcolumntype{C}{>{$}c<{$}} % math version of "C" column type
\newcommand{\dlim}[2]{\displaystyle\lim_{#1\to#2}} % totally not \dfrac ripoff
\newcommand{\dilim}[1]{\dlim{#1}{\infty}} % infinite limits
\newcommand{\ilim}[1]{\lim_{#1\to\infty}}
\usepackage{cancel}

% Auto-number questions
\newcommand{\QType}{Q}
\renewcommand{\theparagraph}{\QType\ifnum\value{paragraph}<10 0\fi\arabic{paragraph}}
\setcounter{secnumdepth}{6}
\newcommand{\question}{\par\refstepcounter{paragraph}\textbf{\theparagraph}.\space}

% Question sections
\titleformat{\section}{\normalsize\bfseries}{\thesection}{1em}{}
\newcommand{\qsection}[2]{%
  \renewcommand{\QType}{#2}
  \section*{#1}
  \refstepcounter{section}
}

\title{MATH 135 Fall 2020: Extra Practice 7}

\begin{document}
\thispagestyle{firstpage}

\textbf{\@title}

\qsection{Warm-Up Exercises}{WE}

\question Find the complete integer solution to $7x+11y=3$.
\begin{proof}[Solution]
  Begin by applying the EEA to determine one solution for $x$ and $y$:
  \begin{center}
    \begin{tabular}{C|C|C|C}
      x  & y  & r  & q  \\ \hline
      1  & 0  & 7  & 0  \\
      0  & 1  & 11 & 0  \\
      -1 & 1  & 4  & -1 \\
      2  & -1 & 3  & 2
    \end{tabular}
  \end{center}
  which gives $7(2)+11(-1)=3$.
  Since 7 and 11 are prime, we immediately know their GCD is 1.
  Now, apply the LDET to determine the complete solution set:
  \begin{equation*}
    \{ (x,y) : x = 2 + 11n, y = -1 - 7n, n\in\Z \} \qedhere
  \end{equation*}
\end{proof}


\question Find the complete integer solution to $28x+60y=10$.
\begin{proof}[Solution]
  Begin by applying the EEA to find the GCD\@:
  \begin{center}
    \begin{tabular}{C|C|C|C}
      y  & x  & r  & q \\ \hline
      1  & 0  & 60 & 0 \\
      0  & 1  & 28 & 0 \\
      1  & -4 & 4  & 2 \\
      -7 & 29 & 0  & 7
    \end{tabular}
  \end{center}
  Therefore, $\gcd(28,60)=4$.
  However, $4 \nmid 10$, so there are no solutions to this equation.
\end{proof}


\qsection{Recommended Problems}{RP}

\question Find all non-negative integer solutions to $12x+57y=423$.
\begin{proof}[Solution]
  Since $12 = 3\times4$ and $57=3\times19$, clearly $\gcd(12,57)=3$.
  We also have that $423 \mid 3$, so solutions exist.
  Applying EEA, we have
  \begin{center}
    \begin{tabular}{C|C|C|C}
      y  & x  & r  & q \\ \hline
      1  & 0  & 57 & 0 \\
      0  & 1  & 12 & 0 \\
      1  & -4 & 9  & 4 \\
      -1 & 5  & 3  & 1
    \end{tabular}
  \end{center}
  so our base solution is $12(5)+57(-1)=3$.
  Multiplying through by $\frac{423}{3}=141$, we have $12(705)+57(-141)=423$.
  By the LDET, we arrive at our solution set in the integers:
  \begin{equation*}
    \{ (x,y) : x = 705 + 19n, y = -141 - 4n, n\in\Z \}
  \end{equation*}
  However, we want to restrict $x \geq 0$ and $y \geq 0$.
  Notice that $x \geq 0$ when $n \geq -\frac{705}{19}$, that is, $n \geq -37$.
  Likewise, $y \geq 0$ when $n \leq -\frac{141}{35}$, that is, $n \leq -36$.

  This just means that $-37 \leq n \leq -36$, or $n=-37,-36$.
  Therefore, the solution set is $(x,y)\in\{(2,7),(21,3)\}$.
\end{proof}


\question Prove or disprove the following implications:
\begin{enumerate}[(a)]
  \item For all integers $a$, $b$, and $c$, if there exists an integer solution to $ax^2+by^2=c$,
        then $\gcd(a,b) \mid c$.
        \begin{proof}
          Let $a$, $b$, and $c$ be integers.
          Suppose there is an integer solution in $x$ and $y$ to the equation $ax^2+by^2=c$.
          Since $x^2$ and $y^2$ are integers, this is a solution to the equation $as+bt=c$
          with integers $s=x^2$ and $t=y^2$.

          It immediately follows from the LDET that $\gcd(a,b) \mid c$.
        \end{proof}
  \item For all integers $a$, $b$, and $c$, if $\gcd(a,b) \mid c$,
        then there exists an integer solution to $ax^2+by^2=c$.
        \begin{proof}
          Consider the counterexample where $a=b=1$ and $c=-2$.
          We have that $\gcd(a,b)=\gcd(1,1)=1$ and clearly $1 \mid -2$.

          We now have the equation $(1)x^2+(1)y^2=-2$.
          From the properties of integers, $x^2 \geq 0$ and $y^2 \geq 0$, so $x^2+y^2 \geq 0$.
          Then, $x^2 + y^2 \geq 0$ but $-2$ is not non-negative.
          Therefore, no solutions to $x^2+y^2=-2$ exist.
        \end{proof}
\end{enumerate}


\question Consider the following statement:
For all integers $a$, $b$, $c$, and $x_0$, there exists an integer $y_0$ such that $ax_0+by_0=c$.
\begin{enumerate}[(a)]
  \item Carefully write down the negation of this statement and prove that this negation is true.
        \begin{proof}
          We prove the negation:
          \begin{center}
            There exist integers $a$, $b$, $c$, and $x_0$ such that for all integers $y_0$, $ax_0+by_0\neq c$.
          \end{center}
          Select $a=x_0=1$, $b=0$, and $c=2$.
          Let $y_0$ be an integer.
          We must show that $(1)(1)+(0)y_0 \neq (2)$.
          This is just $1 \neq 2$, which is true independent of $y_0$.
        \end{proof}
  \item Let $a,b,c\in\Z$.
        Fill in the blank to make the following statement true and prove that it is true.
        % \rule{3cm}{0.15mm}
        \fbox{$b$ is non-zero, $b \mid a$, and $b \mid c$}
        if and only if for all integers $x_0$,
        there exists an integer $y_0$ such that $ax_0+by_0=c$.
        \begin{proof}
          Let $a$, $b$, and $c$ be integers.
          
          We prove the biconditional by proving both implications.

          ($\Rarr$) Suppose $b$ is non-zero, $b \mid a$, and $b \mid c$.
          We break into cases on $a$:

          If $a=0$, then we must show that there exists a $y_0$ such that $by_0=c$.
          This follows immediately from the fact that $b \mid c$.

          If $a$ is non-zero, it follows that $\gcd(a,b)=|b|$.
          Then, since $b \mid c$, we have $\gcd(a,b) \mid c$.
          We may now apply the LDET\@.
          The solution set to the linear Diophantine equation $ax_0+by_0=c$ is
          \begin{equation*}
            \{(x_0,y_0) : x_0=x+\frac{b}{|b|}n, y_0=y+\frac{a}{|b|}n, n\in\Z\}
          \end{equation*}
          for some initial solution $(x,y)$.
          Since $n$ ranges through all integers, we may drop the absolute value bars.
          Then, $x_0=x+n$, so every integer $x_0$ appears in the solution set at $n=x_0-x$,
          with a corresponding $y_0$.

          Alternatively stated, for every integer $x_0$, there exists a $y_0$ such that $ax_0+by_0=c$.

          ($\Larr$) Suppose that for all integers $x_0$, we may choose an integer $y_0$ so $ax_0+by_0=c$.
          Let $x_0$ be an integer.

          Suppose for a contradiction that $b=0$, so $ax_0=c$.
          This is clearly not true for all $a$, $c$, and $x_0$.
          Therefore, $b$ is non-zero.
          
          Now, break into cases on $a$.
          Suppose that $a=0$. 
          Then, we may find $y_0$ such that $by_0=c$, which is the same as saying $b \mid c$.

          Suppose that $a$ is non-zero.
          Since both $a$ and $b$ are non-zero and $ax_0+by_0=c$ is a solution to the LDE $ax+by=c$,
          the LDET applies, giving $\gcd(a,b) \mid c$.

          However, since the LDET applies, there is an entire solution set given by
          \begin{equation*}
            \{(x,y) : x=x_0+\frac{b}{\gcd(a,b)}n, y=y_0+\frac{a}{\gcd(a,b)}n, n\in\Z\}
          \end{equation*}
          Now, recall that $x_0$ is an arbitrary integer.
          Therefore, the values of $x$ given in the set above must also span the integers, that is,
          any arbitrary integer $x$ may be written $x_0+\frac{b}{\gcd(a,b)}n$.

          This implies that $\frac{b}{\gcd(a,b)}=1$, that is, $b=\gcd(a,b)$, since GCD is positive.

          Therefore, $b$ is non-zero, $\gcd(a,b)=b$ divides $c$, and by definition, $b$ divides $a$.
        \end{proof}
\end{enumerate}


\question Suppose $a$ and $b$ are integers.
Prove that $\{ax+by : x,y\in\Z\} = {\{n\gcd(a,b) : n\in\Z\}}$.
\begin{proof}
  Let $a$ and $b$ be integers with GCD $d$.
  We prove $\{ax+by : x,y\in\Z\}=\{nd : n\in\Z\}$ by mutual containment.

  ($\subseteq$) Let $x$ and $y$ be integers.
  Then, since $d \mid a$ and $d \mid b$, $d \mid (ax+by)$.
  This means we may write $ax+by$ as $nd$, as desired.

  ($\supseteq$) Let $n$ be an integer.
  By Bézout's Lemma, we may write $d=xs+yt$ for integers $s$ and $t$.
  Multiplying through by $n$, we have $nd=(ns)x+(nt)y$.
  We may let $a=ns$ and $b=nt$, which are integers, and have $nd=ax+by$ as desired.

  Therefore, since the sets are mutually contained, they are equal.
\end{proof}

\emph{Note}: This is essentially a restatement of Jerry Wang's GCD derivation by subgroups.


\qsection{Challenge}{C}

\question For how many integer values of $c$ does $8x+5y=c$ have exactly one solution
where both $x$ and $y$ are strictly positive integers?


\end{document}