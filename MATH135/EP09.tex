\documentclass{agony}
\usepackage{minted}

\title{MATH 135 Fall 2020: Extra Practice 9}

\begin{document}
\thispagestyle{firstpage}

\textbf{\thetitle}

\qsection{Warm-Up Exercises}{WE}

\question Given the public RSA encryption key $(e, n) = (5, 35)$,
find the corresponding decryption key $(d, n)$.
\begin{proof}[Solution]
  We factor $n$ and find that $n = 5 \times 7$.
  Therefore, $p=5$ and $q=7$.

  We can now find the decryption key $d$ by solving $ed \equiv 1 \pmod{(p-1)(q-1)}$:
  \begin{align*}
    5d \equiv 1 \pmod{24}
  \end{align*}
  By inspection, $d=5$ is a solution.
  Because we have $1 < 5 < (p-1)(q-1)$, this is in fact the decryption key.

  Therefore, the decryption key is $(5, 35)$.
\end{proof}


\qsection{Recommended Problems}{RP}

\question Suppose that in setting up RSA, Alice chooses $p = 47$, $q = 37$, and $e = 25$.
\begin{enumerate}[(a)]
  \item What is Alice's public key?
        \begin{proof}[Solution]
          We have $n = pq = 1739$, so Alice's pubkey is $(25, 1739)$.
        \end{proof}
  \item What is Alice's private key?
        \begin{proof}[Solution]
          We solve the congruence $ed \equiv 1 \pmod{(p-1)(q-1)}$ or $25d \equiv 1 \pmod{1656}$
          which is equivalent to solving the LDE
          \begin{align*}
            25d + 1656y = 1
          \end{align*}
          We do this with the good `ole EEA\@:
          \begin{center}
            \begin{tabular}{C|C|C|C}
              y  & d   & r    & q  \\ \hline
              1  & 0   & 1656      \\
              0  & 1   & 25        \\
              1  & -66 & 6    & 66 \\
              -4 & 265 & 1    & 4
            \end{tabular}
          \end{center}
          and conclude that $d=265$ is a solution to our LDE\@.
          Since $1 < 265 < 1656$, it is in fact the decryption key.
          Therefore, Alice's privkey is $(265,1739)$.
        \end{proof}
  \item Suppose Alice wishes to send Bob the message $M = 20$.
        Bob's public key is $(23, 377)$ and Bob’s private key is $(263, 377)$.
        What is the cipher text corresponding to $M$?
        \begin{proof}[Solution]
          We compute the ciphertext $C$ as $C \equiv M^e \pmod{n}$ where $0 \leq C < n$.

          Substituting, $C \equiv 20^{23} \pmod{377}$.
          We perform the computation by hand like the masochistic math majors we are:
          \begin{align*}
            C & \equiv 20 \times 20^2 \times 20^4 \times 20^{16} \pmod{377} \\
              & \equiv 20 \times 23 \times 23^2 \times 23^8 \pmod{377}      \\
              & \equiv 20 \times 23 \times 152 \times 152^4 \pmod{377}      \\
              & \equiv 20 \times 23 \times 152 \times 107^2 \pmod{377}      \\
              & \equiv 83 \times 152 \times 139 \pmod{377}                  \\
              & \equiv 175 \times 139 \pmod{377}                            \\
              & \equiv 197 \pmod{377}
          \end{align*}
          and since we have $0 \leq 197 < 377$, this is indeed our cyphertext.
        \end{proof}
\end{enumerate}


\question Set up an RSA scheme using two-digit prime numbers.
Select values for the other variables and test encrypting and decrypting messages.
\begin{proof}[Solution]
  Let $p=11$ and $q=13$, the smallest two-digit prime numbers.
  Then, $n=pq=143$.
  Choose $e$ coprime to $(p-1)(q-1)=120$ to be $e=23$.
  To generate $d$, we solve $23d \equiv 1 \pmod{120}$, i.e., $23d + 120y = 1$, with the EEA\@:
  \begin{center}
    \begin{tabular}{C|C|C|C}
      y  & d   & r   & q \\ \hline
      1  & 0   & 120     \\
      0  & 1   & 23      \\
      1  & -5  & 5   & 5 \\
      -4 & 21  & 3   & 4 \\
      5  & -26 & 2   & 1 \\
      -9 & 47  & 1   & 1 \\
    \end{tabular}
  \end{center}
  Therefore, $d=47$, and we have the pubkey $(23,143)$ and privkey $(47,143)$.

  Suppose we want to send the ASCII exclamation mark ``!'', $M=33$.
  Then, we compute the ciphertext $C \equiv M^e \pmod{n}$, i.e., $C \equiv 33^{23} \pmod{143}$.
  Expanding and reducing to the remainder, $C = 132$.

  We decrypt by taking $R \equiv C^d \pmod{n}$, i.e., $R \equiv 132^{47} \pmod{143}$.
  Since in decryption we know $p$ and $q$, we equivalently solve both
  \[ R \equiv 132^{47} \pmod{11} \quad \text{and} \quad R \equiv 132^{47} \pmod{13} \]
  Simplifying by \FLT, we obtain
  \begin{align*}
    R \equiv 132^{7} \equiv 0 \pmod{11} \\
    R \equiv 132^{11} \equiv 7 \pmod{13}
  \end{align*}
  By the CRT, there is a unique solution modulo 143.
  We notice by inspection that $13(2) + 7 = 33 = 11(3)$, so $R=33$ is the received message.
\end{proof}


\qsection{Challenge}{C}
\question Write a computer program to implement RSA encryption and decryption.
\begin{proof}[Solution]
  Allow me to demonstrate just how overpowered Wolfram Mathematica is:
  \begin{minted}{mathematica}
    (* Generates RSA keypair by default *)
    keys = GenerateAsymmetricKeyPair[];
    msg = "This is cheating";
    cyphertext = Encrypt[keys["PublicKey"], msg];
    received = Decrypt[keys["PrivateKey"], cyphertext];
  \end{minted}
  Oh, you meant actually do the calculations? Okay.
  \begin{minted}{mathematica}
    (* Generate random primes below 100 *)
    {p, q} = RandomPrime[100, 2]; n = p*q;
    (* Generate e as a random coprime *)
    m = (p-1)(q-1);
    e = RandomChoice@Pick[Range[m], CoprimeQ[m, Range[m]]];
    (* Solve d automagically *)
    d = D /. Solve[e*D == 1, D, Modulus -> 120][[1]];

    (* Sample encryption/decryption of 42 *)
    C = PowerMod[42,e,n];
    R = PowerMod[C,d,n];
  \end{minted}
\end{proof}

\end{document}