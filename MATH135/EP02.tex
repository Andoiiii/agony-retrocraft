\documentclass{agony}
\title{MATH 135 Fall 2020: Extra Practice 2}

\begin{document}
\thispagestyle{firstpage}

\textbf{\thetitle}

\qsection{Warm-Up Exercises}{WE}

\question Let $A$, $B$ and $C$ be statement variables. Determine the truth table of $(A \land B) \implies \lnot C$.
\begin{center}
  \begin{tabular}{C|C|C|C|C|C}
    A & B & C & A \land B & \lnot C & (A \land B) \implies \lnot C \\ \hline
    T & T & T & T         & F       & F                            \\
    T & T & F & T         & T       & T                            \\
    T & F & T & F         & F       & T                            \\
    T & F & F & F         & T       & T                            \\
    F & T & T & F         & F       & T                            \\
    F & T & F & F         & T       & T                            \\
    F & F & T & F         & F       & T                            \\
    F & F & F & F         & T       & T                            \\
  \end{tabular}
\end{center}


\question State the contrapositive and the converse of the following implication: If Jane is a doctor, then she went to medical school.

\emph{Converse}: If Jane went to medical school, then she is a doctor.
\emph{Contrapositive}: If Jane did not go to medical school, then she is not a doctor.


\qsection{Recommended Problems}{RP}

\question For each of the following statements, identify the four parts of the quantified statement (quantifier, variables, domain, and open sentence).
Next, express the statement in symbolic form using as few words as possible and then write down the negation of the statement
(when possible, without using any negative words such as “not” or the $\lnot$ symbol, but negative math symbols like $\neq$ are okay).
Finally determine if the original statement is true or false. No justification is needed.

\underline{Note:} This is the same question as EP01/RP02(b) and (c).
\begin{enumerate}[(a)]
  \item For all real numbers $x$ and $y$, $x \neq y$ implies that $x^2+y^2 > 0$.
        \[ \forall x\in\R, \forall y\in\R, x \neq y \implies x^2+y^2>0 \]
        Quantifier: universal;
        variable: $x$;
        domain: $\R$;
        open sentence: $x \neq y \Rarr x^2+y^2>0$.
        Negation: $\exists x\in\R, \exists y\in\R, x \neq y \land x^2+y^2\leq0$.
        The statement is \emph{true}.
  \item For every even integer $a$ and odd integer $b$, a rational number $c$ can always be found such that $a<c<b$ or $b<c<a$.
        \[
          \forall a\in\Z, \forall b\in\Z, \exists c\in\Q,
          \left(\frac{a}{2}\in\Z \land \frac{b-1}{2}\in\Z\right) \implies (a<c<b \lor b<c<a)
        \]
        Quantifier: universal/existential;
        variables: $a$, $b$, $c$;
        domain: $\Z$, $\Q$;
        open sentence: $(\frac{a}{2}\in\Z \land \frac{b-1}{2}\in\Z) \Rarr (a<c<b \lor b<c<a)$.
        Negation:
        \[
          \exists a\in\Z, \exists b\in\Z, \forall c\in\Q,
          \qty(\frac{a}{2}\in\Z \land \frac{b-1}{2}\in\Z) \land
          \qty((c \leq a \lor c \geq b) \land (c \leq b \lor c \geq a))
        \]
        The statement is \emph{true}.
\end{enumerate}


\question Let $A$ and $B$ be statement variables. Prove that $(\lnot A) \lor B$ is logically equivalent to $\lnot(A \land \lnot B)$.
\begin{proof}
  Apply De Morgan's law: $(\lnot A) \lor B \equiv \lnot(A \land \lnot B)$.
\end{proof}


\question Let $A$ and $B$ be statement variables. Determine whether $A \implies B$ is logically equivalent to $(\lnot A) \lor B$.
\begin{proof}
  $A \implies B$ is defined as $\lnot(A \land \lnot B)$.
  This is easily verifiable by noticing that an implication is only false when the hypothesis is true but the conclusion is false.
  Expand using De Morgan's law: $\lnot(A \land \lnot B) \equiv (\lnot A \lor B)$.
\end{proof}


\question Assume that it has been established that the following implication is true:
\begin{center}
  If I don’t see my advisor today, then I will see her tomorrow.
\end{center}
For each of the sentences below, determine if it is true or false. No justification is needed.
If you can’t determine the truth value of the sentence, explain why.
\begin{enumerate}[(a)]
  \item I don’t meet my advisor both today and tomorrow. (This is arguably an ambiguous English sentence. Answer the problem using both interpretations.)

        For the case of not today and not tomorrow, the statement is contradictory.
        For the case of today or tomorrow, exclusive, see (c).

  \item I meet my advisor both today and tomorrow.

        Not contradictory, but the truth value is indeterminate because we do not know about meeting ``today''.

  \item I meet my advisor either today or tomorrow (but not on both days).

        Not contradictory, but the truth value is indeterminate because we do not know about meeting ``today''.
\end{enumerate}


\question Let $A$, $B$ and $C$ be statement variables.
Prove the following logical equivalence using a chain of logical equivalences as in Chapter 2.3 of the notes.
\begin{equation*}
  (A \land C) \lor (B \land C) \equiv \lnot((A \lor B) \implies \lnot C)
\end{equation*}

\begin{proof}
  Begin by considering the implication on the right-hand side.
  Recall the definition of an implication $X \implies Y \equiv \lnot X \lor Y$.
  Apply this and simplify:
  \begin{align*}
    \lnot((A \lor B) \implies \lnot C) & \equiv \lnot(\lnot (A \lor B) \lor \lnot C)                                           \\
                                       & \equiv \lnot(\lnot (A \lor B)) \land \lnot(\lnot C) & \text{De Morgan's law}          \\
                                       & \equiv (A \lor B) \land C                           & \text{Double negation}          \\
                                       & \equiv (A \land C) \lor (B \land C)                 & \text{Distributive conjunction}
  \end{align*}
  Hence, the left side is logically equivalent to the right side, so the equivalency holds.
\end{proof}


\question Four friends: Alex, Ben, Gina and Dana are having a discussion about going to the movies.
Ben says that he will go to the movies if Alex goes as well.
Gina says that if Ben goes to the movies, then she will join.
Dana says that she will go to the movies if Gina does.
That afternoon, exactly two of the four friends watch a movie at the theatre.
Deduce which two people went to the movies.
\begin{proof}
  For each friend, let $A$, $B$, $G$, and $D$ be if they go to the movies, respectively.
  We can write our statements as implications: $A \implies B$, $B \implies G$, and $G \implies D$.
  By the transitivity of the implication, $A \implies G$, $A \implies D$, and $B \implies D$.
  Recall that only two of $A$, $B$, $G$, and $D$ are allowed to be simultaneously true.
  If $A$ is true, then all of $B$, $G$, and $D$ are true, which is a contradiction.
  Therefore, $A$ is false.
  If $B$ is true, then both $G$ and $D$ are true, which is a contradiction.
  Therefore, $B$ is false.
  This leaves $G$ (which implies $D$) and $D$ to be true, which satisfies our exclusivity condition.
  Therefore, Gina and Dana atttended the movies.
\end{proof}


\question Consider the following statement.
\begin{center}
  For all $x\in\R$, if $x^6 + 3x^4 - 3x < 0$, then $0 < x < 1$
\end{center}
\begin{enumerate}
  \item Rewrite the given statement in symbolic form. \\
        $\forall x\in\R, x^6 + 3x^4 - 3x < 0 \implies 0 < x < 1$
  \item State the hypothesis of the implication within the given statement. \\
        $x^6 + 3x^4 - 3x < 0$
  \item State the conclusion of the implication within the given statement. \\
        $0 < x < 1$
  \item State the converse of the implication within the given statement. \\
        $0 < x < 1 \implies x^6 + 3x^4 - 3x < 0$
  \item State the contrapositive of the implication within the given statement. \\
        $x \leq 0 \lor x \geq 1 \implies x^6 + 3x^4 - 3x \geq 0$
  \item State the negation of the given statement without using the word ``not'' or the $\lnot$ symbol (but symbols such as $\neq$, $\nmid$, etc.\ are fine). \\
        $\exists x\in\R, x^6 + 3x^4 - 3x < 0 \land (x \leq 0 \lor x \geq 1)$
\end{enumerate}

\end{document}