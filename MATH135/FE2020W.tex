\documentclass{agony}
\title{MATH 135 Winter 2020: Final Assignment}

\begin{document}
\thispagestyle{firstpage}
\textbf{\thetitle}

\begin{prob}
  Let $p,q,r$ be distinct primes.
  Determine $\gcd(p^{10}q^{20}r^{30},(p^2qr^2)^{10})$ in terms of $p,q,r$.
\end{prob}
\begin{sol}
  Since $p$, $q$, and $r$ are distinct, the quantities $p^{10}q^{20}r^{30}$
  and $(p^2qr^2)^{10} = p^{20}q^{10}r^{20}$ are in their UPF form.
  Then, apply GCD PF\@: $\gcd(p^{10}q^{20}r^{30},p^{20}q^{10}r^{20}) = p^{10}q^{10}r^{20}$.
\end{sol}

\begin{prob}
  Given that $[x_0] = [6]$ is a solution to $[12][x] = [8]$ in $\Z_{64}$, write down the complete solution.
  Express your answer(s) in the form $[a]$, where $a$ is an integer and $0 \leq a < 64$.
\end{prob}
\begin{sol}
  By the Modular Arithmetic Theorem, there are $\gcd(12,64) = 4$ solutions,
  which are of the form $[6+\frac{64}{4}k]$ for $0 \leq k < 4$.
  That is, $[x]$ is one of $[6]$, $[22]$, $[38]$, or $[54]$.
\end{sol}

\prob{Determine the units digit (i.e., the ones digit) of $7^{202}$.}
\begin{sol}
  We must evaluate $7^{202} \pmod{10}$.
  Since $7^2 \equiv 49 \ equiv -1 \pmod 10$, it follows that $7^{202} \equiv (7^2)^{101} \equiv (-1)^{101} \equiv -1 \equiv 9 \pmod 10$.

  Therefore, the last digit is 9.
\end{sol}

\prob{Write $(2-2i)^6$ in standard form.}
\begin{sol}
  Notice that $2-2i = 2(1-i) = 2\sqrt{2}\cis(-\frac\pi4)$.
  Then, we distribute and apply DMT\@: $(2\sqrt{2}\cis(-\frac\pi4))^6 = (2\sqrt{2})^6\cis(-\frac{3\pi}{2}) = 512\cis(\frac{\pi}{2})$.

  It follows that in standard form, $(2-2i)^6 = 0 + 512i$.
\end{sol}

\begin{prob}
  Find all $z\in\C$ that satisfy the equation $z^6 = 32z$.
  You may express your solution(s) in polar form.
\end{prob}
\begin{sol}
  We have $z^6 = 32z \iff z^5 = 32$.
  In polar form, $32 = 32 \cis 0$.
  By CNRT, we have the fifth roots of 32 are
  \[ 2\cis 0, 2\cis\frac{2\pi}{5}, 2\cis\frac{4\pi}{5}, 2\cis\frac{6\pi}{5}, 2\cis\frac{8\pi}{5} \qedhere \]
\end{sol}

\begin{prob}
  Determine all integer solutions $(x,y)$ to the linear Diophantine equation $21x+15y=72$
  such that $x \geq 0$ and $y \geq 0$.
\end{prob}
\begin{sol}
  We apply the EEA\@:
  \begin{center}
    \begin{tabular}{C|C|C|C}
      x  & y  & q  & r \\ \hline
      1  & 0  & 21     \\
      0  & 1  & 15     \\
      1  & -1 & 6  & 1 \\
      -2 & 3  & 3  & 2
    \end{tabular}
  \end{center}
  We can stop since $3 \mid 6$ and conclude $\gcd(21,15) = 3$.
  Now, $21(-2) + 15(3) = 3$ and multiplying through by 24, we have $21(-48) + 15(72) = 72$.

  It follows by the LDET that the set of all solutions is given by
  \[ \{ (-48 + 5n, 72 - 7n) : n \in \Z \} \]
  If both $x$ and $y$ are positive, then $-48 + 5n > 0 \iff n > \frac{48}{5} \iff n \geq 10$
  and $72 - 7n > 0 \iff n < \frac{72}{7} \iff n \leq 10$.

  The only such value is $n = 10$ so the only such solution is $x = 2$ and $y = 2$.
\end{sol}

\begin{prob}
  Let $z,w\in\C$ such that $|z| = |w| = 2$ and $z\overline{w}=1+i$.
  Determine $|z-w|^2$.
\end{prob}
\begin{sol}
  Let $z = a+bi$ and $w = c+di$ be complex numbers with modulus 2 where $z\overline{w} = 1+i$.
  Then, by definition, $a^2 + b^2 = c^2 + d^2 = \sqrt{2}$ and $(a+bi)(c-di) = 1+i$.
  From the second equation, we have $(ac + bd) + (bc - ad)i = 1+i$.
  Equating real parts, $ac + bd = 1$.
  Now, \begin{align*}
    |z-w|^2 & = |(a-c) + (b-d)|^2                      \\
            & = (a-c)^2 + (b-d)^2                      \\
            & = a^2 - 2ac + c^2 + b^2 - 2bc + d^2      \\
            & = (a^2 + b^2) + (c^2 + d^2) - 2(ac + bd) \\
            & = \sqrt{2} + \sqrt{2} - 2(1)             \\
            & = 2\sqrt{2} - 2 \qedhere
  \end{align*}
\end{sol}

\begin{prob}
  You are an eavesdropper who has intercepted the ciphertext $C = 9$ sent using RSA\@.
  You have obtained the public key $(29, 91)$ and have managed to factor $n = 91$ as $7 \cdot 13$.

  Determine the original message $M$.
\end{prob}
\begin{sol}
  Let $p = 7$ and $q = 13$, so our secret modulus is $6 \cdot 12 = 72$.
  We determine the privkey $d$ knowing that $ed \equiv 29d \equiv 1 \pmod{72}$.
  Solving by SMT, $29d \equiv 5d \equiv 1 \pmod{8}$ and $29d \equiv 2d \equiv 1 \pmod{9}$.

  From the first congruence, by inspection $d=5$ works, so LCT gives $d\equiv 5 \pmod 8$ as the full solution set. So, $d = 8k + 5$ for some integer $k$.
  Substituting, $2(8k + 5) \equiv 16k + 10 \equiv 7k + 10 \equiv 1 \pmod{9}$.
  Then, $7k \equiv 0 \pmod{9}$ and by inspection $k \equiv 0 \pmod{9}$.
  Finally, $d = 8(9n) + 5 = 72n + 5$ with integer $n$, or, $d \equiv 5 \pmod{72}$. Indeed, $0<d<72$.
  
  Therefore, $d = 5$.

  We decode the message knowing $M \equiv C^d \equiv 9^{5} \pmod{91}$.
  Repeatedly squaring, we have $9^2 \equiv 81 \equiv -10 \pmod{91}$, and $9^4 \equiv 100 \equiv 9 \pmod{91}$.

  Therefore, $M \equiv 9^{4+1} \equiv (9)(9) \equiv 9^2 \equiv -10 \equiv 81 \pmod{91}$, so $M = 81$.
\end{sol}

\begin{prob}
  It is known that $3i$ is a root of the polynomial $f(x) = 2x^5 - 5x^4 + 18x^3 - 44x^2 + 9$.
  \begin{enumerate}[(a)]
    \item Write $f(x)$ as a product of irreducible polynomials in $\C[x]$.
          \begin{sol}
            The CPN gives that $f(x)$ has 5 complex roots, so we must find 5 complex linear factors.
            By the CJRT, $-3i$ is also a root of $f(x)$.
            Then, by the Factor Theorem, $(x-3i)(x+3i) = (x^2+9) \mid f(x)$.
            By long division: \[ \polylongdiv{2x^5 - 5x^4 + 18x^3 - 44x^2 + 9}{x^2 + 9} \]
            Inspecting candidates from the Rational Roots Theorem, we find $f(\frac12) = 0$.
            
            We divide by $(2x-1)$:
            \[ \polylongdiv{2x^3 - 5x^2 + 1}{2x-1} \]
            Finally, the quadratic formula gives $f(1\pm\sqrt{2}) = 0$.
            From these five roots, we multiply the of irreducible first degree factors to get
            \[ f(x) = (x-3i)(x+3i)(2x-1)(x-1+\sqrt{2})(x-1-\sqrt{2}) \qedhere \]
          \end{sol}
    \item Write $f(x)$ as a product of irreducible polynomials in $\R[x]$.
          \begin{sol}
            Since $\R[x] \subsetneq \C[x]$, we can consider the factorization from (a).
            From (a), the only factors not in $\R[x]$ are $(x-3i)$ and $(x+3i)$. Then,
            \[ f(x) = (x^2+9)(2x-1)(x-1+\sqrt{2})(x-1-\sqrt{2}) \qedhere \]
          \end{sol}
    \item Write $f(x)$ as a product of irreducible polynomials in $\Q[x]$.
          \begin{sol}
            Again, $\Q[x] \subsetneq \R[x]$.
            The only factors in (b) not in $\Q[x]$ are $(x-1\pm\sqrt{2})$. Then,
            \[ f(x) = (x^2+9)(2x-1)(x^2-2x-1) \qedhere \]
          \end{sol}
  \end{enumerate}
\end{prob}

\begin{prob}
  True or False. Indicate whether each statement is true or false.
  \begin{enumerate}[(a)]
    \item For all $f(x) \in \R[x]$, if $f(x)$ has no real roots, then $f(x)$ is irreducible in $\R[x]$.

          True \quad \fbox{False} \qquad \emph{Counterexample: take $f(x)=x^4+2x^2+1=(x^2+1)(x^2+1)$}
    \item $\{x \in \Z : \gcd(x,20) = 1\} = \{y \in \Z : \gcd(2y,40) = 2\}$.

          \fbox{True} \quad False \qquad \emph{Use BL to simplify RHS into LHS}
    \item There are infinitely many integers $x$ satisfying the simultaneous congruence
          \begin{align*}
            2x & \equiv 4 \pmod 8 \\ x+1 & \equiv 5 \pmod 7
          \end{align*}
          \fbox{True} \quad False \qquad \emph{Simplifies to $x \equiv 18 \pmod{28}$}
    \item For every $a\in\Z$, the LDE $(2a+1)x + ay = 1$ has a solution.

          \fbox{True} \quad False \qquad \emph{Since $\gcd(2a+1,a) = \gcd(a,1) = 1$}
    \item In $\Z_{48}$, the equation $[9][x] = [4]$ has exactly 3 solutions.

          True \quad \fbox{False} \qquad \emph{There are none.}
    \item For all $d \in \Z$, if $d \mid 10$ and $d \mid 15$
          and $d \mid 10s + 15t$ for some $s,t\in\Z$, then $d = 5$.

          True \quad \fbox{False} \qquad \emph{No special $d$ by DIC}
    \item For all polynomials $f(x)$ with integer coefficients, if $f(\frac{\sqrt{2}}{1+i}) = 0$,
          then $f(\frac{1+i}{\sqrt{2}}) = 0$.

          \fbox{True} \quad False \qquad \emph{Since they are $\frac{\sqrt{2}}{2} \pm \frac{\sqrt{2}}{2}i$}
  \end{enumerate}
\end{prob}

\begin{prob}
  Prove that there does not exist an integer $x$ such that $x^2 \equiv 5 \pmod 6$.
\end{prob}
\begin{prf}
  We exhaust the values of $x \pmod 6$:
  \begin{center}
    \begin{tabular}{C|C|C|C|C|C|C}
      x \pmod{6}   & 0 & 1 & 2 & 3 & 4 & 5 \\ \hline
      x^2 \pmod{6} & 0 & 1 & 4 & 3 & 4 & 1
    \end{tabular}
  \end{center}
  Notice that no $x$ satisfies $x^2 \equiv 5 \pmod 6$.
\end{prf}

\begin{prob}
  Let $p$ be an odd prime, and let $a$ be an odd integer such that $p \nmid a$.
  Prove that \[ a^{p-1} \equiv 1 \pmod{2p}. \]
\end{prob}
\begin{prf}
  Let $p$ be an odd prime, that is, $p \neq 2$, and $a$ be an odd integer not a multiple of $p$.
  By \FLT, $a^{p-1} \equiv 1 \pmod p$.
  Since $a$ is odd, $a \equiv 1 \pmod 2$ and $a^{p-1} \equiv 1 \pmod 2$ by CP\@.
  Then, by SMT, $a^{p-1} \equiv 2 \pmod{2p}$.
\end{prf}

\begin{prob}
  Prove that for all $a,b,c \in \Z$, $c \mid \gcd(a,c) \cdot \gcd(b,c)$ if and only if $c \mid ab$.
\end{prob}
\begin{prf}
  Let $a$, $b$, and $c$ be integers, and say $\gcd(a,c) = g$ and $\gcd(b,c) = h$.
  Then, by Bézout's Lemma, we can write $g = as + ct$ and $h = bu + cv$ for some integers $s,t,u,v$.
  Expanding, $gh = (as + ct)(bu + cv) = asbu + ascv + ctbu + c^2tv = ab(su) + c(asv+tbu+ctv)$.

  ($\Rarr$) Suppose that $c \mid gh$.
  By definition, $g \mid a$ and $h \mid b$.
  Then, $gn = a$ and $hm = b$ for some integers $n$ and $m$.
  It follows that $gh(nm) = ab$ so $gh \mid ab$.
  Finally, by TD, $c \mid ab$.

  ($\Larr$) Suppose that $c \mid ab$.
  Then, since $c \mid ab$ and $c \mid c$, by DIC as $su$ and $asv+tbu+ctv$ are integers, $c \mid gh$,
  finishing the proof.
\end{prf}

\begin{prob}
  Let $\theta \in \R$ be such that $2\sin\theta\cos\theta = \frac{1}{\sqrt{2}}$.
  Prove that $\sin\theta+\cos\theta$ is irrational.
\end{prob}
\begin{prf}
  Let $\theta$ be a real number and $2\sin\theta\cos\theta = \sin 2\theta = \frac{1}{\sqrt{2}}$.
  Then, WLOG, we restrict $0 \leq \theta < 2\pi$,
  so that $2\theta = \frac{\pi}{4}$ and $\theta = \frac{\pi}{8}$.

  Now, recall the half-angle formulae for sine and cosine. We have
  \[ \sin\theta = \sin(\frac{\pi/4}{2}) = \sqrt{\frac{1-\cos(\pi/4)}{2}} = \frac{\sqrt{2-\sqrt{2}}}{2} \]
  and
  \[ \cos\theta = \cos(\frac{\pi/4}{2}) = \sqrt{\frac{1+\cos(\pi/4)}{2}} = \frac{\sqrt{2+\sqrt{2}}}{2} \]
  Then, $\sin\theta + \cos\theta = \frac{\sqrt{2-\sqrt{2}}+\sqrt{2+\sqrt{2}}}{2}$.
  Let $a = \sin\theta+\cos\theta$, so that
  \begin{align*}
    2a          & = \sqrt{2-\sqrt{2}}+\sqrt{2+\sqrt{2}} \\
    4a^2        & = 4+2\sqrt{2}                         \\
    (a^2 - 1)^2 & = 2                                   \\
    0           & = a^4 - 2a^2 -1
  \end{align*}
  Let $f(x) = x^4 - 2x^2 - 1$ so that $f(a) = 0$ and $a$ is a root of $f$.
  Then, the Rational Roots Theorem states that candidates for rational roots of $f$ are $\pm 1$.
  However, $f(1) = -2$ and $f(-1) = -2$.
  Therefore, there are no rational roots of $f$, so $a$ is irrational.
\end{prf}

\end{document}