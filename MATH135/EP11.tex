\documentclass[11pt]{article}

\usepackage{physics}
\usepackage{amsfonts,amsmath,amssymb,amsthm}
\usepackage{enumerate}
\usepackage{titlesec}
\usepackage{fancyhdr}
\usepackage{multicol}

\headheight 13.6pt
\setlength{\headsep}{10pt}
\textwidth 15cm
\textheight 24.3cm
\evensidemargin 6mm
\oddsidemargin 6mm
\topmargin -1.1cm
\setlength{\parskip}{1.5ex}
\parindent=0pt

\author{James Ah Yong}

\pagestyle{fancy}
\fancyhf{}
\fancyfoot[c]{\thepage}
\makeatletter
\lhead{\@title}
\rhead{\@author}

\fancypagestyle{firstpage}{
  \fancyhf{}
  \rhead{\@author}
  \fancyfoot[c]{\thepage}
}

% Sets
\newcommand{\N}{\mathbb{N}}
\newcommand{\Z}{\mathbb{Z}}
\newcommand{\Q}{\mathbb{Q}}
\newcommand{\R}{\mathbb{R}}
\newcommand{\C}{\mathbb{C}}
\newcommand{\U}{\mathcal{U}}
\newcommand{\sym}{\mathbin{\triangle}}

% Functions
\DeclareMathOperator{\sgn}{sgn}
\DeclareMathOperator{\im}{im}

% Operators
\newcommand{\Rarr}{\Rightarrow}
\newcommand{\Larr}{\Leftarrow}
\usepackage{mathtools} % for \DeclarePairedDelimiter macro
\DeclarePairedDelimiter\ceil{\lceil}{\rceil}
\DeclarePairedDelimiter\floor{\lfloor}{\rfloor}

% Macros
% properly typeset ε-δ (epsilon en dash delta)
\newcommand{\epsdel}[1][\delta]{\ensuremath{\epsilon\mathit{\textnormal{--}}#1}}
\newcommand{\by}[1]{& \text{by #1}}
\newcommand{\IH}{\by{inductive hypothesis}}
% multiple choice (remove spacing between items)
\newenvironment{choices}
{\begin{enumerate}[(a)]
    \setlength{\parskip}{0ex}
    }{
  \end{enumerate}}

% Typesetting
\usepackage{array}   % for \newcolumntype macro
\newcolumntype{C}{>{$}c<{$}} % math version of "C" column type
\newcommand{\dlim}[2]{\displaystyle\lim_{#1\to#2}} % totally not \dfrac ripoff
\newcommand{\dilim}[1]{\dlim{#1}{\infty}} % infinite limits
\newcommand{\ilim}[1]{\lim_{#1\to\infty}}
\usepackage{cancel}

% Auto-number questions
\newcommand{\QType}{Q}
\renewcommand{\theparagraph}{\QType\ifnum\value{paragraph}<10 0\fi\arabic{paragraph}}
\setcounter{secnumdepth}{6}
\newcommand{\question}{\par\refstepcounter{paragraph}\textbf{\theparagraph}.\space}

% Question sections
\titleformat{\section}{\normalsize\bfseries}{\thesection}{1em}{}
\newcommand{\qsection}[2]{%
  \renewcommand{\QType}{#2}
  \section*{#1}
  \refstepcounter{section}
}


\title{MATH 135 Fall 2020: Extra Practice 11}

\begin{document}
\thispagestyle{firstpage}

\textbf{\@title}

\qsection{Warm-Up Exercises}{WE}

\question Find a real cubic polynomial whose roots include 1 and $i$.


\question Divide $f(x)=x^3+x^2+x+1$ by $g(x)=x^2+4x+3$ to find the quotient $q(x)$ and remainder $r(x)$
that satisfy the requirements of the \emph{Division Algorithm for Polynomials} (DAP)


\qsection{Recommended Problems}{RP}

\question Let $z\in\C$. Prove that $(x-z)(x-\overline{z}) \in \R[x]$.


\question Prove that there exists a polynomial in $\Q[x]$ with the root $2-\sqrt 7$.


\question For each of the following polynomials $f(x) \in \F[x]$,
write $f(x)$ as a product of irreducible polynomials in $\F[x]$.


\question Let $g(x)=x^3+bx^2+cx+d\in\C[x]$ be a monic cubic polynomial.
Let $z_1$, $z_2$, and $z_3$ be three roots of $g(x)$ such that
\[ g(x) = (x-z_1)(x-z_2)(x-z_3) \]
Prove that \begin{align*}
  z_1 + z_2 + z_3          & = -b \\
  z_1z_2 + z_2z_3 + z_3z_1 & = c  \\
  z_1z_2z_3                & = -d
\end{align*}


\question Using the Rational Roots Theorem, prove that $\sqrt 3 + \sqrt 7$ is irrational.


\question \begin{enumerate}[(a)]
  \item Prove that for every prime $p$, there exists a polynomial $f(x)$ over $\Z_p$,
        of degree $p$, such that every element of $\Z_p$ is a root of $f(x)$.
  \item Prove that for every prime $p$, there exists a polynomial $f(x)$ over $\Z_p$,
        of degree $p$, which has no roots in $\Z_p$.
\end{enumerate}


\question Suppose $f(x) = a_n x^n + a_{n-1} x^{n-1} + \cdots + a_1 x + a_0 \in \C[x]$ with degree $n$.
We say $f(x)$ is \emph{palindromic} if the coefficients $a_j$ satisfy
\[ a_{n-j} = a_j \quad \text{for all} \quad 0 \leq j \leq n \]
Prove that \begin{enumerate}[(a)]
  \item If $f(x)$ is a palindromic polynomial and $c \in \C$ is a root of $f(x)$,
        then $c$ must be non-zero, and $\frac{1}{c}$ is also a root of $f(x)$.
  \item If $f(x)$ is a palindromic polynomial of odd degree, then $f(-1) = 0$.
  \item If $\deg f = 1$ and $f(x)$ is a monic, palindromic polynomial, then $f(x) = x+1$.
\end{enumerate}


\qsection{Challenge}{C}

\question We call a polynomial primitive if the greatest common divisor of all of its coefficients is 1.
Show that the product of two primitive polynomials is again primitive.

\end{document}