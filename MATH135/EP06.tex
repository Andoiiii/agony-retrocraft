\documentclass[11pt]{article}

\usepackage{physics}
\usepackage{amsfonts,amsmath,amssymb,amsthm}
\usepackage{enumerate}
\usepackage{titlesec}
\usepackage{fancyhdr}
\usepackage{multicol}

\headheight 13.6pt
\setlength{\headsep}{10pt}
\textwidth 15cm
\textheight 24.3cm
\evensidemargin 6mm
\oddsidemargin 6mm
\topmargin -1.1cm
\setlength{\parskip}{1.5ex}
\parindent=0pt

\author{James Ah Yong}

\pagestyle{fancy}
\fancyhf{}
\fancyfoot[c]{\thepage}
\makeatletter
\lhead{\@title}
\rhead{\@author}

\fancypagestyle{firstpage}{
  \fancyhf{}
  \rhead{\@author}
  \fancyfoot[c]{\thepage}
}

% Sets
\newcommand{\N}{\mathbb{N}}
\newcommand{\Z}{\mathbb{Z}}
\newcommand{\Q}{\mathbb{Q}}
\newcommand{\R}{\mathbb{R}}
\newcommand{\C}{\mathbb{C}}
\newcommand{\U}{\mathcal{U}}
\newcommand{\sym}{\mathbin{\triangle}}

% Functions
\DeclareMathOperator{\sgn}{sgn}
\DeclareMathOperator{\im}{im}

% Operators
\newcommand{\Rarr}{\Rightarrow}
\newcommand{\Larr}{\Leftarrow}
\usepackage{mathtools} % for \DeclarePairedDelimiter macro
\DeclarePairedDelimiter\ceil{\lceil}{\rceil}
\DeclarePairedDelimiter\floor{\lfloor}{\rfloor}

% Macros
% properly typeset ε-δ (epsilon en dash delta)
\newcommand{\epsdel}[1][\delta]{\ensuremath{\epsilon\mathit{\textnormal{--}}#1}}
\newcommand{\by}[1]{& \text{by #1}}
\newcommand{\IH}{\by{inductive hypothesis}}
% multiple choice (remove spacing between items)
\newenvironment{choices}
{\begin{enumerate}[(a)]
    \setlength{\parskip}{0ex}
    }{
  \end{enumerate}}

% Typesetting
\usepackage{array}   % for \newcolumntype macro
\newcolumntype{C}{>{$}c<{$}} % math version of "C" column type
\newcommand{\dlim}[2]{\displaystyle\lim_{#1\to#2}} % totally not \dfrac ripoff
\newcommand{\dilim}[1]{\dlim{#1}{\infty}} % infinite limits
\newcommand{\ilim}[1]{\lim_{#1\to\infty}}
\usepackage{cancel}

% Auto-number questions
\newcommand{\QType}{Q}
\renewcommand{\theparagraph}{\QType\ifnum\value{paragraph}<10 0\fi\arabic{paragraph}}
\setcounter{secnumdepth}{6}
\newcommand{\question}{\par\refstepcounter{paragraph}\textbf{\theparagraph}.\space}

% Question sections
\titleformat{\section}{\normalsize\bfseries}{\thesection}{1em}{}
\newcommand{\qsection}[2]{%
  \renewcommand{\QType}{#2}
  \section*{#1}
  \refstepcounter{section}
}

\title{MATH 135 Fall 2020: Extra Practice 6}

\begin{document}
\thispagestyle{firstpage}

\textbf{\@title}

\qsection{Warm-Up Exercises}{WE}

\question What is the remainder when $-98$ is divided by 7?

$-98 \divisionsymbol 7 = -14$, so the remainder is 0.

\question Calculate $\gcd(10,-65)$.

We have $10 = 2 \cdot 5$ and $-65 = -1 \cdot 5 \cdot 13$, so the GCD is 5.

\question Let $a,b,c\in\Z$. Consider the implication $S$:
If $\gcd(a,b)=1$ and $c \mid (a+b)$, then $\gcd(b,c)=1$.
Fill in the blanks to complete a proof of $S$.
\begin{enumerate}[(a)]
  \setlength{\parskip}{0ex}
  \item Since $\gcd(a,b)=1$, by \fbox{B\'ezout's Lemma},
        there exist integers $x$ and $y$ such that $ax+by=1$.
  \item Since $c \mid (a+b)$, by \fbox{definition},
        there exists an integer $k$ such that $a+b=ck$.
  \item Substituting $a=ck-b$ into the first equation, we get $1=(ck-b)x+by=b(-x+y)+c(kx)$.
  \item Since 1 is a common divisor of $b$ and $c$ and $-x+y$ and $kx$ are integers,
        $\gcd(b,c)=1$ by \fbox{the GCD Characterization Theorem}.
\end{enumerate}

\question Disprove: For all integers $a$, $b$, and $c$, if $a \mid (bc)$, then $a \mid b$ or $a \mid c$.
\begin{proof}
  We prove the negation, there are integers $a$, $b$, and $c$ where
  $a \mid bc$, $a \nmid b$, and $a \nmid c$.

  Let $a=15$, $b=5$, and $c=3$.
  Clearly, $a \nmid b$ and $a \nmid c$.
  However, $bc=15$, and $15 \mid 15$.
\end{proof}


\qsection{Recommended Problems}{RP}

\question \begin{enumerate}[(a)]
  \item Use the Extended Euclidean Algorithm to find three integers $x$, $y$ and
        $d = \gcd(1112, 768)$ such that $1112x + 768y = d$.
        \begin{proof}[Solution]
          Apply the EEA with $x=1112$ and $y=768$.
          \begin{center}
            \begin{tabular}{C|C|C|C}
              x   & y    & r    & q \\ \hline
              1   & 0    & 1112     \\
              0   & 1    & 768      \\
              1   & -1   & 344  & 1 \\
              -2  & 3    & 80   & 2 \\
              9   & -13  & 24   & 4 \\
              -29 & 42   & 8    & 3 \\ \hline
              96  & -139 & 0    & 3
            \end{tabular}
          \end{center}
          Therefore, we have that $d=\gcd(1112,768)=8$, and that
          \[ 1112(-29) + 768(42) = 8 \]
          That is, our solution is when $x=-29$ and $y=42$.
        \end{proof}
  \item Determine integers $s$ and $t$ such that $768s - 1112t = \gcd(768, -1112)$.
        \begin{proof}[Solution]
          Since the GCD is invariant under sign changes, we immediately know that $\gcd(768,-1112)=8$.
          We also have that $1112(-96) + 768(42) = 8$.
          But this is the same as saying $768(42) - 1112(96) = 8$, so $s=42$ and $t=96$.
        \end{proof}
\end{enumerate}


\question Prove that for all $a\in\Z$, $\gcd(9a + 4, 2a + 1) = 1$.
\begin{proof}
  Let $a$ be an integer.
  We must show that $9a+4$ and $2a+1$ are coprime.
  Recall the Coprimeness Characterization Theorem:
  it suffices to find integers $a$ and $b$ such that $(9a+4)a + (2a+1)b = 1$.

  Choose $a=-2$ and $b=9$. Then,
  \begin{align*}
    (9a+4)a + (2a+1)b & = -2(9a+4)a + 9(2a+1) \\
                      & = -18a-8 + 18a+9      \\
                      & = 1
  \end{align*}
  as desired. Therefore, $\gcd(9a+4,2a+1) = 1$.
\end{proof}


\question Let $\gcd(x, y) = d$ for integers $x$ and $y$.
Express $\gcd(18x + 3y, 3x)$ in terms of $d$ and prove that you are correct.
\begin{proof}
  Let $x$ and $y$ be integers with GCD $d$.

  We may apply GCD With Remainders to reduce $g=\gcd(18x+3y,3x)$.
  We have $18x+3y = 6(3x) + 3y$, so $g=\gcd(3x,3y)$.

  Now, $x \mid d$ and $y \mid d$, so we can find integers $m$ and $n$ where $x = dm$ and $y = dn$.
  Multiplying through by 3, we have $3x = (3d)m$ and $3y = (3d)n$.
  It follows that $3d \mid 3x$ and $3d \mid 3y$, that is, $3d$ is a common divisor of $3x$ and $3y$.

  By Bézout's Lemma, there are integers $s$ and $t$ where $xs+yt=d$.
  Again multiplying through by 3, we have $(3x)s+(3y)t=3d$.

  Therefore, by the GCD Characterization Theorem, $\gcd(3x,3y) = 3d$.
\end{proof}


\question Let $a,b\in\Z$. Prove that if $\gcd(a, b) = 1$, then $\gcd(2a+b, a+2b) \in \{1,3\}$.
\begin{proof}
  Let $a$ and $b$ be coprime integers.

  Applying GCD WR, we have that $2a+b = 2(a+2b)-3b$, so $\gcd(2a+b,a+2b)=\gcd(a+2b,-3b)$.
  The properties of GCD state this is equivalent to $\gcd(3b,a+2b)$.

  The GCD of $3b$ and $a+2b$ must divide both $3b$ and $a+2b$.
  The positive divisors of $3b$ are 1, 3, and any positive divisor $d \geq 2$ of $b$.
  We show that no such divisors of $b$ also divide $a+2b$.

  Suppose for a contradiction that an integer $d \geq 2$ divides both $b$ and $a+2b$.
  Then, by DIC, $d \mid ((a+2b)-2(b))$, that is, $d \mid a$.
  This means that $d$ is a common divisor of $a$ and $b$.
  However, $a$ and $b$ are coprime, meaning $d=1$.
  This is a contradiction since $1 \not\geq 2$.
  Therefore, no positive divisor of $b$, other than 1, also divides $a+2b$.

  It follows that $\gcd(2a+b,a+2b)$ can only be 1 or 3, as desired.
\end{proof}


\question Prove that for all integers $a$, $b$ and $k$, if $b \neq 0$, then $\gcd(a, b) \leq \gcd(ak, b)$.
\begin{proof}
  Let $a$, $b$, and $k$ be integers where $b$ is non-zero.
  Also, let $d=\gcd(a,b)$ and $g=\gcd(ak,b)$. We must show $d \leq g$.

  We will apply the GCD from Prime Factorization.
  For convenience, we define $p_n$ to be the $n$\textsuperscript{th} prime.
  First, by UPF, we are guaranteed to be able to write
  $a=\pf{\alpha_#1}{n}$, $b=\pf{\beta_#1}{n}$, and $k=\pf{\kappa_#1}{n}$,
  with non-negative $\alpha_i$, $\beta_i$, and $\kappa_i$.
  Notice that we may write $ak$ as a product of primes: $\pf{\alpha_#1+\kappa_#1}{n}$.

  Now, by GCD PF, we have $d=\pf{\delta_#1}{n}$, where $\delta_i=\min(\{\alpha_i,\beta_i\})$
  for all integers $1 \leq i \leq k$.
  Likewise, we have $g=\pf{\gamma_#1}{n}$, where $\gamma_i=\min(\{\alpha_i+\kappa_i,\beta_i\})$.

  We will show that $\delta_i \leq \gamma_i$ for all $i$, from which it follows $d \leq g$.
  Let $i$ be arbitrary.

  If $\alpha_i \leq \beta_i \leq \alpha_i+\kappa_i$,
  then we have $\delta_i = \alpha_i$ and $\gamma_i = \beta_i$.
  It follows that $\delta_i\leq\gamma_i$.
  Otherwise, $\beta_i \leq \alpha_i \leq \alpha_i + \kappa_i$,
  so $\delta_i = \beta_i$ and $\kappa_i = \alpha_i$.
  We again have $\delta_i\leq\gamma_i$.

  Therefore, since every exponent in the prime factorization of $d$ is less than or equal to
  the coresponding exponent in the prime factorization of $g$, it must be the case that $d \leq g$.
\end{proof}


\question Prove that for all integers $a$, $b$ and $c$:
if $a \mid c$ and $b \mid c$ and $\gcd(a, b) = 1$, then $ab \mid c$.
\begin{proof}
  Let $a$, $b$, and $c$ be integers such that $a$ and $b$ divide $c$, and $a$ and $b$ are coprime.

  Then, there exist integers $m$ and $n$ such that $am=c$ and $bn=c$.
  Also, by the CCT, there exist integers $s$ and $t$ such that $as+bt=1$.

  Then, $cas+cbt=c$, so $(bn)as+(am)bt=c$.
  It follows that $ab(ns+bt)=c$, so $ab \mid c$.
\end{proof}


\question Let $a,b,c\in\Z$. Prove that if $\gcd(a, b) = 1$ and $c \mid a$, then $\gcd(b, c) = 1$.
\begin{proof}
  Let $a$, $b$, and $c$ be integers such that $\gcd(a,b)=1$ and $c \mid a$.

  Then, $nc = a$ for some integer $n$ and, by Bézout's Lemma, $as+bt=1$.
  Substituting, $(nc)a+bt=bt+c(na)=1$ for integers $t$ and $na$, so by the CCT, $\gcd(b,c)=1$.
\end{proof}


\question Let $a$ and $b$ be integers.
Prove that if $\gcd(a, b) = 1$, then $\gcd(a^m, b^n) = 1$ for all $m,n\in\N$.
You may use the result which is proved in Example 14 in the notes.
\begin{proof}
  Recall that Example 14 proved that for all integers $a$, $b$, and natural numbers $n$,
  if $\gcd(a, b)=1$, then $\gcd(a, b^n)=1$.
  Therefore, it suffices to let $c=b^n$ and prove that $\gcd(a, c)=1$ implies $\gcd(a^m, c)=1$.

  In fact, we may simplify the problem further.
  If we show that the arguments of the GCD are commutative,
  then we may again use the result from Example 14.
  Let $x$ and $y$ be coprime integers, that is, $\gcd(x,y)=1$.
  By Bézout's Lemma, there exist $s$ and $t$ such that $xs+yt=1$.
  Equivalently, $yt+xs=1$, and by the CCT, $\gcd(y,x)=1$.

  Then, $\gcd(a,c)=\gcd(c,a)=1$.
  By Example 14, $\gcd(c,a^m)=1$, that is, $\gcd(a^m,c)=\gcd(a^m,b^n)=1$, as desired.
\end{proof}


\question Suppose $a$, $b$ and $n$ are integers.
Prove that $n \mid \gcd(a, n) \cdot \gcd(b, n)$ if and only if $n \mid ab$.
(sooshi, CS Discord)
\begin{proof}
  Let $a$, $b$, and $n$ be integers.
  Then, let $d=\gcd(a,n)$ and $c=\gcd(b,n)$.
  We prove both implications.

  ($\Larr$) Suppose that $n \mid dc$.
  Recall that by definition, $d \mid a$ and $c \mid b$.
  Then, we may write $dn=a$ and $cm=b$ for some integers $n$ and $m$.
  Multiplying together, $dc(mn)=ab$, that is, since $mn$ is an integer, $dc \mid ab$.
  By the transitivity of divisibility, $n \mid dc$ and $dc \mid ab$ imply $n \mid ab$, as desired.

  ($\Rarr$) Suppose that $n \mid ab$.
  We apply Bézout's Lemma to rewrite $d=as+nt$ and $c=bx+ny$ with integers $s$, $t$, $x$, and $y$.
  Multiplying together gives $dc = absx+asny+bxnt+n^2ty$.
  This factors to $dc = (ab)(sx)+n(asy+bxt+nty)$.
  Since we have both $n \mid ab$ and $n \mid n$, by DIC, $n \mid (ab)(sx)+n(asy+bxt+nty)$.
  However, this is just $n \mid dc$.

  Therefore, since both implications hold, $n \mid dc$ if and only if $n \mid ab$.
\end{proof}


\question How many positive divisors does 33480 have?
\begin{proof}[Solution]
  We may apply prime factorization to get $33480=2^3\cdot3^3\cdot5\cdot31$.
  Then, by DFPF, we have that any positive divisor $d=2^\alpha\cdot3^\beta\cdot5^\gamma\cdot31^\delta$
  for integers $0\leq\alpha\leq3$, $0\leq\beta\leq3$, $0\leq\gamma\leq1$, and $0\leq\delta\leq1$.

  That is, there are 4 choices for each of $\alpha$ and $\beta$, and 2 choices for $\gamma$ and $\delta$.
  Multiplying out, we have $4\cdot4\cdot2\cdot2=64$ positive divisors.
\end{proof}


\question Prove that for all integers $a$ and $b$,
if $9a^2 = b^4$ where $a,b\in\Z$, then 3 is a common divisor of $a$ and $b$.
\begin{proof}
  Let $a$ and $b$ be integers such that $9a^2=b^4$.
  Without loss of generality, let both $a$ and $b$ be positive
  (if $a=b=0$, then, trivially, $3 \mid a$ and $3 \mid b$).

  By UFT, $a=\pf{\alpha_#1}{k}$ for $k$ distinct primes $p_i$ and non-negative integers $\alpha_i$.
  Likewise, $b=\pf{\beta_#1}{k}$ for non-negative integers $\beta_i$.
  Since 3 is prime, there is an $n$ where $p_n=3$.

  It follows that $9a^2$ has $2+2\alpha_n$ factors of 3 and that $b^4$ has $4\beta_n$ factors.
  Since $9a^2=b^4$, by UFT, $2+2\alpha_n=4\beta_n$.

  We have that $4\beta_n = 2+2\alpha_n \geq 2$, so $\beta_n \geq 1$, which means $3 \mid b$.

  However, if $\beta_n \geq 1$, then $2+2\alpha_n = 4\beta_n \geq 4$, which means $\alpha_n \geq 1$.
  That is, $3 \mid a$.

  Therefore, 3 is a common divisor of $a$ and $b$.
\end{proof}


\question Let $n\in\N$. Prove that if $p$ is prime and $p \leq n$, then $p$ does not divide $n! + 1$.
\begin{proof}
  Let $n$ be a natural number, and $p$ be a prime number.

  Since $n!$ is defined as the product of all positive integers up to $n$ and $p \leq n$, $p$ clearly divides $n$.
  Therefore, $n! = kp$ for some integer $k$.
  Then, $k$ is the product of all positive integers up to $n$ \emph{except} $p$.
  Since $p$ is prime, $k \nmid p$.

  Then, we have $n!+1 = p(k+\frac{1}{p})$, so $p \mid (n!+1)$ only if $k+\frac{1}{p}$ is an integer,
  which it clearly is not (since $p \geq 2$).
  Therefore, $p \nmid (n!+1)$.
\end{proof}


\qsection{Challenges}{C}

\question Prove that for any integer $a \neq 1$ and $n\in\N$,
$\gcd\left( \frac{a^n-1}{a-1}, a-1 \right) = \gcd(n, a-1)$.

\question Let $n$ be a positive integer for which
$\gcd(n, n + 1) < \gcd(n, n + 2) < \cdots < {\gcd(n, n + 20)}$.
Prove that $\gcd(n, n + 20) < \gcd(n, n + 21)$.

\question Let $a$ and $b$ be nonnegative integers.
Prove that $\gcd(2^a - 1, 2^b - 1) = 2\gcd(a,b) - 1$.

\question An integer $n$ is \emph{perfect} if the sum of all of its positive divisors
(including 1 and itself) is $2n$.
\begin{enumerate}[(a)]
  \item Is 6 a perfect number? Give reasons for your answer.
  \item Is 7 a perfect number? Give reasons for your answer.
  \item Prove the following statement:
        If $k$ is a positive integer and $2^k-1$ is prime, then $2^{k-1}(2^k-1)$ is perfect.
\end{enumerate}

\question Let $a,b\in\Z$. Prove that $\gcd(a^n, b^n) = \gcd(a, b)^n$ for all $n\in\N$.

\end{document}