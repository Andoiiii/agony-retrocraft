\documentclass[11pt]{article}

\usepackage{physics}
\usepackage{amsfonts,amsmath,amssymb,amsthm}
\usepackage{enumerate}
\usepackage{titlesec}
\usepackage{fancyhdr}
\usepackage{multicol}

\headheight 13.6pt
\setlength{\headsep}{10pt}
\textwidth 15cm
\textheight 24.3cm
\evensidemargin 6mm
\oddsidemargin 6mm
\topmargin -1.1cm
\setlength{\parskip}{1.5ex}
\parindent=0pt

\author{James Ah Yong}

\pagestyle{fancy}
\fancyhf{}
\fancyfoot[c]{\thepage}
\makeatletter
\lhead{\@title}
\rhead{\@author}

\fancypagestyle{firstpage}{
  \fancyhf{}
  \rhead{\@author}
  \fancyfoot[c]{\thepage}
}

% Sets
\newcommand{\N}{\mathbb{N}}
\newcommand{\Z}{\mathbb{Z}}
\newcommand{\Q}{\mathbb{Q}}
\newcommand{\R}{\mathbb{R}}
\newcommand{\C}{\mathbb{C}}
\newcommand{\U}{\mathcal{U}}
\newcommand{\sym}{\mathbin{\triangle}}

% Functions
\DeclareMathOperator{\sgn}{sgn}
\DeclareMathOperator{\im}{im}

% Operators
\newcommand{\Rarr}{\Rightarrow}
\newcommand{\Larr}{\Leftarrow}
\usepackage{mathtools} % for \DeclarePairedDelimiter macro
\DeclarePairedDelimiter\ceil{\lceil}{\rceil}
\DeclarePairedDelimiter\floor{\lfloor}{\rfloor}

% Macros
% properly typeset ε-δ (epsilon en dash delta)
\newcommand{\epsdel}[1][\delta]{\ensuremath{\epsilon\mathit{\textnormal{--}}#1}}
\newcommand{\by}[1]{& \text{by #1}}
\newcommand{\IH}{\by{inductive hypothesis}}
% multiple choice (remove spacing between items)
\newenvironment{choices}
{\begin{enumerate}[(a)]
    \setlength{\parskip}{0ex}
    }{
  \end{enumerate}}

% Typesetting
\usepackage{array}   % for \newcolumntype macro
\newcolumntype{C}{>{$}c<{$}} % math version of "C" column type
\newcommand{\dlim}[2]{\displaystyle\lim_{#1\to#2}} % totally not \dfrac ripoff
\newcommand{\dilim}[1]{\dlim{#1}{\infty}} % infinite limits
\newcommand{\ilim}[1]{\lim_{#1\to\infty}}
\usepackage{cancel}

% Auto-number questions
\newcommand{\QType}{Q}
\renewcommand{\theparagraph}{\QType\ifnum\value{paragraph}<10 0\fi\arabic{paragraph}}
\setcounter{secnumdepth}{6}
\newcommand{\question}{\par\refstepcounter{paragraph}\textbf{\theparagraph}.\space}

% Question sections
\titleformat{\section}{\normalsize\bfseries}{\thesection}{1em}{}
\newcommand{\qsection}[2]{%
  \renewcommand{\QType}{#2}
  \section*{#1}
  \refstepcounter{section}
}

\title{MATH 135 Fall 2020: Extra Practice 5}

\begin{document}
\thispagestyle{firstpage}

\textbf{\@title}

\qsection{Warm-Up Exercises}{WE}

\question What is the remainder when $-98$ is divided by 7?

$-98 \divisionsymbol 7 = -14$, so the remainder is 0.

\question Calculate $\gcd(10,-65)$.

We have $10 = 2 \cdot 5$ and $-65 = -1 \cdot 5 \cdot 13$, so the GCD is 5.

\question Let $a,b,c\in\Z$. Consider the implication $S$:
If $\gcd(a,b)=1$ and $c \mid (a+b)$, then $\gcd(b,c)=1$.
Fill in the blanks to complete a proof of $S$.
\begin{enumerate}[(a)]
  \setlength{\parskip}{0ex}
  \item Since $\gcd(a,b)=1$, by \fbox{B\'ezout's Lemma},
        there exist integers $x$ and $y$ such that $ax+by=1$.
  \item Since $c \mid (a+b)$, by \fbox{definition},
        there exists an integer $k$ such that $a+b=ck$.
  \item Substituting $a=ck-b$ into the first equation, we get $1=(ck-b)x+by=b(-x+y)+c(kx)$.
  \item Since 1 is a common divisor of $b$ and $c$ and $-x+y$ and $kx$ are integers,
        $\gcd(b,c)=1$ by \fbox{the GCD Characterization Theorem}.
\end{enumerate}

\question Disprove: For all integers $a$, $b$, and $c$, if $a \mid (bc)$, then $a \mid b$ or $a \mid c$.
\begin{proof}
  We prove the negation, there are integers $a$, $b$, and $c$ where
  $a \mid bc$, $a \nmid b$, and $a \nmid c$.

  Let $a=15$, $b=5$, and $c=3$.
  Clearly, $a \nmid b$ and $a \nmid c$.
  However, $bc=15$, and $15 \mid 15$.
\end{proof}

\qsection{Recommended Problems}{RP}

\question \begin{enumerate}[(a)]
  \item Use the Extended Euclidean Algorithm to find three integers $x$, $y$ and
        $d = \gcd(1112, 768)$ such that $1112x + 768y = d$.
        \begin{proof}[Solution]
          Apply the EEA with $x=1112$ and $y=768$.
          \begin{center}
            \begin{tabular}{C|C|C|C}
              x   & y    & r    & q \\ \hline
              1   & 0    & 1112     \\
              0   & 1    & 768      \\
              1   & -1   & 344  & 1 \\
              -2  & 3    & 80   & 2 \\
              9   & -13  & 24   & 4 \\
              -29 & 42   & 8    & 3 \\ \hline
              96  & -139 & 0    & 3
            \end{tabular}
          \end{center}
          Therefore, we have that $d=\gcd(1112,768)=8$, and that
          \[ 1112(-29) + 768(42) = 8 \]
          That is, our solution is when $x=-29$ and $y=42$.
        \end{proof}
  \item Determine integers $s$ and $t$ such that $768s - 1112t = \gcd(768, -1112)$.
        \begin{proof}[Solution]
          Since the GCD is invariant under sign changes, we immediately know that $\gcd(768,-1112)=8$.
          We also have that $1112(-96) + 768(42) = 8$.
          But this is the same as saying $768(42) - 1112(96) = 8$, so $s=42$ and $t=96$.
        \end{proof}
\end{enumerate}

\question Prove that for all $a\in\Z$, $\gcd(9a + 4, 2a + 1) = 1$.
\begin{proof}
  Let $a$ be an integer.
  We must show that $9a+4$ and $2a+1$ are coprime.
  Recall the Coprimeness Characterization Theorem:
  it suffices to find integers $a$ and $b$ such that $(9a+4)a + (2a+1)b = 1$.

  Choose $a=-2$ and $b=9$. Then,
  \begin{align*}
    (9a+4)a + (2a+1)b & = -2(9a+4)a + 9(2a+1) \\
                      & = -18a-8 + 18a+9      \\
                      & = 1
  \end{align*}
  as desired. Therefore, $\gcd(9a+4,2a+1) = 1$.
\end{proof}

\question Let $\gcd(x, y) = d$ for integers $x$ and $y$.
Express $\gcd(18x + 3y, 3x)$ in terms of $d$ and prove that you are correct.

\question Let $a,b\in\Z$. Prove that if $\gcd(a, b) = 1$, then $\gcd(2a + b, a + 2b) \in \{1, 3\}$.

\question Prove that for all integers $a$, $b$ and $k$, if $b \neq 0$, then $\gcd(a, b) \leq \gcd(ak, b)$.

\question Prove that for all integers $a$, $b$ and $c$:
if $a \mid c$ and $b \mid c$ and $\gcd(a, b) = 1$, then $ab \mid c$.

\question Let $a,b,c\in\Z$. Prove that if $\gcd(a, b) = 1$ and $c \mid a$, then $\gcd(b, c) = 1$.

\question Let $a$ and $b$ be integers.
Prove that if $\gcd(a, b) = 1$, then $\gcd(a^m, b^n) = 1$ for all $m,n\in\N$.
You may use the result which is proved in Example 14 in the notes.

\question Suppose a, b and n are integers.
Prove that $n \mid \gcd(a, n) \cdot \gcd(b, n)$ if and only if $n \mid ab$.

\question How many positive divisors does 33480 have?

\question Prove that for all integers $a$ and $b$,
if $9a^2 = b^4$ where $a,b\in\Z$, then 3 is a common divisor of $a$ and $b$.

\question Let $n\in\N$. Prove that if $p$ is prime and $p \leq n$, then $p$ does not divide $n! + 1$.


\qsection{Challenges}{C}

\question Prove that for any integer $a \neq 1$ and $n\in\N$,
$\gcd\left( \frac{a^n-1}{a-1}, a-1 \right) = \gcd(n, a-1)$.

\question Let $n$ be a positive integer for which
$\gcd(n, n + 1) < \gcd(n, n + 2) < \cdots < {\gcd(n, n + 20)}$.
Prove that $\gcd(n, n + 20) < \gcd(n, n + 21)$.

\question Let $a$ and $b$ be nonnegative integers.
Prove that $\gcd(2^a - 1, 2^b - 1) = 2\gcd(a,b) - 1$.

\question An integer $n$ is \emph{perfect} if the sum of all of its positive divisors
(including 1 and itself) is $2n$.
\begin{enumerate}[(a)]
  \item Is 6 a perfect number? Give reasons for your answer.
  \item Is 7 a perfect number? Give reasons for your answer.
  \item Prove the following statement:
        If $k$ is a positive integer and $2^k-1$ is prime, then $2^{k-1}(2^k-1)$ is perfect.
\end{enumerate}

\question Let $a,b\in\Z$. Prove that $\gcd(a^n, b^n) = \gcd(a, b)^n$ for all $n\in\N$.

\end{document}