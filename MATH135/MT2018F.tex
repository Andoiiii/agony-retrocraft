\documentclass{agony}
\title{MATH 135 Fall 2018: Midterm Examination}

\begin{document}
\thispagestyle{firstpage}

\textbf{\thetitle}

\question \begin{enumerate}[(a)]
  \item Complete the truth table below
        \begin{center}
          \begin{tabular}{C|C||C|C||C|C}
            P & Q & P \iff Q & (P \iff Q) \lor Q & P \implies Q & (P \implies Q) \land Q \\ \hline
            T & T & T        & T                 & T            & T                      \\
            T & F & F        & F                 & F            & F                      \\
            F & T & F        & T                 & T            & T                      \\
            F & F & T        & T                 & T            & F                      \\
          \end{tabular}
        \end{center}
  \item Is $(P \iff Q) \lor Q$ logically equivalent to $(P \implies Q) \land Q$? \emph{No.}
\end{enumerate}


\question Consider the statement $S$: $\forall a \in\R,\exists b\in\R, a=(b-2)^2-5$.
\begin{enumerate}[(a)]
  \item Give a negation of the statement $S$ without using any words or the symbol $\lnot$.
        \begin{equation*}
          \exists a\in\R, \forall b\in\R, a \neq (b-2)^2-5
        \end{equation*}
  \item Is the statement $S$ true or false? \emph{False}
  \item Prove or disprove the statement $S$.
        \begin{proof}
          We prove by counter-example.

          Let $a = -10$. We must find a $b$ such that $a=(b-2)^2-5$, that is,
          \begin{align*}
            -10         & = (b-2)^2 - 5 \\
            -5          & = (b-2)^2     \\
            \sqrt{-5}+2 & = b
          \end{align*}
          but $\sqrt{-5}$ is undefined in $\R$, so no such $b$ exists.
        \end{proof}
\end{enumerate}


\question The proofs of two different statements are given below.
For each proof, indicate that the proof is correct or identify the fundamental error.
(The statements may or may not be true, but this is not relevant to the question or your answer.)
\begin{enumerate}[(a)]
  \item \emph{Statement:} For all $d,c\in\N$, if $dc \mid d^3$, then $c \mid d$. \\
        \emph{Proof:} Let $d,c\in\N$. Assume that $dc \mid d^3$.
        Then $\exists k\in\Z$ such that $d^3 = kdc$.
        Since $d \neq 0$, we divide both sides of this equation by $d $to get $d^2 = kc$.
        Since $k \in \Z$, it follows that $c \mid d^2$. Thus $c \mid d(d)$.
        Since $d \in \Z$, it follows that $c \mid d$.

        The proof is invalid. It does not immediately follow that $c \mid d$ from $c \mid d^2$.
  \item \emph{Statement:} For all $d,c\in\N$, if $c \mid d$, then $dc \mid d^3$. \\
        \emph{Proof:} Let $d,c\in\N$. Assume that $c \mid d$.
        Then $\exists k \in \Z$, $d=kc$.
        Thus $d^2=k^2c^2$ and $d^3=d(k^2c^2)=dc(k^2c)$.
        Since $k^2c\in\Z$, it follows that $dc \mid d^3$.

        The proof is correct.
\end{enumerate}


\question Let $a \in \N$. Alex is proving by induction that for all non-negative integers $n$,
\begin{equation*}
  \sum_{i=0}^n\binom{a+i-1}{a-1} = \binom{a+n}{a}
\end{equation*}
They have correctly stated the proof below. Complete the rest of the proof.

\begin{quote}
  \emph{Base Case:} When $n=0$, the left-hand side is $\sum^0_{i=0}\binom{a+i-1}{a-1}$
  which evaluates to $\binom{a+0-1}{a-1}=1$.
  The right-hand side is $\binom{a+0}{a}$ which evaluates to 1.
  Since both sides are equal, the base case holds.

  \emph{Inductive Hypothesis:} Assume that for non-negative integer $k$,
  $\sum_{i=0}^k\binom{a+i-1}{a-1}=\binom{a+k}{a}$.
\end{quote}

\emph{Inductive Conclusion:}
We must show that $\sum_{i=0}^{k+1}\binom{a+i-1}{a-1} = \binom{a+k+1}{a}$.
Taking the left hand:
\begin{align*}
  \sum_{i=0}^{k+1}\binom{a+1-1}{a-1} & = \binom{a+k+1-1}{a-1} + \sum_{i=0}^{k}\binom{a+i-1}{a-1} \\
                                     & = \binom{a+k}{a-1} + \binom{a+k}{a} \IH                   \\
                                     & = \binom{a+k+1}{a} \by{Pascal's identity}
\end{align*}
Which is exactly the right-hand side.

Therefore, by the principle of mathematical induction,
$\sum_{i=0}^n\binom{a+i-1}{a-1} = \binom{a+n}{a}$
for all non-negative integers. $\qedsymbol$


\question Let $x,y\in\R$. Consider the implication $S$:
If $x^2-y^2 < 0$, then $x<y$ or $x+y < 0$.
\begin{enumerate}[(a)]
  \item State the hypothesis of $S$.
        \[ x^2-y^2 < 0 \]
  \item State the conclusion of $S$.
        \begin{center}
          $x<y$ or $x+y<0$
        \end{center}
  \item State the converse of $S$.
        \begin{center}
          If $x<y$ or $x+y<0$, then $x^2-y^2<0$
        \end{center}
  \item State the contrapositive of $S$.
        \begin{center}
          If $x\geq y$ and $x+y \geq 0$, then $x^2-y^2 \geq 0$
        \end{center}
  \item State the negation of $S$ in a form that does not contain an implication.
        \begin{center}
          $x^2-y^2 \geq 0$, and $x<y$ or $x+y<0$
        \end{center}
  \item Prove $S$ for all $x,y\in\R$.
        \begin{proof}
          We prove by the contrapositive.
          Let $x$ and $y$ be real numbers such that $x \geq y$ and $x+y \geq 0$.
          Since $x \geq y$, we have $x-y \geq 0$.

          Multiplying, $(x+y)(x-y) \geq 0\cdot 0$, that is, $x^2-y^2 \geq 0$.
        \end{proof}
\end{enumerate}


\question Let $\U=\{1,2,3,4,5\}$.
For each of the following statements indicate clearly whether the statement is
true or false and then prove or disprove the statement.
\begin{enumerate}[(a)]
  \item For all sets $S$ and $T$ which are subsets of $\U$, $\overline{S\cup T}=\overline{S}\cup\overline{T}$.

        Circle one of the following:
        \quad This statement is TRUE
        \quad \fbox{This statement is FALSE}

        \begin{proof}
          Consider the counter-example $S=\{1\}$ and $T=\{1,2\}$.
          We have $\overline{S\cup T} = \overline{(\{1,2\})} = \{3,4,5\}$.
          However, $\overline{S}\cup\overline{T} = \{2,3,4,5\} \cup \{3,4,5\} = \{2,3,4,5\}$.

          These sets are clearly not equal, therefore, the statement is false.
        \end{proof}
  \item There exist sets $S$ and $T$ which are subsets of $\U$, $\overline{S\cup T}=\overline{S}\cup\overline{T}$.

        Circle one of the following:
        \quad \fbox{This statement is TRUE}
        \quad This statement is FALSE

        \begin{proof}
          Let $S=T=\U$. Then, $S$ and $T$ are trivially subsets of $\U$.
          We have $\overline{S\cup T} = \emptyset$ and
          $\overline{S}\cup\overline{T}=\emptyset\cup\emptyset=\emptyset$,
          which are equal by the uniqueness of the empty set.
        \end{proof}
\end{enumerate}


\question Let $A=\{8k : k\in\Z\}\cup\{8j+4 : j\in\Z\}$ and $B=\{x\in\Z : 4 \mid x\}$.
Prove that $A=B$.
\begin{proof}
  Recall that $A=B$ if and only if $A\subseteq B$ and $B \subseteq A$.

  ($\subseteq$) Let $a$ be an arbitrary element of $A$.
  By definition, $a$ is either an element of $\{8k : k\in\Z\}$ or $\{8j+4 : j\in\Z\}$.
  Taking cases,
  \begin{itemize}
    \item If $a\in\{8k : k\in\Z\}$, we may write $a=8k$ for some integer $k$.
          It follows that $a=4(2k)$, and since $2k$ is an integer, 4 divides $a$.
          Therefore, $a$ is an integer such that $4 \mid a$, that is, $a \in B$.
    \item Likewise, if $a\in\{8j+4 : j\in\Z\}$, we may write $a=8j+4$ for some integer $j$.
          It follows that $a=4(2j+2)$. Since $2j+2\in\Z$, we have $4 \mid a$.
          Therefore, $a\in B$.
  \end{itemize}
  Therefore, all elements of $A$ are also elements of $B$, so $A \subseteq B$.

  ($\supseteq$) Let $b$ be an arbitrary element of $B$.
  By definition, $b$ is an integer such that $4 \mid b$, that is, $b = 4k$ for some integer $k$.
  Recall that all integers are either even or odd.
  \begin{itemize}
    \item If $k$ is even, we may write it as $2p$ for another integer $p$.
          Then, $b = 8p$ with integer $p$, which is exactly the definition of a member of $\{8k : k\in\Z\}$.
    \item Likewise, for odd $k$, we may write it as $2q+1$ for integer $q$.
          Then, $b = 8q+4$ with integer $q$, the definition of members of $\{8j+4 : j\in\Z\}$.
  \end{itemize}
  Since $A$ is defined as the union of these sets, $b\in A$.
  Therefore, $B \subseteq A$.
\end{proof}


\question Prove that for all $a,b,c\in\Z$, if $a \mid b$ and $a \nmid c$, then $a \nmid (b+c)$.
\begin{proof}
  Let $a$, $b$, and $c$ be integers where $a \mid b$ and $a \nmid c$.

  Suppose for a contradiction that $a \mid (b+c)$.
  Then, by DIC, since $a \mid ((1)(b+c) + (-1)(b))$, which means $a \mid c$.
  This is a contradiction.
  Therefore, $a \nmid (b+c)$.
\end{proof}


\question Prove that for all $a, b, c \in \Z$, if $a + b^3 + c^5 = 6001$,
then at least one of $a$, $b$ or $c$ is not a multiple of 3.
\begin{proof}
  Let $a$, $b$, and $c$ be integers where $a + b^3 + c^5 = 6001$.
  Suppose for a contradiction that $a$, $b$, and $c$ are all multiples of 3.
  Then, we may write them as $3i$, $3j$, and $3k$ for integers $i$, $j$, and $k$.

  We then have $6001 = a+b^3+c^5 = 3i+27j^3+243k^5 = 3(i+9j^3+81k^5)$
  where $i+9j^3+81k^5$ is an integer.
  Therefore, 3 must divide 6001.
  However, we can see by inspection that 3 does not divide 6001.

  Therefore, $a$, $b$, and $c$ cannot all be multiples of 3, so at least one of them is not.
\end{proof}


\question Prove that there is no solution to $4x^3 - y^2 = 1$ where $x$ and $y$ are integers.
\begin{proof}
  % Let $x$ and $y$ be integers.
  % Notice that the case $x=y=0$ does not hold since $0-0=1$ is false.
  % Without loss of generality, let $y > 0$.

  % Suppose for a contradiction that $4x^3 - y^2 = 1$ holds. Then,
  % \begin{align*}
  %   y^2 & = 4x^3-1        \\
  %   y   & = \sqrt{4x^3-1}
  % \end{align*}
  % This implies that $4x^3-1 \geq 0$, which is true when $x^3 \geq \frac14$.
  % We may take the cube root since it is always increasing and therefore preserves inequalities.
  % Therefore, $x \geq \frac{1}{\sqrt[3]{4}} \approx 0.63$.
  % Since $x$ is an integer, $x \geq 1$.

  Suppose for a counter-example that a solution exists.
  Let $x$ and $y$ be the solution to $4x^3-y^2=1$ in the integers.

  Then, $4x^3 = 1+y^2$.
  Since $x^3$ and $1+y^2$ are integers, $4 \mid (1+y^2)$.
  Recall that all integers are either even or odd.

  If $y$ is even, it may be written as $2k$ for an integer k.
  Then, $4 \mid (1+4k^2)$.
  However, because 4 clearly divides $4k^2$, this implies $4 \mid 1$.
  This is false, therefore $y$ cannot be even.

  Since $y$ is odd, it may be written as $2k+1$ for an integer k.
  Then, $4 \mid (4k^2+4k+2)$.
  Again, because 4 clearly divides $4k^2+4k$, this implies $4 \mid 2$.
  This is false, so $y$ cannot be odd.

  Therefore, no integer values of $y$ exist, and the statement is true by contradiction.
\end{proof}


\question Let $a_1, a_2, a_3, \dots$ be a sequence defined as follows: $a_1 = 1$, $a_2 = 3$, and $a_n = 6a_{n-1} + 5a_{n-2}$ for $n \geq 3$.
Prove by induction that $a_n$ is odd for all $n\in\N$.
\begin{proof}
  We will strongly induct on $n$ the statement $P(n)$, that $a_n$ is odd.

  To verify base cases, note that $a_1=1$ is odd and $a_2=3$ is odd.

  Suppose that $P(n)$ holds for all $1 \leq n < k$ for some natural number $k \geq 3$.
  Specifically, because $P(k-1)$ holds, $n_{k-1}$ is odd and we may write it $2p+1$ with integer $p$.
  Likewise, since $P(k-2)$ holds, $n_{k-2} = 2q+1$ for another integer $q$. Now,
  \begin{align*}
    a_k & = 6a_{k-1} + 5a_{k-2}   \\
        & = 6(2p+1) + 5(2q+1) \IH \\
        & = 12p+10q+11            \\
        & = 2(6p+5q+5) + 1
  \end{align*}
  Because $6p+5q+5$ is an integer, $a_k$ is odd.

  Therefore, by the principle of strong induction,
  $P(n)$ is true for all natural numbers $n$, and all $a_n$ are odd.
\end{proof}


\end{document}