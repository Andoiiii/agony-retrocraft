\documentclass[11pt]{article}

\usepackage{physics}
\usepackage{amsfonts,amsmath,amssymb,amsthm}
\usepackage{enumerate}
\usepackage{titlesec}
\usepackage{fancyhdr}
\usepackage{multicol}

\headheight 13.6pt
\setlength{\headsep}{10pt}
\textwidth 15cm
\textheight 24.3cm
\evensidemargin 6mm
\oddsidemargin 6mm
\topmargin -1.1cm
\setlength{\parskip}{1.5ex}
\parindent=0pt

\author{James Ah Yong}

\pagestyle{fancy}
\fancyhf{}
\fancyfoot[c]{\thepage}
\makeatletter
\lhead{\@title}
\rhead{\@author}

\fancypagestyle{firstpage}{
  \fancyhf{}
  \rhead{\@author}
  \fancyfoot[c]{\thepage}
}

% Sets
\newcommand{\N}{\mathbb{N}}
\newcommand{\Z}{\mathbb{Z}}
\newcommand{\Q}{\mathbb{Q}}
\newcommand{\R}{\mathbb{R}}
\newcommand{\C}{\mathbb{C}}
\newcommand{\U}{\mathcal{U}}
\newcommand{\sym}{\mathbin{\triangle}}

% Functions
\DeclareMathOperator{\sgn}{sgn}
\DeclareMathOperator{\im}{im}

% Operators
\newcommand{\Rarr}{\Rightarrow}
\newcommand{\Larr}{\Leftarrow}
\usepackage{mathtools} % for \DeclarePairedDelimiter macro
\DeclarePairedDelimiter\ceil{\lceil}{\rceil}
\DeclarePairedDelimiter\floor{\lfloor}{\rfloor}

% Macros
% properly typeset ε-δ (epsilon en dash delta)
\newcommand{\epsdel}[1][\delta]{\ensuremath{\epsilon\mathit{\textnormal{--}}#1}}
\newcommand{\by}[1]{& \text{by #1}}
\newcommand{\IH}{\by{inductive hypothesis}}
% multiple choice (remove spacing between items)
\newenvironment{choices}
{\begin{enumerate}[(a)]
    \setlength{\parskip}{0ex}
    }{
  \end{enumerate}}

% Typesetting
\usepackage{array}   % for \newcolumntype macro
\newcolumntype{C}{>{$}c<{$}} % math version of "C" column type
\newcommand{\dlim}[2]{\displaystyle\lim_{#1\to#2}} % totally not \dfrac ripoff
\newcommand{\dilim}[1]{\dlim{#1}{\infty}} % infinite limits
\newcommand{\ilim}[1]{\lim_{#1\to\infty}}
\usepackage{cancel}

% Auto-number questions
\newcommand{\QType}{Q}
\renewcommand{\theparagraph}{\QType\ifnum\value{paragraph}<10 0\fi\arabic{paragraph}}
\setcounter{secnumdepth}{6}
\newcommand{\question}{\par\refstepcounter{paragraph}\textbf{\theparagraph}.\space}

% Question sections
\titleformat{\section}{\normalsize\bfseries}{\thesection}{1em}{}
\newcommand{\qsection}[2]{%
  \renewcommand{\QType}{#2}
  \section*{#1}
  \refstepcounter{section}
}

\title{MATH 135 Fall 2020: Extra Practice 8}

\begin{document}
\thispagestyle{firstpage}

\textbf{\@title}

\qsection{Warm-Up Exercises}{WE}

\question Is 7386458999999992324343123 divisible by 11?
\begin{proof}[Solution]
  We may simply apply Proposition 9 from the course notes: an integer is divisible by 11
  if the difference of the sums of the even and odd digits is divisible by 11.

  The even digits are $7+8+4+8+9+9+9+9+3+4+4+1+3 = 78$
  and the odd digits are $3+6+5+9+9+9+9+2+2+3+3+2 = 62$.
  We have $78 - 62 = 16$ which is not divisible by 11.

  Therefore, $11 \nmid 7386458999999992324343123$.
\end{proof}


\question For each linear congruence, determine the complete solution, if a solution exists.
\begin{enumerate}[(a)]
  \item $3x \equiv 11 \pmod{18}$
        \begin{proof}[Solution]
          Notice that $\gcd(3,18)=3$ and $3 \nmid 11$.
          Therefore, by LCT, there are no solutions.
        \end{proof}
  \item $4x \equiv 5 \pmod{21}$
        \begin{proof}[Solution]
          Notice that $\gcd(4,21)=1$ and $1 \mid 5$.
          Therefore, LCT guarantees a set of solutions where $x \equiv x_0 \pmod{21}$
          for some particular solution $x_0$.

          By inspection, $21+4(-4) = 5$, so $-4 \equiv 5 \pmod{21}$.

          Therefore, the set of solutions is $x \in [-4]_{21} = [17]_{21}$.
        \end{proof}
\end{enumerate}


\question Complete the addition and multiplication tables for $\Z_5$.
\begin{proof}[Solution]
  The elements of $\Z_5$ are $\{[0], [1], [2], [3], [4]\}$:
  \begin{center}
    \begin{tabular}{c|c|c|c|c|c}
      $+$   & $[0]$ & $[1]$ & $[2]$ & $[3]$ & $[4]$ \\ \hline
      $[0]$ & $[0]$ & $[1]$ & $[2]$ & $[3]$ & $[4]$ \\
      $[1]$ & $[1]$ & $[2]$ & $[3]$ & $[4]$ & $[0]$ \\
      $[2]$ & $[2]$ & $[3]$ & $[4]$ & $[0]$ & $[1]$ \\
      $[3]$ & $[3]$ & $[4]$ & $[0]$ & $[1]$ & $[2]$ \\
      $[4]$ & $[4]$ & $[0]$ & $[1]$ & $[2]$ & $[3]$ \\
    \end{tabular}
    \quad
    \begin{tabular}{c|c|c|c|c|c}
      $\times$ & $[0]$ & $[1]$ & $[2]$ & $[3]$ & $[4]$ \\ \hline
      $[0]$    & $[0]$ & $[0]$ & $[0]$ & $[0]$ & $[0]$ \\
      $[1]$    & $[0]$ & $[1]$ & $[2]$ & $[3]$ & $[4]$ \\
      $[2]$    & $[0]$ & $[2]$ & $[4]$ & $[1]$ & $[3]$ \\
      $[3]$    & $[0]$ & $[3]$ & $[1]$ & $[4]$ & $[2]$ \\
      $[4]$    & $[0]$ & $[4]$ & $[3]$ & $[2]$ & $[1]$ \\
    \end{tabular}
  \end{center}
\end{proof}


\question What is the remainder when $14^{43}$ is divided by 41?
\begin{proof}[Solution]
  Since 41 is prime and $41 \nmid 14$, we may apply Fermat's Little Theorem.
  \begin{equation*}
    14^{41-1} = 14^{40} \equiv 1 \pmod{41}
  \end{equation*}
  Now, simply apply modular arithmetic: $14^2 = 196 \equiv -9 \pmod{41}$,
  and $14^3 = 14 \cdot 14^2 \equiv 14 \cdot -9 \equiv -3 \pmod{41}$.
  Finally, $14^{40} \cdot 14^3 = 14^{43} \equiv 1 \cdot -3 \equiv 38 \pmod{41}$.
  Therefore, the remainder is 38.
\end{proof}


\question Solve \begin{align*}
  x & \equiv 7 \pmod{11} \\
  x & \equiv 5 \pmod{12}
\end{align*}
\begin{proof}[Solution]
  We apply the Chinese Remainder Theorem since $\gcd(11,12)=1$.
  Solutions to the first equation are $x \equiv 7,18,29,40,51,62,73,84,95,106,117,128 \pmod{132}$.
  Solutions to the second are $x \equiv 5,17,29,41,53,65,77,89,101,113,125 \pmod{132}$.
  The unique solution common to these is $x \equiv 29 \pmod{132}$.
\end{proof}


\question What is the smallest non-negative integer $x$ such that $2000 \equiv x \pmod{37}$?
\begin{proof}[Solution]
  Simply reduce using the division algorithm, which guarantees a minimal non-negative remainder below 37:
  we have $2000 = 37(54) + 2$, so $2000 \equiv 2 \pmod{37}$.
\end{proof}

\end{document}