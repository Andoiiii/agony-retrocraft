\documentclass[11pt]{article}

\usepackage{physics}
\usepackage{amsfonts,amsmath,amssymb,amsthm}
\usepackage{enumerate}
\usepackage{titlesec}
\usepackage{fancyhdr}
\usepackage{multicol}

\headheight 13.6pt
\setlength{\headsep}{10pt}
\textwidth 15cm
\textheight 24.3cm
\evensidemargin 6mm
\oddsidemargin 6mm
\topmargin -1.1cm
\setlength{\parskip}{1.5ex}
\parindent=0pt

\author{James Ah Yong}

\pagestyle{fancy}
\fancyhf{}
\fancyfoot[c]{\thepage}
\makeatletter
\lhead{\@title}
\rhead{\@author}

\fancypagestyle{firstpage}{
  \fancyhf{}
  \rhead{\@author}
  \fancyfoot[c]{\thepage}
}

% Sets
\newcommand{\N}{\mathbb{N}}
\newcommand{\Z}{\mathbb{Z}}
\newcommand{\Q}{\mathbb{Q}}
\newcommand{\R}{\mathbb{R}}
\newcommand{\C}{\mathbb{C}}
\newcommand{\U}{\mathcal{U}}
\newcommand{\sym}{\mathbin{\triangle}}

% Functions
\DeclareMathOperator{\sgn}{sgn}
\DeclareMathOperator{\im}{im}

% Operators
\newcommand{\Rarr}{\Rightarrow}
\newcommand{\Larr}{\Leftarrow}
\usepackage{mathtools} % for \DeclarePairedDelimiter macro
\DeclarePairedDelimiter\ceil{\lceil}{\rceil}
\DeclarePairedDelimiter\floor{\lfloor}{\rfloor}

% Macros
% properly typeset ε-δ (epsilon en dash delta)
\newcommand{\epsdel}[1][\delta]{\ensuremath{\epsilon\mathit{\textnormal{--}}#1}}
\newcommand{\by}[1]{& \text{by #1}}
\newcommand{\IH}{\by{inductive hypothesis}}
% multiple choice (remove spacing between items)
\newenvironment{choices}
{\begin{enumerate}[(a)]
    \setlength{\parskip}{0ex}
    }{
  \end{enumerate}}

% Typesetting
\usepackage{array}   % for \newcolumntype macro
\newcolumntype{C}{>{$}c<{$}} % math version of "C" column type
\newcommand{\dlim}[2]{\displaystyle\lim_{#1\to#2}} % totally not \dfrac ripoff
\newcommand{\dilim}[1]{\dlim{#1}{\infty}} % infinite limits
\newcommand{\ilim}[1]{\lim_{#1\to\infty}}
\usepackage{cancel}

% Auto-number questions
\newcommand{\QType}{Q}
\renewcommand{\theparagraph}{\QType\ifnum\value{paragraph}<10 0\fi\arabic{paragraph}}
\setcounter{secnumdepth}{6}
\newcommand{\question}{\par\refstepcounter{paragraph}\textbf{\theparagraph}.\space}

% Question sections
\titleformat{\section}{\normalsize\bfseries}{\thesection}{1em}{}
\newcommand{\qsection}[2]{%
  \renewcommand{\QType}{#2}
  \section*{#1}
  \refstepcounter{section}
}

\title{MATH 135 Winter 2020: Midterm Examination}

\begin{document}
\thispagestyle{firstpage}

\textbf{\@title}

\question Let $P$ and $Q$ be logical statements.
\begin{enumerate}[(a)]
  \item Complete the following truth table:
        \begin{center}
          \begin{tabular}{C|C||C|C|C||C}
            P & Q & Q \implies P & P \land (Q \implies P) & (Q \Rarr P) \iff [P \land (Q \Rarr P)] & P \lor Q \\ \hline
            T & T & T            & T                      & T                                      & T        \\
            T & F & T            & T                      & T                                      & T        \\
            F & T & F            & F                      & T                                      & T        \\
            F & F & T            & F                      & F                                      & F
          \end{tabular}
        \end{center}
  \item Are the expressions $(Q \implies P) \iff [P \land (Q \implies P)]$ and $P \lor Q$ logically equivalent?
        Circle one of the options below. No justification required.

        \fbox{Equivalent} \quad Not equivalent
\end{enumerate}


\question Let $x$ and $y$ be real numbers. Consider the following implication $S$:
\begin{center}
  If $x$ is rational, then $y$ is rational or $xy$ is irrational
\end{center}
\begin{enumerate}[(a)]
  \item State the hypothesis of $S$. \begin{center}
          $x$ is rational
        \end{center}
  \item State the conclusion of $S$. \begin{center}
          $y$ is rational or $xy$ is irrational
        \end{center}
  \item State the converse of $S$. \begin{center}
          If $y$ is rational or $xy$ is irrational, then $x$ is rational
        \end{center}
  \item State the contrapositive of $S$. \begin{center}
          If $y$ is irrational and $xy$ is rational, then $x$ is irrational
        \end{center}
  \item State the negation of $S$ in a form that does not contain an implication. \begin{center}
          $x$ and $xy$ are irrational but $y$ is rational
        \end{center}
  \item Indicate whether the statement $\forall x, y \in R$,
        $S$ is true or false by circling one of the options below.
        Then either prove or disprove the statement.

        Circle the correct answer: \quad \fbox{True} \quad False
        \begin{proof}
          We will prove by the contrapositive.
          Consider the hypothesis.
          Because multiplying any real number by an irrational number produces an irrational number,
          we can never have $y$ be irrational but $xy$ be rational.

          Therefore, the contrapositive is vacuously true.
        \end{proof}
\end{enumerate}


\question Let $A = \{2k : k\in\Z\}$ and $B = \{2m + 1 : m\in\Z\}$.
In (a) and (b), indicate whether each statement is true or false by circling one of the options.
Then either prove or disprove the statement.
\begin{enumerate}[(a)]
  \item $\forall a \in A, \exists b \in B, a + b = 7$ \\
        \fbox{True} \quad False
        \begin{proof}
          Let $a$ be an element of $A$, that is, an even integer.
          Then, there exists a $k$ such that $a=2k$.

          Select $b = -2k + 7$.
          This is equal to $2(-k+3)+1$, where $-k+3$ is an integer.
          It follows that $b \in B$.

          Now, $a + b = 2k - 2k + 7 = 7$, as desired.
        \end{proof}
  \item $\exists b \in B, \forall a \in A, a + b = 7$ \\
        True \quad \fbox{False}
        \begin{proof}
          Consider the negation:
          \[ \forall b \in B, \exists a \in A, a + b \neq 7. \]
          Let $b$ be an element of $B$, that is, an odd integer.
          Then, there exists an $m$ such that $b=2m+1$.

          If $m=3$, so $b=7$, then select $a=2$ (i.e.\ $2k$ where $k=1$) and notice $a+b=9 \neq 7$.
          If $m\neq 3$, so $b\neq 7$ then select $a=0$ (i.e.\ $2k$ where $k=0$).
          In this case, $a+b = b \neq 7$.

          Therefore, since the negation is true, the original statement is false.
        \end{proof}
\end{enumerate}


\question \begin{enumerate}[(a)]
  \item Prove that for all integers $a$, if $a \nmid 1$, then $a \nmid 9$ or $a \nmid 17$.
        \begin{proof}
          Consider the contrapositive: if $a \mid 9$ and $a \mid 17$, then $a \mid 1$.

          Let $a$ be an integer that divides both 9 and 17.
          Then, $a$ also divides the integer combination $9(2) - 17(1) = 1$, as desired.

          Because the contrapositive is true, the original statement is also true.
        \end{proof}
  \item Prove that for all integers $a$, $b$, and $c$, if $a \mid (b + c)$,
        then $a^2 \nmid (2b + 3c)$ or $a\mid c$.
        \begin{proof}
          Let $a$, $b$, and $c$ be integers.

          Consider the negation, that $a \mid (b+c)$, $a^2 \mid (2b+3c)$, and $a \nmid c$.
          Suppose for a contradiction that the negation is true.

          Clearly, $a \mid a^2$, so by TD, $a \mid (2b+3c)$.
          But we have $a \mid (b+c)$, so this means that by DIC, $a \mid ((2b+3c)-2(b+c))$.
          This is just $a \mid c$, which is a contradiction.

          Therefore, the negation is false, so the original statement must be true.
        \end{proof}
\end{enumerate}


\question Let $a$, $b$, and $c$ be odd integers.
Prove that there does not exist a right triangle with side lengths $a$, $b$, and $c$.
\begin{proof}
  Suppose for a contradiction that such a right triangle exists.

  Then, there exist $a$, $b$, and $c$ such that $a^2+b^2=c^2$.
  Since they are odd, we may write $a$, $b$, and $c$ as $2r+1$, $2s+1$, and $2t+1$, respectively.
  \begin{align*}
    a^2 + b^2                     & = c^2              \\
    (2r+1)^2 + (2s+1)^2           & = (2t+1)^2         \\
    4r^2 + 4r + 1 + 4s^2 + 4s + 1 & = 4t^2 + 4t + 1    \\
    2(2r^2 + 2r + 2s^2 + 2s + 1)  & = 2(2t^2 + 2t) + 1
  \end{align*}
  Since $2r^2 + 2r + 2s^2 + 2s + 1$ and $2t^2 + 2t$ are both integers,
  the left-hand side represents an even integer and the right-hand side represents an odd integer.
  There are no integers that are both even and odd, so this is a contradiction.

  Therefore, there is no right triangle with three odd side lengths.
\end{proof}


\question Let $a$ be an integer. Prove that if $4 \mid a(a+x)$ for all integers $x$, then $4 \mid a$.
\begin{proof}
  Let $a$ be an integer such that $4 \mid a(a+x)$ for all integers $x$.

  The statement must be true for all integers $x$, and $-a+1$ is an integer, so we may let $x=-a+1$.
  Then, $4 \mid a(a-a+1)$, which is just $4 \mid a$.
\end{proof}


\question \begin{enumerate}[(a)]
  \item Determine the coefficient of $x^4$ in the expansion of $\left(2x^{14} + \dfrac{1}{x^3}\right)^{10}$
        \begin{proof}[Solution]
          Apply the binomial theorem:
          \begin{align*}
            \left(2x^{14} + \frac{1}{x^3}\right)^{10}
             & = \sum_{k=0}^{10} \binom{10}{k} (2x^{14})^k \left(\frac{1}{x^3}\right)^{10-k} \\
             & = \sum_{k=0}^{10} \binom{10}{k} 2^k x^{14k} x^{-3(10-k)}                      \\
             & = \sum_{k=0}^{10} \binom{10}{k} 2^k x^{17k-30}
          \end{align*}
          The exponent on $x$ will be 4 when $17k-30=4$, that is, $k=2$.
          Here, the coefficient is $\binom{10}{2}2^2 = 45 \cdot 4 = 180$.
        \end{proof}
  \item Evaluate the sum $\displaystyle\sum_{i=0}^n\binom{n}{i}\frac{3^{2i}\,5^{n-i}}{2^{3i}}$
        \begin{proof}[Solution]
          Rearrange terms to match the binomial theorem and apply it:
          \begin{align*}
            \sum_{i=0}^n\binom{n}{i}\frac{3^{2i}\,5^{n-i}}{2^{3i}}
             & = \sum_{i=0}^n\binom{n}{i} \frac{(3^2)^i}{(2^3)^i} 5^{n-i}    \\
             & = \sum_{i=0}^n\binom{n}{i} \left(\frac{9}{8}\right)^i 5^{n-i} \\
             & = \left( \frac{9}{8} + 5 \right)^{n}                          \\
             & = \left( \frac{49}{8} \right)^{n}
          \end{align*}
          or, expressed similarly to the original, $\dfrac{7^{2n}}{2^{3n}}$.
        \end{proof}
\end{enumerate}


\question \begin{enumerate}[(a)]
  \item Let $A$, $B$, and $C$ be sets.
        Prove that if $A-B \subseteq C$, then $A - (B \cup C) = \emptyset$.
        \begin{proof}
          Let $A$, $B$, and $C$ be sets such that $A-B \subseteq C$.
          Let $a$ be an element of $A$.

          Suppose for a contradiction that $a$ is in neither $B$ nor $C$.
          Then, because $a\not\in B$, it is in $A-B$.
          However, $A-B \subseteq C$, so $a \in C$.
          This is a contradiction.

          Therefore, $a$ is in either $B$ or $C$, that is, it is in $B \cup C$.
          Thus, since all elements of $A$ are in $B \cup C$, the set difference is the empty set.
        \end{proof}
  \item Consider the sets \begin{equation*}
          A = \{n\in\N : n \geq 2\}, \quad B=\{a\in A : 3 \mid (2a+1)\}, \quad C=\{(2k+5)^2 : k\in\Z\}.
        \end{equation*}
        Prove that $B \cap C \neq \emptyset$.
        \begin{proof}
          It suffices to show the existence of an element of $B \cup C$.
          We propose $x = 49$ and prove it.

          Clearly, $49 \geq 2$, so $x \in A$.
          Also, $2x+1 = 99$, a multiple of 3.
          Therefore, $x \in B$.

          Let $k = 1$, which is an integer.
          Then, $(2k+5)^2 = 7^2 = 49$.
          Therefore, $x \in C$.

          Since $x$ is an element in both $B$ and $C$, it is in their intersection.
          As we have proven that the size of $B \cup C$ is at least 1, it is not the empty set.
        \end{proof}
\end{enumerate}


\question The Fibonacci sequence is defined by $f_0=0$, $f_1=1$,
and $f_n=f_{n-1}+f_{n-2}$ for all integers $n \geq 2$.
Prove that for every non-negative integer $n$,
\[ \sum_{i=0}^n f^2_i = f_n f_{n+1}.\]
\begin{proof}
  Let $P(n)$ be the statement that $\displaystyle\sum_{i=0}^n f^2_i = f_n f_{n-1}$.
  We will prove by induction.

  For the base case $P(0)$, notice that
  \[ \sum_{i=0}^0 f_i^2 = f_0^2 = 0 = 0\cdot 1 = f_0 f_1 \]

  Now, suppose that for an arbitrary non-negative integer $k$, $P(k-1)$ holds. Then,
  \begin{align*}
    \sum_{i=0}^k f^2_i & = f_k^2 + \sum_{i=0}^{k-1} f^2_i \\
                       & = f_k^2 + f_{k-1} f_k \IH        \\
                       & = f_k(f_k + f_{k-1})             \\
                       & = f_k f_{k+1}
  \end{align*}
  which is exactly $P(k)$.

  Therefore, by induction, $P(n)$ holds for all non-negative integers.
\end{proof}


\question Prove that for all $n\in\N$, there exist non-negative integers $a$ and $b$ such that
$3 \nmid b$ and $n = 3^a b$.
\begin{proof}
  We will strongly induct the statement $P(n)$, that $a$ and $b$ exist such that $3 \nmid b$ and $n=3^a b$, on $n$.

  For the base cases $P(1)$ and $P(2)$, let $a=0$ and $b=n$.
  Then, $3 \nmid n$ and $3^a b = b = n$.
  For the base case $P(3)$, let $a=1$ and $b=1$.
  Then, $3 \nmid 1$ and $3^a b = 3$.

  Let $m \geq 4$ be an arbitrary integer.
  Suppose that for all integers $n < m$, $P(n)$ holds.

  If $3 \mid m$, then there exists an integer $k$ such that $m=3k$ we use the fact that $P(k)$ holds.
  This ensures that there exist $a_0$ and $b_0$ such that $3^{a_0}b_0 = k$. Rearranging,
  \begin{align*}
    3^{a_0}b_0 & = k            \\
    3^{a_0}b_0 & = \frac{m}{3}  \\
    m          & = 3^{a_0+1}b_0
  \end{align*}
  which, with $a=a_0+1$ and $b=b_0$, is exactly what we need to show to prove $P(m)$.
\end{proof}

\end{document}