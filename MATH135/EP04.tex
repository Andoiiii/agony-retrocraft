\documentclass{agony}
\title{MATH 135 Fall 2020: Extra Practice 4}

\begin{document}
\thispagestyle{firstpage}

\textbf{\thetitle}

\qsection{Warm-Up Exercises}{WE}

\question Evaluate $\displaystyle\sum^8_{i=3}2^i$ and $\displaystyle\prod^5_{j=1}\frac{j}{3}$.
\begin{proof}[Solution]
  Simply expand along the sum/product:
  \[ \sum^8_{i=3}2^i = 2^3 + 2^4 + 2^5 + 2^6 + 2^7 + 2^8 = 8+16+32+64+128+256 = 504 \]
  and
  \begin{equation*}
    \prod^5_{j=1}\frac{j}{3}
    = \frac13\cdot\frac23\cdot\frac33\cdot\frac43\cdot\frac53
    = \frac{120}{243}
    = \frac{40}{81} \qedhere
  \end{equation*}
\end{proof}


\question Let $x$ be a real number.
Using the Binomial Theorem, expand $\left(x-\frac{1}{x}\right)^7$.
\begin{proof}[Solution]
  Recall the Binomial Theorem, that $(a+b)^n = \sum_{k=0}^n\binom{n}{k}a^{n-k}b^k$.
  Now, substitute $a = x$ and $b = -\frac{1}{x}$.
  \begin{align*}
    \left(x-\frac{1}{x}\right)^7
     & = \sum_{k=0}^7\binom{7}{k}x^{7-k}\left(-\frac{1}{x}\right)^k                          \\
     & = \sum_{k=0}^7\binom{7}{k}x^{7-k}x^{-k}(-1)^k                                         \\
     & = \sum_{k=0}^7\binom{7}{k}(-1)^k x^{7-2k}                                             \\
     & = x^7-7x^{7-2}+21x^{7-4}-35x^{7-6}+35x^{7-8}-21x^{7-10}+7x^{7-12}-x^{7-14}            \\
     & = x^7-7x^5+21x^3-35x+\frac{35}{x}-\frac{21}{x^3}+\frac{7}{x^5}-\frac{1}{x^7} \qedhere
  \end{align*}
\end{proof}


\qsection{Recommended Problems}{RP}

\question Prove the following statements by induction.
\begin{enumerate}[(a)]
  \item For all $n\in\N$, $\displaystyle \sum^n_{i=1} (2i-1) = n^2$.
        \begin{proof}
          We will induct the statement $P(n) \equiv \sum^n_{i=1} (2i-1) = n^2$ on $n$.

          (Base Case) When $n = 1$, the left-hand side is
          \begin{align*}
            \sum^1_{i=1} (2i-1) & = 2(1)-1 \\
                                & = 1      \\
                                & = 1^2
          \end{align*}
          which is the right-hand side, so $P(1)$ holds.

          (Inductive Step) Now, suppose that $P(k)$ holds for an arbitrary $k$.
          Then, we take the left-hand side of $P(k+1)$
          \begin{align*}
            \sum^{k+1}_{i=1} (2i-1)
             & = (2(k+1)-1) + \sum^k_{i=1} (2i-1) \\
             & = (2k+1) + k^2 \IH                 \\
             & = (k+1)^2
          \end{align*}
          as desired to show that if $P(k)$ holds, then $P(k+1)$ holds.

          Therefore, by induction, $P(n)$ holds for all $n$.
        \end{proof}
  \item For all $n\in\N$, $\displaystyle \sum^n_{i=0} r^i = \frac{1-r^{n+1}}{1-r}$ where $r$ is any real number such that $r \neq 1$.
        \begin{proof}
          Let $r$ be an arbitrary real other than 1.
          We will induct the statement $P(n) \equiv \sum^n_{i=0} r^i = \frac{1-r^{n+1}}{1-r}$ on $n$.

          (Base Case) For $n=1$, substitute into the LHS and expand the summation:
          \begin{equation*}
            \sum^1_{i=1} r^i = r^0 + r^1 = 1 + r = (1+r)\frac{1-r}{1-r} = \frac{1-r^2}{1-r}
          \end{equation*}
          This is precisely the RHS of the equality, so $P(1)$ holds.

          (Inductive Step) Now, suppose that $P(k)$ holds for an arbitrary $k$.
          Again, expand the summation but for the LHS of $P(k+1)$:
          \begin{align*}
            \sum^{k+1}_{i=0} r^i
             & = r^{k+1} + \sum^k_{i=0} r^i                  \\
             & = r^{k+1} + \frac{1-r^{k+1}}{1-r} \IH         \\
             & = \frac{(r^{k+1})(1-r) + 1 - r^{k+1}}{1-r}    \\
             & = \frac{r^{k+1} - r^{k+2} + 1 - r^{k+1}}{1-r} \\
             & = \frac{1 - r^{k+2}}{1-r}                     \\
          \end{align*}
          which is the other side of the equality.
          We have proved that if $P(n)$ holds, then $P(n+1)$ holds.
          Therefore, by induction, $P(n)$ holds for all natural $n$.
        \end{proof}
  \item For all $n\in\N$, $\displaystyle \sum^n_{i=1} \frac{i}{(i+1)!} = 1-\frac{1}{(n+1)!}$.
        \begin{proof}
          We will induct the statement $P(n) \equiv \sum^n_{i=1} \frac{i}{(i+1)!} = 1-\frac{1}{(n+1)!}$ on $n$.

          First, verify the base case, $P(1)$.
          Then, we let $n=1$ and have
          \[ \sum^1_{i=1} \frac{i}{(i+1)!} = 1-\frac{1}{2!} \]
          Expanding the summation, we can show that $P(1)$ holds:
          \[ \sum^1_{i=1} \frac{i}{(i+1)!} = \frac{1}{2!} = \frac{1}{2} = 1-\frac{1}{2} = 1-\frac{1}{2!} \]

          Now, suppose $P(k)$ is true for some $k$, and consider $P(k+1)$:
          \[ \sum^{n+1}_{i=1} \frac{i}{(i+1)!} = 1-\frac{1}{(n+2)!} \]
          Like above, we take out a term of the summation and simplify, so we have
          \begin{align*}
            \sum^{k+1}_{i=1} \frac{i}{(i+1)!}
             & = \frac{k+1}{(k+2)!} + \sum^{k}_{i=1} \frac{i}{(i+1)!} \\
             & = \frac{k+1}{(k+2)!} + 1 - \frac{1}{(k+1)!} \IH        \\
             & = 1 + \frac{(k+1) - (k+2)}{(k+2)!}                     \\
             & = 1 - \frac{1}{(k+2)!}
          \end{align*}
          as required.
          We have proven $P(1)$ and that $P(k)$ implies $P(k+1)$, so, by induction, $P(n)$ is true for all natural $n$.
        \end{proof}
  \item For all $n\in\N$, $\displaystyle \sum^n_{i=1} \frac{i}{2^i} = 2-\frac{n+2}{2^n}$.
        \begin{proof}
          For induction on $n$, let $P(n) \equiv \sum^n_{i=1} \frac{i}{2^i} = 2-\frac{n+2}{2^n}$.

          Verify the base case $P(1)$:
          \[ \sum^1_{i=1} \frac{i}{2^i} = \frac{1}{2} = 2 - \frac{3}{2} = 2 -\frac{1+2}{2^1} \]
          Suppose that $P(k)$ holds for some $k$, and consider $P(k+1)$. Now,
          \begin{align*}
            \sum^{n+1}_{i=1} \frac{i}{2^i}
             & = \frac{k+1}{2^{k+1}} + \sum^n_{i=1} \frac{i}{2^i} \\
             & = \frac{k+1}{2^{k+1}} + 2-\frac{k+2}{2^k} \IH      \\
             & = 2 + \frac{k+1 - 2(k+2)}{2^{k+1}}                 \\
             & = 2 - \frac{k+3}{2^{k+1}}
          \end{align*}
          as required.
          Because $P(1)$ holds and $P(k)$ implies $P(k+1)$, by induction, $P(n)$ holds for all $n$.
        \end{proof}
  \item For all $n\in\N$, where $n \geq 4$, $n! > n^2$.
        \begin{proof}
          We will prove by induction on $n$.
          Let $P(n)$ be the statement $n! > n^2$.

          To verify the base case $P(4)$, notice that $4! = 24$, that $4^2 = 16$, and that $24 > 16$.

          Now, suppose that $P(k)$ is true for some $k \geq 4$.
          We must show that $P(k+1)$ holds, i.e., $(k+1)! > (k+1)^2$.

          First, notice that $x^2 > x+1$ for all $x \geq 4$.
          Then, we can state the inductive hypothesis as $k! > k+1$.
          Multiplying both sides by $k+1$ gives $(k+1)! > (k+1)^2$, as desired.

          Therefore, by induction, $n! > n^2$ for all $n \geq 4$.
        \end{proof}
  \item For all $n\in\N$, $4^n-1$ is divisible by 3.
        \begin{proof}
          Induct the statement ``$4^n-1$ is divisible by 3'' on $n$.

          For the base case, let $n=1$ so $4^1-1=3$ and 3 is obviously divisible by 3.

          Now, suppose that $4^k-1$ is divisible by 3 for some natural number $k$.
          By definition, there exists an integer $a$ where $4^k-1 = 3a$.

          Consider when $n = k+1$. Rearranging,
          $4^{k+1}-1 = (4^{k+1} - 4) + 3 = 4(4^k-1)+3$.
          By our inductive hypothesis, this is equal to $4(3a)+3 = 3(4a+1)$.
          Then, since $4^{k+1}-1$ can be written as $3b$ for some integer $b$ (namely, $b = 4a+1$),
          it is by definition divisible by 3.

          Therefore, by induction, $4^n-1$ is divisible by 3 for all $n\in\N$.
        \end{proof}
\end{enumerate}


\question Let $x$ be a real number.
Find the coefficient of $x^{19}$ in the expansion of $(2x^3-3x)^9$.
\begin{proof}[Solution]
  Recall the Binomial Theorem, $(a+b)^n = \sum_{k=0}^n\binom{n}{k}a^{n-k}b^k$.
  Let $a = 2x^3$, $b = -3x$, and $n=9$.
  Then, we have $(2x^3-3x)^9 = \sum_{k=0}^9\binom{9}{k}2^{9-k}(-3)^k x^{27-2k}$.
  We only care about when the exponent on $x$ is 19, i.e., $27-2k=19 \implies k = 4$.
  On this term of the summation, we have $\binom{9}{4}2^5 (-3)^4 x^{19}$.

  The coefficient is $\binom{9}{4}2^5 (-3)^4 = 126 \cdot 32 \cdot 81 = 326592$.
\end{proof}


\question Let $n$ be a non-negative integer.
Prove that $\displaystyle \sum_{k=0}^n \binom{n}{k} = 2^n$.
\begin{proof}
  We will induct the statement $P(n) \equiv \sum_{k=0}^n \binom{n}{k} = 2^n$ on $n \geq 0$.

  For the base case, $P(0)$, we have \[ \sum_{k=0}^0 \binom{0}{k} = \binom{0}{0} = 1 = 2^0. \]

  Now, suppose $P(m)$ is true for some $m \geq 0$.
  Consider the summation in $P(m+1)$:
  \begin{align*}
    \sum_{k=0}^{m+1} \binom{m+1}{k}
     & = \binom{m+1}{m+1} + \sum_{k=0}^m \binom{m+1}{k}                                \\
     & = \binom{m+1}{m+1} + \sum_{k=0}^m \left( \binom{m}{k} + \binom{m}{k-1}  \right)
    \by{Pascal's identity}                                                             \\
     & = \binom{m+1}{m+1} + \sum_{k=0}^m \binom{m}{k} + \sum_{k=0}^m \binom{m}{k-1}    \\
     & = 1 + 2^k + \sum_{k=0}^m \binom{m}{k-1} \IH
  \end{align*}
  Recall that negative binomial coefficients are undefined, so we can change the variable in the summation with $j=k+1$ and ignore the $k=0$ term.
  Add and subtract a $\binom{m}{m}$ term to round out the summation and apply the IH once more:
  \begin{align*}
    \sum_{k=0}^{m+1} \binom{m+1}{k}
     & = 1 + 2^k + \sum_{j=0}^{m-1} \binom{m}{j}                               \\
     & = 1 + 2^k + \sum_{j=0}^{m-1} \binom{m}{j} + \binom{m}{m} - \binom{m}{m} \\
     & = 1 + 2^k + \sum_{j=0}^{m} \binom{m}{j} - 1                             \\
     & = 1 + 2^k + 2^k - 1 \IH                                                 \\
     & = 2^{k+1}
  \end{align*}
  which is what we wanted to show that $P(m+1)$ is true.

  Therefore, by induction, $P(n)$ is true for all non-negative integer $n$.
\end{proof}


\question Let $n$ be a non-negative integer.
Prove by induction on $k$ that $\sum^k_{j=0}\binom{n+j}{j} = \binom{n+k+1}{k}$
for all non-negative integers $k$.
\begin{proof}
  Let $n \geq 0$ be an integer, and let $P(k)$ be the statement $\sum^k_{j=0}\binom{n+j}{j} = \binom{n+k+1}{k}$.
  We will induct $P(k)$ on $k$.

  For the base case, let $k=0$.
  Then, $P(k)$ reads $\sum^0_{j=0}\binom{n+j}{j} = \binom{n+1}{0}$.
  The summation only has one term, so we have $\binom{n}{0} = \binom{n+1}{0}$ which is true for all $n$ (since $\binom{a}{0} = 1$ for all $a$).

  Now, suppose that $P(s)$ holds for some non-negative integer $s$.

  This means that $\sum^s_{j=0}\binom{n+j}{j} = \binom{n+s+1}{s}$.
  Now, consider the left-hand side of $P(s+1)$:
  \begin{align*}
    \sum^{s+1}_{j=0}\binom{n+j}{j}
     & = \binom{n+s+1}{s+1} + \sum^s_{j=0}\binom{n+j}{j} \\
     & = \binom{n+s+1}{s+1} + \binom{n+s+1}{s} \IH       \\
     & = \binom{n+s+2}{s+1} \by{Pascal's identity}
  \end{align*}
  which is exactly the right-hand side.
  Since $P(n)$ is true for $n=0$ and $P(s)$ implies $P(s+1)$, it holds for all $n \geq 0$ by induction.
\end{proof}


\question The sequence $x_1, x_2, x_3, \dots$ is defined recursively by
$x_1=8$, $x_2=32$, and $x_i = 2x_{i-1} + 3x_{i-2}$ for all integers $i \geq 3$.
Prove that for all $n\in\N$, $x_n = 2 \times (-1)^n + 10 \times 3^{n-1}$.
\begin{proof}
  We will strongly induct the statement $P(n)$, $x_n = 2(-1)^n + 10(3)^{n-1}$, on $n$.

  For a base case, let $n = 1$.
  Then, $2(-1)^1 + 10(3)^0 = -2+10 = 8$, which is the defined value of $x_1$.
  For another, let $n = 2$.
  Then, $2(-1)^2 + 10(3)^1 = 2+30 = 32$, which is the defined value of $x_2$.
  Therefore, $P(1)$ and $P(2)$ hold.

  Now, for some $m \geq 3$, suppose $P(n)$ holds for all $n < m$.
  Specifically, $P(m-1)$ and $P(m-2)$ hold.

  Consider the definition of $x_m$:
  \begin{align*}
    x_m & = 2x_{m-1} + 3x_{m-2}                                                                   \\
        & = 2\left( 2(-1)^{m-1} + 10(3)^{m-2} \right) + 3\left( 2(-1)^{m-2} + 10(3)^{m-3} \right) \\
        & = 4(-1)^{m-1} + 20(3)^{m-2} + 6(-1)^{m-2} + 30(3)^{m-3}                                 \\
        & = 4(-1)(-1)^{m-2} + 6(-1)^{m-2} + 20(3)(3)^{m-3}  + 30(3)^{m-3}                         \\
        & = 2(-1)^{m-2} + 90(3)^{m-3}                                                             \\
        & = 2(-1)^2(-1)^{m-2} + 10(3)^2(3)^{m-3}                                                  \\
        & = 2(-1)^m + 10(3)^{m-1}
  \end{align*}
  which is precisely $P(m)$.

  Therefore, by strong induction, $P(n)$ is true for all $n$.
\end{proof}


\question The sequence $t_1, t_2, t_3, \dots$ is defined recursively by
$t_1=2$ and $t_n = 2t_{n-1}+n$ for all integers $n > 1$.
Prove that for all $n\in\N$, $t_n = 5 \times 2^{n-1} - 2 - n$.
\begin{proof}
  Let $P(n)$ be the statement $t_n = 5 \times 2^{n-1} - 2 - n$.
  We will induct $P(n)$ on $n$.

  We first verify base cases: $n = 1$, hypothesized as $t_1 = 5(2)^0-2-1 = 2$, which matches the defined value;
  and $n=2$, for which $t_2$ is defined as $2(2)+2 = 6$ and we hypothesize $t_2 = 5(2)^{1}-2-2 = 6$.

  Now, let $m$ be an integer above 2 and suppose that $P(m-1)$ holds.
  Consider the definition of $t_m$:
  \begin{align*}
    t_m & = 2t_{m-1}+m                                   \\
        & = 2\left( 5(2)^{m-2} - 2 - (m-1) \right)+m \IH \\
        & = 2\left( 5(2)^{m-2} - m - 1 \right)+m         \\
        & = 5(2)^{m-1} - 2m - 2 + m                      \\
        & = 5(2)^{m-1} - 2 - m
  \end{align*}
  This is exactly $P(m)$, so $P(m-1)$ implies $P(m)$.

  Therefore, by induction, $P(n)$ is true for all natural $n$.
\end{proof}


\question The Fibonacci sequence is defined as the sequence $f_1, f_2, f_3, \dots$
where $f_1=1$, $f_2=1$ and $f_i=f_{i-1}+f_{i-2}$ for $i \geq 3$.
Use induction to prove the following statements:
\begin{enumerate}[(a)]
  \item For $n \geq 2$, $f_1 + f_2 + \cdots + f_{n-1} = f_{n+1} - 1$.
        \begin{proof}
          We will induct the statement $P(n)$, $\sum_{i=1}^{n-1} f_i = f_{n+1}-1$ on $n$.

          To verify the base case, $n = 2$, substitute and notice $f_1 = 1 = 2-1 = f_3-1$.

          Now, let $m > 2$ and suppose $P(m)$ holds. Then,
          \begin{align*}
            \sum_{i=1}^{m-1} f_i       & = f_{m+1}-1         \\
            f_m + \sum_{i=1}^{m-1} f_i & = f_m + f_{m+1} - 1 \\
            \sum_{i=1}^{m} f_i         & = f_{m+2}-1
          \end{align*}
          which is $P(m+1)$.

          Therefore, by induction, $P(n)$ is true for all $n \geq 2$.
        \end{proof}
  \item Let $a = \dfrac{1+\sqrt{5}}{2}$ and $b = \dfrac{1-\sqrt{5}}{2}$.
        For all $n\in\N$, $f_n = \dfrac{a^n-b^n}{\sqrt{5}}$.
        \begin{proof}
          Let $P(n)$ be the statement $f_n = \frac{a^n-b^n}{\sqrt{5}}$.
          We will strongly induct $P(n)$ on $n$.

          For the base cases, start with $n=1$.
          $f_1$ is defined to be 1, and $\frac{a-b}{\sqrt{5}} = \frac{\sqrt{5}}{\sqrt{5}} = 1$.
          Likewise, for $n=2$, $f_2$ is defined as 1, and
          $\frac{a^2-b^2}{\sqrt{5}}=\frac{a-b}{\sqrt{5}}(a+b)=(1)(1)=1$.

          For our inductive step, first notice that $a$ and $b$ are the roots of $x^2-x-1=0$.
          Let $x$ be either root.

          Notice that for any $n \geq 2$, we have
          \begin{align*}
            0   & = x^2-x-1                 \\
            0   & = x^{n-2}(x^2-x-1)        \\
            0   & = x^n - x^{n-1} - x^{n-2} \\
            x^n & = x^{n-1} + x^{n-2}
          \end{align*}
          Therefore, $a^n = a^{n-1}+a^{n-2}$ and $b^n = b^{n-1}+b^{n-2}$ for any $n \geq 2$.

          Now, let $m \geq 2$ and suppose $P(m-1)$ and $P(m-2)$ hold. Then, $f_m$ is defined by:
          \begin{align*}
            f_m & = f_{m-1} + f_{m-2}                                                   \\
                & = \frac{a^{m-1}-b^{m-1}}{\sqrt{5}} + \frac{a^{m-2}-b^{m-2}}{\sqrt{5}} \\
                & = \frac{(a^{m-1} + a^{m-2}) - (b^{m-1} + b^{m-2})}{\sqrt{5}}          \\
                & = \frac{a^m - b^m}{\sqrt{5}}
          \end{align*}
          Therefore, by strong induction, $P(n)$ holds for all $n$.
        \end{proof}
\end{enumerate}


\question Each of the following ``proofs'' by induction incorrectly ``proved'' a statement that is actually false.
State what is wrong with each proof.
\begin{enumerate}[(a)]
  \item The proof does not consider the given definition $x_2=20$, and $3(5)^1 = 15 \neq 20$.
        Note that the recursive definition \emph{only} applies to $x_i$ for $i \geq 3$.
  \item The proof erroneously assumes that $n=2$ always falls within the inductive hypothesis.
        However, when proving the case $n=2$ with strong induction, the only given is $n=1$.
\end{enumerate}


\question In a strange country, there are only 4 cent and 7 cent coins.
Prove that any integer amount of currency greater than 17 cents can always be formed.
\begin{proof}
  Let $P(x)$ be the statement that there exist non-negative integer $a$ and $b$ where $x=4a+7b$.
  We will strongly induct on $x > 17$.

  Verify a few base cases: \\
  For $P(18)$ (where $18 = 4(4)+2$), let $a=1$ and $b=2$, so $4(1)+7(2)=18$. \\
  For $P(19)$ (where $19 = 4(4)+3$), let $a=3$ and $b=1$, so $4(3)+7(1)=19$. \\
  For $P(20)$ (where $20 = 4(5)+0$), let $a=5$ and $b=0$, so $4(5)+7(0)=20$. \\
  For $P(21)$ (where $21 = 4(5)+1$), let $a=0$ and $b=3$, so $4(0)+7(3)=21$.

  Now, suppose for some $n > 21$, $P(m)$ holds for all $m < n$.
  Specifically, $P(n-4)$ holds.
  That is, $n-4 = 4a_0 + 7b_0$ for some $a_0$ and $b_0$.
  Equivalently, $n = 4(a_0+1) + 7b_0$.
  If we let $a = a_0+1$ and $b = b_0$, it follows that $P(n)$ holds.


  Therefore, by strong induction, $P(x)$ is true for all $x > 17$.
\end{proof}


\qsection{Challenges}{C}

\question Prove that for every positive integer,
there exists a unique way to write the integer as
the sum of distinct non-consecutive Fibonacci numbers.

\begin{proof}
  Let $f_i$ denote the $i$th Fibonacci number, i.e.,
  $f_1 = 0$, $f_2 = 1$, $f_{n+1} = f_n + f_{n-1}$.

  We begin by proving a lemma, that $f_n$ is greater than all possible
  sums of distinct, non-consecutive $f_i$ with $i < n$.
  Notice the largest such sum is $f_{n-1} + f_{n-3} + \dotsb + f_1$.
  By RP07(a), $f_{n-1} + f_{n-2} + \dotsb + f_1 = f_n - 1 < f_n$,
  so we must have $f_{n-1} + f_{n-3} + \dotsb + f_1 < f_n$
  and generally any distinct, non-consecutive sum is less than that.

  Let $P(n)$ be the statement that all positive integers $x < f_n$,
  $x = \sum_{i=1}^m f_{k_i}$ for unique, distinct, increasing, non-consecutive $k_i$.

  For the base cases $P(1)$, $P(2)$, and $P(3)$ there are no positive integers $x < 0$ or $x < 1$.
  For the base case $P(4)$, the only positive integer less than $f_4 = 2$ is $x = 1$.
  Trivially, we can uniquely write $f_1 + f_3 = 0 + 1 = 1$.

  For the inductive step, suppose that $P(n)$ holds for some $n \geq 4$.
  Let $f_n \leq x < f_{n+1}$.

  If $x$ is $f_n$, then we may write $x = f_1 + f_n$.
  That is, $x = \sum_{i=1}^2 f_{k_i}$
  with distinct, increasing, non-consecutive $k_1 = 1$ and $k_2 = n$.

  Otherwise, write $x = d + f_n$ where $0 < d < f_{n+1} - f_n$.
  Note that by definition, $d < f_{n-1} < f_n$ and $d$ is a positive integer.
  By the inductive hypothesis, $d = \sum_{i=1}^m f_{k_i}$
  for unique, distinct, increasing, non-consecutive $k_i$.
  Then, since $d < f_{n-1} < f_n$, none of the $k_i$s can be $n$ or $n-1$.
  Therefore, we let $k_{m+1} = n$ so that $x = \sum_{i=1}^{m+1} f_{k_i}$
  has distinct, increasing, non-consecutive $f_{k_i}$.

  Now, we show that the integers $k_i$ are unique.
  Suppose $x = \sum_{i=1}^{m+1} f_{k_i} = \sum_{i=1}^{m+1} f_{\ell_i}$.
  We show the terms are identical.

  Since both sums are increasing, the largest term $f_{k_{m+1}}$ is $f_n$.
  If $f_{\ell_{m+1}} > f_n$, then the sum is greater than $f_{n+1}$.
  But $x < f_{n+1}$, so this is a contradiction.
  If $f_{\ell_{m+1}} < f_n$, then by the above lemma, the sum is less than $f_n$.
  But $x \geq f_n$, so this is again a contradiction.
  Thus, $f_{\ell_{m+1}} = f_n = f_{k_{m+1}}$.

  Then, $\sum_{i=1}^m f_{\ell_i} = x - f_n = d$.
  But by the inductive hypothesis, $\sum_{i=1}^m k_i$ is a unique representation of $d$.
  It follows that the remaining $\ell_i = k_i$ for all $i \leq m$.

  Therefore, since we have proven $P(n+1)$, by induction, $P(n)$ holds for all $n$.
\end{proof}


\question Find a formula for the minimum steps required to solve the Tower of Hanoi puzzle with three pegs with $n$ rings.
Prove that your answer is correct.

\end{document}