\documentclass[notes]{agony}
\usepackage{emoji}
\newcommand{\xgets}{\xleftarrow}
\newcommand{\one}{\symbb{1}}
\newcommand{\ord}{\operatorname{ord}}
\newcommand{\leg}[2]{\qty(\frac{#1}{#2})}
\newcommand{\mgrp}[1]{(\Z/#1\Z)^\times}

\title{CO 485/685 Fall 2022: Lecture Notes}
\begin{document}
\renewcommand{\contentsname}{CO 485/685 Fall 2022:\\{\huge Lecture Notes}}
\thispagestyle{firstpage}
\tableofcontents

Lecture notes taken, unless otherwise specified,
by myself during the Fall 2022 offering of CO 485/685,
taught by David Jao.

Chapter/lecture titles are made-up nonsense and do not follow the textbook
or any other published resource.
Actually, scratch that, this entire document is nonsense because
I am literally auditing this course two nested prerequisites behind.

\chapter{Introduction to Cryptography}

\section{(09/07; skipped)}

\begin{markdown}
## Almost-Public Key Cryptosystems (09/09)

- For a symmetric key cryptosystem, require sets of key space $K$, message space $M$, and ciphertext space $C$
    - Define encryption function $Enc : K \to M \to C$ and decryption $Dec : K \to C \to M$
    - Correctness property: for all $k$, $Dec(k)$ is a left inverse of $Enc(k)$
    - Symmetric means that both decryption and encryption use shared secret $k$, which we assume is drawn randomly from $K$
- Public key encryption scheme (Diffie, Hellman, Merkle, c. 1976)
    - Setup similar: message space $M$ and ciphertext space $C$ but with two key spaces $K_1$ of public keys and $K_2$ of private keys
    - Define $Enc : K_1 \to M \to C$ and $Dec : K_2 \to C \to M$
    - Define $KeyGen : \mathbb{1}^\ell \to R \subset K_1 \times K_2$
        - For some reason, let $\mathbb{1}^n$ be the unary representation of $n$??
    - Correctness: for all $(k_1,k_2) \in R$ related, $Dec(k_2)$ is a left inverse of $Enc(k_1)$
- Merkle puzzle (1974) 
    - Each party creates ``puzzle'' which is hard to solve but not too hard
    - Alice generates 1,000,000 puzzles and sends them to Bob
    - Bob solves one of the puzzles arbitrarily and sends half of the answer to Alice
    - Alice knows the answer, so Alice knows the second half of the answer, which becomes the shared secret
    - Eve cannot (realistically) solve 500,000 puzzles in time to intercept
- Diffie--Hellman key exchange
    - Consider the multiplicative group $G = (\Z / p\Z)^* = \{1,\dotsc,p-1\}$ and some arbitrary element $g \in G$ with sufficiently large order
    - Alice privately picks some $x \in \Z$, computes $g^x$, and sends it to Bob
    - Bob privately picks some $y \in \Z$, computes $g^y$, and sends it to Alice
    - Both can now calculate a shared secret $k = g^{xy} = (g^x)^y = (g^y)^x$
    - Eve would have to solve the Diffie--Hellman problem: given $p$, $g$, $g^x$, $g^y$, find $g^{xy}$ which is known to be hard
- Clifford Cocks privately discovered RSA 1973, DH 1974 for GCHQ (if you believe the intelligence community)

## A Public Key Cryptosystem -- RSA (09/12)

- RSA (Rivest, Shamir, Adleman 1977): first cryptosystem and remains secure
- Theoretically secure, but implementations are ass (cf. ``Fuck RSA'')
- MATH 135 review of the algorithm:
    - This ``textbook RSA'' has practical flaws and is insecure
    - $KeyGen : \mathbb{1}^\ell \to (pk, sk) \in R$
        1. Choose random primes $p,q \approx 2^\ell$ where $p$ and $q$ are odd and distinct
        2. Compute $n = pq$
        3. Choose $e \in (\Z / \phi(n)\Z)^\times$ where $\phi(n) = (p-1)(q-1)$
        4. Compute $d = e^{-1} \bmod \phi(n)$
        5. Disclose public key $(n,e)$ and keep secret key $(n,d)$
    - $Enc : K_1 \to M \to C : (n,e) \mapsto m \mapsto m^e \bmod n$ where $M = (\Z/n\Z)^\times = \{ x : \Z / n\Z : \gcd(x,n) = 1\} = C$
        - Weird that $M$ depends on $n$ (part of the key). In practice, it doesn't matter because the only messages that divide $n$ are the primes, which breaks RSA anyways
    - $Dec : K_2 \to C \to M : (n,d) \mapsto c \mapsto c^d \bmod m$
- Correctness: Must show that $(m^e \bmod n)^d \bmod n = m$
    *Proof*. $(m^e \bmod n)^d \bmod n = m^{ed} \bmod n$ (exponentiation under mod). Then, since $d = e^{-1} \bmod \phi(n)$, there exists $k$ such that $de - 1 = k\phi(n)$, we have $m^{\phi(n)k + 1} \equiv (m^{\phi(n)})^km \equiv m \pmod{m}$. This holds by Euler's theorem ($\forall m \in (\Z/n\Z)^\times, m^{\phi(n)} \equiv 1 \pmod n$) or Fermat's Little Theorem + Chinese Remainder Theorem (MATH 135)
- Security: Trivial that factoring $n=pq$ breaks RSA by computing $\phi(n)$
    - Conversely, if you know $\phi(n) = (p-1)(q-1)$ you can take $q\phi(n) = (n-1)(q-1)$ and solve for $q$
        - To avoid this, use the Carmichael exponent $\lambda(n) = \lcm(p-1,q-1)$ instead of $\phi(n)$ which works. Of course, this doesn't work in practice because it's not actually that much different
    - For any non-trivial case, knowing one pair $(e,d)$ also allows factoring $n$
    - Must make an assumption about hardness to prove security:
        - Factoring assumption: factoring random integers is hard
        - RSA factoring assumption: factoring $n=pq$ is hard (see, e.g., elliptical curve algorithm which depends on size of smallest prime in the factorization)
            - Of course, quantum computing fucks all of this to hell (see troll PQRSA which uses many small primes to make terabyte-sized moduli)
        - RSA assumption: given $n$, $e$, $m^e\bmod n$, it is hard to find $m$
    - Can prove RSA assumption $\implies$ RSA works (cannot prove without assumption without better results from complexity theory)

## Security Definitions (09/14)

- Security definitions, e.g., OW-CPA, IND-CPA, IND-CCA (Boneh, Shoup)
- How secure is a cryptosystem? Specify:
    - Allowable interactions between adversaries and parties
        - Second part of abbreviation
    - Computational limits of adversary
        - Not usually specified, usually probabilistic polynomial time
    - Goal of the adversary to ``break'' the cryptosystem
        - First part of abbreviation
- OW-CPA: ``one-way chosen-plaintext attack''
    - Adversary, given public key $pk$ and encryption $c$ of message $m$ under $pk$, wants to determine $m$
    - Formally, given a random $pk$ and $c$ such that $c = Enc(pk, m)$ for some random $m$, it is infeasible for any probabilistic polynomial time algorithm $\mathcal A$ to determine $m$ with non-negligible probability. That is, $\Pr[\mathcal A(pk,c) = m] = O(\frac{1}{\lambda^c})$ for all $c > 0$.
- Easier way to formalize (``Sequences of Games'', Shoup 2004)
    - Two players: challenger $\mathcal C$ and adversary $\mathcal A$
    - Then, OW-CPA is
        1. $\mathcal C$ runs $KeyGen : \one^\lambda \xto{\$} (pk, sk)$
        2. $\mathcal C$ chooses $m \xgets{\$} M$
        3. $\mathcal C$ computes $c \gets Enc(pk, m)$
        4. $m' \xgets{\$} \mathcal A(pk, c)$
        - with the win condition that $m' = m$, and we say that a cryptosystem is OW-CPA if a probabilistic polynomial time adversary $\mathcal A$ cannot win this game with non-negligible probability
    - IND-CPA (Goldmeier, Micoli 1984): indistinguishability
        1. $\mathcal C$ runs $(pk, sk) \xgets{\$} KeyGen(\one^\lambda)$
        2. $(m_0,m_1) \xgets{\$} \mathcal A(\one^\lambda, pk)$
        3. $\mathcal C$ picks $b \xgets{\$} \{0,1\}$
        4. $\mathcal C$ computes $c \xgets{\$} Enc(pk, m_b)$
        5. $b' \xgets{\$} \mathcal A(\one^\lambda, pk, c)$
        - with the win condition $b = b'$, and a cryptosystem is IND-CPA if for all prob.\ poly.\ time $\mathcal A$, $\abs{\frac12 - \Pr[\text{win}]} = O(\frac{1}{\lambda^\varepsilon})$ for all $\varepsilon > 0$
        - Encryption function must be random, otherwise $\mathcal A$ can re-encrypt

## Actual IND-CPA systems (09/16)

- IND-CPA is the standard security definition for symmetric security
    - Ciphertext contains no information about plaintext (except length)
- Design a slightly different equivalent IND-CPA game:
    1. $\mathcal C$ runs $(pk, sk) \xgets{\$} KeyGen(\one^\lambda)$
    2. $(m_0,m_1) \xgets{\$} \mathcal A(\one^\lambda, pk)$
    3. $\mathcal C$ picks $b \xgets{\$} \{0,1\}$
    4. $\mathcal C$ computes $c_1 \xgets{\$} Enc(pk, m_b)$ and $c_2 \xgets{\$} Enc(pk, m_{b-1})$
    5. $b' \xgets{\$} \mathcal A(\one^\lambda, pk, c_1, c_2)$
- Consider textbook RSA: $\mathcal A$ can choose $m_0 \neq m_1$ and compute $Enc(pk, m_0)$ and $Enc(pk, m_1)$ which allows it to win
    - In general, this applies to any scheme with deterministic encryption
- Goldwasser-Micali (``Probabilistic Encryption'' 1982)
    1. Pick $n = pq$ (useful to have $p \equiv q \equiv 3 \pmod 4$)
    2. Pick $r \in (\Z/n\Z)^\times$ such that $r \not\equiv x^2 \pmod p$ and $r \not\equiv x^2 \pmod q$
    3. Define $pk = (n, r)$ and $sk = (p, q)$
    4. Select a message bit $b$ from $M = \{0,1\}$
    5. Encrypt $Enc(b) = r^b y^2$ for some $y \xgets{\$}(\Z/n\Z)^\times$
    - Then, decrypt by determining ciphertext's squareness mod $n$
        - This is easy with the factorization $n=pq$ by Euler's criterion ($a$ is square mod prime $p$ if and only if $a^{(p-1)/2} \equiv 1 \pmod p$)
        - Determining squareness without factorization of $n$ is hard, apparently
    - Since plaintexts are one bit, OW $\iff$ IND and this is provable under the circular-y assumption that determining squareness is hard
    - Also one bit messages are literally useless so who cares
- Elgamal (1984) (sometimes IND-CPA)
    - Publickeycryptosystemified Diffie-Hellman
    1. Setup is the same as DH, take some element $g \in G$ of a group
    2. Define $pk = g^x$ and $sk = x$
    3. Encrypt $Enc(m) = (g^y, g^{xy}\cdot m)$ for $y \xgets{\$} \Z$
    - Then, decrypt $Dec(c_1, c_2) = \frac{c^2}{c_1^x} = \frac{g^{xy}\cdot m}{(g^y)^x} = m$
    - In general, key sharing schemes can be cryptosystemified like this
    - In an IND-CPA game, given $(g^y, g^{xy}m_b)$
        - Divide out $m_0$ to get either $g^{xy}$ (if $m_b = m_0$) or garbage
        - Real challenge is distinguishing $g^{xy}$ from garbage
    - Decisional Diffie-Hellman assumption: in the following game, $\abs{\Pr[\mathcal A\text{ wins}]-\frac12}$ is negligible in $\lambda$
        1. $\mathcal C$ chooses $p \xgets{\$} \Z$ prime, $p \approx 2^\lambda$
        2. $\mathcal C$ chooses $g \in (\Z/p\Z)^\times$
        3. $\mathcal C$ chooses $x,y \xgets{\$} \Z$ and $h \xgets{\$} (\Z/p\Z)^\times$, computes $g_1 = g^x$, $g_2 = g^y$, $g_3 = g^{xy}$
        4. $\mathcal C$ chooses $b \xgets{\$} \{0,1\}$ and $g_4 = g_3$ if $b=0$ and $h$ if $b=1$
        5. $b' \gets \mathcal A(\one^\lambda,p,g,g_1,g_2,g_4)$
    - Can prove: if DDH assumption holds, Elgamal is IND-CPA
- Layers of assumptions here:
    - DLOG: given $g$ and $g^x$, it is hard to find $x$
    - CDH: given $g$, $g^x$, and $g^y$, it is hard to find $g^{xy}$ (equivalent to Elgamal being OW-CPA)
    - DDH: given $g^{xy}$ and garbage, is hard to distinguish the garbage
- How to piss off mathematicians: solving DLOG in $\Z/n\Z$ is easy but in $(\Z/p\Z)^\times$ is hard
    - But $(\Z/p\Z)^\times$ is isomorphic to $\Z/(p-1)\Z$ so DLOG difficulty must not be preserved over isomorphism
    - Specifically, DLOG is as exactly hard as computing the isomorphism (notice that we send $x \mapsto g^x$)
- DDH is actually easy in $(\Z/p\Z)^\times$, need a subgroup $G \subset (\Z/p\Z)^\times$ with $\abs{G}$ prime
\end{markdown}


\chapter{Quadratic Residues}

\begin{markdown}
## Number Theory Background (09/19)
- Recall: RSA primes are gigantic so it takes time to do operations
    - e.g. picking $e \in (\Z/\phi(n)\Z)^\times$ or finding $d = e^{-1} \pmod{\phi(n)}$ using EEA which runs in a logarithmic number of steps
    - e.g. running $Enc(m) = m^e \pmod n$ or $Dec(c) = c^d \pmod n$ using square-and-multiply which runs in a logarithmic number of steps
- Hard: picking non-squares in integers modulo $p$
    - Set of primes $\abs{((\Z/p\Z)^\times)^2} = \frac{p-1}{2}$ for odd $p > 2$
    - This is because $f(x) = x^2$ is a 2-to-1 function on $(\Z/p\Z)^\times$
        - To prove, show $f(a) = f(b) \iff a = \pm b$
        - Apply Euclid's Lemma: $p \mid (x-y)(x+y)$ implies $p \mid x-y$ or $p \mid x+y$, equivalently, $x = y \pmod p$ or $x = -y \pmod p$
        - Also another theorem: for $R$ integral domain, every polynomial of degree $n$ over $R$ has at most $n$ roots
\end{markdown}

\section{Squares Under a Modulus (09/21)}
The big problem: Given $(\Z/n\Z)^\times$ and $x \in (\Z/n\Z)^\times$, when is $x \equiv \square \pmod n$?

For example, for $\Z/15\Z$, 1 and 4 are squares;
for 8: just 1; for 7: 1, 2, and 4; and for 13: 1, 3, 4, 9, 10, and 12.

This breaks down into cases: $n$ composite, $n$ prime power, $n$ prime

\begin{theorem}
  Suppose $n = \prod p_i^{e_i}$.
  Then, $x \equiv \square \pmod n$ if and only if for all $i$,
  $x \equiv \square \pmod{p_i^{e_i}}$.
\end{theorem}
\begin{prf}
  Suppose $x = y^2 \pmod n$ for a unit $y$.
  Then, $n \mid (x - y^2)$ and $p_i^{e_i} \mid (x-y^2)$ by transitivity.
  That is, $x \equiv y^2 \pmod {p_i^{e_i}}$.
  In the reverse direction, if $p_i^{e_i} \mid (x - y^2)$ for all $i$,
  then by UPF (with some omitted detail), $n \mid (x-y^2)$.
\end{prf}

The prime power case reduces to the prime case
under condtions discovered in the homework problems lol.

\begin{theorem}
  The number of squares in $(\Z/p\Z)^\times$
  is $\frac{p-1}{2}$ for primes $p \geq 3$.
\end{theorem}
\begin{prf}
  This is because $x = y^2 = (-y)^2$ and the size of the set is $p-1$.

  Build a table $(x,g^x)$ instead of $(x,x^2)$:

  For $p=13$ and $g=2$, we get $(1,2,4,8,3,6,12=-1,-2,-4,-8,-3,-6,-12=1)$
  and the squares are the even-indexed values $(1,4,3,12,9,10,1)$.

  This works for tables starting with non-squares:
  in fact, if $g \neq \square$, then $g^3 \neq \square$
  (by the contrapositive, if $g^3 = \square$,
  then $g = \frac{g^3}{g^2} = \frac{\square}{\square} = \square$).

  This gives us the result that $g^x = g^y$ when $x \equiv y \pmod{p-1}$
  (note that this is equivalent to Fermat's Little Theorem,
  the reverse direction requires $g$ coprime to $p-1$).
\end{prf}

\begin{defn}[order]
  $\ord(a)$ is the period of $x \mapsto a^x$ for $a \in (\Z/p\Z)^\times$.

  Equivalently, $\ord(a) = \min\{t \in \Z : a^t = 1, t > 0\}$.
\end{defn}

\begin{lemma}
  Given elements $a$ and $b$, numbers $x$ and $y$:
  \begin{itemize}[nosep]
    \item $a^x = 1$ if and only if $\ord(a) \mid x$
    \item $a^x = a^y$ if and only if $x \equiv y \pmod{\ord(a)}$
    \item $\ord(a^x) = \frac{\ord(a)}{\gcd(x,\ord(a))}$
    \item If $\ord(a)$ and $\ord(b)$ are coprime,
          then $\ord(ab) = \ord(a)\ord(b)$.
  \end{itemize}
\end{lemma}
\begin{prf}
  Only prove the last one:

  Let $t=\ord(a)$, $u=\ord(b)$, $v=\ord(ab)$.
  Then, $(ab)^{tu} = a^{tu}b^{tu} = 1^u 1^t = 1$ so we have $v \mid tu$.
  Now, \Wlog, $(ab)^{vu} = 1^u = 1 \implies a^{vu}b^{vu} = a^{vu}1 = a^{vu} = 1$.
  This gives $t \mid vu$ and $t \mid v$ since $\gcd(t,u)=1$.
  Likewise, $u \mid v$ and we can conclude $tu \mid v$ because $\gcd(t,u)=1$.
  That is, $tu = v$.
\end{prf}


\section{Squares cont'd (09/23)}

\begin{defn}[primitive element]
  $g \in G$ where $\{g^n : n \in \N\} = G$.
  Also called a generator.
\end{defn}

Recall: if there exists primitive $g \in (\Z/p\Z)^\times$,
then for all $h \in (\Z/p\Z)^\times$ where $h=g^k$,
$h \equiv \square \iff \text{$k$ even}$.
We can determine squareness using this fact,
but finding $k$ such that $h = g^k$ is doing a discrete log, which is hard.

Whether or not a primitive element exists is a non-trivial observation:

\begin{theorem}[Gauss' primitive root]
  For all primes $p$, $(\Z/p\Z)^\times$ has a primitive element.
\end{theorem}
\begin{prf}
  Observe that for all polynomials $f(x) \neq 0$ over $\Z/p\Z$,
  the number of roots of $f(x)$ is at most $\deg f$.
  Note that factorization fails in $\Z/n\Z$ in general:
  e.g. $x^2 - 1 = (x-1)(x+1) = (x-3)(x-5)$ mod 8
  or something weird like $x = (3x+2)(2x+3)$ mod 6.
  We have this observation because $\Z/p\Z$ is an integral domain (and indeed, a field).

  Consider $a \in (\Z/p\Z)^\times$.

  Claim $t=\ord(a) \mid p-1$.
  Write $p-1=tq+r$. If $r=0$, done.
  If $r > 0$, $\ord(a) = r < t$, contradiction and indeed $r=0$.

  For each divisor $d$ of $p-1$, consider
  $S_d = \{x \in (\Z/p\Z)^\times : \ord(x) = d\}$.
  Then, $\bigcup_{d \mid p-1}S_d = (\Z/p\Z)^\times$
  and this is a disjoint union.
  To prove Gauss' theorem, we just need $\abs{S_{p-1}} > 0$.

  Proceed in general for arbitrary $\abs{S_d} > 0$ for all $d \mid p-1$.

  If $S_d = \varnothing$, then $\abs{S_d} = 0$.
  Otherwise, claim that $\abs{S_d} = \phi(d) = \abs{(\Z/d\Z)^\times}$.

  If $S_d$ is not empty, then $\exists a \in S_d$ where $\ord(a) = d$.
  Consider $x^d - 1$.
  The roots of this polynomial will include all elements of $S_d$ (and others).
  We can write the set of roots as exactly $\{a^0,\dotsc,a^{d-1}\}$.
  So for all $b \in S_d$, $b = a^k$ since $b$ is a root
  and we need only count those powers with order $d$.
  But that is exactly $\ord(a^i) = \frac{\ord(a)}{\gcd(i,d)} = \frac{d}{\gcd(i,d)}$.
  So we are counting the $i$ such that $\gcd(i,d) = 1$,
  which is exactly $\phi(d)$.

  Now, $p-1 = \abs{(\Z/p\Z)^\times} = \abs{\bigcup_{d \mid p-1}S_d} = \sum\abs{S_d} \leq \sum \phi(d)$ which is equal to $p-1$ by M\"obius inversion.
  That last inequality being an equality implies that
  $\abs{S_d} \neq 0$ for any $d \mid p-1$, and in particular $p-1 \mid p-1$.

  Quick combinatorical proof of this fact:
  write out all the $p-1$ fractions over $p-1$,
  then each of $\phi(d)$ is the number of fractions
  where the denominator reduces to $d$. The sum must be $p-1$.
\end{prf}

\section{Applying to DDH (09/26)}

Recall the Decisional Diffie-Hellman problem:
Given $g$, $g^x$, $g^y$, $g^z$, determine if $z = xy$. Formally, as a game:
\begin{itemize}[nosep]
  \item $\mathcal C$ chooses a bit $b \in \{0,1\}$ and $x,y \xgets{\$}\Z$
  \item $b' \gets \mathcal A(g, g^x, g^y, g^z)$ where $z \gets \begin{cases}xy & b = 0 \\ \xgets{\$}\Z & b = 1\end{cases}$
  \item Win condition: $b = b'$ with non-negligible probability
\end{itemize}
Notice that if $g$ is a primitive root, then $\abs{\{g^x : x \in \Z\}}=p-1$.
But bruteforce DLOG takes $\frac{p-1}{2}$ steps on average.
Then, Elgamal is IND-CPA $\iff$ DDH holds.

\begin{prop}
  The Decisional Diffie-Hellman assumption in $(\Z/p\Z)^\times$
  with a primitive base $g$ does not hold.
\end{prop}
\begin{prf}
  We tell squares and non-squares apart.

  Recall from last lecture's theorem we have that if $g$ is a primitive root,
  $g^x \equiv \square \pmod p \iff x \equiv 0 \pmod 2$.
  Then, by Euler's criterion, $a \equiv \square \pmod p \iff a^{(p-1)/2} \equiv 1 \pmod p$.
  Therefore, it is possible to tell the parity of $x$, $y$, and $z$
  in reasonable time using Euler's criterion (since raising to a power is easy).

  If $xy$ is odd only when $x$ and $y$ are odd,
  so if you know the parity of $z$ you can distinguish if $z=xy$
  or random with non-negligible advantage.
\end{prf}

\begin{prop}[Euler's criterion]
  $a \equiv \square \pmod p \iff a^{(p-1)/2} \equiv 1 \pmod p$
\end{prop}
\begin{prf}
  Suppose $a \equiv \square$ iff $a \equiv g^k$ for even $k=2\ell$
  iff $a^{(p-1)/2} = (g^k)^{(p-1)/2} = g^{k(p-1)/2}=(g^{p-1})^\ell=1^\ell=1$ by \FLT{}.

  Otherwise, $a \not\equiv \square$ iff $a = g^k$ for $k=2\ell+1$
  iff $a^{(p-1)/2} = (g^k)^{(p-1)/2} = g^{(p-1)/2 \cdot (2\ell+1)}=g^{(p-1)/2\cdot 2\ell}\cdot g^{(p-1)/2} = g^{(p-1)/2} \neq 1$.
  But in fact $g^{(p-1)/2} = \sqrt{g^{p-1}} = \sqrt{1} = -1$ since it is not positive 1.
\end{prf}

\begin{corollary}
  For $p > 2$, $-1$ is a square mod $p$ if and only if $p \equiv 1 \pmod 4$.
\end{corollary}
\begin{prf}
  For $-1$ to be a square, we need $(-1)^{(p-1)/2} \equiv 1 \pmod p$.
  That is, $\frac{p-1}{2}$ is even and we have $p \equiv 1 \pmod 4$.
\end{prf}

This quantity $g^{(p-1)/2}$ is useful and we give it a name:

\begin{defn}[Legendre symbol]
  For $p > 2$ and $a \in \Z/p\Z$, the quadratic character of $a$,
  written $(\frac{a}{p}) = a^{(p-1)/2}$,
  is 1 if $a \equiv \square$, 0 if $a \equiv 0$, and -1 if $a \not\equiv \square$.

  Equivalently, define $\chi_p : (\Z/p\Z)^\times \to \{\pm 1\} : a \mapsto (\frac{a}{p})$
  and notice that this is a multiplicative homomorphism that preserves $\chi_p(ab) = \chi_p(a)\chi_p(b)$.
\end{defn}

\begin{theorem}[multiplicativity]
  $(\frac{ab}{p}) = (\frac{a}{p})(\frac{b}{p})$
\end{theorem}
\begin{prf}
  $(\frac{ab}{p}) = (ab)^{(p-1)/2} = a^{(p-1)/2}b^{(p-1)/2} = (\frac{a}{p})(\frac{b}{p})$
\end{prf}

\section{Quadratic Characters in the Complex Plane (09/28)}
Recall: we have that for odd primes, $(\frac{-1}{p})=1 \iff p \equiv 1 \pmod 4$
which we proved by applying Euler's criteron.
We have the similar lemma:

\begin{lemma}
  $(\frac{2}{p}) = 1 \iff p \equiv 1,7 \pmod 8$.
\end{lemma}
\begin{prf}
  This is harder because $2^{(p-1)/2}$ is not easy to analyze, i.e.,
  the order of 2 is not easy to derive.

  What numbers, in general, have finite/known order?
  Complex roots of unity $\zeta_n = e^{2\pi i/n}$.

  We can write $\sqrt{2} = \zeta_8 + \zeta_8^7$,
  so $2^{(p-1)/2} = (\zeta_8 + \zeta_8^7)^{p-1}
    = \frac{(\zeta_8 + \zeta_8^7)^p}{\zeta_8 + \zeta_8^7}$.
  The last transformation is helpful since $p$ powers behave well mod $p$.

  Now, notice that $(x+y)^p \equiv x^p + y^p \pmod p$
  because all the other terms will have a factor of $p \mid \binom{p}{i}$.

  Therefore, $(\frac{2}{p}) \equiv 2^{(p-1)/2} \equiv \frac{\zeta_8^p + \zeta_8^{7p}}{\zeta_8+\zeta_8^7} \pmod p$.

  There are four cases for $p \pmod 8$ because we assume $p > 2$:
  \begin{enumerate}[nosep]
    \item[1.] $\zeta_8^p = \zeta_8^1$ and $\zeta_8^{7p} = \zeta_8^7$
    \item[3.] $\zeta_8^p = \zeta_8^3$ and $\zeta_8^{7p} = \zeta_8^5$
    \item[5.] $\zeta_8^p = \zeta_8^5$ and $\zeta_8^{7p} = \zeta_8^3$
    \item[7.] $\zeta_8^p = \zeta_8^7$ and $\zeta_8^{7p} = \zeta_8^1$
  \end{enumerate}
  Clearly, for $p \equiv 1,7 \pmod 8$, we have
  $\frac{\zeta_8^p + \zeta_8^{7p}}{\zeta_8 + \zeta_8^7}
    = \frac{\zeta_8 + \zeta_8^7}{\zeta_8 + \zeta_8^7} = 1$.
  Slightly less intuitively, for $p \equiv 3,5 \pmod 8$,
  notice that $\zeta_8^3 + \zeta_8^5 = -\sqrt{2}$,
  so the fractions go to $-1$.
\end{prf}

Note: We can algebraically extend $\Z/p\Z$ with the necessary complex numbers
to make the proof valid (or simply assert that the necessary roots of unity exist).

The pattern sort of extends:
\begin{itemize}[nosep]
  \item $(\frac{3}{p}) = 1$ if $p = 1,11 \bmod 12$ and $-1$ if $p = 5,7 \bmod 12$.
  \item $(\frac{5}{p}) = 1$ if $p = \pm 1,\pm 9 \bmod 20$ and $-1$ if $p = \pm 3,\pm 7 \bmod 20$.
  \item $(\frac{7}{p}) = 1$ if $p = \pm 1,\pm 3, \pm 9 \bmod 28$ and $-1$ if $p = \pm 5,\pm 11, \pm 13 \bmod 28$.
\end{itemize}
In fact, we have $(\frac{7}{p}) = 1$ if $p = \pm 1, \pm 9, \pm 25$ mod 28.
This flips the question from is 7 a square mod $p$ to asking if $p$ is a square mod 28.

\begin{lemma}
  $(\frac{-3}{p}) = \begin{cases}1 & p \equiv 1 \pmod 3 \\ -1 & p \equiv 2 \pmod 3\end{cases}$
\end{lemma}
\begin{prf}
  Consider again $(-3)^{(p-1)/2} = (\sqrt{-3})^{p-1}$.
  We can notice $\sqrt{-3} = \sqrt{3} i = \zeta_6 + \zeta_3$.

  This gives us $(\sqrt{-3})^{p-1} = \frac{\zeta_3^p - \zeta_3^{2p}}{\zeta_3 - \zeta_3^2}$
  because $\zeta_6 = -\zeta_3^2$.

  If $p \equiv 1 \pmod 3$, then $\frac{\zeta_3^p - \zeta_3^{2p}}{\zeta_3 - \zeta_3^2}
    = \frac{\zeta_3 - \zeta_3^{2}}{\zeta_3 - \zeta_3^2} = 1$
  and if $p \equiv 2 \pmod 3$, $\frac{\zeta_3^p - \zeta_3^{2p}}{\zeta_3 - \zeta_3^2}
    = \frac{\zeta_3^2 - \zeta_3^1}{\zeta_3 - \zeta_3^2} = -1$.
\end{prf}

Notice that to get to $\sqrt{3}$ on the complex plane, we need $\zeta_{12}$,
which explains why we see mod 12 in the rule.
To get $\sqrt{5}$, we can either use the fact that $\cos \frac{2\pi}{5} = \frac14(\sqrt{5}-1)$
or notice that $(\zeta_5 - \zeta_5^2 - \zeta_5^3 + \zeta_5^4)^2
  = (4 - \zeta_5 - \zeta_5^2 - \zeta_5^3 - \zeta_5^4)
  = 5 - (1 + \zeta_5^1 + \zeta_5^2 + \zeta_5^3 + \zeta_5^4) = 5$.
We can then execute the same fraction-by-cases technique, getting our result mod 5.

Aside: This is the Gauss sum for $\sqrt{5} = \sum (\frac{i}{5})\zeta_5^i$.

\section{Quadratic Reciprocity (09/30)}

Recall the pattern from last lecture,
where we noticed that asking if $q$ is a square mod $p$
seems to be like asking if $p$ is a square mod $4q$.
This is almost true, but in fact

\begin{theorem}[Quadratic Reciprocity]
  $(\frac{q}{p}) = (\frac{p}{q})$ for odd primes $p \neq \pm q$
  where at least one is congruent to 1 mod 4 and at least one is positive.
\end{theorem}

Equivalently, for all distinct positive odd primes $p$ and $q$,
$(\frac{p}{q})(\frac{q}{p}) = (-1)^{\frac{p-1}{2}\frac{q-1}{2}}$

The proof follows by Gauss sums and the vague ideas from the last lecture.

This means we can evaluate any Legendre symbol using a modulus as either one of
\begin{align*}
  \qty(\frac{-1}{p}) = (-1)^{\frac{p-1}{2}}
   & = \begin{cases}1 & p \equiv 1 \pmod 4 \\ -1 & p \equiv 3 \pmod 4\end{cases}         \\
  \qty(\frac{2}{p})  = (-1)^{\frac{p^2-1}{8}}
   & = \begin{cases}1 & p \equiv \pm 1 \pmod 8 \\ -1 & p \equiv \pm 3 \pmod 8\end{cases} \\
  \qty(\frac{q}{p})  = \qty(\frac{p}{q})(-1)^{\frac{p-1}{2}\frac{q-1}{2}}
   & = \begin{cases*}
         (\frac{p}{q})  & $p \equiv \pm 1 \pmod 4$ or $q \equiv \pm 1 \pmod 4$ \\
         -(\frac{p}{q}) & $p \equiv q \equiv 3 \pmod 4$
       \end{cases*}
\end{align*}
which is nicer than using Euler's criterion.

\begin{example}
  Is 71 a square mod 101?
\end{example}
\begin{sol}
  Write $\leg{71}{101} = \leg{101}{71} = \leg{30}{71} = \leg{2}{71}\leg{3}{71}\leg{5}{71}$
  by quadratic reciprocity and multiplicativity.

  Then, $\leg{2}{71} = 1$ since $71 \equiv 7 \pmod 8$.

  Also, $\leg{3}{71} = -\leg{71}{3} = -\leg{2}{3} = 1$ since $71 \equiv 3 \pmod 4$.

  Finally, $\leg{5}{71} = \leg{71}{5} = \leg{1}{5} = 1$ since 1 is always a square.

  This gives $\leg{71}{101} = 1\cdot 1 \cdot 1 = 1$ so 71 is a square mod 101.
\end{sol}

Asymptotically, this is not faster than Euler's criterion
because we require factoring. However, it is prettier.

To deal with a random large number, we must consider what to do
after factoring out all the 2s (since we can deal with those quickly).

\begin{defn}[Jacobi symbol]
  For all $m,n \in \N_{>0}$ with $n$ odd, $\leg{m}{n} = \prod_{i=1}^k \leg{m}{p_i}$
  where $\prod_{i=1}^k p_i = n$ is the prime factorization of $n$
\end{defn}

\begin{theorem}[Jacobi]
  For all positive and odd $m$ and $n$,
  \begin{align*}
    \qty(\frac{-1}{n}) = (-1)^{\frac{n-1}{2}}
     & = \begin{cases}1 & n \equiv 1 \pmod 4 \\ -1 & n \equiv 3 \pmod 4\end{cases}         \\
    \qty(\frac{2}{n})  = (-1)^{\frac{n^2-1}{8}}
     & = \begin{cases}1 & n \equiv \pm 1 \pmod 8 \\ -1 & n \equiv \pm 3 \pmod 8\end{cases} \\
    \qty(\frac{m}{n})  = \qty(\frac{n}{m})(-1)^{\frac{n-1}{2}\frac{m-1}{2}}
     & = \begin{cases}
           (\frac{n}{m})  & n \equiv \pm 1 \pmod 4 \text{ or } m \equiv \pm 1 \pmod 4 \\
           0              & \gcd(m,n) \neq 1                                          \\
           -(\frac{n}{m}) & n \equiv m \equiv 3 \pmod 4
         \end{cases}
  \end{align*}
\end{theorem}

Note: For Legendre symbols, $\leg{a}{p} = 1 \iff a \equiv \square \pmod p$.
However, for Jacobi symbols, we only have the one-way implication
$\leg{m}{n} = -1 \implies m \not\equiv \square \pmod n$.

Return now to the application to cryptography,
specifically to Goldwasser--Micali.

\subsubsection{Goldwasser--Micali cryptosystem}

\textbf{Key Generation: } Choose random primes $p$, $q$. Set $n = pq$.

Choose $x \in (\Z/n\Z)^\times$ such that
$\leg{x}{p} = \leg{x}{q} = -1$ (then $\leg{x}{n} = 1$).
Publish $x$.

\textbf{Encrypt:} $m \in \{0,1\}$

Choose some $r \xgets{\$} (\Z/n\Z)^\times$.
Then, $Enc(m) = x^m r^2 = c$.

\textbf{Decrypt:} Determine whether $c$ is a ``fake'' square using the factorization.

The underlying assumption is that it is not easy to distinguish
actual squares mod $n$ and ``fake'' squares mod $n$.

\chapter{Primality}

\section{Primality Testing (10/03)}

Given $n \in \Z$, how can we tell if $n$ is prime?

\begin{lemma}[Fermat test]
  Recall \FLT{}: for a prime $p$, $a \in (\Z/p\Z)^\times \implies a^{p-1} = 1$.
  Therefore, if $a \in (\Z/n\Z)^\times$ and $a^{n-1} \neq 1$,
  then $n$ is not prime.
\end{lemma}

\begin{defn}[Fermat witness]
  Let $n \in \N$, $\alpha \in (\Z/n\Z)^\times$ where $\alpha^{n-1} \neq 1$.
\end{defn}

When $n$ is prime, no Fermat witness can exist.
When $n$ is not prime, only some elements are Fermat witnesses.
The other elements are \emph{Fermat liars}.
How many liars are in $(\Z/n\Z)^\times$?

\begin{theorem}
  For $n > 2$, if there exists one Fermat witness in $(\Z/n\Z)^\times$,
  then there exist at least $\frac{\phi(n)}{2}$ Fermat witnesses.
\end{theorem}
\begin{prf}
  Consider the set $H=\{ \alpha \in \mgrp{n} : \alpha^{n-1} = 1\}$.

  $H$ is a subgroup: $1 \in H$, $ab \in H$, $a^{-1} \in H$ (trivial by exponentiation properties).

  So by Lagrange's theorem, $\abs{H} \mid \abs{\mgrp{n}}$.

  Either (1) $\abs{H} = \phi(n)$, so there are no witnesses,
  or (2) $\abs{H} < \phi(n)$, so $\abs{H} \leq \frac{\phi(n)}{2}$.
\end{prf}

\begin{defn}[Carmichael number]
  $n \in \N$, $n > 2$ such that $n$ is composite and $n$ has no Fermat witnesses.
\end{defn}
Examples: $n = 561 = 3 \times 11 \times 17$.
By \FLT{}, we have $\alpha^{n-1} = \alpha^{560}$ is 1 mod 3, 1 mod 11, and 1 mod 17.

Recall that for $n$ prime:
$a^{\frac{n-1}{2}} \equiv \leg{a}{n} \pmod n$ when $n > 2$, odd, and $a \in \mgrp{n}$.
This gives us the following test:

\begin{lemma}[Solovay--Strassen test]
  If $a^{\frac{n-1}{2}} \not\equiv \leg{a}{n} \pmod n$, then $n$ is not prime.
\end{lemma}

We can calculate $a^{\frac{n-1}{2}}$ by repeated squaring
and $\leg{a}{n}$ by Jacobi reciprocity and factoring out 2's.
We can now define witneses as in the Fermat test.

\begin{defn}[Euler (Solovay--Strassen) witness]
  An element $\alpha \in \mgrp{n}$ where
  $\leg{\alpha}{n} \not\equiv \alpha^{\frac{n-1}{2}} \pmod n$.
  If an element is not an Euler witness, it is an Euler liar.
\end{defn}

Notice that all Euler witnesses must also be Fermat witnesses,
meaning that hopefully we have a more refined test here.

\begin{theorem}
  If $n > 2$ is composite and odd, then there exists at least one Euler witness.
\end{theorem}
\begin{prf}
  Suppose $n$ is composite and $n = p \times k$.

  If $p \nmid k$, then solve $\alpha \equiv \beta \pmod p$ and $\alpha \equiv 1 \pmod k$
  where $\beta$ is a quadratic non-residue mod $p$.
  Now, calculate
  \[
    \leg{\alpha}{n} = \leg{\alpha}{p}\leg{\alpha}{k} = \leg{\beta}{p}\leg{1}{k} = (-1)(1) = -1
  \]
  Suppose $\alpha^{\frac{n-1}{2}}$ is $-1$.
  Then, $\alpha^{\frac{n-1}{2}} \equiv -1 \pmod n$
  and that means $\alpha^{\frac{n-1}{2}} \equiv -1 \pmod k$.
  But we know $\alpha \equiv 1 \pmod k$, so this is a contradiction.

  Otherwise, $p \mid k$.
  Let $\alpha = 1 + k$. Calculate
  \[
    \leg{\alpha}{n} = \leg{1+k}{n} = \leg{1+k}{p}\leg{1+k}{k} = \leg{1}{p}\leg{1}{k} = (1)(1) = 1
  \]
  Suppose $\alpha^{\frac{n-1}{2}} = 1$.
  This implies that $\ord(\alpha) \mid \frac{n-1}{2}$.
  Calculate $\alpha^p = (1+k)^p = 1^p + \underbrace{p k^1 + \dotsb + \binom{p}{p} k^p}_{0 \pmod n} = 1$
  which implies $\ord(\alpha) = p$.
  But $p \mid n \implies p \nmid n-1 \implies p \nmid \frac{n-1}{2}$.

  Therefore, $\alpha$ is an Euler witness.
\end{prf}

This theorem combined with the at-least-$\frac{\phi(n)}{2}$ theorem
means that we have for every odd, composite $n > 2$ there are $\frac{\phi(n)}{2}$ Euler witnesses.

\section{Strong Primality Testing (10/05)}

Recall: for the Fermat test, evaluate $a^{n-1}$ a bunch of times.
If it is equal to 1, prime or liar; otherwise, composite.
For the Solovay--Strassen test, evaluate $a^{\frac{n-1}{2}} = \leg{a}{n}$.
If yes, prime or Euler liar; otherwise, composite.
Also, there are an infinite number of Carmichael numbers that screw with this
but otherwise you have around a 50\% chance of getting a witness.

We can refine this further beyond considering $n-1$ and $\frac{n-1}{2}$.

Write $n-1 = 2^t \cdot s$ so that $s$ is odd.
Then, $a^{n-1}$ is $a^s$ squared $t$ times.
So instead of asking if $a^{2^t s} = 1$, consider if $a^{2^{t-1} s}$
is an ``expected'' square root of 1, i.e., $\pm 1$.
If it is not, it is composite.
If it is and it is $-1$, we have a prime or liar.
If it is and it is 1, keep going back.
If we reach $a^s=1$, we get no information.

\begin{lemma}[Miller--Rabin test]
  Let $x \gets a^s$. Do:
  \begin{itemize}[nosep]
    \item If $x = 1$, stop. Probably prime.
    \item If $x = -1$, stop. Probably prime.
    \item Otherwise, $x \gets x^2$
  \end{itemize}
  while $x \neq a^{2^t s}$. If we reach the end, it is composite.
\end{lemma}

\begin{defn}[Miller--Rabin (strong) liar]
  $a \in \mgrp{n}$ if either $a^s = 1$ or $a^{2^k s} = -1$ for $0 \leq k < t$.
\end{defn}

We call this a ``strong liar'' because every strong liar is an Euler liar,
and every Euler liar is a Fermat liar.

\begin{theorem}
  Suppose $n$ has at least two distinct prime factors.
  Then, the number of Miller--Rabin liars is at most $\frac{\phi(n)}{4}$
  and in general, if $n$ has $\ell$ distinct prime factors,
  there are at most $\frac{\phi(n)}{2^\ell}$ Miller--Rabin liars.
\end{theorem}

We can make these primality tests deterministic by iterating $a = 1,\dotsc,n$.
We do not need to go to $a=n$ and instead
we can establish an upper bound on the smallest witness.
The bound (by Bach) is $O(\log^2 n)$, specifically, $2\log^2n$.
But this requires the Generalized Riemann Hypothesis
which everyone believes anyways, so we just check $a = 1,\dotsc,2\log^2 n$.

To analyze complexity, notice that we have $\log n$ multiplications at each step,
i.e., $\log^{1+\epsilon} n$ bit operations using fast multiplication.
So the complexity is $O(\log^{2 + (1 + \epsilon) + 1} n)$.

Further reading:
\begin{itemize}[nosep]
  \item AKS (Agrawal--Kayal--Saxena; 2004) primality test in $O(\log^6 n)$
        which does not rely on GRH and was an undegrad project(!!)
  \item ECPP (elliptic curve prime proving) notable for not having liars,
        also does not require GRH and runs non-deterministically (Monte Carlo)
        in $O(\log^5 n)$
  \item Cyclotomic primality test in $O((\log n)^{\log \log n})$,
        best until AKS proved that primality is in P.
\end{itemize}

Since there are $\frac{n}{\log n} + O(\sqrt n)$ primes less than $n$,
we can pick random numbers of size $e^\ell$ to get an approximate $\frac{1}{\ell}$
probability of a prime.

\section{Malleability (10/07)}
Recall the Goldwasser--Micali cryptosystem.
It satisfies IND-CPA provided that the quadratic reciprocity problem is hard.
That is, determining whether an $x = pq$ with $\leg{x}{n} = 1$
is actually a square or not (i.e. $\leg{x}{p} = 1$).

However, an adversary can still alter the message without needing to decrypt.
This also applies, for example, to XOR one-time pads (since if $c = k \oplus m$
and we intercept $c \mapsto c \oplus n$, recepient will get $m' = m \oplus n$).
Using MACs can get around this problem (e.g. AES with GCM or Chacha20 with Poly1305).

\begin{defn}[non-malleability]
  Given the game NM-CPA:
  \begin{enumerate}[1.,nosep]
    \item $\mathcal C$ generates $(pk,sk)$
    \item $(m_0, m_1, m_0', m_1') \xgets{\$} \mathcal A(\lambda, pk)$ where $m_0' \neq m_1'$
    \item $\mathcal C$ chooses $b \xgets{\$} \{0,1\}$
    \item $\mathcal C$ computes $c = E(m_b)$
    \item $c' \gets \mathcal A(\lambda, pk, c)$
  \end{enumerate}
  with win condition $D(c') = m_b'$ with non-negligible probability above 50\%.
\end{defn}

Instead of CPA games, consider CCA2 (chosen-ciphertext attack 2) games.
Here, the adversary has a decryption oracle that takes anything except $c$.
In CCA1, the oracle can only be accessed prior to receiving $c$.

\begin{theorem}
  IND-CCA2 is equivalent to NM-CCA2
\end{theorem}

Note that IND-CPA is not equivalent to NM-CPA,
which is instead equal to IND-PCA (parallel ciphertext attack,
where all oracle queries must occur at once).

\section{Factorization Algorithms (10/17)}

Naive approach: trial division by $1,\dotsc,\sqrt{n}$ which is $O(\sqrt{n}) = O(\exp(\frac12\log n))$.
Note that we call this ``exponential'' because we measure with respect
to the size of the input, i.e., $\lg n \approx \log n$.

\begin{prop}
  If $x,y\in \mgrp{n}$ satisfy $x^2 \equiv y^2 \pmod n$ and $x \not\equiv \pm y \pmod n$,
  then $\gcd(n, x-y)$ is a non-trivial factor of $n$.
\end{prop}
\begin{prf}
  Since $x^2 - y^2 \equiv 0$, we have $(x-y)(x+y) \equiv 0$.
  But we know that $x-y \not\equiv 0$ and $x+y \not\equiv 0$
  so there must be some weird hidden factor.

  If $\gcd(n,x-y) = n$, then $n \mid (x-y) \implies x \equiv y \pmod n$
  and if $\gcd(n,x-y) = 1$, then $n \mid (x-y)(x+y)$ which implies $n \mid (x+y)$
  by Gauss' Lemma which gives the same contradiction.
  Therefore, since the GCD must divide $n$, it is non-trivial.
\end{prf}

Using this, we can find non-trivial factors of $n$ by finding $x$ and $y$
and then applying the EEA.
How to find $x$ and $y$?

\subsubsection{Random squares (Dixon)}

\begin{defn}[$B$-smoothness]
  $n \in \N$ where the largest prime factor is less than $B$
\end{defn}

Chose $x_i \xgets{\$} \mgrp{n}$.
For each $x_i$, compute $x_i^2 \pmod n$ and keep the $B$-smooth squares.
We can tell if a number is $B$-smooth by trial division (since $B$ is small).

We need at least $t+1$ squares that are $B$-smooth.\footnote{Where $t=\pi(B)$ is the prime-counting function.}

This gives us squares $x_1^2 \bmod n = p_1^{e_{1,1}}\dotsb p_t^{e_{t,1}}$
up to $x_{t+1}^2 \bmod n = p_1^{e_{1,t+1}}\dotsb p_t^{e_{t,t+1}}$.

Take the subset product $\prod x_i^2 \bmod n = p_1^{\sum e_{1,i}}\dotsb p_t^{\sum e_{t,i}}$.

We can define $b_i = \begin{cases}0 & i \in S \\ 1 & i \not\in S\end{cases}$
so that $\sum_{i\in S}e_{j,i} = \sum_{i=1}^{t+1}e_{j,i}b_i = 0 \bmod 2$
to find squares.

Solve this homogeneous linear system over $\Z/2\Z$ (where the $b_i$ are variables).
We know there exists a non-trivial solution because there are more variables (at least $t+1$) than equations (exactly $t$).

That gives a square subset product $x^2 = \prod\limits_{i=1}^{t+1} (x_i^2)^{b_i} \bmod n
  = \prod\limits_{j=1}^t p_j^{\sum_{i=1}^{t+1} e_{j,i} b_i} \bmod n = y^2$.

The LHS and RHS are unrelated except for the fact that they are equal mod $n$.
In fact, with about 50\% probability, $x \not\equiv \pm y \bmod n$.
The probability can be improved by increasing $t+1$ to like $t+10$.
Since $t \approx B$ is large, this is negligible.

Picking $B$: large $B$ makes it more likely to find $B$-smooth squares, however,
the amount of work $t+1$ is proportional to $B$.

We want to pick $B$ such that the probability of squares being $B$-smooth is $\frac1B$.
This depends on $n$.

From analytic number theory, the probability that a random $y \xgets{\$} \Z/n\Z$
is $L(\alpha,c)$-smooth is $L(1-\alpha,\frac{1-\alpha}{c})$.\footnote{Where $L_n(\alpha,c) = O(\exp(c(\log n)^\alpha(\log\log n)^{1-\alpha}))$.
  Notably, $L_n(1,1) = n$ and $L_n(1,c) = n^c$. Then, $\sqrt{n} = L_n(1,\frac12)$.
  Also, $L_n(0,c) = (\log n)^c$.
  That is, we interpolate between $L_n(0,c)$ polynomial time and $L_n(1,c)$ exponential time}
So we set a bound on $B$ of $L_n(\frac12,\frac{\sqrt2}{2})$.
Since $(\log n)^k \ll B \ll \sqrt{n}$, we call this subexponential.

\section{Better Sieves (10/19)}

What is the probability that a particular $x^2 \bmod n$ is $B$-smooth?
Vanishingly small for large $n$ (in the hundreds of digits)
and small-ish $B$ (around $10^9$).
However, we can prove that the runtime for random squares is $L_n(\frac12,2\sqrt2)$
using results from analytic number theory, i.e., probabilistic subexponential time.

How can we improve?
Pick $x$ such that $x^2 \bmod n$ is small (and more likely to be $B$-smooth).
Naively: small numbers stay small (but are useless).
Instead, pick $x \approx \sqrt{n}$ so that $x^2 \bmod n = x^2 - n$.

Then, if $x = \sqrt{n} + k$, $x^2 = (\sqrt n + k)^2 - n = 2k\sqrt n + k^2
  = O(k\sqrt{n})$, i.e., around half the size of $n$ and much smaller than $n$.

This is the quadratic sieve.
We can bound $B < L_n(\frac12, 1)$ and prove runtime $L_n(\frac12,\sqrt2)$.

Suppose we write $\sqrt{n} = a_0 + \frac{1}{a_1 + \frac{1}{a_2 + \frac{1}{\ddots}}} = [a_0,a_1,\dotsc]$
as a continued fraction.

Define $\frac{P_i}{Q_i} = [a_0,\dotsc,a_i]$.
These fractions rapidly approach $\sqrt{n}$ (and are in fact the best rational approximations).
That is, $P_i^2 - nQ_i^2$ rapidly approaches 0.
We can prove that $0 < P_i - nQ_i^2 < 2\sqrt n + 1$.
Then, we can take $P_i^2 \bmod n$ and sieve guaranteed that the squares are $O(2\sqrt n)$.

Comparing the continued fraction sieve and quadratic sieve,
$O(2\sqrt n)$ appears better than $O(k\sqrt n)$.
However, if the quadratic sieve considers consecutive numbers to square,
we can do a sieve of Eratosthenes-like search to find good $B$-smooth candidates.
This is faster.

One more improvement step: number field sieve.

Choose $d \approx 6 \in \Z$ and $m \approx n^{1/d}$.
Write $n$ in base $m$: $n = a_0 + a_1m + \dotsb + a_5m^5$
and consider the polynomials $f(x) = a_0 + a_1x + \dotsb + a_5 x^5$
and $g(x) = x-m$.

We know that $f(m) = n \equiv 0 \pmod n$ and $g(m) = 0$.
That is, $m$ is a root of both $f$ and $g$ mod $n$.
The coefficients are also all around $m = O(n^{1/d})$ in size.

Consider $\alpha$ a complex root of $f$
(but consider it as part of a number field $\alpha \in \Z[\alpha]$).
We pick $a_i,b_i \in \Z$ such that $a_i+b_i\alpha = \prod\beta_i$ is smooth in $\Z[\alpha]$
and $a_i+b_im = \prod q_i$ is smooth in $\Z$.

Pick a subset $S$ such that $\prod_{i\in S}(a_i + b_i\alpha) = \square$ in $\Z[\alpha]$
and $\prod_{i\in S}(a_i + b_i m) = \square \bmod n$ in $\Z$.
We can expand the first sum, then replace $\alpha$ with $m \bmod n$
to get congruent squares for a sieve.
In fact, $\alpha \mapsto m \bmod n$ is a ring homomorphism from $\Z[\alpha] \to \Z/n\Z$.

Since the numbers are smaller, we have complexity $L_n(\frac13, \sqrt[3]{\frac{64}{9}})$.

\section{(10/21)}

\section{Index Calculus (10/26)}

There is a connection between runtimes of
factoring algorithms and DLOG algorithms:
\begin{center}
  \begin{tabular}{c|c}
    Factoring $n$                                       & DLOG in $\Z_p^*$ or $\F_q^*$ where $q = p^k$                  \\ \hline
    Trial ($O(n)$)                                      & Trial ($O(p)$)                                                \\
    $O(\sqrt{n})$                                       & Pollard's rho ($O(\sqrt{p})$)                                 \\
    Random squares ($L_p(\frac12,\sqrt2)$)              & Index calculus ($L_p(\frac12,\sqrt2)$)                        \\
    NFS ($L_n(\frac13,\sqrt[3]{\frac{64}{9}})$)         & NFS for DLOG ($L_p(\frac13,\sqrt[3]{\frac{64}{9}})$)          \\
    Special NFS ($L_n(\frac13,\sqrt[3]{\frac{32}{9}})$) & Tower NFS ($L_q(\frac13,\sqrt[3]{\frac{32}{9}})$ if $k < 50$) \\
    ???                                                 & $L_q(\varepsilon, c)$ for $p < 10$
  \end{tabular}
\end{center}

\begin{theorem}[Shoup]
  For a generic group, classical probabilistic DLOG algorithms
  require $\Omega(\sqrt{p})$ group operations.
\end{theorem}

What we mean by generic here is that the group ``interface''
is exposed (multiplication, inversion, equality) but we don't
know anything about the elements/structure.

\subsubsection*{Index calculus}

Consider $\Z_p^* = \{1,2,\dotsc,p-1\}$, $g$, $h = g^\alpha$.
We want to find $\alpha$, the ``index''.
We construct ``random index calculus'' from the random squares algorithm.
Pick random $x_i$ and calculate:
\begin{align*}
  g^{x_1} \bmod p     & = p_1^{e_{1,1}}\dots p_t^{e_{t,1}}     \\
                      & \vdots                                 \\
  g^{x_{t+1}} \bmod p & = p_1^{e_{1,t+1}}\dots p_t^{e_{t,t+1}}
\end{align*}
where we keep $B$-smooth $g^{x_i} \bmod p \approx O(p)$
until we get more equations than primes $p_i$.

If we take log base $g$ on both sides: $x_1 \equiv \sum e_{i,1}\log_g p_i \pmod{p-1}$.\footnote{
  $\log_g p_i$ always exists. If $g$ is not a generator
  since \emph{some} generator $h$ exists and we have
  $\frac{\log_h p_i \bmod p-1}{\log_h g \bmod p-1}$.}
Since we know the $x_i$ and $e_{i,j}$, we can solve the system of linear equations
for the discrete logs $\log_g p_i$ (since there are at least $t+1$ equations and $t$ variables).

Now, take random $y$ find an $h^y = p_1^{f_1}\dots p_t^{f_t}$ that is $B$-smooth.
Taking logs as above, $y\alpha = \sum f_i \log_g p_i$ and we can solve for $\alpha$.

Since this is basically the same process as random squares,
it is no surprise it has similar time complexity $L_p(\frac12,\sqrt2)$.
Practically, it's slightly harder than factoring.

\chapter{Signatures}

\section{Hash Functions (10/28)}

To establish something that is NM-CCA2 secure, we need to somehow ``sign''
the ciphertext to distinguish ``authenticated'' ciphertexts.
We can do this with MACs (e.g., AES-GCM or ChaCha20-Poly1305)
but we will do something different.

Consider a hybrid encryption scheme:
use public-key encryption to send a symmetric key that encrypts the message.
This is CO 487 content.

\subsubsection*{Hash functions}

Most common hash functions are the SHA family:
SHA0 (broken 2005), SHA1 (broken 2017), SHA2 (actually used),
SHA3 (not really used, made in anticipation of SHA2 breaking).
Again, CO 487 content beyond the scope of this course.

\begin{defn}[hash function]
  Function $H : S \to T$ (typically, $S = \{0,1\}^*$ and $T = \{0,1\}^\lambda$)
\end{defn}

Ideally, a hash function is a random oracle, i.e.,
$H \xgets{\$} \{ f : (f : S \to T) \}$.
This is useful, e.g., for making hashed RSA signatures
existentially unforgeable under chosen message attack.

There is no way to easily construct a random oracle
because (1) we can't construct the set of all functions and
(2) we run into measure theory issues with defining a probability distribution on that set.
Instead we construct with desired properties:
\begin{enumerate}[1.]
  \item Preimage resistant: Given $t \in T$, it is infeasible to find $s \in H^{-1}(t)$.
  \item Second preimage resistant: Given $s \in S$, it is infeasible to find
        $s' \in S$ such that $s \neq s'$ and $H(s) = H(s')$.
  \item Collision resistant: It is infeasible to find $s \neq s'$
        such that $H(s) = H(s')$.
\end{enumerate}

\begin{example}
  Are all preimage resistant functions second preimage resistant?
\end{example}
\begin{sol}
  Consider $f(x) = x^2 \bmod n$.
  To find a preimage, take $x = \sqrt{y}$ (hard).
  To find a second preimage, take $x' = -x \neq x$ so $(-x)^2 = x^2$ (easy).
\end{sol}

To be formal, use games.
For example, with collision resistance:
Suppose we have a family of hash functions $HashGen : \one^\lambda \mapsto H_\lambda$.
Play the game:
\begin{enumerate}[1.]
  \item Pick a hash function $H_\lambda \xgets{\$} HashGen(\one^\lambda)$
  \item $(s, s') \gets \mathcal A(\one^\lambda, H_\lambda)$
\end{enumerate}
with win condition $H_\lambda(s) = H_\lambda(s')$ and $s \neq s'$.
We define $\{H_\lambda : \lambda \in \N\}$
to be collision-resistant if no probabilistic polynomial time adversary $\mathcal A$
can win this game with non-negligible probability in $\lambda$.

We can construct collision-resistant hash functions
from claw-free permutations by Damg\aa{}rd.

\begin{defn}[claw-free permutation]
  Given a set $X$, the pair of permutations $(f,g)$ is claw-free if
  it is infeasible to find $x_1,x_2 \in X$ such that $f(x_1) = g(x_2)$.
\end{defn}

The wrong way:
Given claw-free permutations $f : X \to X$ and $g : X \to X$,
we define $H : \{0,1\}^* \to X$ with $H(\varepsilon) = x_\varepsilon$.
Inductively, $H(b_1b_2\cdots b_n) = h(H(b_1\cdots b_{n-1}))$
where $h = f$ if $b_n = 0$ and $g$ if $b_n = 1$.
Claim this is collision-resistant because if there is a collision
$H(m) = H(m')$ and $m \neq m'$, we have a claw at some point,
which is a contradiction.
Unfortunately, we could run into a loop back to $x_\varepsilon$.

Instead, pick $x_0 \in X$ and define $x_\varepsilon = g(f(x_0))$
and define $H(b_0\cdots b_n) = h(h(H(b_0 \cdots b_{n-1})))$ as above.
Then, we cannot arrive at $x_\varepsilon$
because generating pairs of $f(f(\dots))$ and $h(h(\dots))$ cannot create $g(f(\dots))$.

\section{Signature Schemes (10/31)}

Consider some RSA modulus $n = pq$, $p > 2$, $q > 2$, $p \neq q$
where $p \equiv 3 \pmod 4$ and $q \equiv 3 \pmod 4$ (Blum integers,
notable for use in the Blum--Blum--Shub generator).

Let $y_p$ and $-y_p$ be square roots of $y$ mod $p$ (and for $q$).
Notice that $\leg{-1}{p} = -1$ and $\leg{-1}{q} = -1$.
Then, exactly one of $\{y_p,-y_p\}$ is a square mod $p$ (and for $q$).

Finally, combining gives exactly one of the square roots of $y$ mod $n$
is a square mod $n$.
This means that $f(x) = x^2$ is a permutation on $(\mgrp n)^2$.

Choose $a \in \mgrp n$ such that $\leg{a}{n}=-1$
Then, define $g(y) = a^2 y^2$ which is also a permutation.

Note: suppose $f(x) = g(y)$.
Then, $x^2 = (ay)^2$ and $x \neq \pm ay$
because if $x = \pm ay$ then $\leg{x}{n} = \leg{\pm 1}{n}\leg{a}{n}\leg{y}{n}$
but this is $1 = (1)(-1)(1)$, contradiction.
From this, we can factor $n$ (by Fermat).

Overall: claw-free permutations $\to$ collision-resistant hash functions
$\to$ \{secure digital signatures, CCA2-secure encryption, etc.\}

How do we generate secure digital signatures?

Suppose we have RSA $pk = (n,e)$ and $sk = (n,d)$.
Then, define signing and verification as
$Sign : \mgrp{n} \to \mgrp{n} m \mapsto \sigma := m^d \bmod n$
and $Verify : (m,\sigma) \mapsto \sigma^e \bmod n \stackrel{?}{=} m$.

\begin{defn}[signature schemes]
  A signature (scheme) is a tuple $(KeyGen, Sign, Verify)$ where
  \begin{itemize}[nosep]
    \item $KeyGen : \one^\lambda \mapsto (pk, sk)$
    \item $Sign : (sk, m) \mapsto \sigma$
    \item $Verify : (pk, m, \sigma) \mapsto \{0,1\}$
  \end{itemize}
  and we have that if $(pk, sk) \xgets{\$} KG(\one^\lambda)$
  and $\sigma \xgets{\$} S(sk,m)$, then $V(pk, m, \sigma) = 1$.
\end{defn}

Under Textbook RSA, it is trivial to forge junk (but valid) signatures,
i.e., given random signature $\sigma$, it signs some calculable message.

Example security definition game: EUF-CMA \\
Existential unforgeability (EUF): adversary produces a valid signature \\
Chosen-message attack (CMA): adversary can always use a signing oracle
\begin{enumerate}[1., nosep]
  \item $(pk, sk) \xgets{\$} KeyGen(\one^\lambda)$
  \item \textbf{for} $i=1\dots q$ \textbf{do:} \\
        \hspace*{1cm} $m_i \xgets{\$} \mathcal A(\one^\lambda, pk, (m_1,\sigma_1),\dotsc,(m_{i-1},\sigma_{i-1}))$ \\
        \hspace*{1cm} $\sigma_i \xgets{\$} Sign(sk, m_i)$ \\
        \textbf{end}
  \item $(m,\sigma) \xgets{\$} \mathcal A(\one^\lambda, pk, (m_1,\sigma_1),\dotsc,(m_q,\sigma_q))$
\end{enumerate}
with win condition $Verify(pk,m,\sigma)=1$ and for all $i$, $m \neq m_i$.

\begin{defn}[EUF-CMA]
  A signature scheme is EUF-CMA if there does not exist a probabilistic polynomial time
  adversary $\mathcal A$ which wins the EUF-CMA game with non-negligible probability.
\end{defn}

\paragraph{Hashed RSA} $KeyGen : \one^\lambda \mapsto ((n,e), (n,d))$ \\
$Sign : m \mapsto H(m)^d \bmod n$ for hash $H : \{0,1\}^* \to \mgrp n$ (i.e., a claw-free permutation) \\
$Verify : (m, \sigma) \mapsto H(m) \stackrel{?}{=} \sigma^e \bmod n$

We can prove that if the RSA assumption holds\footnote{Given $n$, $e$, $m^e$, it is infeasible to find $m$.}
and the hash function $H$ is a random oracle,
then Hashed RSA is EUF-CMA.

\section{Hashed RSA (11/02)}

Recall EUF-CMA and Hashed RSA. We want to prove

\begin{theorem}
  Hashed RSA is EUF-CMA assuming:
  \begin{itemize}[nosep]
    \item The RSA assumption holds
    \item The hash functions $H$ are random oracles
  \end{itemize}
\end{theorem}
\begin{prf}
  For a contradiction, let $\mathcal A$ be an adversary that wins the EUF-CMA game,
  generating a forged signature $(m_*, \sigma_*)$.
  Note that we must expose the hash function $H$ to the adversary.

  Consider when $H$ has the property that
  for some $\sigma \in \mgrp{n}$, $H(m_*) = m\sigma^e$.
  Then, $\sigma_* = H(m_*)^d = (m\sigma^e)^d = m^d\sigma$
  so $\sigma_*\sigma^{-1} = m^d$.
  We could return $H(m_*) = m\sigma^e$
  but we still have to respond to signing queries of $\mathcal A$ somehow.

  To respond to a query for $m_i$,
  pick a random $\sigma_i \xgets{\$} \mgrp{n}$
  and set $H(m_i) = \sigma_i^e$ and respond with $\sigma_i$.
  Note that the challenger must maintain a table of $H(m_i)$ to respond to duplicates.

  To make this work somehow, we define $H(m) = \begin{cases*}
      m\sigma^e & with probability $\frac{1}{q+1}$ \\
      \sigma^e  & with probability $\frac{q}{q+1}$
    \end{cases*}$.

  Then, notice that the adversary will make at most $q+1$ relevant hash function requests
  ($q$ for the signing queries, 1 for $m_*$).
  Now, the probability that we get what we want, i.e., calculate $m^d$, is
  $\qty(\frac{q}{q+1})^q\frac{1}{1+1}Adv(\mathcal A) > \frac{1}{(q+1)\exp(1)}Adv(\mathcal A)$
  which is non-negligible since $q$ is polynomial and $Adv(\mathcal A)$ is non-negligible.

  That is, we can break RSA in probabilistic polynomial time with
  non-negligible probability, violating the RSA assumption.
  Therefore, $\mathcal A$ cannot exist.
\end{prf}

Note: non-negligible means that there exists an $n$ such that
$\Pr[\mathcal A\text{ wins}] = f(\lambda) \in \Omega(\frac{1}{\lambda^n})$.

Further reading: EdDSA (Schnorr), ``Short signatures without random oracles'' (Boneh--Boyen)

\section{Zero-Knowledge Proofs (11/04)}

Suppose that $x \in \mgrp{n}$ where $n = pq$.

Claim: there exists a $y$ such that $x = y^2 \bmod n$.

If Alice knows that $x = y^2$ and sends $y$ to Bob,
that is a full-knowledge proof.
A zero-knowledge proof would not send $y$.

Instead, Alice chooses a random $r \xgets{\$} \mgrp{n}$
and computes $xr^2 = y^2r^2 = (yr)^2$.
If she sends $\beta = xr^2$ and $\alpha = yr$,
Bob can verify that $\alpha^2 = \beta$.
However, Bob cannot trust that $\alpha$ is in fact $yr$
and cannot prove that $\frac{\beta}{x} = r^2$ without sending $r$.

\paragraph{Protocol} For Alice to prove that she knows $y$ such that $y^2 = x$,
\begin{enumerate}[1.,nosep]
  \item Alice picks $r \xgets{\$} \mgrp{n}$ and sends $xr^2$
  \item Bob picks $b \xgets{\$} \{0,1\}$ and sends $b$
  \item Alice sends $\rho = y^b r$ and sends $\rho$
  \item Bob verifies that $\rho^2 = \beta x^{b-1}$
\end{enumerate}
Then, if $b=0$, Bob can catch a forged $y$ and if $b=1$, Bob is more certain that $y$ exists.
The chance that Alice is cheating and avoids being caught
in $\lambda$ iterations is $2^{-\lambda}$.

Suppose Alice does not know a square root $y = \sqrt{x}$. She could:
\begin{itemize}[nosep]
  \item Choose $r$ randomly, send $\beta = xr^2$, and hope that $b=0$ to send $r$
  \item Choose $\alpha$ randomly, send $\beta = \alpha^2$, and hope that $b=1$ to send $\alpha$
\end{itemize}
meaning that Alice can forge with success probability 50\%,
and indeed Bob could fool himself half the time by doing this himself.
That is, a zero-knowledge proof does not introduce any new information
that Bob could not have produced on his own.

Then, a security definition for a ZKP protocol requires
\begin{enumerate}[1.]
  \item Correctness: With an honest prover and honest verifier,
        the proof succeeds with probability 100\%.
  \item Soundness: With a dishonest prover and honest verifier,
        then there is a non-negligible probability that they get caught.
  \item Zero-knowledge: With an honest prover and dishonest verifier,
        then the verifier can simulate correct proofs with non-negligible probability.
        This means that the verifier cannot actually use any information
        for anything else (e.g., cannot factor a number even if the ZKP proves
        that the prover knows the factors).
\end{enumerate}
where non-negligible means any useful number (e.g., 50\%, 25\%, 30\%, etc.).

From a ZKP, we can construct a signature scheme. Generate a key $(x, y)$ where $y^2=x$.
Signing is done by:
\begin{enumerate}[1.,nosep]
  \item Alice picks random $r$ and sends $x$ and $\beta = xr^2$.
  \item Bob picks random $b$ and sends $b$
  \item Alice calculates $\rho = ry^b$ and sends $\rho$
\end{enumerate}
To verify, ensure that $\rho^2 = \beta x^{-1}$.

Alternatively, if we want to use DLOG (i.e., with $g$ and $g^x$, prove that you know $x$):
\begin{enumerate}[1.,nosep]
  \item Alice picks random $r$ and sends $g$, $g^x$, and $\beta = g^r$
  \item Bob picks random bit $b$ and sends $b$
  \item Alice calculates $\rho = r + bx$ and sends $\rho$
\end{enumerate}
To verify, ensure that $g^\rho = \beta(g^x)^b$.

We call these $\Sigma$ protocols because the back-and-forth looks like a $\Sigma$.

\begin{defn}[Fiat--Shamir transformation]
  To transform a ZKP protocol to a signing scheme,
  set $b = H(\beta, m)$ where $H$ is a random oracle.
  Then, the signature of $m$ is $(\beta,\rho)$.
  To verify, assert $\rho$ satisfies the ZKP protocol given $\beta$.
\end{defn}

Notice that there is only one bit of entropy, so it is forgeable 50\% of the time.
If we try increasing entropy in the DLOG scheme by making $b \in \Z$,
correctness and soundness still hold but zero-knowledge might not.

\section{ZKP Signatures (11/07)}

Recall the Fiat--Shamir transform:

Given a $\Sigma$ protocol,
Peggy sends Victor the problem $\pi$
and a commitment $c$ (usually some sort of randomized value).
Victor returns a challenge $b$.
Peggy's response $r$ depends on $b$.

From the perspective of a signature scheme,
we generate a private key (statement to be proved)
and public key $\pi$.

To sign, choose a commitment at random $c \xgets{\$} C$.
Set the challenge to be a deterministic but random function $b = H(m, c)$.
Calculate a response for $b$.
Then, let $\sigma = (c, r)$.

To verify, recalculate the challenge and verify with the response.

Notice that if $b \in \{0,1\}$, then signature forgery is permitted half the time.
To actually use this, the challenge space must be large.

Generically, the signer repeats the protocol $\lambda$ times
and initially commit to a commitment vector $\vb c = (c_1,\dotsc,c_\lambda)$.
They also make a challenge vector $\vb b = (b_1,\dotsc,b_\lambda) = H(m,\vb c)$.
Then, to successfully fake the proof (and forge a signature),
the signer would need to very luckily get a $\vb b$
that matches perfectly with a malicious $\vb c$.
Finally, generate a response vector $\vb r$ and return $\sigma = (\vb c, \vb r)$.

This cannot be proved to be ZK because the proof cannot be simulated.
We assume that the heuristic (that ZKP's produce ZKP's) holds.

\begin{example}
  Why do we need to generate the vectors at once?

  \WLOG, suppose we want to prove knowledge of $x$ (in the DLOG problem).

  The public key is $\pi = g^x$.
  The commitment is $c = g^y$ for random $y$.
  The challenge is a bit $b$.
  The response is $r = y + bx$.

  If Peggy predicts $b = 0$,
  pick $y$ randomly, set $c = g^y$,
  and Victor verifies $r = y$.
  Otherwise, if she predicts $b = 1$,
  pick $r_0$ randomly, set $c = \frac{g^{r_0}}{g^x}$,
  and Victor verifies $r = r_0$.

  If Peggy continuously generates random $y$,
  she only has to try twice until getting desired $b=0$.
  Then, she only needs to do $2\lambda$ work instead of $2^\lambda$ work.
\end{example}

What we're describing is the Schnorr signature.

\paragraph{Schnorr scheme}
Given a group $G$ and $g \in G$,
generate keys $(pk, sk) = (g^x, x)$.

A signature of $m$ is a proof of knowledge of $x$.
First, generate a commitment $r \xgets{\$} \Z$, $c = g^r$.
Then, calculate a challenge $b \gets H(m,c) = H(m,g^r)$.
The response is $r+bx$, so return $(c, \sigma) = (g^r, r+bx)$.

To verify a signature $(c, \sigma)$,
check if $g^\sigma \stackrel{?}{=} c(g^x)^{H(m, g^r)}$.
That is, compute $b' \gets H(m, c) = H(m, g^r)$
and check if $g^{r+bx} = g^r (g^x)^b \stackrel{?}{=} c(pk)^{b'}$.

Alternatively, send $(b, \sigma)$.
Then, calculate $c' \gets \frac{g^\sigma}{(g^x)^b}$
and check if $H(m, c') \stackrel{?}{=} b$.
This is better since $b$ is an integer,
which is easier to serialize and send than a group element $c$.

\begin{theorem}[Schnorr]
  Assuming that $H$ is a random oracle and that DLOG is hard in $G$,
  the Schnorr signature scheme is EUF-CMA.
\end{theorem}
\begin{prf}
  Suppose we are an adversary $\mathcal A_1$ trying to solve DLOG.
  Let $g$, $g^\alpha$ be a challenge from $\mathcal C_1$.
  Suppose also that we are a challenger $\mathcal C_2$
  with access to an adversary $\mathcal A_2$ that breaks Schnorr.

  Give $\mathcal A_2$ the parameter $g^\alpha$.
  Then, the adversary forges a signature $(b, r + b\alpha)$.
  We want to isolate $\alpha$, so we need two signatures
  with the same public key, same commitment, but with different hashes $b$ and $b'$.
  Using the \emph{forking lemma}, we stop execution before the hashing
  and swap out the hash function $H$.

  Then, we have $(b, r + b\alpha)$ and $(b', r + b'\alpha)$.
  We can now solve for $\alpha$ and return it to $\mathcal C_1$.

  Since $\mathcal A_2$ runs in poly.\ time,
  we ($\mathcal A_1$) ran in poly.\ time,
  meaning that DLOG is easy.
\end{prf}

\section{CCA2-Secure Signature Schemes (11/09)}

Recall: in IND-CCA2, $\mathcal A$ can use a decryption oracle,
then produce two messages.
$\mathcal C$ picks a random one of the two and encrypts it.
Then, $\mathcal A$ gets access to the decryption oracle
and wins if they can distinguish which message was encrypted.

In Textbook RSA, $E(m) = m^e \bmod n$,
so an attacker can pick garbage $k$ and ask for the decryption of $m^e k^e$.
The core issue here is that $E$ is a group homomorphism, i.e.,
$E(m_1m_2) = E(m_1)E(m_2)$.

\begin{remark}
  Any homomorphic cryptosystem is not CCA2-secure.
\end{remark}

For example, Rabin encryption $E(m) = m^2 \bmod n$ is homomorphic and
Elgamal $E(m) = (g^y, g^{xy}m)$ is also homomorphic in each entry.

\paragraph{Symmetric + asymmetric hybrid}
Let $KeyGen$, $Enc : M \to C$, and $Dec : C \to M$ be a public key cryptosystem.
Also let $\mathcal{Enc}$ and $\mathcal{Dec}$ be a symmetric key cryptosystem.

Suppose Alice wants to send to Bob.
Bob generates $(pk, sk) \xgets{\$} KeyGen$ and publishes $pk$.

Alice picks random $\sigma \xgets{\$} M$,
encrypts both $c = Enc(pk, \sigma)$ and $d = \mathcal{Enc}(\sigma, m)$,
and sends $(c, d)$.
Notice that we can reinterpret $\sigma$ as a key for the SKC
by just treating it as an appropriately-sized bistring.

Bob can now decrypt $(c, d)$ by first decrypting $\sigma = Dec(sk, c)$
and then $m = \mathcal{Dec}(\sigma, d)$.

\paragraph{Fujisaki--Okamoto} (1999) is a CCA2-secure one-time pad (OTP) hybrid.
Let $KeyGen$, $Enc$, $Dec$ be a PKC.
Then, make a pseudo-OTP $\mathcal{Enc}(k,m) = m \oplus H_1(k)$
and add a MAC $H_2(k,m)$.

Generate $(pk, sk) \xgets{\$} KeyGen(\one^\ell)$
and pick $\sigma \xgets{\$} M$.

Then, $E(m) = (Enc(pk, \sigma), m \oplus H_1(\sigma), H_2(\sigma,m))$.

To invert, $D(c, d, e) = d \oplus H_1(Dec(sk, c)) = m$
and check $H_2(\sigma, m) = e$.
If the MAC does not check out, either explicitly error
or implicitly output random garbage $H_3(s,(c,d,e))$ with a secret seed $s$.

Then, the CCA2 oracle is sabotaged.

\section{Proving Fujisaki--Okamoto Security (11/11)}

Recall the Fujisaki--Okamoto inputs:
$KGen : \one^\ell \mapsto (pk, sk)$,
$Enc : M \to C$,
$Dec : C \to M$,
$H_1 : M \to \{0,1\}^n$, and
$H_2 : \{0,1\}^n \times M \to T$.

Then, $\mathcal{Enc}(pk, m) = (Enc(pk, r), m \oplus H_1(\sigma), H_2(m, \sigma))$
where $m \in \{0,1\}^n$ and $\sigma \xgets{\$} M$.

\begin{theorem}
  If the original PKC is OW-CPA and $H_1$, $H_2$ are reandom oracles,
  then this basic Fujisaki--Okamoto is IND-CPA.
\end{theorem}
\begin{prf}
  Let $\mathcal{A}$ be an adversary that can win IND-CPA for FO.
  Recall IND-CPA: let $m_0,m_1 \gets \mathcal{A}(\one^\ell, pk)$
  and $b' \gets \mathcal{A}(\one^\ell, pk, \mathcal{Enc}(pk, m_b))$.
  Then, $\mathcal{A}$ can find $b = b'$ with non-negligible probability.

  Notice that the second term $m \oplus H_1(\sigma)$ is garbage
  since $\sigma$ is random so $m$ is randomly scrambled.
  Therefore, it is information-theoretically indistinguishable from random garbage.
  So the only way to get any information about $m$ is to find $\sigma$.

  Therefore, $\Pr[\text{$\mathcal{A}$ wins}] \leq \Pr[\text{$\mathcal{A}$ finds $\sigma$}]$.

  Suppose we are challenged to break the PKC in the OW-CPA game
  and are given $(pk,\sigma)$.

  Then, we can challenge $\mathcal{A}$
  with $(Enc(pk, \sigma), \tau, \mu)$ with random garbage $\tau$, $\mu$.
  Then, at some point $\mathcal{A}$ must call $H_1(\sigma)$.
  We intercept all the calls to $H_1$ (since we control $H_1$)
  and respond with $\sigma$ with non-negligible probability.
  Therefore, if FO is not IND-CPA, then the PKC is not OW-CPA.
\end{prf}

This reduction is not \emph{tight} because we randomly pick potential $\sigma$ candidates.
If the original PKC is deterministic,
then we can re-encrypt all potential $\sigma$ to find the right one.

\begin{theorem}
  If $Enc$ is deterministic, the PKS is OW-CPA, and $H_1$, $H_2$ are random oracles,
  then FO is IND-CCA2.
\end{theorem}
\begin{prf}
  First, notice that the IND-CCA2 game
  without the decryption oracle is the IND-CPA game.

  However, in FO, we claim the decryption oracle is ``useless''
  because there is no information-theoretic use of it.
  Therefore, since FO is IND-CPA, it is also IND-CCA2.

  To prove the claim, consider $\mathcal{Enc}(pk,m) = (Enc(pk,\sigma), m\oplus H_1(\sigma), H_2(m,\sigma))$.

  Then, $\mathcal{Dec}(sk, (c_1,c_2,c_3)) = \begin{cases}
      \underbrace{H_1(\overbrace{Dec(sk, c_1)}^{\sigma'}) \oplus c_2}_{m'} & \text{otherwise}          \\
      \bot                                                                 & c_3 \neq H_2(m', \sigma')
    \end{cases}$

  Since encryption is deterministic,
  the only way to construct a valid ciphertext that gets a return value
  is to know both $m$ and $\sigma$ to calculate $Enc(pk,m)$ and $H_2(m,\sigma)$.

  We can simulate this for the adversary by intercepting calls to $H_2$
  and checking if $\sigma$ matches the encryption of $m$ (i.e. $c_1$).
  Therefore, there is no difference between IND-CPA and IND-CCA2.
\end{prf}

\paragraph{Full Fujisaki--Okamoto}
Instead of using $Enc(pk, \sigma)$,
randomize to $Enc(pk, \sigma; r)$.
For example, in Elgamal, $Enc(g^x, \sigma; r) = (g^r, g^{xr}\sigma)$.

Then, $\mathcal{Enc}(pk,m) = (Enc(pk,\sigma;H_2(m,\sigma)), m \oplus H_1(\sigma))$
for $\sigma \xgets{\$} M$.
That is, we use the tag as the randomness.

Finally, $\mathcal{Dec}(sk, (c_1,c_2)) = \begin{cases}
    \underbrace{H_1(\overbrace{Dec(sk, c_1)}^{\sigma'}) \oplus c_2}_{m'} & \text{otherwise}                      \\
    \bot                                                                 & c_1 \neq Enc(pk, \sigma'; H_2(m',s'))
  \end{cases}$

\begin{theorem}
  If the PKC is OW-CPA and $H_1$, $H_2$ are random oracles,
  then full FO is IND-CCA2.
\end{theorem}

What if we don't have random oracles?
Cramer--Shoup (1998) gets IND-CCA2 using DDH and a collision-resistant hash function.
It is also stupid complicated.

Given a group $\abs{G} = q$ with two generators $g_1$, $g_2$ where $\expval{g_1} = \expval{g_2} = G$.

The $sk = (x_1,x_2,y_1,y_2,z) \in (\Z/q\Z)^5$
and $pk = (c,d,h) = (g_1^{x_1}g_2^{x_2}, g_1^{y_1}g_2^{y_2}, g_1^z)$.

Encryption is $Enc(pk, m) = (g_1^r, g_2^r, h^r m, c^r d^{rH(g_1^m, g_2^m, h^r m)})$
for $m \in G$ and $r \xgets{\$} \Z/q\Z$.

Then, the last part $c^r d^{r\alpha}$ acts as a checksum.
To generate a valid ciphertext and use the CCA2 oracle,
an adversary must generate this, which breaks DDH.

\chapter{Elliptic Curve Cryptography}

\section{Elliptic Curves (11/14)}

Recall the conic sections:
$y^2 = 1 - x^2$ (circles), $y^2 = x^2 - 1$ (hyperbola), etc.
If we replace the quadratic in $x$ with a cubic, we get an elliptic curve.
For our purposes,

\begin{defn}[curve]
  The set of points satisfying $f(x,y) = 0$ where $f \in K[x,y]$ for a field $K$.
\end{defn}

where $K$ is a (usually finite) field, e.g., $\Z/p\Z$ or something funny like $\Z/3\Z[i] = \F_9$.

Note that we can rewrite any cubic $ax^3 + bx^2 + cx + d$
by first dividing through by $a$ to get $x^3 + b'x^2 + c'x + d'$.
Then, send $x \mapsto x - \frac{b'}{3}$ to get $x^3 + c''x + d''$.
This only works if $3 \neq 0$ so we can divide by 3,
i.e., the characteristic of $K$ is not 3.

To simplify the quadratic in $y$, we can complete the square as long as
$2 \neq 0$, i.e., the characteristic of $K$ is not 3.

\begin{defn}[elliptic curve]
  A solution set to an equation of the form $y^2 = x^3 + ax + b$
  where $a,b \in K$ and $\operatorname{char}(K) \neq 2,3$.
\end{defn}

Consider an ellipse centered at the origin with semimajor axes $a$ and $b$.
Then, the arc length is $\int_0^{\pi/2}\sqrt{a^2\cos^2t + b^2\sin^2t} \dd{t}$.
Make the substitution $u = \sin t$, $\dd{u} = \cos t \dd{t}$ to get
$\int \sqrt{a^2 - \frac{(a^2-b^2)u^2}{1-u^2}} \dd{u}$.
Then, $k^2 = 1-\frac{b^2}{a^2}$ gives $\int a\sqrt{\frac{1-k^2u^2}{1-u^2}}\dd{u}$
and finally $x = 1-k^2u^2$ for $\frac12\int_{1-k^2}^1\frac{x\dd{x}}{\sqrt{x(x-1)(x-(1-k^2))}}$.
This is our \emph{elliptic integral}.

Generally, elliptic integrals of the first kind $\int \frac{\dd{x}}{\sqrt{x^3+\dotsb}}$
and of the second kind $\int \frac{x\dd{x}}{\sqrt{x^3+\dotsb}}$.

Just as $\int \frac{\dd{x}}{\sqrt{x^2 + \dotsb}}$ gives $\sin^{-1}(x)$,
the inverse of a periodic function, complex analysis shows that
elliptic integrals of the first kind gives the inverse of the doubly periodic
Weierstrass function $\wp^{-1}(x)$.

By analogy to circles which can be defined by $f^2 + f'^2 = 1$,
elliptic integrals of the first kind satisfy $\wp'^2 = 4\wp^3 + c_1\wp + c_2$.

Elliptic curves have a somewhat natural group law.

\begin{lemma}
  Every line intersects an elliptic curve in exactly three places
  (up to multiplicity).
\end{lemma}
\begin{prf}
  Let $y^2 = x^3 + ax + b$ be an elliptic curve.

  Consider a line $L$ through $P = (x_P, y_P)$ and $Q = (x_Q, y_Q)$.
  Then, the slope of $L$ is $\frac{y_Q-y_P}{x_Q-x_P} = m$.
  Substitute $y = mx + c$ into the elliptic curve to get a cubic in $x$.
  The cubic has three roots.

  Then, it is $(x-x_P)(x-x_Q)(x-x_R)$.
  Taking the coefficient on $x^2$, we have $x_R = m^2 - x_P - x_Q$
  and $y_R = m((m^2-x_P-x_Q)-x_P)+y_P$.
\end{prf}

Define the group law as $P + Q = (x_R, -y_R)$.

What is $P+P$? Take the tangent line to $P$, i.e., ``$\lim_{Q \to P}(P+Q)$''.

What if there is no tangent line (self-intersection or cusps)?
To ensure this cannot happen, assert that the discriminant $4a^3 + 27b^2 \neq 0$.

What is the identity? Not really a point, denote $\infty$ or $[0:1:0]$ in Sage.

\section{(11/16)}

\section{Attacks on ECDH (11/18)}

Recall: we have an elliptic curve $y^2 = x^3 + ax + b$ over a field $K$
and a group law $x_{P+Q} = m^2 - x_P - x_Q$ and $y_{P+Q} = y_P - m(x_{P+Q}-x_P)$
where $m = \frac{y_Q - y_P}{x_Q - x_P}$ if $x_P \neq x_Q$
and $m = \frac{3x_P^2+a}{2y_P}$ if $P = Q$.

We denote this as $E(K) = \{\overbrace{(x,y) \in K^2 : y^2 = x^3 + ax + b}^{\text{affine (finite) points}}\} \cup \{\infty\}$,
so $E(K) \subset K^2 \cup \{\infty\}$.

We can show by Hasse--Weil that for any elliptic curve over $\F_q$
with prime power $q$, $\abs{E(\F_q)} = q + 1 - t$ for a trace of Frobenius $\abs{t} \leq 2\sqrt{q}$.

Consider the size of $E(\F_p)$.
We want to find square roots of $x^3+ax+b$ for all $x$ to find $y$.
There are two square roots $(x, \pm y)$, one square root $(x, 0)$, or potentially none.
We can show that \[ \abs{\{P \in E : x_P = x_0\}} = \leg{x_0^3+ax_0+b}{p}+1 \]
This gives us \[
  \abs{E(\F_p)} = 1 + \sum_{x \in \F_p}\qty(1 + \leg{x^3+ax+b}{p})
  = p + 1 + \sum_{x \in \F_p}\leg{x^3+ax+b}{p}
\]
and we can say that $t = -\sum\leg{x^3+ax+b}{p}$.\footnote{Consider that $\sum\leg{x}{p} \leq \sqrt{p}\ln p$}

Then, $\abs{E(\F_p)} \approx O(p)$,
so a generic DLOG algorithm over $E$ should take around $O(\sqrt{p})$ steps.
For some elliptic curves, this is the best we can do.

This is very attractive.
To get time approximately $2^{128}$, $p$ only needs to be $2^{256}$
whereas the NFS would require $n \geq 2^{3072}$.
This means faster computation with smaller keys.

There are some issues.

\paragraph{CRT attack}
Suppose that $\abs{E(\F_p)} = p+1-t = p_1p_2$ for small primes
(or in general, that it is $q$-smooth).
Then, Pohlig--Hellman allows us to find DLOG by the CRT
in about $O(\sqrt{p_1}+\sqrt{p_2})$ time.

This is because $E \simeq \Z/p_1p_2\Z \simeq \Z/p_1\Z \times \Z/p_2\Z$.
We can compute these isomorphisms.
Let $P \in E$ with order $p_1p_2$ and $Q = \alpha P$.
Then, $p_1Q$ has order $p_2$ and $p_2Q$ has order $p_1$.

That is, $p_1\alpha P = p_1 Q$ is in $\Z/p_2\Z$ and we can solve DLOG here
to obtain $\alpha \bmod p_2$.
Likewise with $p_2$ to find $\alpha \bmod p_1$.
Then, by CRT, we can find $\alpha$.

\paragraph{Invalid curve attack}
Let an otherwise secure curve $y^2=x^3+ax+b$ over $\abs{E(\F_p)} = q$.
Notice that the equations to calculate $P + Q$ do not use $B$.

Suppose Alice generates $A = \alpha P$ and Bob $B = \beta P$
so that they calculate $\alpha B$ and $\beta A$, respectively.
If Bob instead sends $B \in E' : y^2 = x^3 + ax + b'$
where $\abs{E'(\F_p)} = 3 \cdots$ (or smooth or otherwise insecure).
Then, Alice instead computes $\alpha B \in E'$ and Bob can find $\alpha \bmod 3$.
Repeating, Bob can recover Alice's key by CRT.

To avoid this, just check that $B \in E$.
Alternatively, express $P+Q$ using $b$.
When we are doubling $P + P$, $m^2 = \frac{(3x_P^2 + a)^2}{4y_P^2}
  = \frac{(3x_P^2 +a)^2}{4(x_P^3 + ax_P + b)}$
so then $x_{2P} = \frac{(3x_P^2 +a)^2}{4(x_P^3 + ax_P + b)} - 2x_P$.

Since this relies only on $x_P$, we can only send the $x$-coordinate in ECDH.
That is, Alice sends $x_{\alpha P}$ (i.e., $\pm \alpha P$)
and Bob sends $x_{\beta P}$ (i.e., $\pm \beta P$).
They both calculate $\pm \alpha \beta P = \pm \beta \alpha P$,
so $x_{\alpha\beta P}$ is the shared secret.

\section{Pairing-Based Cryptography (11/21)}

Recall: starting at an elliptic curve $E$,
we get a group $E(\F_p)$ and from there get ECDLOG and ECDH.
Applying the idea of Schnorr signatures gives us EdDSA.\footnote{
  There is a slight difference, where instead of hashing $H(m,g^r)$, we hash $H(m,g^r,g^\alpha)$}
CCA2 security can be achieved with Cramer--Shoup.

After basic ECC developed, pairing-based
and post-quantum isogeny\footnote{
  SIDH and SIKE broken but CSIDH and SQIsign still unbroken.
}-based cryptosystems developed.

\begin{defn}[cryptographic pairing]
  Bilinear\tablefootnote{
    i.e., $e(g^\alpha,g^\beta) = e(g,g)^{\alpha\beta} = \text{\emoji{egg}}^{\alpha\beta}$
    (or in additive notation, $e(\alpha P, \beta P) = e(P,P)^{\alpha\beta}$)
  } non-degenerate\tablefootnote{
    i.e., for all $g$, $(\forall h, e(g,h) = 1_T) \implies g = 1_G$
    and for all $h$, $(\forall g, e(g,h) = 1_T) \implies h = 1_G$
  } map $e : G \times G \to G_T$
  where (usually) $\abs{G} = \abs{G_T} = p$.
\end{defn}\spewnotes

\paragraph{MOV attack} (Menezes--Okamoto--Vanstone)
Suppose $E$ is an elliptic curve admitting a cryptographic pairing.
Consider a ECDLOG problem $P, \alpha P \to \alpha$.

Let $e(P,P) = g$ so that $e(P,\alpha P) = g^\alpha$ by bilinearity.
Then, we can consider the DLOG for $g$ and $g^\alpha$ in the new group $G_T$
which is some finite field $\F_q^*$.
But the whole point of ECDLOG is that it is harder than DLOG on a similarly sized finite field.
Transferring from $E$ to $\F_q^*$ made it easy again.

Over time, people found enough use in pairings to make it worth using
large enough curves admitting pairings.

\paragraph{Joux} (2000) 3-party Diffie-Hellman setup.
Suppose Alice, Bob, and Carol have keys $g^a$, $g^b$, and $g^c$.
Alice and Bob can generate a shared secret $g^{ab}$ by normal DH
but can't easily add Carol.

But with pairings, each one calculates $e(g,g)^{abc}
  = \underbrace{e(g^a,g^b)^c}_{\text{Carol}}
  = \underbrace{e(g^b,g^c)^a}_{\text{Alice}}
  = \underbrace{e(g^a,g^c)^b}_{\text{Bob}}$.

For this to work, we must assume the bilinear Diffie--Hellman assumption:
given $g^a$, $g^b$, $g^c$, it is infeasible to compute $e(g,g)^{abc}$.

Likewise, define bilinear DDH as given $g^a,g^b,g^c \in G$ and $h \in G_T$,
it is infeasible to compute $h \stackrel{?}{=} e(g,g)^{abc}$.

Note that normal DDH does not hold in a pairing, i.e.,
given $g^a,g^b,g^z \in G$, is $z = ab$?
Simply take $e(g,g^z) = e(g,g)^z \stackrel{?}{=} e(g,g)^{ab} = e(g^a,g^b)$.

\begin{prop}
  CDH $\geq_P$ BDH
\end{prop}
\begin{prf}
  Suppose CDH is broken, i.e., we can find $g^{ab}$ from $g^a$ and $g^b$.
  Then, we can take $e(g^{ab},g^c) = e(g,g)^{abc}$.
\end{prf}

\begin{prop}
  CDH\textsubscript{T} $\geq_P$ BDH
\end{prop}
\begin{prf}
  Suppose CDH\textsubscript{T} is broken.
  Then, we can find $e(g^a,g^b) = e(g,g)^{ab}$ and $e(g,g^c) = e(g,g)^c$
  normally but use CDH\textsubscript{T} to get $e(g,g)^{abc}$.
\end{prf}

\section{Divisors (11/23)}

\begin{defn}[divisor]
  Formal sum $\sum\limits_{P \in E} a_P(P)$
  of points $P \in E$ with integer coefficients $a_P \in \Z$
  where only finitely many $a_P$ are non-zero.
\end{defn}
\begin{example}
  Given points $P$ and $Q$ in $E$,
  $(P)$, $-(P)$, $2(P)$, and $3(P)-(Q)$ are divisors.
\end{example}

The set of all divisors $\operatorname{Div}(E)$ is a free $\Z$-module with basis $E(K)$.\footnote{
  Recall: a vector space $V$ over a field $K$ with basis $B$
  is $V = \{\sum_{b \in B} k_b b : k_b \in K, \text{finite $k_b$ are non-zero}\}$.
  By analogy, a $\Z$-module is a ``vector space'' over $\Z$
  (because $\Z$ is not a field).
}

\begin{example}
  Let $E : y^2 = x^3 - x$ over some finite field $\F_p$.
  Then, there are roots $P = (-1,0)$, $Q = (0,0)$, and $R = (1,0)$.
  We can make divisors $(P) + (Q) + (R)$ or $(P) - (R)$.
  Notice that $(P) - (R) \neq (P) + (-(R))$.
  Likewise, $(P + Q) \neq (P) + (Q)$.
\end{example}

If we define the \term{empty divisor} $\varnothing = \sum\limits_{P \in E} 0(P)$,
notice that we get a group.

We can treat divisors as prime factorizations in disguise.
Consider that if $n = p_1^{\alpha_1}\cdots p_k^{\alpha_k}$,
then $\log n = \alpha_1 \log p_1 + \dotsb + \alpha_k \log p_k$.

Then, let $(n) = \log n$ and we get $(n) = \alpha_1 (p_1) + \dotsb + \alpha_k (p_k)$.
We have $(1) = \log 1 = 0$, the ``empty divisor''.

\begin{defn}[degree]
  If $D = \sum a_P(P)$, then $\deg(D) = \sum a_P$.
\end{defn}
\begin{example}
  $\deg((P) - (\infty)) = 0$ and $\deg(2(P) - 4(Q)) = -2$.
\end{example}

\newcommand{\Div}{\operatorname{Div}}
\begin{prop}
  $\deg : \Div(E) \to \Z$ is a homomorphism, i.e.,
  $\deg(D_1 + D_2) = \deg(D_1) + \deg(D_2)$.
\end{prop}

Since this is a homomorphism, we get a kernel
$\ker \deg = \{ D \in \Div(E) : \deg(D) = 0 \}$
which is the degree zero divisors $\Div^0(E)$.

\begin{defn}[rational function]
  A quotient of polynomials.
\end{defn}

Note: On $E : y^2 = x^3 + ax + b$, a rational function
$\frac{f_1(x,y)}{f_2(x,y)}$ is written in two variables.

\begin{example}
  Consider the polynomial $2x + 3y + 5y^2 + 7x^2y + y^3$
  over the elliptic curve $y^2 = x^3 + ax + b$.

  Substituting gives $2x + 5(x^3 + ax + b) + (3 + 7x^2 + x^3 + ax + b)y$,
  i.e., $f_0(x) + f_1(x)y$ for some polynomials in only $x$.
\end{example}

That is, we can always split into a ``real'' part $f_0(x)$ and ``imaginary'' part $f_1(x)$
and only consider them the way we would consider only $a + bi$ in $\Z[i]$.

Therefore, the set of polynomials $K[E] = \{f_0 + f_1 y : f_0,f_1 \in K[x]\}$.
Then, the set of rational functions $K(E) = \{\frac{f}{g} : f,g \in K[E]\}$
where $\frac{f}{g} = \frac{f_0+f_1y}{g_0+g_1y}\cdot\frac{g_0-g_1y}{g_0-g_1y}
  = \frac{f_0g_0 + 2f_1g_1 + f_1g_1 y^2}{g_0^2-g_1^2y^2} = \frac{F_0}{G} + \frac{F_1}{G}y$
because we can again substitute out the $y^2$ terms.

(since these are commutative rings, PMATH 446 says we can get this just by localizing)

\renewcommand{\div}{\operatorname{div}}
\begin{defn}[divisor of a polynomial]
  Consider the set of polynomials $K[x]$ over an algebraically closed field $K$.
  Then, $\div(f) = \sum_{i=1}^n e_i(r_i)$ where $r_i$ are the roots of $f$,
  $n$ is the number of roots of $f$, and $e_i$ is the multiplicity of the corresponding root.
\end{defn}
\begin{example}
  $\div(x^3 + 2x^2) = \div(x^2(x+2)) = 2(0) + 1(-2)$.

  By analogy, $\log(x^3 + 2x^2) = \log x + \log x + \log(x+2)$.
  Then, if $(r) = \log(x-r)$, we have $(0) + (0) + (-2) = 2(0) + 1(-2)$.
\end{example}

\textbf{Key Observation.} A prime factor of multiplicity $r$
corresponds exactly with a root of multiplicity $r$.

\begin{defn}[divisor of a rational function]
  $\div(\frac{f}{g}) = \div f - \div g$.
\end{defn}

\begin{defn}[order of vanishing]
  For $f \in K(E)$ and $P \in E$,
  $\ord_P(f)$ is the multiplicity of the ``prime'' $P$ in the factorization of $f$,
  i.e., the coefficient in the divisor.
\end{defn}

\begin{theorem}
  $\ord_P(f \cdot g) = \ord_P(f) + \ord_P(g)$
\end{theorem}

Suppose we have polynomials $f = f_0 + f_1 y$ and its ``conjugate'' $g = f_0 - f_1 y$.
Then, $\ord_P(f \cdot g) = \ord_P(f_0^2 - f_1^2 y^2)$
which is a polynomial in only $x$,
so we can find the multiplicity normally.
Using a symmetry argument, we can then derive other orders.

\section{(11/25)}

Let $e_n(P, Q) = \frac{f_P}{f_Q}$ where $f_P = \div(P)$

Recall the degree of a single-variable polynomial $\deg(a + a_1 x + \dotsb + a_d x^d) = d$.
This definition breaks for multi-variable polynomials,
so instead define the degree as the number of roots.\footnote{Assuming the polynomial is separable and the field is algebraically closed, counting multiplicities.}

For a polynomial $f \in K[E]$, we can \Wlog write $f(x,y) = f_0(x) + f_1(x)y$
and say that $\deg f$ is the number of roots of $f$.

\begin{example}
  $\deg x = \deg (x-\alpha) = 2$,
  $\deg y = \deg (y-\alpha) = 3$
\end{example}

We have properties of normal degrees:
$\deg(f \cdot g) = \deg f + \deg g$, $\deg(f+g) = \max\{\deg f, \deg g\}$.

\begin{defn}[conjugation]
  If $f = f_0 + f_1y$, then $\bar f = f_0 - f_1y$.
\end{defn}

Since conjugation is an automorphism, $\deg f = \deg \bar f$.
Then, $\deg(f\cdot\bar f) = 2\deg f$
but we have that $f\cdot\bar f = f_0^2 - f_1^2 y^2
  = f_0^2 - f_1^2(x^3 + ax + b) \in K[x]$.
We can factor the usual way to get $n$ linear factors,
which each have degree 2, so $\deg f = \deg(f_0^2 - f_1^2(x^3+ax+b))$.

There are three points where $y = 0$, i.e., $P_i = (r_i, 0)$
where $r_i$ are the roots of the cubic.
Claim that $\div y = (P_1) + (P_2) + (P_3)$.

Relate every point $P = (\alpha, \beta)$
and relate it to a maximal prime ideal generated by $(x-\alpha, y-\beta)$,
i.e., $K[x,y]/(x-\alpha,y-\beta) = K$.
Considering the principal ideal generated by
$(y) = (x-r_1, y-0)(x-r_2,y-0)(x-r_3,y-0)$ which factors into prime ideals.
This is why we have $\div y = (P_1) + (P_2) + (P_3)$
since divisors correspond with prime factorizations.

But we need to find $\ord_{\infty}(y)$.
With projective coordinates $x = \frac{X}{Z}$ and $y = \frac{Y}{Z}$,
we get $Y^2Z = X^3 + aXZ^2 + bZ^3$.
Finally, if $\tilde x = \frac{X}{Y}$ and $\tilde z = \frac{Z}{Y}$,
then we have $\tilde z = \tilde x^3 + a\tilde x\tilde z^2 + b\tilde z^3$.
Then, we have $\ord_{\infty}(y) = \ord_{\infty}(\frac{1}{\tilde z}) 
  = -\ord_{\infty}(\tilde z) = -\ord_{(0,0)}(\tilde z) = -3$.

So $\div y = (P_1) + (P_2) + (P_3) - 3(\infty)$
and $\deg(\div y) = 0$.
In fact, $\deg(\div f) = 0$ for all $f$.

\begin{example}
  Calculate $\div x$.
\end{example}
\begin{sol}
  Likewise, $\div x = (Q_1) + (Q_2) - 2(\infty)$
  because we have two points on the line $x=0$ and $\ord_{\infty}(x) = \ord_{(0,0)}(\frac{\tilde x}{\tilde z})
  = \ord_{(0,0)}(\tilde x) - \ord_{(0,0)}(\tilde z) = 1 - 3 = -2$.
\end{sol}

\begin{example}
  Calculate $\div(\frac{x}{x^2+y})$.
\end{example}
\begin{sol}
  First, we have $\div x - \div(x^2 + y)$.

  Using the conjugate trick, $\div(x^2+y) + \div(x^2-y) = \div(x^4 - (x^3+ax+b))$
  which factors $\div \prod (x-e_i) = \sum \div (x-e_i)$ with roots $e_i$.
  Then, we have $\sum((P_i) + (-P_i)) - 8(\infty)$.
  By symmetry, $\div(x^2+y) = \sum(\pm (P_i)) - 4(\infty)$.
\end{sol}

\end{document}