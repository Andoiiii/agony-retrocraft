\documentclass{article}
\documentclass{report}
\usepackage{amsmath}
\usepackage{amsfonts}
\usepackage{amssymb}
\usepackage{enumitem}
\usepackage{hyperref}
\usepackage{geometry}
\usepackage{listings}
\usepackage{graphicx}
\usepackage{xcolor}
\usepackage{color,soul}



\lstset{
    basicstyle=\ttfamily,
    framesep=3pt,
    language=C,
    showstringspaces=false,
    tabsize=2,
    frame = single,
    breaklines=true
}
\geometry{a4paper, margin=1in}

\begin{document}

\title{Final Practice - CS136 (SOLUTION)}
\author{Prepared and compiled by Tommy Pang, special thank to Benny Wu for review }
\date{Last update: April 13th, 2024}

\maketitle



\noindent
\textbf{Disclaimer:} \\
\vspace{3pt}
\noindent
The practice problems is an independent effort and is \textbf{not affiliated} with the University of Waterloo. The materials provided are intended solely for practice purposes, as we recognize its potential benefits for first-year students. Should there be any concerns regarding rights or violations, please contact us immediately for removal.

\vspace{20pt}
\title{\textbf{\textcolor{red}{NOTE: WE CANNOT GUARANTEE CORRECTNESS TO ALL QUESTIONS}}}


\vspace{10pt}

\section*{Conceptual Questions}

\begin{enumerate}
    \item Give three advantages of using modularization and describe each.
    \vspace{16mm}
    \item Which of the following statement(s) are true?
    \begin{enumerate}[label=\alph*.]
        \item \hl{Clients may require functions from modules.} 
        \item Clients provide implementation to modules.
        \item \hl{Modules provide functions to clients.}
    \end{enumerate}
    \item Which of the following statement(s) are true?
    \begin{enumerate}[label=\alph*.]
        \item \hl{.o files represent source files that are compiled into machine code.}
        \item Only one module is permitted in a program.
        \item \hl{We can combine multiple machine code files to build a program.}
        \item \hl{A program must have exactly one function called main.}
    \end{enumerate}
    \item Discuss declaration vs. definition in C.
    \vspace{16mm}
    \item Describe the \texttt{extern} keyword.
    \vspace{16mm}
    \item True or \hl{False}: A module \texttt{my\_module.c} will not compile if you do not include \texttt{my\_module.h}.
    
 
\item Consider the below module, label each identifier (\texttt{x}, \texttt{score}, \texttt{score\_update}, \texttt{main}, \texttt{run\_game}, \texttt{MAX\_SCORE}) with their scopes (Local, Program, Module scope):

    
    \noindent
    
    
    \begin{minipage}{.5\textwidth}
    \begin{lstlisting}
    // main.c
    extern int score;
    const int MAX_SCORE = 100;
    
    void run_game(void);
    
    int main(void) {
        run_game();
        printf("Score: %d", score);
    }
    \end{lstlisting}
    \end{minipage}%
    \begin{minipage}{.5\textwidth}
    \begin{lstlisting}
    // module.c
    int score = 0;
    static int direction = 0;
    
    static void score_update(int n) {
        score += n;
    }
    
    void run_game(void) {
        int x = 0;
        // ...
    }
    \end{lstlisting}
    \end{minipage}

    \begin{enumerate}
        \item x - \hl{local}
        \item score - \hl{program}
        \item score\_update - \hl{module}
        \item main - \hl{program}
        \item run\_game - \hl{program}
        \item MAX\_SCORE - \hl{program}
    \end{enumerate}


    \item Describe an Opaque structure.
    \vspace{16mm}
    \item What constant(s) does the module \texttt{limits.h} provide?
    \vspace{16mm}
    \item What constant(s) does the module \texttt{stdlib.h} provide?
    % \vspace{16mm}
    % \item Describe high cohesion.
    % \vspace{16mm}
    % \item Describe low coupling.
    \vspace{16mm}
    \item Compare Interface vs. implementation.
    \vspace{16mm}

    % \item Describe an Abstract Data Type (ADT) at a high level, when we may use it, and give an example.
    % \vspace{16mm}
    \item Why is the actual data structure \& implementation hidden from the client in an ADT?
    \vspace{16mm}
    \item Is this a valid array definition: \texttt{char a[5] = "array";}?
    \item Is this a valid array definition: \texttt{char a[5] = \{0\};}?
    \item Is this a valid array definition: \texttt{int a[3] = \{0, 1, 2, 3\};}?
    \item Given definition \texttt{char a[5] = \{0\};} what does \texttt{strlen(a)} return?
    \vspace{16mm}
    \item True or \hl{False}: Given algorithm A has worst case complexity $O(n^2)$, and algorithm B has worst case complexity $O(n\log n)$, then A runs faster than B in all instances (assume of size $n$).
    \item State the correct order for the running time: $100000 + 0.0001n + 0.001\log n$.
    \vspace{16mm}
    \item Insertion Sort, Selection Sort, and Quicksort, which is more efficient?
    \begin{enumerate}[label=\alph*.]
        \item Insertion Sort
        \item Selection Sort
        \item Quicksort
        \item \hl{They all have the same efficiency}
    \end{enumerate}
    \item True/\hl{False}: C has a String type that’s built-in.
    \item True/\hl{False}: The length returned by \texttt{strlen} includes the null-terminator.
    \item True/\hl{False}: The character ‘0’ is the null terminator.
    \item State the result of the below \texttt{strcmp} calls:
    \begin{enumerate}[label=\alph*.]
        \item \texttt{strcmp("", "x");}
        \item \texttt{strcmp("2", "1");}
        \item \texttt{strcmp("abcd", "abc");}
    \end{enumerate}
    \vspace{16mm}

    \item What’s the result of the length:
    \begin{lstlisting}
    char arr[5] = {'a', 'x', '0', 'x'};
    printf("%d\n", strlen(arr));
    \end{lstlisting}
    \vspace{16mm}
    \item What’s the output of the following code:
    \begin{lstlisting}
    char s1[] = "str";
    char s2[] = "str";
    
    if (s1 == s2) {
        printf("equal");
    }
    else {
        printf("not equal");
    }
    \end{lstlisting}
    \vspace{16mm}
    \item Describe string literals.
    \vspace{16mm}
    \item Consider this code, write out the output produced, or up to the point of error occurrence if you think there’s an error, and describe what’s the error (e.g. heap-overflow):
    \begin{lstlisting}
    int i = 0;
    const char* s = "abc\0aaa\0bbb";
    while (i < strlen(s)) {
        printf("%c", s[i]);
        i++;
    }
    for (int j = 1; j <= 3; ++j) {
        printf("%c", s[i + j]);
    }
    \end{lstlisting}
    \vspace{16mm}
    \item Why this is an issue:
    \begin{lstlisting}
    void dumb_string_op(const char* a, const char* b) {
        strcpy(a, b);
    }
    \end{lstlisting}
    \vspace{16mm}
    \item In the following code snippet, which line will error occur at runtime in EdX?
    \begin{lstlisting}
    int main(void) {
        int *j = malloc(sizeof(int)); // line 1
        free(j); // line 2
        *j = 43; // line 3
        return 0; // line 4
    }
    \end{lstlisting}
    Options:
    \begin{enumerate}[label=\alph*.]
        \item Line 1
        \item Line 2
        \item \hl{Line 3}
        \item Line 4
    \end{enumerate}
    \item In the following code snippet, which line will error occur at runtime in EdX?
    \begin{lstlisting}
    int main(void) {
        int *j = malloc(sizeof(int)); // line 1
        int *k = j; // line 2
        free(k); // line 3
        *j = 43; // line 4
        return 0; // line 5
    }
    \end{lstlisting}
    Options:
    \begin{enumerate}[label=\alph*.]
        \item Line 1
        \item Line 2
        \item Line 3
        \item \hl{Line 4}
        \item Line 5
    \end{enumerate}
    \item Why is it a good practice to set a pointer to NULL after freeing it?
    \vspace{16mm}
    \item What occurs when the malloc function is unable to allocate the requested amount of memory?

    Options:
    \begin{enumerate}[label=\alph*.]
        \item Program end with non 0 exit code
        \item malloc allocate the maximum \# bytes that’s affordable
        \item \hl{malloc returns NULL}
        \item Undefined behavior
    \end{enumerate}
    \item What is a dangling pointer, and provide an example.
    \vspace{16mm}
    \item Compare stack and heap data/memory.
    \vspace{16mm}
    \item List two advantages of using heap memory.
    \vspace{16mm}

    \item Is there anything wrong with the below function that destroys the linked list?
    \begin{lstlisting}
    struct Node {
        const void* val;
        struct Node* next;
    };

    struct List {
        struct Node* front;
    };

    void destroy_linked_list(struct List* lst) {
        struct Node* cur = lst->front;
        while (cur) {
            free(cur);
            cur = cur->next;
        }
    }
    \end{lstlisting}
    \vspace{16mm}
    \item Will the below code result in a dangling pointer? \hl{NO}
    \begin{lstlisting}
    void bruhdanglingptr(int n) {
        char *a = malloc(n * sizeof(char));
        char *b = malloc(n * sizeof(char));
        b = a;
    }
    \end{lstlisting}

    \item Will the below code result in a dangling pointer? \hl{NO}
    \begin{lstlisting}
    void bruhdanglingptr(int n) {
        char *a = malloc(n * sizeof(char));
        a = malloc(2 * n * sizeof(char));
    }
    \end{lstlisting}
    
    \item Will the below code result in a dangling pointer? \hl{YES}
    \begin{lstlisting}
    void bruhdanglingptr(int n) {
        char *a = malloc(n * sizeof(char));
        char *c = realloc(a, 2 * n * sizeof(char));
    }
    \end{lstlisting}
    
    \item Will the below code for sure result in a memory leak? \hl{YES}
    \begin{lstlisting}
    void bruhmemleak(int n) {
        char *a = malloc(n * sizeof(char));
        char *b = malloc(n * sizeof(char));
        b = a;
    }
    \end{lstlisting}
    
    \item Will the below code for sure result in a memory leak? \hl{NO}
    \begin{lstlisting}
    void bruhmemleak(int n) {
        char *a = malloc(n * sizeof(char));
        char *c = realloc(a, 2 * n * sizeof(char));
    }
    \end{lstlisting}

    \item Is it true that the worst case complexity of pushing to a stack ADT seen in class is \( O(n) \)? \hl{TRUE}
    
    \item Is it true that if a program runs \( O(1) \) amortized, then its worst case complexity cannot be worse than \( O(n) \)? \hl{FALSE}
    
    \item Is it true that amortized analysis is only applicable to data structures and cannot be used for analyzing algorithms? \hl{FALSE}
    
    \item Is it true that the amortized cost of an operation is always equal to the worst case cost of that operation? \hl{FALSE}
    
    \item Is it true that the amortized runtime/cost of an operation is always better than the worst case cost/runtime of that operation? \hl{FALSE}
    
    \item Beside each print statement, write the corresponding output (address), if there’s an error, describe why. See the code here: \url{https://pastebin.com/qeqgGQz4}. \hl{RUN IT YOURSELF LOL}
    \vspace{10pt}
    % Repeat for the other cases...
    % Add all your other questions following the same pattern.

    % Add the questions regarding complexity analysis and code outputs.
    
\end{enumerate}

\newpage
\section*{Complexity Analysis}

\begin{enumerate}
    
    \item What’s the runtime worst case running time in terms of parameter \( n \) for the following function:
    \begin{lstlisting}
    bool is_prime(int n) {
        if (n <= 1) {
            return false;
        } else if (n == 2) {
            return true;
        }
        for (int i = 3; i * i <= n; i += 2) {
           if (n % i == 0) {
            return false;
           }
        }
        return true;
    }
    \end{lstlisting}
    Options:
    \begin{enumerate}[label=\alph*.]
        \item \( O(n) \)
        \item \( O(\log n) \)
        \item \( O(\sqrt{n}) \) \hl{\checkmark}
        \item \( O(n^2) \)
    \end{enumerate}
    % Add the other runtime analysis questions in a similar manner...

    \item What’s the worst case time complexity/efficiency of this code:
    \begin{lstlisting}
    void subset_sums(int i, int n, int val, int a[]) {
        if (i == n) {
            printf("%d\n", val);
            return;
        }
        
        subset_sums(i + 1, n, val, a);
        if (i % 2 == 1) {
            subset_sums(i + 1, n, val + a[i], a);
        }
    }
    \end{lstlisting}
    Options:
    \begin{enumerate}[label=\alph*.]
        \item \( O(n^2) \)
        \item \( O(n) \)
        \item \( O(n!) \)
        \item \( O(2^n) \) \hl{\checkmark}
    \end{enumerate}

    \newpage

    \item What’s the worst case time complexity/efficiency of this code:
    \begin{lstlisting}
    int cringe_search() {
        int lo = 0, hi = 1000000, ans = -1;
        while (lo <= hi) {
            int mid = lo + (hi - lo) / 2;
            if (mid % 2 == 1) {
                ans = mid;
                hi = mid - 1;
            } else {
                lo = mid + 1;
            }
        }
        return ans;
    }
    \end{lstlisting}
    Options:
    \begin{enumerate}[label=\alph*.]
        \item \( O(\log n) \)
        \item \( O(n) \)
        \item \( O(1) \) \hl{\checkmark}
        \item \( O(n^2) \)
    \end{enumerate}
    % Continue with the rest of the questions in the same manner...
    
    
    \item What’s the worst case complexity of the following pseudocode:
    \begin{lstlisting}
    p = 1
    s = 0
    for i = 1 to n do
        p = p * 2
        for j = 1 to p do
            s = s + 1
    \end{lstlisting}
    Options:
    \begin{enumerate}[label=\alph*.]
        \item \( O(2^n) \) \hl{\checkmark}
        \item \( O(n \log n) \)
        \item \( O(n 2^n) \)
        \item \( O(n^2) \)
    \end{enumerate}

    You may find geometric series summation formula useful:
    \[
    s_n = a r^0 + ar^1 + \ldots + ar^{n-1} = \sum_{k=0}^{n-1} ar^k = \sum_{k=1}^{n} ar^{k-1} = 
    \begin{cases} 
    a \frac{1-r^n}{1-r} & \text{if } r \neq 1 \\
    an & \text{otherwise}
    \end{cases}
    \]
    
    \newpage
    
    \item What's the runtime worst case running time in terms of parameter \( n \):
    \begin{lstlisting}
    int fibonacci(int n) {
        if (n <= 1) {
            return n;
        }
    
        int a = 0, b = 1;
        for (int i = 2; i <= n; i++) {
            int temp = a + b;
            a = b;
            b = temp;
        }
        return b;
    }
    \end{lstlisting}
    Options:
    \begin{itemize}
        \item \( O(n) \) \hl{\checkmark}
        \item \( O(1) \)
        \item \( O(\sqrt{n}) \)
        \item \( O(n^2) \)
    \end{itemize}

    \item What's the runtime worst case running time in terms of parameter \( n \):
    \begin{lstlisting}
    void f(int n) {
        int a = 0;
        for (int i = 0; i < n; i++) {
            for (int j = n; j > i; j /= 2) {
                a = a + 1;
            }
        }
    }
    \end{lstlisting}
    Options:
    \begin{itemize}
        \item \( O(n) \)
        \item \( O(n \log n) \) \hl{\checkmark}
        \item \( O(n \sqrt{n}) \)
        \item \( O(n^2) \) 
    \end{itemize}

    \item Give the exact number of iterations performed, i.e., what’s the value printed:
    \begin{lstlisting}
    void f(int n) {
        int value = 0;
        for (int i = 0; i < n; i++) {
            for (int j = 0; j < i; j++) {
                value += 1;
            }
        }
        printf("%d\n", value);
    }
    \end{lstlisting}
    Options:
    \begin{itemize}
        \item \( n(n+1) \)
        \item \( \frac{n(n-1)}{2} \) \hl{\checkmark}
        \item \( n^2 \)
        \item \( n \)
    \end{itemize}
    \newpage

    \item What’s the worst case complexity of the function \texttt{dumb\_binary\_search}:
    \begin{lstlisting}
    int binary_search(int arr[], int lo, int hi, int target) {
        while (lo <= hi) {
            int mid = lo + (hi - lo) / 2;
            if (arr[mid] == target) {
                return mid;
            } else if (arr[mid] < target) {
                lo = mid + 1;
            }
            else {
                hi = mid - 1;
            }
        }
        return -1;
    }
    
    int dumb_binary_search(int arr[], int n, int target) {
        int lo = 0, hi = n - 1;
        while (lo <= hi) {
            int mid = lo + (hi - lo) / 2;
            if (arr[mid] == target) {
                return mid;
            } else if (arr[mid] < target) {
                lo = mid + 1;
            }
            else {
                int result = binary_search(arr, lo, mid - 1, target);
                if (result != -1) {
                    return -1;
                }
                hi = mid - 1;
            }
        }
        return -1;
    }
    \end{lstlisting}
    Options:
    \begin{itemize}
        \item \( O(\log n) \)
        \item \( O((\log n)^2) \) \hl{\checkmark}
        \item \( O(n \log n) \)
        \item \( O(\log (n^2)) \)
    \end{itemize}

    \newpage
    
    \item What’s the worst case complexity of the function \texttt{weird} (be exact):
    \begin{lstlisting}
    void weird(int n, int m) {
        for (int i = 2; i <= n; i++) {
            for (int j = 1; j < i % m; j++) {
                printf("*");
            }
        }
    }
    \end{lstlisting}
    Options:
    \begin{itemize}
        \item \( O(n^2) \)
        \item \( O(n) \)
        \item \( O(m) \)
        \item \( O(nm) \) \hl{\checkmark}
    \end{itemize}

    \item What is the worst case complexity of the following function \texttt{weird2}?
    \begin{lstlisting}
    void weird2(int n, int m) {
        for (int i = 2; i <= n; i++) {
            for (int j = 1; j < i % 1000000; j++) {
                printf("*");
            }
        }
    }
    \end{lstlisting}
    Options:
    \begin{itemize}
        \item \( O(n) \) \hl{\checkmark}
        \item \( O(n^2) \)
        \item \( O(n \log n) \)
        \item \( O(\log n) \)
    \end{itemize}


    \item What’s the worst case complexity of the function  \texttt{weird\_recurrence2}? \textbf{Hint: Geometric Series Sum}
    \begin{lstlisting}
    void weird_recurrence2(int n) {
        if (n == 0) return;
        weird_recurrence(n / 2);

        int s = 0;

        for (int i = 0; i < n; ++i) {
            for (int j = i + 1; j < n; ++j) {
                s++;
            }
        }

        printf("%d\n", s);
    }
    \end{lstlisting}
    Options:
    \begin{itemize}
        \item \( O(n^2 \log n) \)
        \item \( O(n^3) \)
        \item \( O(n^2) \) \hl{\checkmark}
        \item \( O(n \log n) \)
    \end{itemize}

    
    \item What’s the worst case complexity of the function \texttt{a\_string\_function}?
    
    \begin{lstlisting}
    void a_string_function(char* s) {
        const char* dummy = "abc";
        char *res = "";

        for (int i = 0; i < strlen(s); ++i) {
            if (strcmp(dummy, s)) {
                strcat(res, s);
            }
        }
        printf("gg %s", res);
    }
    \end{lstlisting}
    Options:
    \begin{itemize}
        \item \( O(n^2) \)
        \item \( O(n^3) \) \hl{\checkmark}
        \item \( O(n^4) \)
        \item \( O(n^5) \)
    \end{itemize}

    \newpage

    \item What’s the worst case complexity of the function \texttt{a\_string\_function2}?
    \begin{lstlisting}
    void a_string_function2(char* s, char* s2, char*res) {
        const char* tmp = "sheeeesh";

        for (int i = 0; i < strlen(s); ++i) {
            for (int j = 0; j < strlen(tmp); j *= 2) {
                strcopy(res, s2);
            }
        }
        printf("%s", res);
    }
    \end{lstlisting}
    Options:
    \begin{itemize}
        \item \( O(nm^2) \)
        \item \( O(nm \log n) \)
        \item \( O(n^2m^2) \)
        \item \( O(n^2m) \) \hl{\checkmark}
        \item \( O(n^2m \log n) \)
    \end{itemize}

    \item What’s the worst case complexity of the function \texttt{two\_ptr}?
    \begin{lstlisting}
    int two_ptr(int nums[], int k, int n) {
        int ans = n * (n + 1) / 2, max = 0;
        for (int i = 0; i < n; i++) {
            if (nums[i] > max) {
                max = nums[i];
            }
        }

        int cnt[100005] = {0};

        for (int i = 0, j = 0; j < n; ++j) {
            cnt[nums[j]]++;
            while (i <= j && cnt[max] >= k) {
                cnt[nums[i]]--;
                i++;
            }
            ans -= (j - i + 1);
        }
        return ans;
    }
    \end{lstlisting}
    Options:
    \begin{itemize}
        \item \( O(n^2) \)
        \item \( O(n) \) \hl{\checkmark}
        \item \( O(n \log n) \)
        \item \( O(n \sqrt{n}) \)
    \end{itemize}

    \newpage
    \item What’s the worst case complexity of the function \texttt{nxt\_greater}, assuming a stack ADT implemented with dynamic array is available.
    \begin{lstlisting}
    void nxt_greater(int nums[], int n) {
        struct stack* stk = stack_create();
        for (int i = n - 1; i >= 0; --i) {
            while (!stack_empty(stk) && stack_top(stk) <= nums[i]) {
                stack_pop(stk);
            }
            if (stack_empty(stk))
                printf("%d has no next greater element\n", nums[i]);
            else 
                printf("%d has next greater element - %d \n", nums[i], stack_top(stk));
            stack_push(stk, nums[i]);
        }
    }
    \end{lstlisting}
    Options:
    \begin{itemize}
        \item \( O(n) \) \hl{\checkmark}
        \item \( O(n^2) \)
        \item \( O(n \log n) \)
    \end{itemize}

\end{enumerate}

\newpage
\section*{Programming Questions}

\begin{enumerate}
    \item 

    \begin{lstlisting}
        
    int pow(int b, int p) {
        if (p == 0) { // base case
            return 1;
        }
        int half = pow(b, p / 2); // b ^ k
        if (p % 2 == 0) { // b^2k = (b^k) ^ 2
            return half * half; // b^2k = (b^k) ^ 2
        } else { // b^(2k+1) = b^2k * b
            return half * half * b;
        }
    }
    \end{lstlisting}
    \newpage
    \item
    \begin{lstlisting}
    void special_sort(int A[], int n) {
        const int max = 1000;
        int cnt[1001] = {0};
        for (int i = 0; i < n; i++) {
            cnt[A[i]]++;
        }
        int idx = 0;
        for (int i = 1; i <= max; i++) {
            for (int j = 0; j < cnt[i]; j++) {
                A[idx] = i;
                idx++;
            }
        }
    }
    \end{lstlisting}
    \url{https://www.geeksforgeeks.org/counting-sort/}

    \newpage
    \item

    \begin{lstlisting}
        
struct stack;
// malloc a stack in heap memory
struct stack *create();
// check if stack is empty
bool empty(const struct stack *stk);
// push element to top of stack
void push(void *item, struct stack *stk);
// get the top element of the stack
const void *top(const struct stack *stk);
// remove the top element of the stack
void *pop(struct stack *stk);
// free the resources used by stack
void destroy(struct stack *s);

bool balanced(char* s, int n) {
    struct stack* stk = create();
    for (int i = 0; i < n; i++) {
        if (s[i] == ')') {
            if (!empty(stk) && top(stk) == '(') {
                pop(stk);
            } else {
                return false;
            }
        } else if (s[i] == ']') {
            if (!empty(stk) && top(stk) == '[') {
                pop(stk);
            } else {
                return false;
            }
        } else if (s[i] == '}') {
            if (!empty(stk) && top(stk) == '{') {
                pop(stk);
            } else {
                return false;
            }
        } else if (s[i] == '(' || s[i] == '[' || s[i] == '{') {
            push(s[i], stk);
        }
    }
    bool empty = empty(stk);
    destroy(stk);
    return empty;
}


    \end{lstlisting}

    \newpage
    \item \\

    \begin{lstlisting}
    char most_frequent(const char* s, int n) {
        int cnt[128] = {0};
        for (int i = 0; i < n; i++) {
            cnt[s[i]]++;
        }
        char res = '\0';
        int max_cnt = 0;
        for (int c = 0; c < 128; c++) {
            if (cnt[c] > max_cnt) {
                max_cnt = cnt[c];
                res = (char) c;
            }
        }
        return res;
    }
    \end{lstlisting}

    \newpage
    \item
    \begin{lstlisting}
struct Node {
    const void* val;
    Node* nxt;
};

struct List {
    Node* front;
};

struct hashtable {
    int size;
    int bucket_length;
    int (*hash_func)(const void *);
    int (*key_cmp)(const void *, const void *);
    void (*key_print)(const void *);
    struct List **buckets;
};

struct hashtable *table_create(int M, int (*hash_func)(const void *), int (*key_cmp)(const void *, const void *), void (*key_print)(const void *)) {
    struct hashtable *ht = malloc(sizeof (struct hashtable));
    ht->size = 0;
    ht->bucket_length = M;
    ht->hash_func = hash_func;
    ht->key_cmp = key_cmp;
    ht->key_print = key_print;
    ht->buckets = malloc(sizeof (struct List *) * M);
    for (int i = 0; i < M; i++) {
        ht->buckets[i] = malloc(sizeof (struct List));
        ht->buckets[i]->front = NULL;
    }
    return ht;

}

void list_insert(const void* val, struct List* bucket) {
    assert(val);
    assert(bucket);
    struct Node* n = malloc(sizeof (struct Node));
    n->nxt = NULL;
    n->val = val;
    if (bucket->front == NULL) {
        bucket->front = n;
    } else {
        n->nxt = bucket->front;
        bucket->front = n;
    }
}

bool table_insert(const void* x, struct hashtable* ht) {
    assert(x);
    assert(ht);
    int hsh = ht->hash_func(x);
    struct Node* p = ht->buckets[hsh]->front;
    while (p) { 
        if (ht->key_cmp(x, p->val) == 0) {
           return false; 
        }
        p = p->nxt;
    }

    list_insert(x, ht->buckets[hsh]);  
    ht->size++;

    return true;
}

bool table_search(const void* x, struct hashtable* ht) {
    assert(x);
    assert(ht);
    int hsh = ht->hash_func(x);
    struct List* l = ht->buckets[hsh];
    struct Node* ptr = l->front;
    while (ptr) {
        if (ht->key_cmp(x, ptr->val) == 0) {
            return true;
        }
        ptr = ptr->nxt;
    }
    return false;
}

bool table_remove(const void* x, struct hashtable* ht) {
    int hsh = ht->hash_func(x);
    struct List* l = ht->buckets[hsh];
    if (ht->key_cmp(l->front->val, x) == 0) { // remove from head
        struct Node* tmp = l->front;
        l->front = l->front->nxt;
        ht->size--;
        free(tmp); // must free
    }
    struct Node* ptr = l->front;
    struct Node* prev = NULL;
    while (ptr) {
        if (ht->key_cmp(x, ptr->val) == 0) {
            prev->nxt = ptr->nxt;
            free(ptr);
            ht->size--;
            return true;
        }
        prev = ptr;
        ptr = ptr->nxt;
    }
    return false;
}

void table_print(struct hashtable* ht) {
    for (int i = 0; i < ht->bucket_length; i++) {
        struct List* lst = ht->buckets[i];
        if (lst->front == NULL) {
            printf("EMPTY Bucket\n");
            continue;
        }
        struct Node* p = 
lst->front;
        while (p) {
            ht->key_print(p->val);
            p = p->nxt;
        }
        printf("\n");
    }
}


    \end{lstlisting}

    \newpage
    \item 
    \begin{lstlisting}
struct Node {
    int val;
    struct Node* left;
    struct Node* right;
    struct Node* parent;
};

struct Node* find_lca(struct Node* n1, struct Node* n2) {
    struct Node* p1 = n1;
    struct Node* p2 = n2;
    int dep1 = 0, dep2 = 0;
    while (p1->parent != NULL) {
        p1 = p1->parent;
        dep1++;
    }
    while (p2->parent != NULL) {
        p2 = p2->parent;
        dep2++;
    }
    if (p1 != p2) { // not on same tree
        return NULL;
    }
    p1 = n1;
    p2 = n2;
    while (dep1 > dep2) {
        p1 = p1->parent;
        dep1--;
    }
    while (dep2 > dep1) {
        p2 = p2->parent;
        dep2--;
    }
    while (p1 != p2) {
        p1 = p1->parent;
        p2 = p2->parent;
    }
    return p1;
}


    \end{lstlisting}



    \newpage
    \item Assume we have a variant of the binary tree, called a \( d \)-ary tree, where each node have exactly \( d \) children except the leaf nodes. Write a function \texttt{find\_val} that returns the value of the \( k \)th ( \( 1 \leq k \leq n \) ) element from the left in the  \( j \)th level (root is level 1). The runtime should be $O(n)$ where n is the total number of elements. (\textbf{Challenge: Solve in O(h) where h is the height of the d-ary tree})

    
    \begin{lstlisting}
    struct DaryTreeNode {
        int value;
        struct DaryTreeNode** children; // Array of pointers to children
        int numChildren;                // Actual number of children
    };
    
    
    /**
     * @param root The root of the d-ary tree.
     * @param j The depth of the element to find.
     * @param k The position of the element to find.
     */
    int find_kth(struct DaryTreeNode* root, int j, int k) {
      int answer = 0;
      int index = 0;
      find_kth_helper(root, j, k, 0, &index, &answer);
      return answer; 
    }
    
    void find_kth_util(struct DaryTreeNode* node, int j, int k, int current_depth, int* index, int* answer) {
      if (*index == k) return;
      
      if (current_depth == j) {
        (*index)++;
        if (k == *index) {
          *answer = node->val;
        }
        return;
      }
    
      for (int i = 0; i < node->d; i++) {
        find_kth_helper(node->children[i], j, k, current_depth + 1, index, answer);
      }
    }
\end{lstlisting}


    
    \newpage
    \item 
    \begin{lstlisting}
    char* substr(char* s, int i, int j) {
       char* ans = malloc((j - i + 1) * sizeof(char));
    
    
       for (int t = i; t < j; t++) {
           ans[t - i] = s[i];
       }
       ans[j - i] = '\0';
      
       return ans;
    }
    \end{lstlisting}


    \newpage
    \item 
    \url{https://pastebin.com/u3jjQcFQ}

    
\end{enumerate}



\end{document}
