We cannot apply fundamental theorem of LPs to IPs, since some IPs may be
feasible, bounded, and still lack an optimal solution.

\begin{example}
  $\max\{x_1-\sqrt2x_2 : x_1-\sqrt2x_2 \leq 0, x \geq \one, x \in \Z^2\}$
  has a feasible solution $x = (1,1)\trans$,
  is bounded since $(1,-\sqrt2)\trans x \leq 0$,
  but has no optimal solution.
\end{example}

\begin{defn}[convex hull]
  The smallest convex set that contains all
  points in a set. Formally,
  $\operatorname{conv}(x^1,\dotsc,x^k) = \{\sum \lambda_i x^i : \sum \lambda_i = 1, \lambda \geq \zero \}$
\end{defn}

\begin{theorem}[Fundamental Theorem of Integer Programming]
  Given a polyhedron $P = \{x \in \R^n : Ax \leq b\}$ with $A,b \in \textbf{M}(\Q)$,
  the convex hull $\operatorname{conv}(\Z^n \cap P)$ is another polyhedron
  $Q = \{x \in \R^n : A'x \leq b'\}$ with $A',b' \in \textbf{M}(\Q)$.
\end{theorem}

This reduces solving an IP to solving a related LP, since the
extreme points of $Q$ are just integer points in $P$.

\lecture{July 11}
\begin{theorem}
  Given (IP) $\max\{c\trans x : Ax \leq b : x \in \Z^n\}$,
  let $\{x : A' x \leq b'\}$ be the convex hull of solutions of (IP).
  Let (LP) be $\max\{c\trans x : A'x \leq b'\}$. Then:
  \begin{itemize}[nosep]
    \item (IP) infeasible if and only if (LP) infeasible
    \item (IP) unbounded if and only if (LP) unbounded
    \item Every optimal solution of (IP) is an optimal solution of (LP)
    \item Every optimal solution of (LP) that is an extreme point is an optimal solution of (IP)
  \end{itemize}
\end{theorem}

Not a great idea in general because finding the convex hull is hard.

We need to find feasible solutions in order to define the convex hull anyways
and the number of constraints to define convex hull is $O(2^n)$ in the original constraints.

There is no finite convex hull if coefficients are irrational.

\begin{defn}[LP relaxation]
  An IP without the integer constraint.
\end{defn}

\begin{example}
  Find $\max\qty{\mqty(0&1\\3&-1\\-5.5&-4)x \leq \mqty(5.5\\10.5\\-22) : x \in \Z^3}$
\end{example}
\begin{sol}

  The optimal solution of the relaxation is
  $(\frac{128}{35},\frac{33}{70})\trans$.

  To make a cutting plane, need (1) current non-integral optimal
  solution outside of halfspace and (2) all integral feasible
  solutions inside halfspace.

  For example, shrink the feasible region with $x_2 \geq 2$ to get
  new optimal solution $(\frac{28}{11},2)\trans$ which is ``better''

  To find cutting plane, note that simplex BFS
  $\bar x = (\frac{128}{35}, \frac{33}{70}, \frac{176}{35},0,0)^t$
  with objective value $-\frac{289}{70}$.

  Take floor of coefficients on LHS, change $=$ to $\leq$:
  $x_1 + \floor{\frac{8}{35}}x_4 + \floor{-\frac{2}{35}}x_5 = x_1 - x_5 \leq \frac{128}{35}$

  Since $x \geq \zero$, LHS only decreases and all solutions are preserved.

  Take floor of RHS constant: $x_1 - x_5 \leq \floor{\frac{128}{35}} \leq 3$.

  Since $x$ integral, coefficients integral, we can take floor
  without losing integral solutions.
  This also eliminates $\bar x$ since it satisfied the original
  constraint with equality.

  Then, we have $x_1 - x_5 \leq 3$ a cutting plane.
\end{sol}

\lecture{July 13}

Summarize the algorithm:

\begin{algorithm}
  \caption{Cutting plane algorithm}
  \begin{algorithmic}[1]
    \State Relax the IP by removing the integrality constraint
    \State Run simplex on the resulting LP to get non-integral BFS $\bar x$
    \State Select a constraint $\sum a_i x_i = b_j$ where $b_j \not\in \Z$
    \State Construct cutting plane $\sum \floor{a_i} x_i \leq \floor{b_j}$
    \State Add slack variable to turn into equality and append to LP
    (will have to run either two-phase simplex or dual simplex to find a new BFS)
  \end{algorithmic}
\end{algorithm}
