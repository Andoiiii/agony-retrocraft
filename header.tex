\documentclass[11pt]{article}

\usepackage{physics}
\usepackage{amsfonts,amsmath,amssymb}
\usepackage{systeme}
\sysdelim..
\usepackage{polynom}
\usepackage{enumerate}
\usepackage{titlesec}
\usepackage{fancyhdr}
\usepackage{multicol}
\usepackage[dvipsnames]{xcolor}

\headheight 13.6pt
\setlength{\headsep}{10pt}
\textwidth 15cm
\textheight 24.3cm
\evensidemargin 6mm
\oddsidemargin 6mm
\topmargin -1.1cm
\setlength{\parskip}{1.5ex}
\parindent=0pt

\author{James Ah Yong}

\pagestyle{fancy}
\fancyhf{}
\fancyfoot[c]{\thepage}
\makeatletter
\lhead{\@title}
\rhead{\@author}

\fancypagestyle{firstpage}{
  \fancyhf{}
  \rhead{\@author}
  \fancyfoot[c]{\thepage}
}

% Sets
\newcommand{\N}{\ensuremath{\mathbb{N}}}
\newcommand{\Z}{\ensuremath{\mathbb{Z}}}
\newcommand{\Q}{\ensuremath{\mathbb{Q}}}
\newcommand{\R}{\ensuremath{\mathbb{R}}}
\newcommand{\C}{\ensuremath{\mathbb{C}}}
\newcommand{\F}{\ensuremath{\mathbb{F}}}
\newcommand{\U}{\ensuremath{\mathcal{U}}}
\newcommand{\sym}{\mathbin{\triangle}}

% Functions
\DeclareMathOperator{\sgn}{sgn}
\DeclareMathOperator{\im}{im}
\DeclareMathOperator{\lcm}{lcm}
\DeclareMathOperator{\cis}{cis}

% Operators
\newcommand{\Rarr}{\Rightarrow}
\newcommand{\Larr}{\Leftarrow}
\newcommand{\Harr}{\Leftrightarrow}
\newcommand{\harr}{\leftrightarrow}
\usepackage{mathtools} % for \DeclarePairedDelimiter macro
\DeclarePairedDelimiter\ceil{\lceil}{\rceil}
\DeclarePairedDelimiter\floor{\lfloor}{\rfloor}
\newcommand{\dyx}{\dv{y}{x}}

% Macros
% properly typeset ε-δ (epsilon en dash delta)
\newcommand{\epsdel}[1][\delta]{\ensuremath{\epsilon\mathit{\textnormal{--}}#1}}
\newcommand{\by}[1]{& \text{by #1}}
\newcommand{\IH}{\by{inductive hypothesis}}
\newcommand{\pf}[2]{%
\let\tmp\relax\newcommand\tmp[1]{#1}
\ensuremath{p_1^{\tmp{1}}p_2^{\tmp{2}}\cdots p_#2^{\tmp{#2}}}}
\newcommand{\FLT}{F\ensuremath{\ell}T}
% multiple choice (remove spacing between items)
\newenvironment{choices}
{\begin{enumerate}[(a)]
    \setlength{\parskip}{0ex}
    }{
  \end{enumerate}}

% Typesetting
\usepackage{array}   % for \newcolumntype macro
\newcolumntype{C}{>{$}c<{$}} % math version of "C" column type
\newcommand{\dlim}[2]{\lim\limits_{#1\to#2}} % totally not \dfrac ripoff
\newcommand{\dilim}[1]{\dlim{#1}{\infty}} % infinite limits
\newcommand{\ilim}[1]{\lim_{#1\to\infty}}
\newcommand\at[2]{\left.#1\right|_{#2}}
\usepackage{cancel}

% Linear Algebra
\newcommand{\xto}{\xrightarrow} % \xto{R_1 \harr R_2}

% Links
\usepackage{hyperref,cleveref}
\hypersetup{
  colorlinks,
  linkcolor=RoyalBlue,
  linktoc=all
}

% Question/Problem theorem styles
\usepackage{amsthm,thmtools}
\titleformat{\section}{\normalsize\bfseries}{\thesection}{1em}{}
\titleformat{\subsection}{\normalsize\bfseries}{\thesubsection}{1em}{}

\newcommand{\QType}{Q}
\newcounter{question}[subsection]
\renewcommand{\thequestion}{\QType\ifnum\value{question}<10 0\fi\arabic{question}}
\newcommand{\question}{\par\refstepcounter{question}\textbf{\thequestion}.\space}

\newcommand{\qsection}[2]{%
  \renewcommand{\QType}{#2}
  \section*{#1}
  \refstepcounter{section}
}
\declaretheoremstyle[
  spaceabove=6pt,spacebelow=6pt,
  headfont=\normalfont\itshape,bodyfont=\normalfont,
  qed=\qedsymbol]{proof}
\declaretheorem[sibling=question,style=definition,name=,
  refname={problem,problems},Refname={Problem,Problems}]{prob}
\declaretheorem[numberwithin=subsection,style=definition,name=Exercise,
  refname={exercise,exercises},Refname={Exercise,Exercises}]{xca}
\declaretheorem[name=Proof,style=proof,unnumbered]{prf}
\declaretheorem[name=Solution,style=proof,unnumbered]{sol}

\usepackage{tikz,pgfplots}
\pgfplotsset{compat=1.15}
