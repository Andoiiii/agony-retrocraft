\documentclass[class=co487,notes]{agony}

\title{CO 487 Winter 2024: Lecture Notes}
\begin{document}
\renewcommand{\contentsname}{CO 487 Winter 2024:\\{\huge Lecture Notes}}
\thispagestyle{firstpage}
\tableofcontents

Lecture notes taken, unless otherwise specified,
by myself during the Winter 2024 offering of CO 487,
taught by Alfred Menezes.

\begin{multicols}{2}
  \listoflecture
\end{multicols}

\chapter{Introduction}
\lecture{Jan 8}

Cryptography is securing communications in the presence of malicious adversaries.
To simplify, consider Alice and Bob communicating with the eavesdropper Eve.
Communications should be:
\begin{itemize}
  \item Confidential: Only authorized people can read it
  \item Integral: Ensured that it is unmodified
  \item Origin authenticated: Ensured that the source is in fact Alice
  \item Non-repudiated: Unable to gaslight the message existing
\end{itemize}
Examples: TLS for intenet browsing, GSM for cell phone communications,
Bluetooth for other wireless devices.

\paragraph{Overview: Transport Layer Security} The protocol used by browsers
to visit websites.
TLS assures an individual user (a \term{client})
of the authenticity of the website (a \term{server})
and to establish a secure communications \term{session}.

TLS uses \term{symmetric-key cryptography}.
Both the client and server have a shared secret $k$ called a \term{key}.
They can then use AES for encryption and HMAC for authentication.

To establish the shared secret, use \term{public-key cryptography}.
Alice can encrypt the session key $k$ can be encrypted with Bob's RSA public key.
Then, Bob can decrypt it with his private key.

To ensure Alice is getting an authentic copy of Bob's public key,
a \term{certification authority} (CA) signs it using the CA's private key.
The CA public key comes with Alice's device preinstalled.

Potential vulnerabilities when using TLS:
\begin{itemize}
  \item Weak cryptography scheme or vulnerable to quantum computing
  \item Weak random number generation for the session key
  \item Fraudulent certificates
  \item Implementation bugs
  \item Phishing attacks
  \item Transmission is secured, but the endpoints are not
\end{itemize}
These are mostly the purview of cybersecurity,
of which cryptography is a part.
Cryptography is not typically the weakest link in the cybersecurity chain.

\end{document}
