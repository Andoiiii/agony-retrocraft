\documentclass[11pt]{article}

\usepackage{physics}
\usepackage{amsfonts,amsmath,amssymb,amsthm}
\usepackage{enumerate}
\usepackage{titlesec}
\usepackage{fancyhdr}
\usepackage{multicol}

\headheight 13.6pt
\setlength{\headsep}{10pt}
\textwidth 15cm
\textheight 24.3cm
\evensidemargin 6mm
\oddsidemargin 6mm
\topmargin -1.1cm
\setlength{\parskip}{1.5ex}
\parindent=0pt

\author{James Ah Yong}

\pagestyle{fancy}
\fancyhf{}
\fancyfoot[c]{\thepage}
\makeatletter
\lhead{\@title}
\rhead{\@author}

\fancypagestyle{firstpage}{
  \fancyhf{}
  \rhead{\@author}
  \fancyfoot[c]{\thepage}
}

% Sets
\newcommand{\N}{\mathbb{N}}
\newcommand{\Z}{\mathbb{Z}}
\newcommand{\Q}{\mathbb{Q}}
\newcommand{\R}{\mathbb{R}}
\newcommand{\C}{\mathbb{C}}
\newcommand{\U}{\mathcal{U}}
\newcommand{\sym}{\mathbin{\triangle}}

% Functions
\DeclareMathOperator{\sgn}{sgn}
\DeclareMathOperator{\im}{im}

% Operators
\newcommand{\Rarr}{\Rightarrow}
\newcommand{\Larr}{\Leftarrow}
\usepackage{mathtools} % for \DeclarePairedDelimiter macro
\DeclarePairedDelimiter\ceil{\lceil}{\rceil}
\DeclarePairedDelimiter\floor{\lfloor}{\rfloor}

% Macros
% properly typeset ε-δ (epsilon en dash delta)
\newcommand{\epsdel}[1][\delta]{\ensuremath{\epsilon\mathit{\textnormal{--}}#1}}
\newcommand{\by}[1]{& \text{by #1}}
\newcommand{\IH}{\by{inductive hypothesis}}
% multiple choice (remove spacing between items)
\newenvironment{choices}
{\begin{enumerate}[(a)]
    \setlength{\parskip}{0ex}
    }{
  \end{enumerate}}

% Typesetting
\usepackage{array}   % for \newcolumntype macro
\newcolumntype{C}{>{$}c<{$}} % math version of "C" column type
\newcommand{\dlim}[2]{\displaystyle\lim_{#1\to#2}} % totally not \dfrac ripoff
\newcommand{\dilim}[1]{\dlim{#1}{\infty}} % infinite limits
\newcommand{\ilim}[1]{\lim_{#1\to\infty}}
\usepackage{cancel}

% Auto-number questions
\newcommand{\QType}{Q}
\renewcommand{\theparagraph}{\QType\ifnum\value{paragraph}<10 0\fi\arabic{paragraph}}
\setcounter{secnumdepth}{6}
\newcommand{\question}{\par\refstepcounter{paragraph}\textbf{\theparagraph}.\space}

% Question sections
\titleformat{\section}{\normalsize\bfseries}{\thesection}{1em}{}
\newcommand{\qsection}[2]{%
  \renewcommand{\QType}{#2}
  \section*{#1}
  \refstepcounter{section}
}

\title{MATH 137 Fall 2020: Practice Assignment 3}

\begin{document}
\thispagestyle{firstpage}

\textbf{\@title}

\question Assuming $\dilim{n}a_n = \dilim{n}b_n = 0$ where $a_n \geq 0$ and $b_n \geq 0$, determine
\[ \ilim{n}\left(a_n \sin(n) + b_n \cos(n)\right) \]

\begin{proof}
  First, notice that given a convergent sequence $\{k_n\}$ with limit $L$,
  by the product and constant arithmetic rules for limits of sequences,
  $\{-k_n\}$ has limit $-L$.
  Therefore, $\dilim{n} -a_n = \dilim{n} -b_n = 0$.

  Now, consider the term $a_n \sin n$.
  Because $-1 \leq \sin n \leq 0$ and $a_n \geq 0$, $-a_n \leq a_n \sin n \leq a_n$.
  By the squeeze theorem, $a_n \sin n \to 0$.

  Applying the same argument to $\{b_n\}$, we find $b_n \cos n \to 0$.

  By the sum rule, $\dilim{n}\left(a_n \sin(n) + b_n \cos(n)\right)=0+0=0$.
\end{proof}


\question Let's examine how absolute values and limits interact.
\begin{enumerate}[(a)]
  \item The statement
        \begin{center}
          If $\dilim{n}\abs{a_n}=\abs{L}$ then $\dilim{n}a_n = L$.
        \end{center}
        is false in general. Provide a counter-example.
        \begin{proof}
          Let $a_n=1$ if $n$ is even, and $a_n=-1$ otherwise.
          $|a_n|$ is constant for all $n$, so the limit is that constant, i.e.\ 1.
          However, $a_n$ clearly has no limit.
        \end{proof}

  \item The statement
        \begin{center}
          If $\dilim{n}a_n = L$ then $\dilim{n}\abs{a_n}=\abs{L}$.
        \end{center}
        is true. Show this using the definition of limits.

        Hint: $\abs{|a| - |b|} \leq \abs{a-b}$.
        (Even though it is not necessary for this question, you should be able to show that the hint is true.)
        \begin{proof}
          Let $\epsilon > 0$.
          There is then an $N$ such that $n \geq N$ implies $|a_n-L| > \epsilon$.
          We must find an $N$ so $n \geq N$ implies $\abs{|a_n| - |L|}>\epsilon$.
          Consider the same $N$ and the hint:
          \begin{equation*}
            \abs{|a_n| - |L|} \leq \abs{a_n-L} < \epsilon \qedhere
          \end{equation*}
        \end{proof}

  \item Is the statement
        \begin{center}
          If $\dilim{n}\abs{a_n}=0$ then $\dilim{n}a_n = 0$.
        \end{center}
        true? If so, argue why, if not, provide a counterexample.
        \begin{proof}
          Let $\epsilon > 0$.
          There is then an $N$ such that $n \geq N$ implies $\abs{|a_n| - 0|} < \epsilon$.
          However, $\abs{|a_n| - 0}=\abs{|a_n|}=|a_n|=|a_n-0|$.
          This means that $n\geq N\implies|a_n-0|<\epsilon$, which is precisely what must be shown to prove that $a_n\to0$.
        \end{proof}
\end{enumerate}


\question Compute the following limits using any method.
\begin{enumerate}[(a)]
  \item $\dilim{n} \frac{\sin(n^2)}{n^2}$
        \begin{proof}
          First, consider the limit of $\frac{1}{n^2}$.
          This is trivially zero (let $N = \epsilon^{-1/2}$).
          By the constant multiple rule, $-\frac{1}{n^2}\to0$ as well.

          Now, recall that $-1 \leq \sin n^2 \leq 1$ for all $n$.
          Since $\frac{1}{n^2}\geq 0$, we can also say that $-\frac{1}{n^2} \leq \frac{\sin n^2}{n^2} \leq \frac{1}{n^2}$ for all $n$.
          By the squeeze theorem, the limit is 0.
        \end{proof}

  \item $\dilim{n} \frac{3n - (-1)^n}{n}$
        \begin{proof}
          Let $a_n = \frac{3n - (-1)^n}{n}$.

          When $n$ is odd, $(-1)^n=-1$, so $a_n = \frac{3n+1}{n}$.
          Otherwise, $(-1)^n=1$, so $a_n = \frac{3n-1}{n}$.

          The limits of both of these are 3, since $\frac{3n\pm1}{n}=3\pm\frac{1}{n}$, and $\frac{1}{n}$ converges to zero.

          Notice that $\frac{3n+1}{n} > \frac{3n-1}{n}$ for all $n$.
          Therefore, $\frac{3n+1}{n} \geq a_n \geq \frac{3n-1}{n}$ for all $n$.
          By the squeeze theorem, the limit is 3.
        \end{proof}

  \item $\dilim{n} \frac{n!}{n^n}$

        Hint: Write $n!=1\cdot 2\cdot 3\dots n$ and $n^n=n\cdot n\cdot n\dots n$ and use the fact that $0 < \frac{n!}{n^n}$

        \begin{proof}
          Let $a_n=\frac{n!}{n^n}$.
          Expand using the definitions of the factorial and exponentiation:
          \begin{equation*}
            a_n = \frac{n!}{n^n}
            = \frac{1\cdot 2 \cdot 3\dots n}{n\cdot n\cdot n\dots n}
            = \frac{1}{n} \cdot \frac{2}{n} \cdots \frac{n-1}{n} \cdot \cancel{\frac{n}{n}}
            = \frac{1}{n} \cdot \frac{2}{n} \cdots \frac{n-1}{n}
            = \prod^{n-1}_{k=1} \frac{k}{n}
          \end{equation*}
          For each term in the product, since $k$ is constant with respect to $n$, $\frac{k}{n} \to k\frac{1}{n} \to 0$.
          By the product rule for limits, $a_n \to \prod 0 = 0$.
        \end{proof}

  \item $\dilim{n} \frac{3n^3 + 2n^2 - n - 1}{n^3 + n + 3}$
        \begin{proof}
          Divide through by $n^3$ and cancel all trivially zero terms:
          \begin{align*}
            \ilim{n} \frac{3n^3 + 2n^2 - n - 1}{n^3 + n + 3}
             & = \ilim{n} \frac{\frac{3n^3}{n^3} + \frac{2n^2}{n^3} - \frac{n}{n^3} - \frac{1}{n^3}}{\frac{n^3}{n^3} + \frac{n}{n^3} + \frac{3}{n^3}} \\
             & = \ilim{n} \frac{3 + \frac{2}{n} - \frac{1}{n^2} - \frac{1}{n^3}}{1 + \frac{1}{n^2} + \frac{3}{n^3}}                                   \\
             & = \ilim{n} \frac{3 + 0 - 0 - 0}{1 + 0 + 0}                                                                                             \\
             & = 3 \qedhere
          \end{align*}
        \end{proof}

  \item $\dilim{n} \frac{n^2 - 2n - 6}{n + 1}$
        \begin{proof}
          Let $a_n = \frac{n^2 - 2n - 6}{n + 1}$.
          Perform polynomial division to find $a_n = n-3 - \frac{3}{1 + n}$.
          Notice that $\frac{3}{1+n} \leq 3$, so $a_n \geq n-6$.
          $n-6$ obviously diverges, so $a_n$ diverges to $\infty$.
        \end{proof}
\end{enumerate}


\question Define a sequence $\{a_n\}$ by $a_1=1$ and $a_{n+1}=\dfrac{7+a_n}{6}$ for $n \geq 1$.
\begin{enumerate}[(a)]
  \item By induction, show that $\{a_n\}$ is an increasing sequence that is bounded above by 2.
        \begin{proof}
          Let $n = 1$. $a_n = 1$ and $a_{n+1} = \frac{7+1}{6} = \frac{4}{3}$.
          Therefore, $a_1 < a_2 < 2$.

          Suppose that $a_n < a_{n+1} < 2$ for some $n$. Then,
          \begin{alignat*}{2}
            a_n             & < a_{n+1}             & < 2           \\
            7+a_n           & < 7+a_{n+1}           & < 9           \\
            \frac{7+a_n}{6} & < \frac{7+a_{n+1}}{6} & < \frac{3}{2} \\
            a_{n+1}         & < a_{n+2}             & < 2
          \end{alignat*}

          By induction, $a_n < a_{n+1} < 2$ for all $n$.
          Therefore, $\{a_n\}$ is increasing and bounded above by 2.
        \end{proof}

  \item Prove that this sequence is convergent and find $\dilim{n}a_n$.
        \begin{proof}
          By the monotone convergence theorem, because $\{a_n\}$ is non-decreasing and bounded above, it must converge.
          We can therefore let $L = \dilim{n} a_n = \dilim{n} a_{n+1}$.
          \begin{align*}
            L  & = \ilim{n} \frac{7+a_n}{6}  \\
               & = \frac{7+\dilim{n} a_n}{6} \\
               & = \frac{7+L}{6}             \\
            6L & = 7+L                       \\
            L  & = \frac{7}{5} \qedhere
          \end{align*}
        \end{proof}
\end{enumerate}


\question Define a sequence $\{a_n\}$ by $a_1=\sqrt{2}$ and $a_{n+1}=\sqrt{2+a_n}$ for $n \geq 1$.
\begin{enumerate}[(a)]
  \item By induction, show that $\{a_n\}$ is an increasing sequence that is bounded above by 3.
        \begin{proof}
          Let $n=1$, so $a_n=\sqrt{2}$ and $a_{n+1}=\sqrt{2+\sqrt{2}} \approx 1.85$.
          As required, $a_n < a_{n+1} < 3$.

          Suppose $a_n < a_{n+1} < 3$ for some $n$. Then,
          \begin{alignat*}{2}
            a_n            & < a_{n+1}            & < 3        \\
            2 + a_n        & < 2 + a_{n+1}        & < 5        \\
            \sqrt{2 + a_n} & < \sqrt{2 + a_{n+1}} & < \sqrt{5} \\
          \end{alignat*}
          By induction, $a_n < a_{n+1} < 3$ for all $n$.
          Therefore, $\{a_n\}$ is increasing and bounded above by 3.
        \end{proof}

  \item Prove that this sequence is convergent and find $\dilim{n}a_n$.
        \begin{proof}
          By the monotone convergence theorem, because $\{a_n\}$ is bounded above and non-decreasing, it converges to a limit $L$.
          Recall that if $a_n \to L$, then $a_{n+1} \to L$:
          \begin{align*}
            L   & = \ilim{n} \sqrt{2+a_n}   \\
                & = \sqrt{2+ \dilim{n} a_n} \\
                & = \sqrt{2+L}              \\
            L^2 & = 2 + L                   \\
            0   & = L^2 - L - 2             \\
            0   & = (L-2)(L+1)
          \end{align*}
          So $L$ is either $-1$ or $2$.
          Notice the recursive definition of $a_n$ uses a square root, so $a_n \geq 0$.
          Therefore, $L$ must be 2.
        \end{proof}
\end{enumerate}

\end{document}