\documentclass[class=cs245,nogeometry]{agony}
\usepackage{multirow}
\usepackage{geometry}
\geometry{margin=1cm,landscape}
\pagenumbering{gobble}

\newcommand{\law}[2]{\text{#1}&\left\{\begin{aligned}#2\end{aligned}\right.\\}
\newcommand{\dfn}[3]{\textit{#1} (\textsection #2, #3).}
\newcommand{\thm}[3][Theorem]{\textbf{#1} (\textsection #2, #3).}
\newcommand{\lem}[2]{\textit{Lemma} (\textsection #1, #2).}

% Fancy list shorthand for using inside tabular
\makeatletter
\NewDocumentCommand\ls{sO{\\}m}{\ensuremath{%
  \global\def\rc@delim{#2}%
    \!\IfBooleanTF#1{}{\inalign@true}\begin{aligned}
      \rc@vector #3;\relax\noexpand\@eolst%
    \end{aligned}\IfBooleanTF#1{}{\inalign@true}}}
\def\rc@vector #1;#2\@eolst{%
  \ifx\relax#2\relax
    #1
  \else
    #1\rc@delim[-1.7\jot]
    \rc@vector #2\@eolst%
  \fi}
\makeatother
\newcommand{\br}{&}
\newcommand{\D}{&\d}
\newcommand{\DD}{&\dd}

\title{Cheat Sheet}
\begin{document}

\begin{multicols}{2}
  \section*{Essential Laws of Propositional Logic}
  \begin{align*}
    \law{Double Negation}{\n(\n p) \tt p}
    \law{Excluded Middle}{p \o \n p \tt 1}
    \law{Contradiction}{p \a \n p \tt 0}
    \law{Idempotence}{
    p \a p                  & \tt p                    \\
    p \o p                  & \tt p                    \\
    }
    \law{Identity}{
    p \a 1                  & \tt p                    \\
    p \o 0                  & \tt p                    \\
    }
    \law{Domination}{
    p \a 0                  & \tt 0                    \\
    p \o 1                  & \tt 1                    \\
    }
    \law{Commutativity}{
    p \a q                  & \tt q \a p               \\
    p \o q                  & \tt q \o p               \\
    p \e q                  & \tt q \e p               \\
    }
    \law{Associativity}{
    p \a (q \a r)           & \tt (p \a q) \a r        \\
    p \o (q \o r)           & \tt (p \o q) \o r        \\
    }
    \law{Distributivity}{
    p \a (q \o r)           & \tt (p \a q) \o (q \a r) \\
    p \o (q \a r)           & \tt (p \o q) \a (q \o r) \\
    }
    \law{Implication}{p \i q \tt \n p \o q}
    \law{Contrapositive}{p \i q \tt \n q \i \n p}
    \law{Equivalence}{p \e q \tt (p \i q) \a (q \i p)}
    \law{De Morgan}{
    \n(p \a q)              & \tt \n p \o \n q         \\
    \n(p \o q)              & \tt \n p \a \n q         \\
    }
    \law{Absorption I}{
    p \a (p \o q)           & \tt p                    \\
    p \o (p \a q)           & \tt p                    \\
    }
    \law{Absorption II}{
    (p \o q) \a (\n p \o q) & \tt q                    \\
    (p \a q) \o (\n p \a q) & \tt q                    \\
    }
  \end{align*}
  \section*{Eleven Rules of Formal Deduction (\texttt{+} Six of First Order Logic)}
  \begin{center}
    \begin{deduceinternal}
      \begin{tabular}{r|C|C|l}
        (Abbr.)                     & \text{From}                                              &
        \text{Conclude}             & Rule                                                       \\ \hline
        (Ref)                       & \0                                                       &
        A |- A                      & Reflexivity                                                \\
        (+)                         & \S |- A                                                  &
        \S,\S' |- A                 & Addition of premises                                       \\
        (! -)                       & \ls{\S,!A |- B; \S,!A |- !B}                             &
        \S |- A                     & ! elimination                                              \\
        (-> -)                      & \ls{\S |- A -> B; \S |- A}                               &
        \S |- B                     & \ls*{\br\text{-> elimination}; \br\text{(modus ponens)}}   \\
        (-> +)                      & \S, A |- B                                               &
        \S |- A \i B                & -> introduction                                            \\
        (and -)                     & \S |- A and B                                            &
        \ls{\S |- A;\S |- B}        & and elimination                                            \\
        (and +)                     & \ls{\S |- A; \S |- B}                                    &
        \S |- A and B               & and introduction                                           \\
        (or -)                      & \ls{\S,A |- C; \S,B |- C}                                &
        \S, A or B |- C             & or elimination                                             \\
        (or +)                      & \S |- A                                                  &
        \ls{\S|- A\o B; \S|- B\o A} & or introduction                                            \\
        \multirow{2}{*}{(<-> -)}    & \ls{\S |- A <-> B; \S |- A}                              &
        \S |- B                     & \multirow{2}{*}{<-> elimination}                           \\
                                    & \ls{\S |- A <-> B; \S |- B}                              &
        \S |- A                     &                                                            \\
        (<-> +)                     & \ls{\S,A |- B; \S,B |- A}                                &
        \S |- A <-> B               & <-> introduction                                           \\ \hline
        (\A -)                      & \S |- \A[x] A(x)                                         &
        \S |- A(t)                  & \A elimination                                             \\
        (\A +)                      & \ls*{\S |- A(u); \text{$u$ not in \S}}                   &
        \S |- \A[x] A(x)            & \A introduction                                            \\
        (\X -)                      & \ls*{\S, A(u) |- B; \text{$u$ not in $\S,B$}}            &
        \S, \X[x] A(x) |- B         & \X elimination                                             \\
        (\X +)                      & \S |- A(u)                                               &
        \S |- \X[x] A(x)            & \X introduction                                            \\
        (~ -)                       & \ls{\S |- A(t_1); \S |- t_1 = t_2}                       &
        \S |- A(t_2)                & ~ elimination                                              \\
        (~ +)                       & \0                                                       &
        \0 |- u = u                 & ~ introduction
      \end{tabular}
    \end{deduceinternal}
  \end{center}
  \underline{N.B.:} In (\A +) and (\X +), $A(x)$ is $A(u)$ with some but not
  necessarily all occurrences of $u$ replaced by $x$.
\end{multicols}
\pagebreak
\begin{multicols}{2}
  \section*{Proved Theorems}
  \begin{center}
    \begin{deduceinternal}
      \begin{tabular}{r|C|C|l}
        (Abbr.)               & \text{From}                                   &
        \text{Conclude}       & Theorem                                         \\ \hline
        (\E)                  & A \in \S                                      &
        \S |- A               & Membership                                      \\
        (Tr.)                 & \ls{\S |- \S'; \S' |- A}                      &
        \S |- A               & Transitivity                                    \\
        (! +)                 & \ls{\S,A |- B; \S,A |- !B}                    &
        \S |- !A              & Reductio ad absurdum                            \\
        (Repl.)               & \ls{A \DD A'; C\br=B[A'/A]}                   &
        B |-| C               & Replaceability                                  \\
                              & A |- B                                        &
        !B |- !A              & Flip-Flop                                       \\
        $(~ -)'$              & \ls{\S |- A(t_1); \S |- t_2 = t_1}            &
        \S |- A'(t_2)         & Partial substitution                            \\
        EQSubs                & \S |- t_1 = t_2                               &
        \S |- r(t_1) = r(t_2) &                                                 \\
        EQTrans               & \ls{\S |- t_i = t_{i+1}; i \br= 1, \dotsc, k} &
        \S |- t_1 = t_k       &
      \end{tabular}
    \end{deduceinternal}
  \end{center}
  \begin{deduce*}
    !!A                       |-| A                                                  & \tag{Double Negation}        \\
    \0                        |-  A or !A                                            & \tag{Excluded Middle}        \\
    \0                        |-  !(A and !A)                                        & \tag{Non-Contradiction}      \\
    A, !A                     |-  B                                                  & \tag{Inconsistency Rule}     \\
    A -> B, B -> C            |-  A -> C                                             & \tag{Hypothetical Syllogism} \\
    A or B, !B                |-  A                                                  & \tag{Disjunctive Syllogism}  \\
    A -> B                    |-| !A or B                                            & \tag{Implication}            \\
    A -> B                    |-| !B -> !A                                           & \tag{Contrapositive}         \\
    \ls{!(A and B);!(A or B)} &   \;\ls{\dequiv !A or !B;\dequiv !A and !B}          & \tag{De Morgan}              \\
    \0                        |-  \A[x](x=x)                                         & \tag{Reflexivity}            \\
    \0                        |-  \A[x]\A[y]((x=y) -> (y=x))                         & \tag{Symmetry}               \\
    \0                        |-  \A[x]\A[y]\A[z](((x=y) and (y=z)) -> (x=z))        & \tag{Transitivity}           \\
    \0                        |-  \A[x](s(x) \neq x)                                 & \tag{Not own succ.}          \\
    \0                        |-  \A[x](x = 0 or \X[y](s(y) = x))                    & \tag{Exist. of pred.}        \\
    \0                        |-  \A[x]\A[y]\A[z](x \leq y and y \leq z -> x \leq z) & \tag{Trans. of $\leq$}       \\
    \0                        |-  \A[x](s(0) \cdot x = x)                            & \tag{Left Mult. Identity}    \\
    \0                        |-  \A[x]\A[y]\A[z]((x+y)+z = x+(y+z))                 & \tag{Assoc. Add.}
  \end{deduce*}

  \columnbreak
  \section*{Axioms of Peano Arithmetic}
  \begin{deduceinternal}
    \begin{enumerate}[{PA}1.,nosep]
      \item $\A[x] !(s(x) = 0)$
      \item $\A[x]\A[y] ((s(x) = s(y)) -> (x = y))$
      \item $\A[x] (x+0 = x)$
      \item $\A[x]\A[y] (x+s(y) = s(x+y))$
      \item $\A[x] (x \cdot 0 = 0)$
      \item $\A[x]\A[y] (x \cdot s(y) = x \cdot y + x)$
      \item $(A(0) and \A[x](A(x) -> A(s(x)))) -> \A[x] A(x)$ (induction)
    \end{enumerate}
  \end{deduceinternal}

  \section*{Inference Rules for Program Verification}
  \begin{center}
    \AXC{}
    \RL{(assignment)}
    \UIC{\Hoare{Q[E/x]}\texttt{x = E;}\Hoare{Q}}
    \DP

    \AXC{\Hoare{P}$C_1$\Hoare{Q}}
    \AXC{\Hoare{Q}$C_2$\Hoare{R}}
    \RL{(composition)}
    \BIC{\Hoare{P}$C_1$; $C_2$\Hoare{R}}
    \DP

    \AXC{\logic{P -> P'}}
    \AXC{\Hoare{P'}$C$\Hoare{Q}}
    \RL{(implied)}
    \BIC{\Hoare{P}$C$\Hoare{Q}}
    \DP

    \AXC{\Hoare{P}$C$\Hoare{Q'}}
    \AXC{\logic{Q' -> Q}}
    \RL{(implied)}
    \BIC{\Hoare{P}$C$\Hoare{Q}}
    \DP

    \AXC{\Hoare{P and B}$C_1$\Hoare{Q}}
    \AXC{\Hoare{P and !B}$C_2$\Hoare{Q}}
    \RL{(if-then-else)}
    \BIC{\Hoare{P}\texttt{if (B) $C_1$ else $C_2$}\Hoare{Q}}
    \DP

    \AXC{\Hoare{P and B}$C$\Hoare{Q}}
    \AXC{\logic{P and !B -> Q}}
    \RL{(if-then)}
    \BIC{\Hoare{P}\texttt{if (B) $C$}\Hoare{Q}}
    \DP

    \AXC{\Hoare{I and B}$C$\Hoare{I}}
    \RL{(partial-while)}
    \UIC{\Hoare{I}\texttt{while (B) $C$}\Hoare{I and !B}}
    \DP
  \end{center}
\end{multicols}
\pagebreak
\begin{multicols}{2}
  \section*{Definitions}

  \dfn{Alphabet}{16a}{13} Finite set of symbols $\S$.
  $\S^*$ contains all possible strings, including empty string $\lambda$.
  Subset of $\S^*$ is language.

  \dfn{Propositional language}{02}{3}
  $\Lp$ has $\S = \{p,q,r,\ldots,\n,\a,\o,\i,\e,(,)\}$. \\
  \dfn{First-order language}{11}{4}
  $\L$ extends above with quantifier ($\A,\X$),
  free ($u,v,w$) and bound variable ($x,y,z$), individual ($a,b,c,\dotsc$),
  relation ($F,G,H$), and function $(f,g,h)$ symbols.

  \dfn{Expression}{02}{4} Element of language, including empty expression $\epsilon$.

  \dfn{Segment}{02}{5} $V$ segment of $U$ if $U = W_1VW_2$.
  Initial if $W_1 = \epsilon$, final if $W_2 = \epsilon$.
  Proper if $V \neq U$.

  \dfn{Term}{11}{6} $\Term(\L)$ is smallest set that contains
  all individual symbols, free variable symbols, and functions of terms.

  \dfn{Atom}{02}{6} $\Atom(\Lp)$ has expressions that are one proposition symbol. \\
  \dfn{Atom}{11}{8} $\Atom(\L)$ has $F(t_1,t_2,\dotsc,t_n)$ and $t_1 \eq t_2$ for $t_i \in \Term(\L)$.

  \dfn{Formula}{02}{6}
  $\Form(\Lp)$ defined by: $\Atom(\Lp) \subset \Form(\Lp)$. \\
  $A,B\in\Form(\Lp) \Rarr (\n A), (A \a B), (A \o B), (A \i B), (A \e B) \in \Form(\Lp)$. \\
  No other expressions are in $\Form(\Lp)$. \\
  \dfn{Formula}{11}{9} $\Form(\L)$ has same formation rules but also includes
  $A(u)\in\Form(\L), x\not\in A(u) \Rarr \A[x]A(x), \X[x]A(x) \in \Form(\L)$

  \dfn{Sentence}{11}{17} $\Sent(\L) \subset \Form(\L)$ with no free variables.

  \dfn{Scope}{02}{45}
  In $(\n A)$, $A$ is the scope of $\n$.
  In $(A \star B)$, $A$ is the left scope and $B$ is the right scope of $\star$.

  \dfn{Truth valuation}{03}{6} A function $t : \Atom(\Lp)\i\{0,1\}$. \\
  \dfn{Valuation}{12}{7} A domain $D$ and function $v$ such that \\
  $a^v, u^v \in D$ for all individual symbols $a$ and free variables $u$ \\
  $w^{v(u/d)} = d$ if $w = u$ and $w^v$ otherwise for free variables $u$, $w$, $d$ \\
  $F^v \subseteq D^n$ for all $n$-ary relations $F$ with $\eq^v = \{(x,x),x\in D\}$ \\
  $f^v : D^n \i D$ for all $n$-ary functions $f$

  \dfn{Satisfiablity}{03}{9}
  $t$ satisfies $A$ if $A^t = 1$.
  Set satisfied if members satisfied.

  \dfn{Tautology/Universally valid}{03, 14}{\textsection12, 23} For all $t$, $A^t = 1$. \\
  \dfn{Contradiction/Unsatisfiability}{03}{14} For all $t$, $A^t = 0$. \\
  \dfn{Contingent}{03}{14} $A$ is neither a tautology nor contradiction.

  \dfn{Tautological consequence}{03}{21}
  $\S \t A$ if for all $t$, $\S^t = 1$ gives $A^t = 1$. \\
  \dfn{Tautological equivalence}{03}{25}
  $A \tt B$ if $A \t B$ and $B \t A$.

  \dfn{Literal}{04}{10}
  A formula of the form $p$ or $\n p$.

  \dfn{Disjunctive clause}{04}{12}
  Disjunction with literal disjuncts. \\
  \dfn{Conjunctive clause}{04}{12}
  Conjunction with literal conjuncts.

  \dfn{DNF}{04}{13} Disjunction with conjunctive clause disjuncts. \\
  \dfn{CNF}{04}{13} Conjunction with disjunctive clause conjuncts.

  \dfn{Definability/Reducibility}{05}{2}
  Connective $\star$ reducible to set $\mathcal S$ if $A \star B \tt C$
  where $C$ uses only $A$, $B$, and connectives in $\mathcal S$.

  \dfn{Adequate}{05}{7}
  Connectives which express any truth table/connective.

  \dfn{Formal deducibility}{06}{15}
  $\S \d A$ generated by finite deduction rules. \\
  \dfn{Syntactic equivalence}{06}{15}
  $A \dd B$ if $A \d B$ and $B \d A$.

  \dfn{Consistency}{06}{67}
  There is no $F$ such that $\S\d F$ and $\S\d\n F$.

  \dfn{Resolution}{07}{8}
  $\logic{C or p, D or !p |-_r C or D}$ if $C$ and $D$ disjunctive clauses.
  Resolve parent clauses over $p$ to resolvent $C \o D$.
  Commutativity and idempotence allowed.
  Note $\logic{p, !p |-_r \{\}}$ (empty clause, representing contradiction).

  \dfn{Set-of-Support Strategy}{07}{19}
  Let $\S' = \n C$. Resolve all premises against $\S'$ and add resolvents to $\S'$.
  Repeat until \{\}. \\
  \dfn{Davis-Putnam Procedure}{07}{29}
  For each $p$: discard tautolgies from \S,
  resolve all pairs over $p$, discard clauses with $p$.
  Always outputs \0 or \{\}.

  \dfn{Prenex Normal Form}{15}{4}
  All quantifiers at start.
  Prefix (quantifiers) + matrix (rest).
  If no \X{}, it is \X{}-free PNF.

  \dfn{Skolem function}{15}{14}
  Given $\A[x_i]\X[y]$, a function $f(x_i) = y$.
  Note that Skolemized sentence is not equivalent to original.

  \dfn{Unification}{15}{26}
  $A$ unifies with $B$ if exists unifiers $x_i \coloneqq t_i$
  (replacing variable $x_i$ with term $t_i$) that make $A$ equal $B$.

  \dfn{Turing machine}{16a}{13}
  Finite set of states $S$, alphabet $I$ containing blank $B$,
  start state $s_0 \in S$,
  transition function $f : S \times I \to S \times I \times \{L,R\}$ or
  transition rules $(s,x,s',x',d)$.
  State $s$ final if no rules for $s$.
  Halts if no rule for $(s,x)$, accepts if $s$ final (reject otherwise).
  TM total if always halts.

  \dfn{Decidability}{16a}{22}
  Decision problem can be solved by TM that accepts ``yes'' answers.
  Function that can be computed by TM is computable.

  \dfn{$P$ and $NP$}{16b}{42}
  Problem in $P$ if DTM can solve in polynomial time.
  In $NP$ if NDTM can solve (or DTM can verify) in polynomial time.
  Problem $NP$-complete if any $NP$ can be reduced to it in polynomial time.

  \dfn{Theory}{17}{7}
  A set of premises (axioms) $A_T$ and a formal deduction system $\d$.
  Axioms are decidable, consistent, syntactically complete
  ($F$ xor $\n F$ provable).
  Denote $\S \d_T F$ to mean $\S \cup A_T \d F$.

  \dfn{Hoare triple}{18}{17}
  Program $C$ in state satisfying $P$ ends in state satisfying $Q$.
  A specification is the triple \Hoare{P}$C$\Hoare{Q}.

  \dfn{Partial correctness}{18}{22}
  $\t_{par} \Hoare{P}C\Hoare{Q}$
  means if start state satisfies $P$ and $C$ halts,
  then end state satisfies $Q$.
  This problem is undecidable. \\
  \dfn{Total correctness}{18}{24}
  $\t_{tot} \Hoare{P}C\Hoare{Q}$
  means if start state satisfies $P$,
  then $C$ halts and end state satisfies $Q$.
  This is also undecidable.

  \section*{Theorems (\ldots{}the general kind)}

  \lem{02}{29} Every formula has equal number of left/right parentheses.

  \thm[Unique Readability Theorem]{02}{32}
  Every formula in $\Form(\Lp)$ is exactly one form of exactly one of
  $(\n A)$, $(A \a B)$, $(A \o B)$, $(A \i B)$, $(A \e B)$. \\
  \thm{11}{15} The same applies to $\Form(\L)$ plus $(\A[x] A(x))$, $(\X[x] A(x))$.
  \thm{11}{15} Also to $\Term(\L)$ with free/individual/function symbols.

  \lem{03}{23} Equivalent statements of $\{A_i\} \t C$: \\
  Argument with premises $A_i$ and conclusion $C$ is valid.\\
  $(\bigwedge A_i) \i C$ is a tautology;
  $(\n C \a \bigwedge A_i)$ is a contradiction. \\
  Formula $(\n C \a \bigwedge A_i)$ or set $\{\n C, A_i\}$ is not satisfiable.

  \thm[Replaceability of tautologically equivalent formulas]{03, 44}{\textsection13, 16}
  If $A \tt A'$ and $A$ is a subformula of $B$,
  then $B \tt B'$ where $B'$ is $B$ with some of the $A$s replaced by $A'$.

  \thm[Duality Theorem]{03}{44}
  If $A$ has only $\n$, $\a$, $\o$ and
  $\Delta(A)$ replaces atoms with negations and swaps $\a$ with $\o$,
  then $\n A \tt \Delta(A)$. \\
  \thm[Duality in FoL]{13}{16}
  If $A$ is as above plus \A, \X, and
  $\Delta(A)$ replaces atoms with negations, swaps \a with \o, and swaps \A with \X,
  then $\n A \tt \Delta(A)$.

  \thm{04}{22} All formulas have $F \tt \operatorname{DNF}(F)$ based on truth table. \\
  \thm{04}{24} All formulas have $F \tt \operatorname{CNF}(F) = \Delta(\operatorname{DNF}(\n F))$.

  \thm{05}{8} The set $S_0 = \{\n, \a, \o\}$ is adequate.

  \thm[Finiteness of Premise Set]{06}{31}
  If $\S \d A$, $\S^0 \d A$ with finite $\S^0 \subseteq \S$.

  \thm[Soundness Theorem]{06}{45} If $\S \d A$ then $\S \t A$. \\
  \thm[Completeness Theorem]{06}{49} If $\S \t A$ then $\S \d A$.

  \lem{06}{67} $\S$ is satisfiable if and only if $\S$ is consistent.

  \thm{07}{20} $\d_r$ is complete with the set of support strategy. \\
  \thm{07}{35} $\d_r$ is sound and complete with DPP. \\
  \thm{15}{31} $\d_r$ in FoL is sound and complete with unification.

  \thm{12}{17} All terms in $\Term(\L)$ have $t^v \in D$. \\
  \thm{12}{20} All formulas in $\Form(\L)$ have $A^v \in \{0,1\}$.

  \thm{12}{28} There is no algorithm to decide validity or satisfiability in first-order logic.

  \thm{14}{33} First order formal deduction is sound and complete.

  \thm{15}{19} All sentences $F$ in $\Sent(\L)$ have a corresponding \X{}-free PNF $F'$
  such that $F$ satisfiable iff $F'$ satisfiable.

  \thm{15}{20} All \X{}-free PNFs $F$ can construct a finite set of disjunctive clauses
  $C_F$ such that $F$ satisfiable iff $C_F$ satisfiable.

  \thm{15}{21} Argument $\S \t A$ valid iff $C_{\n A} \cup \bigcup_{F\in \S}C_F$ unsatisfiable.

  \thm{16a}{7} The halting problem is undecidable. \\
  \lem{16a}{28} Satisfiablity and validity problems for FoL undecidable. \\
  \lem{16b}{50} Satisfiablity for propositional logic $NP$-complete.

  \thm{16a}{31} If $P_1$ reduces to $P_2$ and $P_1$ undecidable, $P_2$ undecidable.

  \thm[G\"odel's Incompleteness Theorem]{17}{32} Given axiomatic theory $T$
  with decidable axioms and capacity to express Peano Arithmetic,
  there exists statement $G_T$ which cannot be proved or disproved by $T$
\end{multicols}

\end{document}
