\section{Recurrence Relations}

\begin{xca}
  For each of the sets of compositions from \Cref{xca:2-15}, do the following:
  \begin{itemize}[nosep]
    \item Derive a recurrence relation and initial conditions for the coefficients of the corresponding generating series.
    \item Calculate the coefficients for $n=0$ up to 9.
  \end{itemize}
\end{xca}
\begin{enumerate}
  \item $A(x) = \frac{x-2x^2+x^3}{1-3x+3x^2-2x^3}$
        \begin{sol}
          Apply Theorem 4.8 to read:
          \[
            a_n - 3a_{n-1} + 3a_{n-2} - 2a_{n-3} = \begin{cases}
              0  & n = 0    \\
              1  & n = 1    \\
              -2 & n = 2    \\
              1  & n = 3    \\
              0  & n \geq 4
            \end{cases}
          \]
          and calculate the initial conditions
          \begin{align*}
            a_0                          & = 0 \\
            a_1 = 1 + 3a_0               & = 1 \\
            a_2 = -2 + 3a_1 - 3a_0       & = 1 \\
            a_3 = 1 + 3a_2 - 3a_1 + 2a_0 & = 1
          \end{align*}
          as desired.
        \end{sol}
  \item $B(x) = \frac{x^2-x^3}{1-3x+3x^2-2x^3}$
        \begin{sol}
          Apply Theorem 4.8 to read:
          \[
            b_n - 3b_{n-1} + 3b_{n-2} - 2b_{n-3} = \begin{cases}
              0  & n = 0    \\
              0  & n = 1    \\
              1  & n = 2    \\
              -1 & n = 3    \\
              0  & n \geq 4
            \end{cases}
          \]
          and calculate the initial conditions
          \begin{align*}
            b_0                           & = 0 \\
            b_1 = 0 + 3b_0                & = 0 \\
            b_2 = 1 + 3b_1 - 3b_0         & = 1 \\
            b_3 = -1 + 3b_2 - 3b_1 + 2b_0 & = 2
          \end{align*}
          as desired.
        \end{sol}
  \item $C(x) = \frac{1}{1 - x^2 - 2x^3 - 3x^4 - 2x^5 - x^6}$
        \begin{sol}
          Apply Theorem 4.8 to read:
          \[
            c_n - c_{n-2} - 2c_{n-3} - 3c_{n-4} - 2c_{n-5} - c_{n-6} = \begin{cases}
              1 & n = 0    \\
              0 & n \geq 1
            \end{cases}
          \]
          and calculate the initial conditions
          \begin{align*}
            c_0                                  & = 1  \\
            c_1                                  & = 0  \\
            c_2 = c_0                            & = 1  \\
            c_3 = c_1 + 2c_0                     & = 2  \\
            c_4 = c_2 + 2c_1 + 3c_0              & = 4  \\
            c_5 = c_3 + 2c_2 + 3c_1 + 2c_0       & = 6  \\
            c_6 = c_4 + 2c_3 + 3c_2 + 2c_1 + c_0 & = 12
          \end{align*}
          as desired.
        \end{sol}
  \item $D(x) = \frac{x^2 - 2x^3 + x^4}{1 - 3x + 3x^2 - x^3 - x^4 + x^5}$
        \begin{sol}
          Apply Theorem 4.8 to read:
          \[
            d_n - 3d_{n-1} + 3d_{n-2} - d_{n-3} - d_{n-4} + d_{n-5} = \begin{cases}
              0  & n = 0    \\
              0  & n = 1    \\
              1  & n = 2    \\
              -2 & n = 3    \\
              1  & n = 4    \\
              0  & n \geq 5
            \end{cases}
          \]
          and calculate the initial conditions
          \begin{align*}
            d_0                                 & = 0 \\
            d_1 = 3d_0                          & = 0 \\
            d_2 = 1 + 3d_1 - 3d_0               & = 1 \\
            d_3 = -2 + 3d_2 - 3d_1 + d_0        & = 1 \\
            d_4 = 1 + 3d_3 - 3d_2 + d_1 + d_0   & = 1 \\
            d_5 = 3d_4 - 3d_3 + d_2 + d_1 - d_0 & = 1 \\
          \end{align*}
          as desired.
        \end{sol}
  \item $E(x) = \frac{1+x-2x^2-x^3+x^4}{1-2x^2-x^3+x^4}$
        \begin{sol}
          Apply Theorem 4.8 to read:
          \[
            e_n - 2e_{n-2} - e_{n-3} + e_{n-4} = \begin{cases}
              1  & n = 0    \\
              1  & n = 1    \\
              -2 & n = 2    \\
              -2 & n = 3    \\
              1  & n = 4    \\
              0  & n \geq 5
            \end{cases}
          \]
          and calculate the initial conditions
          \begin{align*}
            e_0                        & = 1 \\
            e_1                        & = 1 \\
            e_2 = -2 + 2e_0            & = 0 \\
            e_3 = -2 + 2e_1 + e_0      & = 1 \\
            e_4 = 1 + 2e_2 + e_1 - e_0 & = 1
          \end{align*}
          as desired.
        \end{sol}
\end{enumerate}

\begin{xca}
  Let $\rl K$ be the set of compositions $\gamma = (c_i)$
  with at least one part such that the first part is odd.
  Let $K(x)$ be the generating series for $\rl K$ with respect to size.
\end{xca}
\begin{enumerate}
  \item Show that $K(x) = \frac{x}{(1+x)(1-x)}$.
        \begin{sol}
          Write $\rl K = \{1,3,5,\dotsc\}\times\{1,2,3,\dotsc\}^*$.

          The generating series for the odd parts is $\frac{x}{1-x^2}$
          and for the normal parts $\frac{x}{1-x}$.

          By the String Lemma, we get $\frac{1}{1-\frac{x}{1-x}} = \frac{1-x}{1-2x}$.

          Finally, by the Product Lemma, we have $K(x) =
            \frac{x(1-x)}{(1-x^2)(1-2x)} = \frac{x-x^2}{1-2x-x^2+2x^3}$.
        \end{sol}
  \item Use part (a) to show that among all $2^{n-1}$ compositions of size $n \geq 1$,
        the fraction of these compositions in the set $\rl K$
        is $\frac{2}{3}+\frac{1}{3}(\frac{-1}{2})^{n-1}$.
        \begin{sol}
          Use partial fractions to decompose $\frac{x-x^2}{(1-x)(1+x)(1-2x)}
            = \frac{1/3}{1-2x}-\frac{1/3}{1+x}$.

          Then, $k_n = \frac13[x^n]\frac{1}{1-2x} - \frac13[x^n]\frac{1}{1+x}
            = \frac13\cdot 2^n - \frac13 (-1)^n$
          so that $\frac{k_n}{2^{n-1}} = \frac23 + \frac13(\frac{-1}{2})^{n-1}$
          as desired.
        \end{sol}
\end{enumerate}

\begin{xca}
  Consider the power series $\sum_{n=0}^\infty c_n x^n
    = \frac{1-2x^2}{1-5x+8x^2-4x^3}
    = 1 + 5x + 15x^2 + 39x^3 + \dotsb$
\end{xca}
\begin{enumerate}
  \item Give a linear recurrence relation that
        (together with the initial conditions above)
        determines the sequence of coefficients $(c_n : n \geq 0)$ uniquely.
        \begin{sol}
          Apply Theorem 4.8:
          \[
            c_n - 5c_{n-1} + 8c_{n-2} - 4c_{n-3} = \begin{cases}
              1  & n = 0    \\
              0  & n = 1    \\
              -2 & n = 2    \\
              0  & n \geq 3
            \end{cases}
          \]
          and we are given initial conditions 1, 5, 15, 39.
        \end{sol}
  \item Derive a formula for $c_n$ as a function of $n \geq 0$.
        \begin{sol}
          Write $\frac{1-2x^2}{1-5x+8x^2-4x^3} = \frac{1-2x^2}{(1-x)(1-2x)^2}$.

          Then, partial fractions gives $\frac{-1}{1-x} + \frac{1}{1-2x} + \frac{1}{(1-2x)^2}$.

          Extracting coefficients gives $c_n = -1 + 2^n + (n+1)2^n = (n+2)2^n - 1$.
        \end{sol}
\end{enumerate}

\begin{xca}
  Consider the power series
  $A(x) = \sum_{n=0}^\infty a_nx^n = \frac{x+7x^2}{1-3x^2-2x^3}$
\end{xca}
\begin{enumerate}
  \item Write down a linear recurrence relation and enough initial conditions
        to determine the sequence $(a_n : n \in \N)$ uniquely.
        \begin{sol}
          Apply Theorem 4.8 to get $a_n - 3a_{n-2} - 2a_{n-3} = \begin{cases}
              0 & n = 0    \\
              1 & n = 1    \\
              7 & n = 2    \\
              0 & n \geq 3
            \end{cases}$
        \end{sol}
  \item Given that $1 - 3x^2 - 2x^3 = (1-2x)(1+x)^2$,
        obtain a formula for $a_n$ as a function of $n \in \N$.
        \begin{sol}
          The inverse roots are 2 (multiplicity 1) and $-1$ (multiplicity 2).
          By the Main Theorem, $a_n = A2^n + (Bn+C)(-1)^n$ for some coefficients $A$, $B$, and $C$.

          The initial conditions are $a_0 = 0$, $a_1 = 1$,
          and $a_2 = 3a_0 + 7 = 7$.

          Using the initial conditions, we have the system
          \[ \systeme[ABC]{A+C=0,2A-B-C=1,4A+2B+C=7} \]
          which solves to $A=1$, $B=2$, $C=-1$.

          Therefore, $a_n = 2^n + (2n-1)(-1)^n$ for all $n$.
        \end{sol}
\end{enumerate}

\begin{xca}
  Consider the power series $\sum_{n=0}^\infty c_nx^n = \frac{3-11x+11x^2}{1-4x+5x^2-2x^3} = 3 + x + x^3 + 6x^4 + 19x^5 + \dotsb$
\end{xca}
\begin{enumerate}
  \item Give a linear recurrence relation that
        (together with the initial conditions above)
        determines the sequence of coefficients $(c_n : n \geq 0)$ uniquely.
        \begin{sol}
          Apply Theorem 4.8: $c_n - 4c_{n-1} + 5c_{n-2} - 2c_{n-3} = \begin{cases}
              3   & n=0      \\
              -11 & n=1      \\
              11  & n=2      \\
              0   & n \geq 3
            \end{cases}$
        \end{sol}
  \item Derive a formula $c_n$ as a function of $n \geq 0$.
        \begin{sol}
          Observe that the denominator factors as $-(x-1)^2(2x-1)$.
          The inverse roots are 2 (multiplicity 1) and 1 (multiplicity 2).
          By the Main Theorem, we have $c_n = A(2)^n + (Bn+C)1^n = A(2)^n + Bn + C$.

          The initial conditions give us:
          \[ \systeme[ABC]{A + C = {c_0=3}, 2A + B + C = {c_1=1}, 4A + 2B + C = {c_2=0}} \]
          which solves to $A = 1$, $B = -3$, and $C = 2$.

          Therefore, $c_n = 2^n - 3n + 2$ for all $n \geq 0$.
        \end{sol}
\end{enumerate}

\begin{xca}
  A sequence of integers is determined by the initial conditions
  $g_0 = 1$, $g_1 = 2$, $g_2 = 3$, and $g_n = 2g_{n-1} - g_{n-2} + 2g_{n-3}$ for $n \geq 3$.
\end{xca}
\begin{enumerate}
  \item Obtain a rational function formula for the generating series
        $G(x) = \sum_{n=0}^\infty g_n x^n = 1 + 2x + 3x^2 + 6x^3 + 13x^4 + 26x^5 + 51x^6 + \dotsb$
        \begin{sol}
          The denominator will be $1 - 2x + x^2 - 2x^3$ (based on the recurrence).
          Using the initial conditions, we can find the numerator coefficients:
          \begin{align*}
            g_0 - 2g_{-1} + g_{-2} - 2g_{-3} & = g_0              & = 1 \\
            g_1 - 2g_{0} + g_{-1} - 2g_{-2}  & = g_1 - 2g_0       & = 0 \\
            g_2 - 2g_{1} + g_{0} - 2g_{-1}   & = g_2 - 2g_1 + g_0 & = 0
          \end{align*}
          and notice the numerator is just 1.

          Therefore, $G(x) = \frac{1}{1-2x+x^2-2x^3}$.
        \end{sol}
  \item Obtain a formula for the coefficient $g_n$ as a function of $n \in \N$.
        \begin{sol}
          Notice that the denominator factors as $-(2x-1)(x^2+1)$.
          The inverse roots are 2, $i^{-1} = -i$, and $(-i)^{-1} = i$ (all multiplicity 1).
          By the Main Theorem, we can write $g_n = \alpha(2)^n + \beta(-i)^n + \gamma(i)^n$
          for complex coefficients $\alpha$, $\beta$, and $\gamma$.

          The initial conditions give us:
          \[
            \systeme[\alpha\beta\gamma]{
              1\alpha + \beta + \gamma = 1,
              2\alpha - i\beta + i\gamma = 2,
              4\alpha - \beta - \gamma = 3
            }
          \]
          which solves to $\alpha = \frac45$, $\beta = \frac{1}{10} + \frac{1}{5}i$,
          and $\gamma = \bar\beta$.

          Therefore, $g_n = \frac{1}{10}\qty[8(2)^n + (1+2i)(-i)^n + (1-2i)(i)^n]$.
        \end{sol}
\end{enumerate}

\begin{xca}
  Define a sequence of numbers $(c_n : n \in \N)$ by
  $c_0 = 1$, $c_1 = 2$, $c_2 = 3$, and the recurrence relation
  $c_n = -c_{n-1} + 2c_{n-2} + 2c_{n-3}$ for all $n \geq 3$.
\end{xca}
\begin{enumerate}
  \item Obtain an algebraic formula for the rational function
        $C(x) = \sum_{n=0}^\infty c_n x^n = 1 + 2x + 3x^2 + 3x^3 + 7x^4 + 5x^5 + \dotsb$
        \begin{sol}
          Based on the recurrence, the denominator will be $1 + x - 2x^2 - 2x^3$.
          The numerator will have coefficients:
          \begin{align*}
            c_0 + c_{-1} - 2c_{-2} - 2c_{-3} & = c_0              & = 1 \\
            c_1 + c_{0} - 2c_{-1} - 2c_{-2}  & = c_1 + c_0        & = 3 \\
            c_2 + c_{1} - 2c_{0} - 2c_{-1}   & = c_2 + c_1 - 2c_0 & = 3
          \end{align*}
          That is, it will be $1 + 3x + 3x^2$.

          Therefore, $C(x) = \frac{1+3x+3x^2}{1+x-2x^2-2x^3}$.
        \end{sol}
  \item Obtain a formula for $c_n$ as a function of $n \in \N$.
        \begin{sol}
          Factor the denominator as $-(x+1)(2x^2-1) = -(x+1)(\sqrt{2}x-1)(\sqrt{2}x+1)$.
          This has inverse roots $-1$, $\sqrt{2}$ and $-\sqrt{2}$ (all multiplicity 1).
          Therefore, by the Main Theorem, $c_n = A(-1)^n + B(\sqrt{2})^n + C(-\sqrt{2})^n$.

          The initial conditions give us:
          \[
            \systeme[ABC]{A+B+C = 1, -A + {\sqrt{2}}B - {\sqrt{2}}C = 2, A + 2B + 2C = 3}
          \]
          which solves to $A = -1$, $B = 1 + \frac{\sqrt{2}}{4}$, and $C = 1 - \frac{\sqrt{2}}{4}$.

          Therefore, $c_n = (-1)^{n+1} + (1+\frac{\sqrt{2}}{4})(\sqrt{2})^n + (1-\frac{\sqrt{2}}{4})(-\sqrt{2})^n$.
        \end{sol}
\end{enumerate}

\begin{xca}\end{xca}
\begin{enumerate}
  \item Obtain a formula for the coefficients of the rational function
        $B(x) = \sum_{n=0}^\infty b_n x^n = \frac{1+3x-x^2}{1-3x^2-2x^3}$.
        \begin{sol}
          Since the denominator factors as $-(2x-1)(x+1)^2$,
          we can write as a partial fraction
          $B(x) = - \frac{A}{2x-1} - \frac{B}{(x+1)^2} - \frac{C}{(x+1)}$.
          Then, multiplying by the denominator,
          we have $1+3x-x^2 = A(x+1)^2 + B(2x-1) + C(x+1)(2x-1)$.

          When $x=-1$, we have $-3 = -3B$, i.e., $B=1$.

          When $x=\frac12$, we have $\frac94 = \frac{9}{4}A$, i.e., $A=1$.

          When $x=0$, we have $1 = A - B - C$, i.e., $C = -1$.

          Therefore, we can write
          \begin{align*}
            B(x) & = \frac{1}{2x-1} + \frac{1}{(x+1)^2} - \frac{1}{x+1}                                          \\
                 & = \sum_{n=0}^\infty (2x)^n + \sum_{n=0}^\infty\binom{n+1}{1}(-x)^n - \sum_{n=0}^\infty (-x)^n \\
                 & = \sum_{n=0}^\infty (2^n - (n+1)(-1)^n + (-1)^n)x^n                                           \\
                 & = \sum_{n=0}^\infty (2^n - n(-1)^n)x^n
          \end{align*}
          and conclude that $b_n = 2^n - n(-1)^n$.
        \end{sol}
  \item Derive a recurrence relation and use it to check your answer.
        \begin{sol}
          Read off the recurrence relation $b_n - 3b_{n-2} - 2b_{n-3} = \begin{cases}
              1  & n = 0 \\
              3  & n = 1 \\
              -1 & n = 2
            \end{cases}$

          We get initial conditions $b_0 = 1$, $b_1 = 3$, $b_2 = -1 + 3(1) = 2$.

          Generating some coefficients by both methods yields
          1, 3, 2, 11, 12, 37, 58, 135.
        \end{sol}
\end{enumerate}

\begin{xca}
  Define a sequence $(h_n : n \in \N)$ by
  $h_0 = 1$, $h_1 = 2$, $h_2 = 0$, $h_3 = 5$,
  and the recurrence relation $h_n = -2h_{n-1} + h_{n-2} + 4h_{n-3} + 2h_{n-4}$
  for all $n \geq 4$.
\end{xca}
\begin{enumerate}
  \item Obtain an algebraic formula for the rational function
        $H(x) = \sum_{n=0}^\infty h_n x^n = 1 + 2x + 0x^2 + 5x^3 + 0x^4 + 9x^5 + \dotsb$
        \begin{sol}
          Reading the recurrence relation, by the Main Theorem,
          the denominator will be $1 + 2x - x^2 - 4x^3 - 2x^4$.
          The coefficients of the numerator will be:
          \begin{align*}
            h_0 + 2h_{-1} - h_{-2} - 4h_{-3} - 2h_{-4} & = 1  \\
            h_1 + 2h_{0} - h_{-1} - 4h_{-2} - 2h_{-3}  & = 4  \\
            h_2 + 2h_{1} - h_{0} - 4h_{-1} - 2h_{-2}   & = 3  \\
            h_3 + 2h_{2} - h_{1} - 4h_{0} - 2h_{-1}    & = -1
          \end{align*}
          so we have $H(x) = \frac{1 + 4x + 3x^2 - x^3}{1 + x - x^2 - 4x^3 - 2x^4}$.
        \end{sol}
  \item Obtain a formula for $h_n$ as a function of $n \in \N$.
        \begin{sol}
          The denominator factors as $(1+x)^2(1-2x^2)$.
          This has inverse roots $-1$ (multiplicity 2)
          and $\pm\sqrt{2}$ (multiplicity 1).
          Therefore, by the Main Theorem,
          $h_n = (An+B)(-1)^n + C(\sqrt2)^n + D(-\sqrt2)^n$.

          Construct the linear system for $n=0,1,2,3$:
          \[
            \systeme[ABCD]{
            B+C+D = 1,
            -A-B+{\sqrt2}C-{\sqrt2}D = 2,
            2A + B + 2C + 2D = 0,
            -3A - B + {2\sqrt2}C - {2\sqrt2}D = 5
            }
          \]
          and solve with a computer to get
          $A = -1$, $B = 0$, $C = \frac12 + \frac{\sqrt{2}}{4}$,
          $D = \frac12 - \frac{\sqrt{2}}{4}$.

          Therefore, $h_n = (-1)^{n+1} + (\frac12 + \frac{\sqrt{2}}{4})(\sqrt{2})^n
            + (\frac12 - \frac{\sqrt{2}}{4})(-\sqrt{2})^n$.
        \end{sol}
\end{enumerate}
