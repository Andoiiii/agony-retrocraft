\documentclass[11pt]{article}

\usepackage{physics}
\usepackage{amsfonts,amsmath,amssymb,amsthm}
\usepackage{enumerate}
\usepackage{titlesec}
\usepackage{fancyhdr}
\usepackage{multicol}

\headheight 13.6pt
\setlength{\headsep}{10pt}
\textwidth 15cm
\textheight 24.3cm
\evensidemargin 6mm
\oddsidemargin 6mm
\topmargin -1.1cm
\setlength{\parskip}{1.5ex}
\parindent=0pt

\author{James Ah Yong}

\pagestyle{fancy}
\fancyhf{}
\fancyfoot[c]{\thepage}
\makeatletter
\lhead{\@title}
\rhead{\@author}

\fancypagestyle{firstpage}{
  \fancyhf{}
  \rhead{\@author}
  \fancyfoot[c]{\thepage}
}

% Sets
\newcommand{\N}{\mathbb{N}}
\newcommand{\Z}{\mathbb{Z}}
\newcommand{\Q}{\mathbb{Q}}
\newcommand{\R}{\mathbb{R}}
\newcommand{\C}{\mathbb{C}}
\newcommand{\U}{\mathcal{U}}
\newcommand{\sym}{\mathbin{\triangle}}

% Functions
\DeclareMathOperator{\sgn}{sgn}
\DeclareMathOperator{\im}{im}

% Operators
\newcommand{\Rarr}{\Rightarrow}
\newcommand{\Larr}{\Leftarrow}
\usepackage{mathtools} % for \DeclarePairedDelimiter macro
\DeclarePairedDelimiter\ceil{\lceil}{\rceil}
\DeclarePairedDelimiter\floor{\lfloor}{\rfloor}

% Macros
% properly typeset ε-δ (epsilon en dash delta)
\newcommand{\epsdel}[1][\delta]{\ensuremath{\epsilon\mathit{\textnormal{--}}#1}}
\newcommand{\by}[1]{& \text{by #1}}
\newcommand{\IH}{\by{inductive hypothesis}}
% multiple choice (remove spacing between items)
\newenvironment{choices}
{\begin{enumerate}[(a)]
    \setlength{\parskip}{0ex}
    }{
  \end{enumerate}}

% Typesetting
\usepackage{array}   % for \newcolumntype macro
\newcolumntype{C}{>{$}c<{$}} % math version of "C" column type
\newcommand{\dlim}[2]{\displaystyle\lim_{#1\to#2}} % totally not \dfrac ripoff
\newcommand{\dilim}[1]{\dlim{#1}{\infty}} % infinite limits
\newcommand{\ilim}[1]{\lim_{#1\to\infty}}
\usepackage{cancel}

% Auto-number questions
\newcommand{\QType}{Q}
\renewcommand{\theparagraph}{\QType\ifnum\value{paragraph}<10 0\fi\arabic{paragraph}}
\setcounter{secnumdepth}{6}
\newcommand{\question}{\par\refstepcounter{paragraph}\textbf{\theparagraph}.\space}

% Question sections
\titleformat{\section}{\normalsize\bfseries}{\thesection}{1em}{}
\newcommand{\qsection}[2]{%
  \renewcommand{\QType}{#2}
  \section*{#1}
  \refstepcounter{section}
}

\title{MATH 137 Fall 2020: Practice Assignment 6}

\begin{document}
\thispagestyle{firstpage}

\textbf{\@title}

\question For $f(x)=\dfrac{x+1}{x-1}$, find $f'(x)$ using the limit definition.
\begin{proof}[Solution]
  Apply the Newton quotient:
  \begin{align*}
    f'(x) & = \dlim{x}{a} \frac{f(x)-f(a)}{x-a}                           \\
          & = \dlim{x}{a} \frac{\frac{x+1}{x-1} - \frac{a+1}{a-1}}{x-a}   \\
          & = \dlim{x}{a} \frac{(x+1)(a-1) - (a+1)(x-1)}{(x-a)(x-1)(a-1)} \\
          & = \dlim{x}{a} \frac{(xa+a-x-1) - (xa-a+x-1)}{(x-a)(x-1)(a-1)} \\
          & = \dlim{x}{a} \frac{-2(x-a)}{(x-a)(x-1)(a-1)}                 \\
          & = \dlim{x}{a} \frac{-2}{(x-1)(a-1)}                           \\
          & = -\frac{2}{(x-1)^2} \qedhere
  \end{align*}
\end{proof}


\question Let $f(x)=\dfrac{ax+b}{ax-b}$ where $a \neq 0$, $b \neq 0$.
\begin{enumerate}[(a)]
  \item Find $f'(x)$ using any method.
        \begin{proof}[Solution]
          First, notice that $f(x)$ is undefined at $x=\frac{b}{a}$,
          so we differentiate along all $x \neq \frac{b}{a}$.
          Apply the quotient and linear function rules:
          \begin{align*}
            \dv{x}(\frac{ax+b}{ax-b})
             & = \frac{(ax-b)\dv{x}(ax+b) - (ax+b)\dv{x}(ax-b)}{(ax-b)^2} \\
             & = \frac{(ax-b)a - (ax+b)a}{(ax-b)^2}                       \\
             & = \frac{a(-2b)}{(ax-b)^2}                                  \\
             & = -\frac{2ab}{(ax-b)^2} \qedhere
          \end{align*}
        \end{proof}
  \item Show that for $x \neq \frac{b}{a}$, $abf'(x)<0$.
        \begin{proof}
          Let $a$ and $b$ be non-zero reals, and let $x \neq \frac{b}{a}$. Then,
          \begin{align*}
            abf'(x) & = ab\left(\frac{2ab}{(ax-b)^2}\right) \\
                    & = -\frac{2a^2b^2}{(ax-b)^2}
          \end{align*}
          Recall that the square of any non-zero number is positive.
          Then, we have that $a^2 > 0$, $b^2 > 0$, and $(ax-b)^2 > 0$.
          The last one also implies $\frac{1}{(ax-b)^2} > 0$. Multiplying,
          \begin{align*}
            \frac{a^2b^2}{(ax-b)}   & > 0          \\
            -2\frac{a^2b^2}{(ax-b)} & < 0          \\
            abf'(x)                 & < 0 \qedhere
          \end{align*}
        \end{proof}
\end{enumerate}


\question In each case, find $f'(x)$ using any method.
\begin{enumerate}[(a)]
  \item $f(x)=5^x\sin x+(x^3+x^2)\cos x$.
        \begin{proof}[Solution]
          Apply arithmetic rules and recall that $\dv{x}a^x=a^x\ln a$:
          \begin{align*}
            f'(x) & = \dv{x}(5^x\sin x)+\dv{x}((x^3+x^2)\cos x)                                         \\
                  & = (5^x\dv{x}\cos x + \cos x\dv{x}5^x)+(\cos x\dv{x}(x^3+x^2)+(x^3+x^2)\dv{x}\cos x) \\
                  & = 5^x\sin x + \ln 5\cos x5^x + \cos x(3x^2+2x)+(x^3+x^2)\sin x                      \\
                  & = \sin x(x^3+x^2+5^x) + \cos x(5^x\ln 5 + 3x^2+2x) \qedhere
          \end{align*}
        \end{proof}
  \item $f(x)=\dfrac{x^2+x-2}{x^3+6}$.
        \begin{proof}[Solution]
          Apply the quotient rule, excepting $x=\sqrt[3]{-6}$ from the domain:
          \begin{align*}
            f'(x) & = \frac{(x^3+6)\dv{x}(x^2+x-2)-(x^2+x-2)\dv{x}(x^3+6)}{(x^3+6)^2} \\
                  & = \frac{(x^3+6)(2x+1)-(x^2+x-2)3x^2}{(x^3+6)^2}                   \\
                  & = -\frac{x^4+2x^3-6x^2-12x-6}{(x^3+6)^2} \qedhere
          \end{align*}
        \end{proof}
  \item $f(x)=\sqrt{2\tan^2x+3}$.
        \begin{proof}[Solution]
          Apply the chain rule, recalling that $\dv{x}\tan x=\sec^2 x$.
          \begin{align*}
            f'(x)
             & = \dv{\sqrt{2\tan^2x+3}}{(2\tan^2x+3)}\cdot \dv{x}(2\tan^2x+3)                              \\
             & = \frac{1}{2\sqrt{2\tan^2x+3}} \cdot 2\dv{\tan^2x}{\tan x} \cdot \dv{x}\tan x               \\
             & = \frac{1}{2\sqrt{2\tan^2x+3}} \cdot \left(2\dv{\tan^2x}{\tan x} \cdot \dv{x}\tan x \right) \\
             & = \frac{1}{2\sqrt{2\tan^2x+3}} \cdot 4\tan x \sec^2x                                        \\
             & = \frac{2\tan x \sec^2x}{\sqrt{2\tan^2x+3}} \qedhere
          \end{align*}
        \end{proof}
  \item $f(x)=2^{\sin(\sec x)}$.
        \begin{proof}[Solution]
          Again, simply apply the chain rule repeatedly.
          \begin{align*}
            f'(x)
             & = \dv{(2^{\sin(\sec x)})}{(\sin(\sec x))}\cdot\dv{\sin(\sec x)}{(\sec x)}\cdot\dv{x}\sec x \\
             & = 2^{\sin(\sec x)}\ln(2)\cos(\sec x)\sec(x)\tan(x) \qedhere
          \end{align*}
        \end{proof}
\end{enumerate}


\question In each case, determine the equation of the tangent to $y=f(x)$ at the point where $x=a$.
\begin{enumerate}[(a)]
  \item $f(x)=x^2$, $a=3$.
        \begin{proof}[Solution]
        \end{proof}
  \item $f(x)=\cos x$, $a=-\dfrac{3\pi}{4}$.
        \begin{proof}[Solution]
        \end{proof}
  \item $f(x)=e^x$, $a=\ln\pi$.
        \begin{proof}[Solution]
        \end{proof}
  \item $f(x)=4^x$, $a=-3$.
        \begin{proof}[Solution]
        \end{proof}
\end{enumerate}


\question Compute $\displaystyle\dv{y}{x}$ and $\displaystyle\dv[2]{y}{x}$ in each case.
\begin{enumerate}[(a)]
  \item $y=\cos x^2$.
        \begin{proof}[Solution]
        \end{proof}
  \item $y=\cos^2 x$.
        \begin{proof}[Solution]
        \end{proof}
\end{enumerate}


\question \begin{enumerate}[(a)]
  \item Use the Chain Rule to prove that the derivative of an even function is odd.
        \begin{proof}
        \end{proof}
  \item Using ONLY the Chain Rule and the Product Rule (and not the Reciprocal/Quotient rules),
        give an alternative proof of the Quotient Rule.
          [Hint: $\dfrac{f(x)}{g(x)}=f(x)(g(x))^{-1}$].
        \begin{proof}
        \end{proof}
\end{enumerate}


\question If $y=f(u)$ and $u=g(x)$ where $f$ and $g$ are twice differentiable functions, prove that
\[\dv[2]{y}{x} = \dv[2]{y}{u}\left(\dv{u}{x}\right)^2 + \dv{y}{u}\dv[2]{u}{x}. \]
\begin{proof}
\end{proof}

\end{document}