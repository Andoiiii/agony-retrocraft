\documentclass{agony}
\title{MATH 137 Fall 2019: Practice Final Exam}

\begin{document}
\thispagestyle{firstpage}
\textbf{\thetitle}

\qsection{Multiple Choice}{MC}

\prob{$\dlim{x}{3} \ln|x-3| = $}
\begin{choices}
  \item 0.
  \item $\infty$.
  \item \fbox{$-\infty$.} \emph{Vertical asymptote}
  \item None of the above.
\end{choices}
\prob{If $f$ is continuous on $[a,b]$ and differentiable on $(a,b)$, then}
\begin{choices}
  \item \fbox{for any $x_1,x_2\in(a,b)$ where $x_1 < x_2$, exists $c\in(x_1,x_2)$ so that
    $f'(c) = \frac{f(x_1)-f(x_2)}{x_1-x_2}$.} \\
  \emph{Statement of the MVT}
  \item $f(a) \leq f(x) \leq f(b)$ for all $x\in[a,b]$.
  \item $f'(x)$ is continuous on $(a,b)$.
  \item None of the above.
\end{choices}
\prob{$\dlim{x}{0}\frac{\cos x + \sin x - 1}{x}=$}
\begin{choices}
  \item -1.
  \item 0.
  \item 2.
  \item \fbox{None of the above.} \emph{And is equal to 1}
\end{choices}
\begin{prob}
  If $\dlim{x}{a} f(x) = L \in \R$ and $\{f(x_n)\}$ is a sequence such that
  $x_n \to a$ as $n \to \infty$, and $x_n \neq a$ for all $n \in \N$, then
\end{prob}
\begin{choices}
  \item $\dilim{n}f(x_n)$ does not exist.
  \item $f$ is continuous at $x=a$.
  \item \fbox{$\dilim{n}f(x_n) = L$.} \emph{Statement of SCL}
  \item None of the above.
\end{choices}
\prob{For a function $f$ and $a\in\R$, if $f''(a) = 0$ and $f'(a) = 0$, then}
\begin{choices}
  \item $x=a$ is a point of inflection for $f$.
  \item \fbox{$x=a$ is a critical point for $f$.} \emph{By definition}
  \item $f$ cannot have a local maximum at $x=a$.
  \item None of the above.
\end{choices}


\qsection{True/False}{TF}
\setcounter{question}{5}
\prob{For $a\in\R$, $|x-a| \leq 1$ defines a closed interval of length 1.} \\
False. \emph{The interval has length 2}
\prob{$f(x) = 3x^4 - 2x - 1$ haas a root on $[0,1]$.} \\
True. \emph{Notice that $f(1) = 0$}
\prob{If $f'(x) = \cos x$ then $f(x) = \sin x$.} \\
False. \emph{Missing constant of integration}
\prob{Let $a_n = f(n)$ where $f$ is a continuous function defined on \R.
  If $\dilim{n} a_n = L$ then $\dilim{x} f(x) = L$.} \\
False. \emph{Let $f(n+\frac12) = L+1$ for all $n$}
\prob{If $f$ is not differentiable at $x=a\in\R$,
  then for $k\in\R$, $g(x)=f(x)+k$ is not differentiable at $x=a$} \\
True. \emph{The limit does not exist}


\qsection{Short Answer}{SA}

\prob{For $f(x) = \ln(e+x)$, find $L_0^f(x)$.}
\begin{sol}
  We know $f(0) = \ln e = 1$ and $f'(x) = \frac{1}{e+x}$ so $f'(0) = \frac{1}{e}$. Therefore,
  \[ L_0^f(x) = 1 + \frac{1}{e}x \qedhere \]
\end{sol}

\begin{prob}
  If $f$ is a differentiable function such that $f(0) = 1$ and $f'(x)\in[1,5]$ for all $x\in\R$,
  use the Bounded Derivative Theorem to write down an interval that $f(3)$ must lie in.
\end{prob}
\begin{sol}
  Apply BDT\@: $f(3) \in [1+1(3),1+5(3)] = [4,16]$.
\end{sol}

\begin{prob}
  Give an example of a differentiable function $f$ that is concave up everywhere,
  but $f''(0)$ does not exist.
\end{prob}
\begin{sol}
  Let $f(x) = \begin{cases} x^2 & x \geq 0 \\ x^4 & x < 0 \end{cases}$.

  At $x=0$, the function remains differentiable since both one-sided limits and derivatives are 0.
  However, $f''(0)$ does not exist since the one-sided derivatives do not agree.
  For all points other than $x=0$, $f''(x) > 0$, so $f$ is concave up for positive and negative $x$.
  In fact, $f$ is concave up everywhere.
\end{sol}

\begin{prob}
  Give an example of a function $f$ that is differentiable on $(0,1)$,
  both $f(0)$ and $f(1)$ are defined, but the Mean Value Theorem cannot be applied to $f$.
\end{prob}
\begin{sol}
  We are given the hypotheses of MVT except continuity. Make $f$ discontinuous:
  \[ f(x) = \begin{cases}
      x & x \neq 0 \\
      1 & x = 0
    \end{cases} \qedhere \]
\end{sol}

\begin{prob} If $f(3) = 1$ and $f'(3) = \pi$, find $(f^{-1})'(1)$. \end{prob}
\begin{sol}
  We know that $f^{-1}(1) = 3$.
  Then, by the IFT, $(f^{-1})'(1) = \frac{1}{f'(3)} = \frac{1}{\pi}$.
\end{sol}


\qsection{Long Answer}{LA}

\begin{prob}
  Find each of the following sequence limits, if they exist.
  If they do not exist, prove it.
  \begin{enumerate}[(a)]
    \item $\dilim{n} \frac{\sin n\pi}{\sin n}$
          \begin{sol}
            Recall that $\sin n\pi = 0$ for all $n$.
            Then, $\frac{\sin n\pi}{\sin n} = 0$ so the limit is 0.
          \end{sol}
    \item $\dilim{n} \frac{\sin n}{n^2+1}$
          \begin{sol}
            We know that $-1 \leq \sin n \leq 1$, so we have
            \[ -\frac{1}{n^2+1} \leq \frac{\sin n}{n^2+1} \leq \frac{1}{n^2+1} \]
            Now, both of these converge to 0, so by the Squeeze Theorem, the limit is 0.
          \end{sol}
    \item $\dilim{n} \frac{n^3+n+1}{3n^3+n^2}$
          \begin{sol}
            Divide through by $n^3$: \begin{align*}
              \ilim{n} \frac{n^3+n+1}{3n^3+n^2}
               & = \ilim{n} \frac{1+\frac{1}{n^2}+\frac{1}{n^3}}{3+\frac{1}{n}} \\
               & = \ilim{n} \frac{1+0+0}{3+0}                                   \\
               & = \ilim{n} \frac{1}{3} \qedhere
            \end{align*}
          \end{sol}
  \end{enumerate}
\end{prob}

\begin{prob}
  Prove that if $\{a_n\}$ and $\{b_n\}$ are sequences such that $\{a_n\}$ is bounded
  and $b_n \to 0$ as $n \to \infty$, then $\dilim{n}(a_n b_n) = 0$.
\end{prob}
\begin{prf}
  Since $a_n$ is bounded, we have $x = \sup a_n$ and $y = \inf a_n$ such that $x \leq a_n \leq y$ for all $n$.
  Multiplying through by $b_n$, we have $xb_n \leq a_n b_n \leq yb_n$.
  As $b_n$ converges to 0, so too do $xb_n$ and $yb_n$ by the arithmetic rules.

  Therefore, by the Squeeze Theorem, the limit is 0.
\end{prf}

\begin{prob}
  For each of the following functions, compute $f'(x)$ using any method.
  You do not need to simplify your answers.
  \begin{enumerate}[(a)]
    \item $f(x) = x^2 e^x \ln x$
          \begin{sol}
            Repeatedly apply the product rule: \begin{align*}
              f'(x) & = 2x e^x \ln x + x^2(e^x \ln x)'                         \\
                    & = 2x e^x \ln x + x^2(e^x \ln x + \frac{e^x}{x}) \qedhere
            \end{align*}
          \end{sol}
    \item $f(x) = \tan(\cos x)$
          \begin{sol}
            Apply the chain rule: \begin{align*}
              f'(x) & = \sec^2(\cos x)(\cos x)'        \\
                    & = -\sec^2(\cos x)\sin x \qedhere
            \end{align*}
          \end{sol}
  \end{enumerate}
\end{prob}

\begin{prob}
  \begin{enumerate}[(a)]
    \item Find $y'$ if $\ln x + \ln y = xy$.
          \begin{sol}
            Implicitly differentiate with respect to $x$: \begin{align*}
              \frac{1}{x} + \frac{y'}{y} & = xy' + y                                      \\
              y'(\frac{1}{y} - x)        & = y - \frac{1}{x}                              \\
              y'                         & = \frac{y-\frac{1}{x}}{\frac{1}{y}-x} \qedhere
            \end{align*}
          \end{sol}
    \item Find $\dyx$ if $y = (\sin x)^{\ln x}$ for $0 < x \leq \pi$.
          \begin{sol}
            Taking the logarithm of both sides, $\ln y = \ln x \ln(\sin x)$.
            Then, implicitly differentiating with respect to $x$: \begin{align*}
              \frac{y'}{y} & = \frac{\ln(\sin x)}{x} + \frac{\ln x \cos x}{\sin x}                        \\
              \dyx         & = (\sin x)^{\ln x}\left(\frac{\ln(\sin x)}{x} + \ln x \cot x\right) \qedhere
            \end{align*}
          \end{sol}
  \end{enumerate}
\end{prob}

\begin{prob}
  For $f(x) = (x-1)|x+2| - 3$, determine all global extrema on the interval $[-3,0]$, if they exist.
\end{prob}
\begin{sol}
  For $x < -2$, $f(x) = -(x-1)(x+2) - 3 = - x^2 - x - 1$ and $f'(x) = -2x - 1$.
  Likewise, if $x > -2$, $f(x) = (x-1)(x+2) - 3 = x^2 + x - 5$ and $f'(x) = 2x + 1$.

  The critical points are at $x = -2$ (undefined) and $x = -\frac12$ (zero).
  Testing the function values here and at the endpoints, we have
  $f(-3) = -7$, $f(-2) = -3$, $f(-\frac12) = -\frac{21}{4}$, and $f(0) = -5$.

  Therefore, the global extrema are at $(-3, -7)$ and $(-2, -3)$.
\end{sol}

\begin{prob}
  \begin{enumerate}[(a)]
    \item State the Intermediate Value Theorem for a function $f$.

          \textbf{Theorem.} If $f$ is continuous on $[a,b]$ and $f(a) < f(b)$,
          then for any value $k\in(f(a),f(b))$ there exists a $c\in(a,b)$ where $f(c) = k$.
          Likewise, if $f(b) < f(a)$, then for any value $k\in(f(b),f(a))$,
          there exists a $c\in(a,b)$ where $f(c)=k$.

    \item Find an interval of length at most 1 that contains a root of $f(x) = x^3 + 3x + 1$.
          \begin{sol}
            We know that $f(0) = 1$ and $f(-1) = -3$, so by the IVT, a root exists on $(0,1)$.
          \end{sol}

    \item Using $x_1 = 0$, perform two iterations of Newton's Method to find $x_2$ and $x_3$
          to approximate the root of $f(x) = x^3 + 3x + 1$.
          \begin{sol}
            We have $f'(x) = 3x^2 + 3$
            Then, $x_1 = 0$ and $x_2 = x_1 - \frac{f(x_1)}{f'(x_1)} = -\frac{f(0)}{f'(0)} = -\frac13$.
            Again, $x_3 = x_2 - \frac{f(x_2)}{f'(x_2)} = -\frac13 - \frac{-1/27}{10/3} = -\frac{29}{90}$.

            Therefore, $f(-\frac{29}{90}) \approx 0$.
          \end{sol}
  \end{enumerate}
\end{prob}

\begin{prob}
  Let $f(x) = \ln(x^2+1)$.
  \begin{enumerate}[(a)]
    \item Determine the intervals of increase/decrease for $f$.
          \begin{sol}
            We have $f'(x) = \frac{2x}{x^2+1}$.
            Since $x^2+1$ is always positive, $f'(x)$ has the same sign as $x$.
            Therefore, $f$ is increasing for positive $x$ and decreasing for negative $x$.
          \end{sol}
    \item Determine the intervals of concavity for $f$.
          \begin{sol}
            Taking the derivative from above, $f''(x) = -\frac{2(x-1)(x+1)}{(x^2+1)^2}$.
            The denominator is always positive so this is well-defined.
            It is zero at $x = \pm 1$, and from a sign analysis,
            we determine that $f''(x)$ is positive on $(-1,1)$ so $f$ is concave up,
            and that $f''(x)$ is negative on $(-\infty,-1)$ and $(1,\infty)$ so $f$ is concave down.
          \end{sol}
  \end{enumerate}
\end{prob}

\begin{prob}
  Prove that if $f$ is a differentiable function with no critical points,
  then it can have at most one real root.
\end{prob}
\begin{prf}
  Let $f$ be a differentiable function with no critical points.
  Suppose for a contradiction that $f$ contains more than one real root, namely, $a$ and $b$.
  Then, by Rolle's Theorem, since $f(a) = 0$ and $f(b) = 0$, there exists a point $c\in(a,b)$
  such that $f'(c) = 0$, i.e., $c$ is a critical point of $f$.
  However, $f$ has no critical points.

  Therefore, $f$ has at most one real root.
\end{prf}

\begin{prob}
  In each case, compute the limit using any method.
  \begin{enumerate}[(a)]
    \item $\dlim{x}{1^+}(\ln x)^{x-1}$
          \begin{proof}[Solution]
            This is indeterminate of the form $0^0$.
            We can rewrite the quantity in the limit as $e^{\ln((\ln x)^{x-1})} = e^{(x-1)\ln(\ln x)}$.
            Since $e^x$ is continuous, we can push through the limit and consider only the limit
            \[ \dlim{x}{1^+}(\ln x)^{x-1} = e^{\dlim{x}{1^+}(x-1)\ln(\ln x)} \]
            We now have the form $0\cdot-\infty$.
            We rewrite as $-\frac{\infty}{\infty}$ and apply l'Hôpital's Rule:
            \begin{align*}
              \dlim{x}{1^+}(x-1)\ln(\ln x)
               & = \dlim{x}{1^+}\frac{\ln(\ln x)}{\frac{1}{x-1}}                                             \\
               & = \dlim{x}{1^+}\frac{\frac{1}{x\ln x}}{-\frac{1}{(x-1)^2}}                                  \\
               & = \dlim{x}{1^+}\frac{(x-1)^2}{x \ln x}                                                      \\
               & = \dlim{x}{1^+}\frac{(x-1)^2}{x \ln x}                     & \text{(of the form $\frac00$)} \\
               & = \dlim{x}{1^+}\frac{2(x-1)}{1+\ln x}                                                       \\
               & = \frac{0}{1} = 0
            \end{align*}
            Now, $e^0 = 1$, so the limit is equal to 1.
          \end{proof}
    \item $\dlim{x}{0^+}(\sqrt{x})^{\frac{1}{3\sqrt{x}}}$
          \begin{proof}[Solution]
            The limit is of the form $0^\infty$, so we can say that it is equal to 0.
          \end{proof}
    \item $\dlim{x}{0^+}(1+\sqrt{x})^{\frac{1}{3\sqrt{x}}}$
          \begin{proof}[Solution]
            The limit is indeterminate and of the form $1^{\infty}$.
            Doing our good ol$e$' trickery, we rewrite as $e^{\frac{\ln(1+\sqrt{x})}{3\sqrt{x}}}$,
            and after pushing through, we have a form $\frac{0}{0}$ so we can apply l'Hôpital's Rule:
            \begin{align*}
              \dlim{x}{0^+} \frac{\ln(1+\sqrt{x})}{3\sqrt{x}}
               & = \dlim{x}{0^+} \frac{\frac{1}{2\sqrt{x}(1+\sqrt{x})}}{\frac{3}{2\sqrt{x}}} \\
               & = \dlim{x}{0^+} \frac{1}{3(1+\sqrt{x})}                                     \\
               & = \frac13
            \end{align*}
            We can conclude the limit is $\sqrt[3]{e}$.
          \end{proof}
  \end{enumerate}
\end{prob}

\begin{prob}
  Find values of $a$ and $b$ so that $f$ is differentiable everywhere, where
  \[ f(x) = \begin{cases}
      \sin ax      & x \geq 0 \\
      x^2 + 2x + b & x < 0
    \end{cases} \]
\end{prob}
\begin{sol}
  For $f$ to be differentiable, the one-sided derivatives must agree.
  These are $a\cos ax$ and $2x + 2$.
  At $x=0$, these are equal to $a$ and $2$, so $a=2$
  Differentiability implies continuity, so we check the one-sided limits.
  Then, $\sin 0 = (0)^2 + 2(0) + b$, and we conclude $b=0$.

  Therefore, $(a,b) = (2,0)$.
\end{sol}

\begin{prob}
  \begin{enumerate}[(a)]
    \item Prove that if $f'(x) = g'(x)$ for all $x$ in some open interval $I$,
          then there exists $k \in \R$ so that $f(x) = g(x) + k$ for all $x \in I$.
          \begin{prf}
            This follows directly from the Constant Function Theorem.
          \end{prf}
    \item Use part (a) to prove that if $f'(x) - g'(x) = 2x$ on $I$,
          then $f(x) = g(x) + x^2 + k$ for all $x \in I$ for some $k \in \R$.
          \begin{prf}
            Notice that $f'(x) = g'(x) + 2x = (g(x)+x^2)'$.

            Then, from (a), $f(x) = g(x) + x^2 + k$.
          \end{prf}
  \end{enumerate}
\end{prob}

\begin{prob} Consider the function $f(x) = \ln(1+x)$.
  \begin{enumerate}[(a)]
    \item Find the second-degree Taylor polynomial for $f$ centred at $x=0$, $T_{2,0}(x)$.
          \begin{sol}
            We have that $f'(x) = \frac{1}{1+x}$ and $f''(x) = -\frac{1}{(1+x)^2}$.
            Then, $f(0) = 0$, $f'(0) = 1$, and $f''(0) = -1$.
            Now, applying the formula,
            \begin{align*}
              T_{2,0}(x) & = \frac{f''(0)}{2}x^2 + f'(0)x + f(0) \\
                         & = -\frac{1}{2}x^2 + x \qedhere
            \end{align*}
          \end{sol}
    \item Use $T_{2,0}$ to approximate $\ln 2$.
          \begin{sol}
            From above, evaluate $\ln 2 = \ln(1+1) \approx T_{2,0}(1) = \frac12$.
          \end{sol}
    \item Use Taylor's Theorem to write down what $f(x) - T_{2,0}(x)$ is equal to
          (in terms of $x$ and $c$) for $x > 0$.
          \begin{sol}
            For some $c\in(0,1)$, we have that $f(x) - T_{2,0}(x) = R_{2,0} = \frac{f^{(3)}(c)}{3!}x^3$.
            We can calculate $f^{(3)}(x) = \frac{2}{(1+x)^3}$, so we can expand this as
            \[ R_{2,0}(x) = \frac{x^3}{3(1+c)^3} \qedhere \]
          \end{sol}
    \item Find an upper bound on the error in your approximation in part (b).
          \begin{sol}
            From above, $R_{2,0}(1) = \frac{1}{3(1+c)^3}$.
            This is always positive and maximized when $c=0$, so $R_{2,0}(1) \leq \frac{1}{3}$.
          \end{sol}
    \item Is the estimate in part (b) an over or under estimate?
          \begin{sol}
            From above, the difference is positive, so it is an underestimate.
          \end{sol}
    \item Give an interval that $\ln 2$ must lie in, be as specific as possible.
          \begin{sol}
            From part (d), it lies in $[\frac12,\frac12+\frac13] = [\frac12,\frac56]$.
          \end{sol}
  \end{enumerate}
\end{prob}

\end{document}