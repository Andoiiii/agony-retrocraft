\documentclass{agony}
\title{MATH 137 Fall 2020: Practice Assignment 10}

\begin{document}
\thispagestyle{firstpage}
\textbf{\thetitle}

\question For each of the following functions find the intervals where the function
is concave up or concave down and state any points of inflection.
\begin{enumerate}[(a)]
  \item $f(x) = x^4+6x^3-60x^2+6x-60$
        \begin{proof}[Solution]
          Take the second derivative: $f''(x) = 12x^2 + 36x - 120$.
          Factoring, $f''(x) = 12(x+5)(x-2)$
          Then, $f''(x) \geq 0$ and $f$ is concave up on $[-\infty,-5]$ and $[2,\infty]$,
          and $f''(x) \leq 0$ and $f$ is concave down on $[-5,2]$.

          The inflection points are $(-5,f(-5)) = (-5,-1715)$ and $(2,f(2)) = (2,-224)$.
        \end{proof}
  \item $f(x) = 1+\frac{1}{x}-\frac{1}{x^3}$
        \begin{proof}[Solution]
          We have $f'(x) = -\frac{1}{x^2} + \frac{3}{x^4}$ and $f''(x) = \frac{2}{x^3} - \frac{12}{x^5}$.
          Then, $f''(x) = \frac{2x^2-12}{x^5}$ which is zero or undefined at $x=0,\pm\sqrt6$.
          In fact, this is a rational function with zeroes at $x=\pm\sqrt6$ and an asymptote at $x=0$.
          Therefore, $f$ is concave up on $[-\sqrt6,0)$ and $[\sqrt6,\infty)$,
          and $f$ is concave down on $(-\infty,-\sqrt6]$ and $(0,\sqrt6]$.

          The inflection points are $(-\sqrt6,f(-\sqrt6)) \approx (-2.4495, 0.6598)$
          and $(\sqrt6,f(\sqrt6)) \approx (2.4495,1.3402)$ since there is none at the asymptote.
        \end{proof}
  \item $f(x) = \frac{x^2}{x+x^3}$
        \begin{proof}[Solution]
          There is a removable discontinuity at $x=0$, so $f'$ is well-defined everywhere.
          Then, $f'(x) = -\frac{x^2-1}{(x^2+1)^2}$ and $f''(x) = \frac{2x(x^2-3)}{(x^2+1)^2}$.
          This is zero at $x=0,\pm\sqrt3$.
          Skipping the sign analysis cause I'm lazy, $f$ is concave up on $[-\sqrt3,0]$ and $[\sqrt3,\infty)$
          and $f$ is concave down on $(-\infty,-\sqrt3]$ and $[0,\sqrt3]$.

          Since $f$ is discontinuous at $x=0$, there cannot be an inflection point there, so we have only
          $(-\sqrt3,f(-\sqrt3)) \approx (-1.7321, -0.4330)$ and $(\sqrt3,f(\sqrt3)) \approx (1.7321, 0.4330)$.
        \end{proof}
  \item $f(x) = \frac{\sin x}{1+\cos x}$
        \begin{proof}[Solution]
          Use a derivative calculator (am lazy):
          $f'(x) = \frac{1}{1+\cos x}$ and $f''(x) = \frac{\sin x}{(1+\cos x)^2}$.
          Clearly, $(1+\cos x)^2 \geq 0$ for any $x$,
          so this changes sign when $\sin x = 0$ or $x = k\pi$ for some integer $k$.
          However, if $\cos x = -1$, i.e., when $k$ is odd, $f$ is undefined so there is no inflection point.

          Therefore, $f$ is concave up on $[2k\pi,2k\pi+\pi)$, concave down on $(2k\pi-\pi,2k\pi]$ and
          has an inflection point at $(2k\pi, 0)$ for any integer $k$.
        \end{proof}
  \item $f(x) = x^2-9x^{\frac{1}{3}}$
        \begin{proof}[Solution]
          Chug some derivatives: $f'(x) = 2x - 3x^{-\frac23}$
          and $f''(x) = 2 + 2x^{-\frac53} = \frac{2x^{5/3}+2}{x^{5/3}}$.
          This is zero at $x=-1$ and undefined at $x=0$.
          Do some sign analysis and find that $f$ is concave up on $(\infty,-1]$ and $(0,\infty)$,
          $f$ is concave down on $[-1,0)$, and an inflection point occurs at $(-1,10)$.
        \end{proof}
  \item $f(x) = x\ln(x^2-2x)$
        \begin{proof}[Solution]
          First, notice that the function is only defined when $x^2-2x > 0$ or $x \not\in [0,2]$.
          Take $f'(x) = \ln(x^2-2x) + \frac{2x-2}{x-2}$ and $f''(x) = \frac{2x^2-8x+4}{x(x-2)^2}$.
          This is zero at $x=2\pm\sqrt2$ but only $x=2+\sqrt2$ is in our domain.
          So too are both undefined points.

          We test the sign of $f''$ in $(-\infty,0)$, $(2,2+\sqrt2)$ and $(2+\sqrt2,\infty)$
          to conclude that $f$ is concave up on $[2+\sqrt2,\infty)$,
          concave down on $(-\infty,0)$ and $(2,2+\sqrt2)$,
          and has an inflection point at approximately $(3.4142, 5.3758)$.
        \end{proof}
\end{enumerate}


\question Locate and classify the critical points of each of the following functions:
\begin{enumerate}[(a)]
  \item $f(x) = 3x^5-20x^3+15$
        \begin{proof}[Solution]
          We have $f'(x) = 15x^4-60x^2 = 15x^2(x^2-4)$, so $f'(x)=0$ exactly when $x=0,\pm2$.
          Apply the Second Derivative Test: $f''(x) = 60x^3-120x$, so we have
          $f''(0) = 0$ (not a local extremum), $f''(-2) = -240$ (local maxima), and $f''(2)=240$ (local minima).
        \end{proof}
  \item $f(x) = \frac{x+1}{x-1}-x$
        \begin{proof}[Solution]
          We have $f'(x) = -\frac{x^2-2x+3}{(x-1)^2}$, so $f'(x)$ is never zero and undefined at $x=1$.
          This creates an asymptote, which is not a local extremum.
        \end{proof}
  \item $f(x) = \frac{x+1}{x-1}+x$
        \begin{proof}[Solution]
          We have $f'(x) = -\frac{x^2-2x-1}{(x-1)^2}$, so $f'(x)$ is zero at $x=1\pm\sqrt2$ and undefined at $x=1$.
          As above, we have an asymptote at $x=1$ which is not a local extrema.
          Since $(x-1)^2 \geq 0$ for all $x$ and the numerator is a quadratic, we can deduce that
          $1-\sqrt2$ is the local max and $1+\sqrt2$ is the local min.
        \end{proof}
  \item $f(x) = 12x^{\frac{1}{3}}+|x|$
        \begin{proof}[Solution]
          Again, $f'(x) = 4x^{-\frac23} + \sgn x$.
          Notice that $4x^{-\frac23}$ is always positive.
          Then, $f'(x) = 0$ only if $x$ is negative and $4x^{-\frac23} = -1$, which occurs at $x=-8$.
          We also know that $f'(x)$ is undefined at $x=0$.
          From this, we conclude that $x=-8$ is a local min.
        \end{proof}
  \item $f(x) = 4e^x-2e^{2x}-x$
        \begin{proof}[Solution]
          Derivative time: $f'(x) = 4e^x - 4e^{2x} - 1$.
          This is a quadratic in $e^x$ which factors to $-(2e^x - 1)^2$.
          The critical point is then at $e^x = \frac12$ or $x = -\ln 2$.
          However, since $f'(x) \leq 0$ for all $x$, this is not a local extremum.
        \end{proof}
  \item $f(x) = \ln(x+1) + x^2 - 6x$
        \begin{proof}[Solution]
          WolframAlpha says $f'(x) = \frac{1}{x+1} + 2x - 6 = \frac{2x^2 - 4x - 5}{x+1}$.
          This is undefined at $x=-1$ and zero at $x=\frac{2\pm\sqrt{14}}{2}$.
          However, $x=-1$ is not in the domain of $f$ and is not a critical point.
          Some sign analysis later, we find that $\frac{2-\sqrt{14}}{2}$ is the local max
          and $\frac{2+\sqrt{14}}{2}$ is the local min.
        \end{proof}
  \item $f(x) = |x|+\sin^2 x$
        \begin{proof}[Solution]
          Another janky $|x|$ derivative, $f'(x) = \sgn x + 2\sin x \cos x = \sgn x + \sin 2x$.

          This is equal to zero for positive $x$ when $\sin 2x = -1$, i.e., $x = (4k+3)\pi$ for some integer $k$.
          For negative $x$, $\sin 2x = 1$, i.e., $x = (4k-1)\pi$ for some integer $k$.
          These are in fact the same so critical points exist at all $x = (4k+3)\pi$.

          Notice that for positive $x$, $\sgn x = 1$ and $\sin 2x \leq 1$ so $f'(x) \geq 0$.
          Likewise, for negative $x$, $f'(x) \leq 0$.
          Therefore, no local extrema exist for $x \neq 0$.

          At $x = 0$, the sign of $f'(x)$ changes from negative to positive, so it is a local min.
        \end{proof}
\end{enumerate}


\question Consider the function
\[ f(x) = \begin{cases}
    \frac{x^3}{|x|} & x \neq 0 \\
    0               & x = 0
  \end{cases} \]
\begin{enumerate}[(a)]
  \item Compute $f'(x)$ for all possible $x$. Your answer can be piecewise if necessary.
        \begin{proof}[Solution]
          Notice that $f(0)=0$ and $\dlim{x}{0}f(x)=0$, so $f$ is continuous.
          We notice that for positive $x$, $f(x)=x^2$ and for negative $x$, $f(x)=-x^2$.
          Taking the respective derivatives,
          \[ f'(x)=\begin{cases} x & x\geq0 \\ -x & x<0 \end{cases} \]
          which is in fact by definition $f'(x)=|x|$.
        \end{proof}
  \item Compute $f''(x)$ for all possible $x$. Your answer can be piecewise if necessary.
        \begin{proof}[Solution]
          Following the same process as above, $f''(x)=\sgn x$.
          Notice that $f''(x)$ is undefined at $x=0$.
        \end{proof}
  \item Does $f$ have an inflection point at $x=0$?
        \begin{proof}[Solution]
          Yes, because for $x < 0$, $f''(x) < 0$ and $f$ is concave down,
          and for $x > 0$, $f''(x) > 0$ and $f$ is concave up.
        \end{proof}
\end{enumerate}


\question Let $f$ and $g$ be twice differentiable functions
(that is, both $f''(x)$ and $g''(x)$  exist for all $x\in\R$).
Prove or disprove (that is, find a counterexample to) the following statements:
\begin{enumerate}[(a)]
  \item If the graph of $f$ is concave up on $\R$, the graph of $g$ is concave down on $\R$,
        and $f(x) > g(x)$ for all $x\in\R$, then the graph of $f+g$ is concave up on $\R$.
        \begin{proof}[Solution]
          For a counterexample, let $f(x)=x^4$ and $g(x)=-x^2-1$.
          Then, $f(x) > g(x)$, $f''(x)=12x^2 \geq 0$, and $g''(x)=-2 < 0$ for all $x\in\R$.
          However, $(f+g)''(x) = 12x^2-2$ is negative at $x=0$, so $f+g$ is concave down,
          i.e., not concave up on all $\R$.
        \end{proof}
  \item If the graph of $f$ is concave up on $\R$, and the graph of $g$ is concave down on $\R$,
        then the graph of $f+g$ is neither concave up nor concave down on all of $\R$.
        \begin{proof}[Solution]
          For a counterexample, let $f(x)=x^4+x^2$ and $g(x)=-x^2-1$.
          Then, $f(x) > g(x)$, $f''(x)=12x^2+2 > 0$, and $g''(x)=-2 < 0$ for all $x\in\R$.
          However, $(f+g)''(x) = 12x^2 \geq 0$, so $f+g$ is concave up for all $\R$.
        \end{proof}
\end{enumerate}


\question Prove that if $f$ is a twice differentiable function such that $f''(x) \neq 0$ for all $x\in\R$,
$f$ is positive on $\R$, and the graph of $f$ is concave up on $\R$,
then the graph of $g = (f)^2$ is concave up on $\R$.
\begin{proof}
  Let $f$ be a twice differentiable function.
  Since $f$ is concave up and $f'' \neq 0$, we have that $f''$ is always positive.

  Now, consider $g''$. Apply the product rule: $g'(x) = f'(x)f(x) + f(x)f'(x) = 2f'(x)f(x)$.
  Then, $g''(x) = 2(f''(x)f(x) + f'(x)f'(x)) = 2f''(x)f(x) + (f'(x))^2$.

  Since $f''(x) > 0$ and $f(x) > 0$, the product $2f''(x)f(x) > 0$.
  Since it is a square, $(f'(x))^2 \geq 0$.
  Therefore, the sum is positive and $g$ is concave up.
\end{proof}


\question Let $f(x)=ax^3+bx^2+cx+d$ be a cubic equation.
\begin{enumerate}[(a)]
  \item Find $a,b,c,d\in\R$ such that $f(x)$ has a local maximum at $(-2,3)$ and local minimum at $(1,0)$.
        \begin{proof}[Solution]
          This requires that $f'(-2)=f'(1)=0$, $f(1)=0$, and $f(-2)=3$.
          Notice that $f'(x) = 3ax^2+2bx+c$, a quadratic.
          Therefore, we must have $f'(x)=k(x+2)(x-1)=kx^2+kx-2k$.
          Then, $a=\frac13k$, $b=\frac{k}{2}$, and $c=-2k$, so our cubic is now
          \[ f(x) = \frac13kx^3 + \frac12kx^2 - 2kx + d \]
          We solve for $d$ where $f(1)=0$:
          \begin{align*}
            0 & = \frac13k + \frac12k - 2k + d \\
            d & = \frac76k
          \end{align*}
          and $k$ where $f(-2)=3$:
          \begin{align*}
            3 & = \frac13k(-2)^3 + \frac12k(-2)^2 - 2k(-2) + d \\
            3 & = -\frac83k + 2k + 4k + \frac76k               \\
            3 & = \frac92k                                     \\
            k & = \frac23
          \end{align*}
          so, backsubstituting, we have the cubic equation
          \[ f(x) = \frac{2}{9}x^3 + \frac{1}{3}x^2 - \frac{4}{3}x + \frac{7}{9} \qedhere \]
        \end{proof}
  \item Use (a) to find the intervals of concavity and the inflection points of $f(x)$.
        \begin{proof}[Solution]
          Recall that $f'(x)=kx^2+kx-2k$, so $f''(x)=2kx+k$.
          Then, $f''(x)=0$ exactly when $x=-\frac12$.
          Since $f$ is a cubic with positive $a$, we can say $x=-\frac12$ is an inflection point,
          $f$ is concave up on $(-\infty,-\frac12)$, and $f$ is concave down on $(-\frac12,\infty)$.
        \end{proof}
\end{enumerate}


\question For each of the following functions perform all the steps we outlined
to sketch the curve with calculus, and then sketch the curve.
This includes listing the domain, any asymptotes, intervals of increase/decrease,
intervals of concavity, and along with the sketch plotting any critical points,
points of inflection, and intercepts.
\begin{proof}[Existential Crisis]
  Yeah, uh, not sure how to do this in \LaTeX.
\end{proof}

\end{document}