\documentclass[11pt]{article}

\usepackage{physics}
\usepackage{amsfonts,amsmath,amssymb,amsthm}
\usepackage{enumerate}
\usepackage{titlesec}
\usepackage{fancyhdr}
\usepackage{multicol}

\headheight 13.6pt
\setlength{\headsep}{10pt}
\textwidth 15cm
\textheight 24.3cm
\evensidemargin 6mm
\oddsidemargin 6mm
\topmargin -1.1cm
\setlength{\parskip}{1.5ex}
\parindent=0pt

\author{James Ah Yong}

\pagestyle{fancy}
\fancyhf{}
\fancyfoot[c]{\thepage}
\makeatletter
\lhead{\@title}
\rhead{\@author}

\fancypagestyle{firstpage}{
  \fancyhf{}
  \rhead{\@author}
  \fancyfoot[c]{\thepage}
}

% Sets
\newcommand{\N}{\mathbb{N}}
\newcommand{\Z}{\mathbb{Z}}
\newcommand{\Q}{\mathbb{Q}}
\newcommand{\R}{\mathbb{R}}
\newcommand{\C}{\mathbb{C}}
\newcommand{\U}{\mathcal{U}}
\newcommand{\sym}{\mathbin{\triangle}}

% Functions
\DeclareMathOperator{\sgn}{sgn}
\DeclareMathOperator{\im}{im}

% Operators
\newcommand{\Rarr}{\Rightarrow}
\newcommand{\Larr}{\Leftarrow}
\usepackage{mathtools} % for \DeclarePairedDelimiter macro
\DeclarePairedDelimiter\ceil{\lceil}{\rceil}
\DeclarePairedDelimiter\floor{\lfloor}{\rfloor}

% Macros
% properly typeset ε-δ (epsilon en dash delta)
\newcommand{\epsdel}[1][\delta]{\ensuremath{\epsilon\mathit{\textnormal{--}}#1}}
\newcommand{\by}[1]{& \text{by #1}}
\newcommand{\IH}{\by{inductive hypothesis}}
% multiple choice (remove spacing between items)
\newenvironment{choices}
{\begin{enumerate}[(a)]
    \setlength{\parskip}{0ex}
    }{
  \end{enumerate}}

% Typesetting
\usepackage{array}   % for \newcolumntype macro
\newcolumntype{C}{>{$}c<{$}} % math version of "C" column type
\newcommand{\dlim}[2]{\displaystyle\lim_{#1\to#2}} % totally not \dfrac ripoff
\newcommand{\dilim}[1]{\dlim{#1}{\infty}} % infinite limits
\newcommand{\ilim}[1]{\lim_{#1\to\infty}}
\usepackage{cancel}

% Auto-number questions
\newcommand{\QType}{Q}
\renewcommand{\theparagraph}{\QType\ifnum\value{paragraph}<10 0\fi\arabic{paragraph}}
\setcounter{secnumdepth}{6}
\newcommand{\question}{\par\refstepcounter{paragraph}\textbf{\theparagraph}.\space}

% Question sections
\titleformat{\section}{\normalsize\bfseries}{\thesection}{1em}{}
\newcommand{\qsection}[2]{%
  \renewcommand{\QType}{#2}
  \section*{#1}
  \refstepcounter{section}
}

\title{MATH 137 Fall 2020: Practice Assignment 9}

\begin{document}
\thispagestyle{firstpage}
\textbf{\@title}

\question Find the intervals over which the following functions are increasing/decreasing.
\begin{enumerate}[(a)]
  \item $f(x) = x^4 - 8x^2$
        \begin{proof}[Solution]
          We take $f'(x) = 4x^3 - 16x = 4x(x^2-4) = 4x(x-4)(x+4)$.
          The critical points are $x=0,\pm 4$.
          Since $f'$ is an odd-degree polynomial with a positive leading coefficient and linear factors,
          we can say that $f$ is decreasing on $(-\infty,-4)\cup(0,4)$
          and increasing on $(-4,0)\cup(4,\infty)$.
        \end{proof}
  \item $f(x) = \frac{1}{x^2-1}$
        \begin{proof}[Solution]
          We take $f'(x) = -\frac{2x}{(x^2-1)^2} = -\frac{2x}{(x-1)(x+1)}$
          and find critical points $x=0,\pm1$.
          Analyzing the signs of the factors of $f'$:
          \begin{center}
            \begin{tabular}{C|C|C|C|C}
                    & (-\infty,-1) & (-1,0) & (0,1) & (1,\infty) \\ \hline
              -2x   & +            & +      & -     & -          \\
              (x-1) & -            & -      & -     & +          \\
              (x+1) & -            & +      & +     & +          \\ \hline
              f'    & +            & -      & +     & -          \\
            \end{tabular}
          \end{center}
          Then, $f$ is decreasing on $(-1,0)\cup(1,\infty)$ and increasing on $(-\infty,-1)\cup(0,1)$.
        \end{proof}
  \item $f(x) = e^x + e^{-x+1}$
        \begin{proof}[Solution]
          We have $f'(x) = e^x - e^{-x+1}$. This is defined on $\R$, so we solve $f'(x)=0$:
          \begin{align*}
            f'(x) & = 0        \\
            e^x   & = e^{-x+1} \\
            x     & = -x + 1   \\
            x     & = \frac12
          \end{align*}
          Therefore, our only critical point is at $x=\frac12$.
          For large positive $x$, the $e^x$ term dominates and for large negative $x$, the $e^{-x}$ term dominates.
          It follows that $f$ is decreasing on $(-\infty,\frac12)$ and increasing on $(\frac12,\infty)$.
        \end{proof}
  \item $f(x) = x^4 - 4x^3 + 16x - 7$
        \begin{proof}[Solution]
          Taking the derivative, $f'(x) = 4x^3 - 12x^2 + 16 = 4(x+1)(x-2)^2$,
          and the critical points are $x=-1,2$.

          Since $(x-2)^2$ is always non-negative, it does not affect the sign of $f'$.
          From the sign of $(x+1)$, we can say $f$ is increasing on $(-\infty,-1)$ and increasing on $(-1,\infty)$.
        \end{proof}
\end{enumerate}


\question Show that if $f$ is increasing and differentiable on $(a,b)$
then $f'(x) \geq 0$ for all $x\in(a,b)$.

\textbf{Hint:} You may wish to use the result \begin{center}
  If $g(x) > 0$ for all $x \neq a$ and $\dlim{x}{a}g(x)=L$, then $L \geq 0$.
\end{center}
\begin{proof}
  Let $f$ be an increasing and differentiable function on $(a,b)$, and let $x\in(a,b)$.

  Since $f$ is differentiable, $f'(x)$ exists and is equal to
  $\lim\limits_{h\to0}\frac{f(x+h)-f(x)}{h}$.

  Since the limit exists, the one-sided limits exist and are equal.
  Consider the right-handed limit. Then, $h > 0$ and $x+h > x$.
  Because $f$ is increasing, $f(x+h) > f(x)$ and $f(x+h)-f(x) > 0$.
  Therefore, the Newton quotient is positive for all $h$,
  so the limit, i.e., the derivative, is positive.
\end{proof}


\question Suppose $f$ is a differentiable function that satisfies $f(1) = 3$ and $2 \leq f'(x) \leq 7$.
Use the Bounded Derivative Theorem to find an interval for $f(3)$.
\begin{proof}[Solution]
  Since the lower bound of $f'$ is 2, over a distance $3-1=2$, $f$ can increase by at least 4.
  Likewise, as the upper bound of $f'$ is 7, over a distance 2, $f$ can increase by at most 14.
  Therefore, we have the range $f(3)\in[3+4,3+14] = [7,17]$.
\end{proof}


\question Assume $f$ is a differentiable function on $\R$.
\begin{enumerate}[(a)]
  \item Prove that if $|f'(x)| \leq M$ for all $x \in \R$,
        then $|f(x) - f(y)| \leq M|x-y|$ for all $x,y\in\R$.
          [Functions with this property are called \emph{Lipschitz}].
        \begin{proof}
          Let $f$ be differentiable and let $x$, $y$, and $M \geq 0$ be real numbers.
          Suppose $|f'(n)| \leq M$, that is, $-M \leq f'(n) \leq M$ for all $n$.

          Then, $f(y)$ is at most $f(x) + M|x-y|$ and at least $f(x) - M|x-y|$.

          That is, $f(y) - f(x) \leq \pm M|x-y|$, or, $|f(x)-f(y)| \leq M|x-y|$.
        \end{proof}
  \item Is the converse of part (a) true? Prove it or give a counterexample.
        \begin{proof}[Solution]
          Yes. Suppose $|f(x) - f(y)| \leq M|x-y|$ for all $x,y\in\R$. Then,
          \begin{align*}
            |f'(x)| = \lim_{h\to0}\frac{\abs{f(x+h)-f(x)}}{|h|} \leq \frac{M|x+h-x|}{|h|} = M
          \end{align*}
          since $|x|$ is continuous.
        \end{proof}
\end{enumerate}


\question Let $f(x) = \sqrt{x}$ and let $g(x) = 1 + \ln x$.
\begin{enumerate}[(a)]
  \item Show that there is at least one point of intersection of $f$ and $g$ between $e^2$ and $e^4$.
        \begin{proof}
          Consider the function $h(x) = f(x) - g(x)$.
          Since $h$ is composed of continuous functions, it is continuous on its domain ($x > 0$).
          Then, $h(e^2) = \sqrt{e^2} - 1 - \ln e^2 = e-3$ and $h(e^4) = \sqrt{e^4} - 1 - \ln e^4 = e^2-5$.

          As $e-3 < 0$ and $e^2-5 > 0$, by the IVT, there exists a $c\in(e^2,e^4)$ where $h(c)=0$,
          that is, $f(c) = g(c)$.
        \end{proof}
  \item Show that there is exactly one point of intersection of $f$ and $g$ between $e^2$ and $e^4$.
        Call this point $x=b$.
        \begin{proof}
          Let $b_0,b_1\in(e^2,e^4)$.
          Suppose for a contradiction that $h(b_0) = 0$ and $h(b_1) = 0$.
          Then, by the MVT, there exists some $c\in(b_0,b_1)\subsetneq(e^2,e^4)$ where $h'(c) = 0$.
          Now,
          \begin{align*}
            h'(c) & = f'(c) - g'(c)                     \\
            0     & = \frac{1}{2\sqrt{c}} - \frac{1}{c} \\
            0     & = \frac{\sqrt{c} - 2}{2c}           \\
            c     & = 4
          \end{align*}
          (since $0 \not\in(b_0,b_1)$) but $4 \not\in (e^2,e^4)$.
          Therefore, there cannot be a second point of intersection.
        \end{proof}
  \item Show that for all $x > b$ we have $f(x) > g(x)$.
        That is, there are no more intersection points after $x=b$.
        \begin{proof}
          Notice from above that $h'(x) = 0$ only when $x=4$.
          When $x > 4$, $h'(c) < 0$, and as $h'$ is continuous on its domain, $h$ is decreasing.

          Since $b \in (e^2,e^4)$, we have $4 < e^2 < b$, $h$ is decreasing for all $x>b$.
          Then, $h(b) > h(x) = f(x)-g(x)$, so $f(x) > g(x)$ for all $x > b$.
        \end{proof}
\end{enumerate}


\question Evaluate the following limits, you may use any method.
\begin{enumerate}[(a)]
  \item $\dlim{x}{0}\frac{\tan x + x^2 - x}{\sin^2 x}$.
        \begin{proof}[Solution]
          We evaluate the fraction and find that it is of the form
          $\frac{\tan 0+0^2-0}{\sin^2 0}=\frac{0}{0}$.
          Repeatedly applying l'Hôpital's rule:
          \begin{align*}
            \lim_{x\to0} \frac{\tan x + x^2 - x}{\sin^2 x}
             & = \at{\frac{\dv{x}(\tan x + x^2 - x)}{\dv{x}(\sin^2 x)}}{x=0} \\
             & = \lim_{x\to0} \frac{\sec^2 x + 2x - 1}{2\sin 2x}             \\
             & = \at{\frac{\dv{x}(\sec^2 x + 2x - 1)}{\dv{x}(\sin 2x)}}{x=0} \\
             & = \lim_{x\to0} \frac{2\sec^2 x \tan x + 2}{2\cos 2x}          \\
             & = \frac{0 + 2}{2(1)}                                          \\
             & = 1 \qedhere
          \end{align*}
        \end{proof}
  \item $\dlim{x}{1}\left(\frac{x}{x-1} - \frac{1}{\ln x}\right)$.
        \begin{proof}[Solution]
          Simplify the fraction and apply l'Hôpital's Rule to forms $\frac{0}{0}$:
          \begin{align*}
            \lim_{x\to1} \left(\frac{x}{x-1} - \frac{1}{\ln x}\right)
             & = \lim_{x\to1} \frac{x \ln x - (x-1)}{\ln x(x-1)}               \\
             & = \at{\frac{\dv{x}(x \ln x - x+1)}{\dv{x}(\ln x(x-1))}}{x=1}    \\
             & = \lim_{x\to1} \frac{\ln x}{\ln x + \frac{x-1}{x}}              \\
             & = \at{\frac{\dv{x}(\ln x)}{\dv{x}(\ln x + \frac{x-1}{x})}}{x=1} \\
             & = \lim_{x\to1} \frac{1}{x\frac{1}{x} + \frac{1}{x^2}}           \\
             & = \frac{1}{2} \qedhere
          \end{align*}
        \end{proof}
  \item $\dilim{x}\left(1 + \frac{1}{x}\right)^{2x}$.
        \begin{proof}[Solution]
          This is of the form $1^\infty$ so we take the logarithm:
          \begin{align*}
            \lim_{x\to\infty} \ln \left(1 + \frac{1}{x}\right)^{2x}
             & = \lim_{x\to\infty} 2x\ln \left(1 + \frac{1}{x}\right)                            \\
             & = \lim_{x\to\infty} \frac{2\ln \left(1 + \frac{1}{x}\right)}{\frac1x}             \\
             & = \at{\frac{\dv{x}(2\ln \left(1 + \frac{1}{x}\right))}{\dv{x}\frac1x}}{x=\infty}  \\
             & = \lim_{x\to\infty} \frac{-2\frac{1}{1+\frac{1}{x}}\frac{1}{x^2}}{-\frac{1}{x^2}} \\
             & = \lim_{x\to\infty} 2\frac{1}{1+\frac{1}{x}}                                      \\
             & = 2 \qedhere
          \end{align*}
        \end{proof}
\end{enumerate}


\question Let $f(x) = x + \sin x \cos x$ and let $g(x) = f(x)e^{\sin x}$.
\begin{enumerate}[(a)]
  \item Argue why $\dilim{x}\frac{f(x)}{g(x)}$ does not exist.
        \begin{proof}
          Note that $\frac{f(x)}{g(x)} = \frac{f(x)}{f(x)e^{\sin x}} = \frac{1}{e^{\sin x}}$.
          Since $\sin x$ is periodic and has no infinite limit,
          $\frac{1}{e^{\sin x}}$ oscillates between the values
          $\frac{1}{e}$ for $x = \frac{\pi+4k}{2}$ and $e$ for $x = \frac{3\pi+4k}{2}$, $k\in\Z$,
          which are not equal.

          Therefore, picking some sequence with those values, limit cannot exist.
        \end{proof}

  \item Prove that $\dilim{x}f(x) = \infty$ and $\dilim{x}g(x) = \infty$.
        \begin{proof}
          Note that $\sin x\cos x \geq -1$ for all $x$.
          Then, $f(x) \geq x - 1$ for all $x$, but $x-1$ diverges to infinity.
          Therefore, $\dilim{x}f(x) = \infty$.

          Now, $\sin x$ has range $[-1,1]$, so $e^{\sin x}$ has range $[e^{-1},e]$.
          Then, $g(x) \geq f(x)e^{-1}$, but we established that $f(x)$ diveres, so $\dilim{x}g(x) = \infty$.
        \end{proof}
  \item Prove that $\dilim{x}\frac{f'(x)}{g'(x)} = 0$.
        \begin{proof}
          Take some derivatives to get $f'(x) = 1 + \cos 2x$ and
          \begin{align*}
            g'(x) & = f'(x)e^{\sin x} + f(x)e^{\sin x}\cos x                         \\
                  & = (1 + \cos 2x)e^{\sin x} + (x + \sin x \cos x)e^{\sin x}\cos x  \\
                  & = e^{\sin x}(1 + \cos x + \frac{\sin 2x}{2}) + (e^{\sin x}\cos x)x
          \end{align*}
          Note that $0 \leq f'(x) \leq 2$ for all $x$, so $0 \leq \abs{\frac{f'(x)}{g'(x)}} \leq 2$.
          The first term in $g'(x)$ is also clearly bounded.
          However, the second term is a bounded term multiplied by $x$, so it is unbounded.
          Therefore, $|g'(x)|$ can be made arbitrarily large.
          It follows by some squeeze theorem bullshit that $\dilim{x}\frac{f'(x)}{g'(x)} = 0$.
        \end{proof}
  \item Why is the above not a contradiction to l'Hôpital's Rule?
        \begin{proof}[Answer]
          $f'(x)$ does not go to 0 or $\infty$,
          so the limit not of the form $\frac{0}{0}$ or $\frac{\infty}{\infty}$.
        \end{proof}
\end{enumerate}

\end{document}