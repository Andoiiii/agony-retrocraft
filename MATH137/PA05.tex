\documentclass[11pt]{article}

\usepackage{physics}
\usepackage{amsfonts,amsmath,amssymb,amsthm}
\usepackage{enumerate}
\usepackage{titlesec}
\usepackage{fancyhdr}
\usepackage{multicol}

\headheight 13.6pt
\setlength{\headsep}{10pt}
\textwidth 15cm
\textheight 24.3cm
\evensidemargin 6mm
\oddsidemargin 6mm
\topmargin -1.1cm
\setlength{\parskip}{1.5ex}
\parindent=0pt

\author{James Ah Yong}

\pagestyle{fancy}
\fancyhf{}
\fancyfoot[c]{\thepage}
\makeatletter
\lhead{\@title}
\rhead{\@author}

\fancypagestyle{firstpage}{
  \fancyhf{}
  \rhead{\@author}
  \fancyfoot[c]{\thepage}
}

% Sets
\newcommand{\N}{\mathbb{N}}
\newcommand{\Z}{\mathbb{Z}}
\newcommand{\Q}{\mathbb{Q}}
\newcommand{\R}{\mathbb{R}}
\newcommand{\C}{\mathbb{C}}
\newcommand{\U}{\mathcal{U}}
\newcommand{\sym}{\mathbin{\triangle}}

% Functions
\DeclareMathOperator{\sgn}{sgn}
\DeclareMathOperator{\im}{im}

% Operators
\newcommand{\Rarr}{\Rightarrow}
\newcommand{\Larr}{\Leftarrow}
\usepackage{mathtools} % for \DeclarePairedDelimiter macro
\DeclarePairedDelimiter\ceil{\lceil}{\rceil}
\DeclarePairedDelimiter\floor{\lfloor}{\rfloor}

% Macros
% properly typeset ε-δ (epsilon en dash delta)
\newcommand{\epsdel}[1][\delta]{\ensuremath{\epsilon\mathit{\textnormal{--}}#1}}
\newcommand{\by}[1]{& \text{by #1}}
\newcommand{\IH}{\by{inductive hypothesis}}
% multiple choice (remove spacing between items)
\newenvironment{choices}
{\begin{enumerate}[(a)]
    \setlength{\parskip}{0ex}
    }{
  \end{enumerate}}

% Typesetting
\usepackage{array}   % for \newcolumntype macro
\newcolumntype{C}{>{$}c<{$}} % math version of "C" column type
\newcommand{\dlim}[2]{\displaystyle\lim_{#1\to#2}} % totally not \dfrac ripoff
\newcommand{\dilim}[1]{\dlim{#1}{\infty}} % infinite limits
\newcommand{\ilim}[1]{\lim_{#1\to\infty}}
\usepackage{cancel}

% Auto-number questions
\newcommand{\QType}{Q}
\renewcommand{\theparagraph}{\QType\ifnum\value{paragraph}<10 0\fi\arabic{paragraph}}
\setcounter{secnumdepth}{6}
\newcommand{\question}{\par\refstepcounter{paragraph}\textbf{\theparagraph}.\space}

% Question sections
\titleformat{\section}{\normalsize\bfseries}{\thesection}{1em}{}
\newcommand{\qsection}[2]{%
  \renewcommand{\QType}{#2}
  \section*{#1}
  \refstepcounter{section}
}

\title{MATH 137 Fall 2020: Practice Assignment 5}

\begin{document}
\thispagestyle{firstpage}

\textbf{\@title}

\question Compute the following limits using the fact that $\dilim{x}\frac{\ln x}{x^p}=0$
and $\dilim{x}\frac{x^p}{e^x}=0$ for any $p>0$.
\begin{enumerate}[(a)]
  \item $\dilim{x} \frac{\sqrt{x} + \ln x - x}{1 - \ln e^{2x}}$
        \begin{proof}[Solution]
          Divide through by $x$: \begin{align*}
            \ilim{x} \frac{\sqrt{x} + \ln x - x}{1 - \ln e^{2x}}
             & = \ilim{x} \frac{\sqrt{x} + \ln x - x}{1 - 2x}                              \\
             & = \ilim{x} \frac{\frac{\sqrt{x}}{x} + \frac{\ln x}{x} - 1}{\frac{1}{x} - 2} \\
             & = \frac{0 + 0 - 1}{0 - 2}                                                   \\
             & = \frac{1}{2} \qedhere
          \end{align*}
        \end{proof}

  \item $\dilim{x} e^{-x}\left(1-x\sqrt{e^x}\right)$
        \begin{proof}[Solution]
          Distribute and simplify: \begin{align*}
            \ilim{x} e^{-x}\left(1-x\sqrt{e^x}\right)
             & = \ilim{x} \left(e^{-x}-e^{-x}x\sqrt{e^x}\right) \\
             & = \ilim{x} \left(0- e^{-x} x e^{x/2} \right)     \\
             & = \ilim{x} \frac{x}{e^{x/2}}                     \\
             & = 0 \qedhere
          \end{align*}
        \end{proof}

  \item $\dilim{x} \frac{\frac{\ln x}{x^p}}{\frac{x^p}{e^x}}$ for $p > 0$.
        \begin{proof}[Solution]
          Recall that a product of divergences diverges.
          \begin{align*}
            \ilim{x} \frac{\frac{\ln x}{x^p}}{\frac{x^p}{e^x}}
             & = \ilim{x} e^x\ln x = \infty \qedhere
          \end{align*}
        \end{proof}

  \item $\dilim{x} \frac{(\ln x)^e}{x}$
        \begin{proof}[Solution]
          Rewrite as $\left(\frac{\ln x}{x^{1/e}}\right)^e$.
          This follows the given pattern, so the limit is $0^e=0$.
        \end{proof}
\end{enumerate}

\question Find all asymptotes (both horizontal and vertical) of $f(x) = \dfrac{1}{-2x^2+2}$.
\begin{proof}[Solution]
  Notice that the limit as $x \to \infty$ is 0, so $y=0$ is an asymptote.
  Vertical asymptotes of a rational function occur only when the numerator is 0 but the denominator is non-zero.
  The denominator factors to $-2(x-1)(x+1)$, so $x=\pm 1$ are asymptotes.
\end{proof}

\question Prove that the function $f(x)=2x^2+9$ is continuous at $x=2$ using the \epsdel{} definition of continuity.
\begin{proof}
  We must show that $\dlim{x}{a}f(x) = f(a)$ for $a=2$.
  Specifically, $\dlim{x}{2} (2x^2+9) = 17$.

  This means that for any $\epsilon > 0$, we can find a $\delta$ where $0 < |x-2| < \delta$
  implies $|2x^2-8| < \epsilon$.

  Let $\epsilon > 0$.
  Choose $\delta = \min(\{\frac{\epsilon}{8},2\})$, limiting $\delta$ to be at most 2.
  Also, suppose that $0 < |x-2| < \delta$.
  Then, we have $|x+2| < 4$ and $|x-2| < \frac{\epsilon}{8}$.
  Multiplying:
  \begin{align*}
    \abs{x-2}\abs{x+2} & < \frac{\epsilon}{8} \cdot 4 \\
    \abs{x^2-4}        & < \frac{\epsilon}{2}         \\
    \abs{2x^2-8}       & < \epsilon
  \end{align*}
  which is exactly what we needed to show.

  Therefore, by the \epsdel{} definition of continuity, $f(x)$ is continuous at $x=2$.
\end{proof}

\question Let $f$ be a function defined as
\begin{equation*}
  f(x) = \begin{dcases}
    \frac{x^2-4}{x^2+x-6} \cos x^2 & x \neq -3,2 \\
    0                              & x=-3,2
  \end{dcases}
\end{equation*}
Find the intervals where $f$ is continuous. Justify your answer.
\begin{proof}
  First, simplify $f$ by factoring:
  \begin{align*}
    f(x) & = \begin{dcases}
      \frac{(x-2)(x+2)}{(x-2)(x+3)} \cos x^2 & x \neq -3,2 \\
      0                                      & x=-3,2
    \end{dcases} \\
         & = \begin{dcases}
      \frac{x+2}{x+3} \cos x^2 & x \neq -3,2 \\
      0                        & x=-3,2
    \end{dcases}
  \end{align*}
  Note that cancelling the $x-2$ factors is allowed since the term is only defined when $x\neq 2$.

  By the arithmetic rules for continuity, $f$ is continuous everywhere except possibly at $x=-3$ and $x=2$.

  Consider these two points.

  At $x=-3$, we define $f(-3) = 0$.
  However, the limit from above blows up to negative infinity (and from below to positive infinity).
  Since the limit does not exist, the function is not continuous.

  At $x=2$, we again define $f(2)=0$.
  Applying arithmetic limit rules, the limit is $\frac{2+2}{2+3}\cos 2^2 = \frac{4}{5}\cos 4$.
  This does not equal the value of the function, 0, so the function is not continuous.

  Therefore, $f$ is continuous on $\R \setminus \{-3,2\}$.
\end{proof}

\question Let
\begin{equation*}
  f(x) = \begin{cases}
    cx^2+2x & x > 2    \\
    x^3-cx  & x \leq 2
  \end{cases}
\end{equation*}
Find the value $c$ such that $f(x)$ is continuous on $\R$. Justify your answer.
\begin{proof}
  By the arithmetic rules for continuity, $f(x)$ is clearly continuous on $\R \setminus \{2\}$.

  For $f(x)$ to be continuous at $x=2$, the one-sided limits must agree and equal $f(2)$.
  Since $f(2)$ is defined using the definition for $x > 2$, we need only compare the two cases.

  By the limit rules for polynomials, we have:
  \begin{align*}
    \dlim{x}{2^+} (cx^2+2x) & = \dlim{x}{2^-}(x^3-cx) \\
    c(2)^2+2(2)             & = (2)^3-c(2)            \\
    4c+4                    & = 8-2c                  \\
    c                       & = \frac{2}{3} \qedhere
  \end{align*}
\end{proof}

\question Show that if a function is continuous at $x=0$ and satisfies the following, it is it is continuous everywhere.

\textbf{Hint:} You may use the fact that $\dlim{x}{a}f(x)=f(a)$ is equivalent to $\dlim{h}{0}f(a+h)=f(a)$.
\begin{enumerate}[(a)]
  \item $f(x+y)=f(x)+f(y)$
        \begin{proof}
          Let $f$ be a function continuous at $x=0$.
          By definition, $\dlim{x}{0}f(x)=f(0)$.
          We must show that for all $a$, $\dlim{x}{a} f(x) = f(a)$.

          Let $a$ be an arbitrary value in the domain of $f$.
          Recall that $f(a+h) = f(a)+f(h)$, so
          \[ \dlim{h}{0}f(a+h) = \dlim{h}{0}f(a) + \dlim{h}{0}f(h) = f(a) + f(0) = f(a+0) = f(a) \]
          By the above hint, this is equivalent to saying $\dlim{x}{a}f(x)=f(a)$.
        \end{proof}
  \item $f(x+y)=f(x)f(y)$
        \begin{proof}
          Let $f$ be a function continuous at $x=0$, i.e., $\dlim{x}{0}f(x)=f(0)$.
          Again, we must show that for all $a$, $\dlim{x}{a} f(x) = f(a)$.

          Let $a$ be an arbitrary value in the domain of $f$.
          Since $f(x+y)=f(x)f(y)$:
          \[ \dlim{h}{0} f(a+h) = \dlim{h}{0} f(a) \cdot \dlim{h}{0} f(h) = f(a) \cdot f(0) = f(a+0) = f(a) \]
          By the above hint, this is equivalent to saying $\dlim{x}{a}f(x)=f(a)$.
        \end{proof}
\end{enumerate}

\end{document}