\documentclass[class=pmath370,tikz,notes]{agony}

\title{PMATH 370 Winter 2024: Lecture Notes}
\begin{document}
\renewcommand{\contentsname}{PMATH 370 Winter 2024:\\{\huge Lecture Notes}}
\thispagestyle{firstpage}
\tableofcontents

Lecture notes taken, unless otherwise specified,
by myself during the Winter 2024 offering of PMATH 370,
taught by Blake Madill.

\begin{multicols}{2}
  \listoflecture
\end{multicols}

\chapter{Iteration and Orbits}

\section{Orbits}
\lecture{Jan 8}

\begin{defn}[iteration]
  Let $f : A \to \R$ such that $A \subseteq \R$ and $f(A) \subseteq A$.
  For $a \in A$ we may \term[iteration]{iterate} the function at $a$:
  \[ x_1 = a, x_2 = f(a), x_3 = \underbrace{f(f(a))}_{f^2(a)}, \dotsc, x_i = f^{i-1}(a), \dotsc. \]
  The sequence $(x_n)_{n=1}^\infty$ is the \term[orbit]{orbit of $a$ under $f$}
  (abbreviated $(x_n)$ without limits).
\end{defn}

\begin{example}
  Let $f(x) = x^4 + 2x^2 - 2$, $a = -1$. What is the orbit of $a$ under $f$?
\end{example}
\begin{sol}
  $a = -1$, $f(a) = 1$, $f(f(a)) = f(1) = 1$, so we have $-1,1,1,1,\dotsc$.
  We call this eventually constant.
\end{sol}

\begin{example}
  Let $f(x) = -x^2 - x + 1$, $a = 0$. What is the orbit of $a$ under $f$?
\end{example}
\begin{sol}
  Calculate: $0, 1, -1, 1, -1, 1, \dotsc$.
  We call this eventually periodic (with period 2).
\end{sol}

\begin{example}
  Let $f(x) = x^3 - 3x + 1$, $a = 1$. What is the orbit of $a$ under $f$?
\end{example}
\begin{sol}
  Calculate the first few terms: $1, -1, 3, 19, \dotsc$ (too big).
  This is a divergence to infinity.
\end{sol}

\begin{example}
  Let $f(x) = x^2 + 2x$, $a = -0.5$. What is the orbit of $a$ under $f$?
\end{example}
\begin{sol}
  Calculate: $-0.5, -0.75, -0.9375, -0.9961\dots$
  and we make an educated guess that this converges to $-1$
  since $f(-1) = -1$, a fixed point.
\end{sol}

\begin{example}
  Let $f(x) = x^3 - 3x$, $a = 0.75$. What is the orbit of $a$ under $f$?
\end{example}
\begin{sol}
  Calculate: $0.75, -1.828, -0.625, 1.631, -0.552, \dotsc$.
  There is no clear pattern, so we call this chaotic.
  In fact, the orbit is dense in a neighbourhood of 0.
\end{sol}

We can start to formalize the examples.

\begin{defn}[fixed point]
  Let $f : A \to \R$ such that $f(A) \subseteq A$.
  A point $a \in A$ is fixed if $f(a) = a$.

  Then, the orbit of $a$ under $f$ is $(a,a,a,\dotsc)$
  which is \term[orbit!constant]{constant}.
\end{defn}

\begin{example}
  Find all fixed points of $f(x) = x^2 + x - 4$.
\end{example}
\begin{sol}
  We find points where $f(x) = x$, i.e., $x^2 + x - 4 = x$.
  \begin{equation*}
    x^2 + x - 4 = x \iff x^2 = 4 \iff x = \pm 2 \qedhere
  \end{equation*}
\end{sol}

\begin{example}
  How many fixed points does $f(x) = 2\sin x$ have?
\end{example}
\begin{sol}
  Consider where the curve $y = 2\sin x$ meets $y = x$:
  \begin{center}
    \begin{tikzpicture}
      \begin{axis}[axis lines=middle,
          xlabel=$x$,
          ylabel=$y$,
          enlargelimits,
          ytick=\empty,
          xtick=\empty,
          samples=60]
        \addplot[name path=F,blue,domain={-6:6}] {2*sin(deg(x))} node[pos=.9, below]{$y = 2\sin x$};
        \addplot[name path=G,ForestGreen,domain={-4:4}] {x} node[pos=.8, left]{$y = x$};
      \end{axis}
    \end{tikzpicture}
  \end{center}
  We can see there are three fixed points.
\end{sol}

\begin{example}
  Prove that $f(x) = x^4 - 3x + 1$ has a fixed point.
\end{example}
\begin{prf}
  We must show there is a solution to $x^4 - 3x + 1 \iff x^4 - 4x + 1 = 0$.
  Let $g(x) = x^4 - 4x + 1$.
  Since $g(x)$ is continuous, $g(0) = 1 > 0$, and $g(1) = -2 < 0$,
  by the Intermediate Value Theorem, there must exist a root of $g$ on the interval $(0,1)$.
  That is, a fixed point of $f$.
\end{prf}

\begin{defn}[periodicity]
  Let $f : A \to \R, f(A) \subseteq A$.
  \begin{enumerate}[nosep]
    \item A point $a \in A$ is \term[point!periodic]{periodic} for $f$ if its orbit is periodic.
          An orbit is \term[orbit!periodic]{periodic} if for some $n \in \N$, $f^n(a) = a$.
          The smallest $n$ is the \term{period} of (the orbit of) $a$.
    \item An orbit (of a point) is \term[orbit!eventually periodic]{eventually periodic}
          if there exists $n < m$ such that $f^n(a) = f^m(a)$.
          The smallest difference $m-n$ is the period of the orbit.
  \end{enumerate}
\end{defn}

\pagebreak
\phantomsection\addcontentsline{toc}{chapter}{Back Matter}
\renewcommand{\listtheoremname}{List of Named Results}
\phantomsection\addcontentsline{toc}{section}{\listtheoremname}
\listoftheorems[ignoreall,onlynamed={theorem,lemma,corollary}]
\printindex

\end{document}
