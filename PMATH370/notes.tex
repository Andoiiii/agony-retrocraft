\documentclass[class=pmath370,tikz,notes]{agony}

\title{PMATH 370 Winter 2024: Lecture Notes}
\begin{document}
\renewcommand{\contentsname}{PMATH 370 Winter 2024:\\{\huge Lecture Notes}}
\thispagestyle{firstpage}
\tableofcontents

Lecture notes taken, unless otherwise specified,
by myself during the Winter 2024 offering of PMATH 370,
taught by Blake Madill.

\begin{multicols}{2}
  \listoflecture
\end{multicols}

\chapter{Iteration and Orbits}

\section{Orbits}
\lecture{Jan 8}

\begin{defn}[iteration]
  Let $f : A \to \R$ such that $A \subseteq \R$ and $f(A) \subseteq A$.
  For $a \in A$ we may \term[iteration]{iterate} the function at $a$:
  \[ x_1 = a, x_2 = f(a), x_3 = \underbrace{f(f(a))}_{f^2(a)}, \dotsc, x_i = f^{i-1}(a), \dotsc. \]
  The sequence $(x_n)_{n=1}^\infty$ is the \term[orbit]{orbit of $a$ under $f$}
  (abbreviated $(x_n)$ without limits).
\end{defn}

\begin{example}
  Let $f(x) = x^4 + 2x^2 - 2$, $a = -1$. What is the orbit of $a$ under $f$?
\end{example}
\begin{sol}
  $a = -1$, $f(a) = 1$, $f(f(a)) = f(1) = 1$, so we have $-1,1,1,1,\dotsc$.
  We call this eventually constant.
\end{sol}

\begin{example}
  Let $f(x) = -x^2 - x + 1$, $a = 0$. What is the orbit of $a$ under $f$?
\end{example}
\begin{sol}
  Calculate: $0, 1, -1, 1, -1, 1, \dotsc$.
  We call this eventually periodic (with period 2).
\end{sol}

\begin{example}
  Let $f(x) = x^3 - 3x + 1$, $a = 1$. What is the orbit of $a$ under $f$?
\end{example}
\begin{sol}
  Calculate the first few terms: $1, -1, 3, 19, \dotsc$ (too big).
  This is a divergence to infinity.
\end{sol}

\begin{example}
  Let $f(x) = x^2 + 2x$, $a = -0.5$. What is the orbit of $a$ under $f$?
\end{example}
\begin{sol}
  Calculate: $-0.5, -0.75, -0.9375, -0.9961\dots$
  and we make an educated guess that this converges to $-1$
  since $f(-1) = -1$, a fixed point.
\end{sol}

\begin{example}
  Let $f(x) = x^3 - 3x$, $a = 0.75$. What is the orbit of $a$ under $f$?
\end{example}
\begin{sol}
  Calculate: $0.75, -1.828, -0.625, 1.631, -0.552, \dotsc$.
  There is no clear pattern, so we call this chaotic.
  In fact, the orbit is dense in a neighbourhood of 0.
\end{sol}

We can start to formalize the examples.

\begin{defn}[fixed point]
  Let $f : A \to \R$ such that $f(A) \subseteq A$.
  A point $a \in A$ is fixed if $f(a) = a$.

  Then, the orbit of $a$ under $f$ is $(a,a,a,\dotsc)$
  which is \term[orbit!constant]{constant}.
\end{defn}

\begin{example}
  Find all fixed points of $f(x) = x^2 + x - 4$.
\end{example}
\begin{sol}
  We find points where $f(x) = x$, i.e., $x^2 + x - 4 = x$.
  \begin{equation*}
    x^2 + x - 4 = x \iff x^2 = 4 \iff x = \pm 2 \qedhere
  \end{equation*}
\end{sol}

\begin{example}
  How many fixed points does $f(x) = 2\sin x$ have?
\end{example}
\begin{sol}
  Consider where the curve $y = 2\sin x$ meets $y = x$:
  \begin{center}
    \begin{tikzpicture}
      \begin{axis}[axis lines=middle,
          xlabel=$x$,
          ylabel=$y$,
          enlargelimits,
          ytick=\empty,
          xtick=\empty,
          samples=60]
        \addplot[name path=F,blue,domain={-6:6}] {2*sin(deg(x))} node[pos=.9, below]{$y = 2\sin x$};
        \addplot[name path=G,ForestGreen,domain={-4:4}] {x} node[pos=.8, left]{$y = x$};
      \end{axis}
    \end{tikzpicture}
  \end{center}
  We can see there are three fixed points.
\end{sol}

\begin{example}
  Prove that $f(x) = x^4 - 3x + 1$ has a fixed point.
\end{example}
\begin{prf}
  We must show there is a solution to $x^4 - 3x + 1 \iff x^4 - 4x + 1 = 0$.
  Let $g(x) = x^4 - 4x + 1$.
  Since $g(x)$ is continuous, $g(0) = 1 > 0$, and $g(1) = -2 < 0$,
  by the Intermediate Value Theorem, there must exist a root of $g$ on the interval $(0,1)$.
  That is, a fixed point of $f$.
\end{prf}

\begin{defn}[periodicity]
  Let $f : A \to \R, f(A) \subseteq A$.
  \begin{enumerate}[nosep]
    \item A point $a \in A$ is \term[point!periodic]{periodic} for $f$ if its orbit is periodic.
          An orbit is \term[orbit!periodic]{periodic} if for some $n \in \N$, $f^n(a) = a$.
          The smallest $n$ is the \term{period} of (the orbit of) $a$.
    \item An orbit (of a point) is \term[orbit!eventually periodic]{eventually periodic}
          if there exists $n < m$ such that $f^n(a) = f^m(a)$.
          The smallest difference $m-n$ is the period of the orbit.
  \end{enumerate}
\end{defn}

\lecture{Jan 10}
\begin{defn}[doubling function]
  $D : [0,1) \to [0,1) : x \mapsto 2x-\floor{2x}$
  returns the fractional part of $2x$.
\end{defn}
\begin{example}
  $D(0.4) = 0.8$, $D(0.6) = 0.2$, $D(0.8) = 0.6$, $D(0.5) = 0$.
\end{example}

This is a nice function that gives lots of periodic orbits for funsies.

\begin{example}\label{ex:orbit1}
  Find the orbit of $a=\frac15$ under $D$.
\end{example}
\begin{sol}
  Double until we pass 1: $\frac15, \frac25, \frac45, \frac85 \to \frac35, \frac65 \to \frac15$.
  The period is $\abs{\{\frac15,\frac25,\frac45,\frac35\}} = 4$.
\end{sol}

\begin{example}
  Find the orbit of $a=\frac1{20}$ under $D$.
\end{example}
\begin{sol}
  Double: $\frac1{20}, \frac1{10}, \frac15$ and we can stop
  because \cref{ex:orbit1} showed $\frac15$ is periodic.

  So this is eventually periodic with period 4.
\end{sol}

\begin{problem}
  Given $f$ and $a$, does $f^n(a)$ tend towards some limit $L$?
\end{problem}

To solve this problem, we need to rigorously define ``tend'' and ``limit''.

\section{Real Analysis Review}

\begin{notation}
  If $(x_n)_{n=1}^\infty$ is a sequence of real numbers,
  we write $(x_n) \subseteq \R$.
\end{notation}

\begin{defn*}[convergence of a sequence]
  Let $(x_n) \subseteq \R$, $x \in \R$.

  We say $(x_n)$ \term[sequence!convergence]{converges} to $x$ if
  for all $\varepsilon > 0$, there exists $N \in \N$
  such that $\abs{x_n - x} < \varepsilon$ for all $n \geq N$.

  Then, we write $x_n \to x$ or $\lim x_n = x$.
\end{defn*}

\begin{example}
  Show that $\frac1n \to 0$.
\end{example}
\begin{prf}
  Let $\varepsilon > 0$. Consider $N = \frac{2}{\varepsilon} > \frac{1}{\varepsilon}$.
  For $n \geq N$, we have
  \[ \abs{\frac1n - 0} = \frac1n < \varepsilon \]
  Therefore, $\frac1n \to 0$.
\end{prf}

\begin{example}
  Prove that $\frac{2n}{n+3} \to 2$.
\end{example}
\begin{prf}
  Let $\varepsilon > 0$.
  Since we know $\frac1n \to 0$,
  let $N \in \N$ such that $\frac1N < \frac{\varepsilon}{6}$.

  For $n \geq N$,
  \begin{align*}
    \abs{\frac{2n}{n+3} - 2} = \abs{\frac{2n}{n+3} - \frac{2n+6}{n+3}}
    = \abs{\frac{-6}{n+3}}
    = \frac{6}{n+3}
    < \frac{6}{n}
    \leq \frac{6}{N}
    < 6\cdot\frac{\varepsilon}{6}
    = \varepsilon
  \end{align*}
  Therefore, $\frac{2n}{n+3} \to 2$.
\end{prf}

\begin{defn*}[bounded sequence]
  A sequence $(x_n)$ is \term[sequence!bounded]{bounded} (by $M$)
  if there exists $M > 0$ such that $\forall n\in\N$, $\abs{x_n} \leq M$.
\end{defn*}

\begin{prop}[convergence implies boundedness]
  If $(x_n)$ is convergent, then $(x_n)$ is bounded.
\end{prop}
\begin{prf}
  Suppose $x_n \to x$.
  Then, there exists $N \in \N$ such that if $n \geq N$,
  then $\abs{x_n - x} < 1$.

  For $n \geq N$, $\abs{x_n} - \abs{x} \leq \abs{x_n-x} < 1$.
  That is, $\abs{x_n} < 1 + \abs{x}$.

  Let $M = \max\{\abs{x_1},\dotsc,\abs{x_{n-1}},1+\abs{x}\}$.
  Then, for both all $n < N$ and $n \geq N$, we have $\abs{x_n} \leq M$.
\end{prf}

The converse is not true. Notice that $x_n = (-1)^n$ is bounded by 1
but obviously not convergent.

\begin{prop}[limit laws]
  Let $x_n \to x$ and $y_n \to y$. Then:
  \begin{enumerate}[(1)]
    \item $x_n + y_n \to x + y$
    \item $x_ny_n \to xy$
  \end{enumerate}
\end{prop}
\begin{prf}
  (1) Let $\varepsilon > 0$.
  Then, since $x_n \to x$ and $y_n \to y$,
  there exist $N_1,N_2 \in \N$ such that $n \geq N_1 \implies \abs{x_n - x} < \frac\varepsilon2$
  and $n \geq N_2 \implies \abs{y_n - y} < \frac\varepsilon2$.

  For $N = \max\{N_1,N_2\}$ and $n \geq N$,
  \begin{align*}
    \abs{(x_n + y_n) - (x + y)}
     & = \abs{(x_n - x) + (y_n - y)}           \\
     & \leq \abs{x_n-x} + \abs{y_n-y}          \\
     & < \frac\varepsilon2 + \frac\varepsilon2 \\
     & = \varepsilon
  \end{align*}
  That is, $x_n + y_n \to x+y$.

  (2) Let $\varepsilon > 0$. Notice that:
  \begin{equation*}
    \abs{x_ny_n - xy} = \abs{x_ny_n - x_ny + x_ny - xy}
    \leq \abs{x_n}\cdot\abs{y_n-y} + \abs{y}\cdot\abs{x_n-x}
    \tag{\ast}
  \end{equation*}
  Since $x_n$ is bounded, there exists $M > 0$ such that
  $\abs{x_n} \leq M$ for all $n$.

  Let $N_1,N_2 \in \N$ such that
  \begin{align*}
    n \geq N_1 & \implies \abs{x_n - x} \leq \frac{\varepsilon}{2(\abs{y}+1)}\text{ and} \\
    n \geq N_2 & \implies \abs{y_n-y} < \frac{\varepsilon}{2M}.
  \end{align*}
  Then, for $n \geq N := \max\{N_1,N_2\}$,
  $\abs{x_ny_n - xy} < \frac\varepsilon2 + \frac\varepsilon2 = \varepsilon$
  by $(\ast)$.
\end{prf}

\pagebreak
\phantomsection\addcontentsline{toc}{chapter}{Back Matter}
\renewcommand{\listtheoremname}{List of Named Results}
\phantomsection\addcontentsline{toc}{section}{\listtheoremname}
\listoftheorems[ignoreall,onlynamed={theorem,lemma,corollary,prop}]
\printindex

\end{document}
