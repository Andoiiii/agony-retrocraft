\documentclass[11pt]{article}

\usepackage{physics}
\usepackage{amsfonts,amsmath,amssymb,amsthm}
\usepackage{enumerate}
\usepackage{titlesec}
\usepackage{fancyhdr}
\usepackage{multicol}

\headheight 13.6pt
\setlength{\headsep}{10pt}
\textwidth 15cm
\textheight 24.3cm
\evensidemargin 6mm
\oddsidemargin 6mm
\topmargin -1.1cm
\setlength{\parskip}{1.5ex}
\parindent=0pt

\author{James Ah Yong}

\pagestyle{fancy}
\fancyhf{}
\fancyfoot[c]{\thepage}
\makeatletter
\lhead{\@title}
\rhead{\@author}

\fancypagestyle{firstpage}{
  \fancyhf{}
  \rhead{\@author}
  \fancyfoot[c]{\thepage}
}

% Sets
\newcommand{\N}{\mathbb{N}}
\newcommand{\Z}{\mathbb{Z}}
\newcommand{\Q}{\mathbb{Q}}
\newcommand{\R}{\mathbb{R}}
\newcommand{\C}{\mathbb{C}}
\newcommand{\U}{\mathcal{U}}
\newcommand{\sym}{\mathbin{\triangle}}

% Functions
\DeclareMathOperator{\sgn}{sgn}
\DeclareMathOperator{\im}{im}

% Operators
\newcommand{\Rarr}{\Rightarrow}
\newcommand{\Larr}{\Leftarrow}
\usepackage{mathtools} % for \DeclarePairedDelimiter macro
\DeclarePairedDelimiter\ceil{\lceil}{\rceil}
\DeclarePairedDelimiter\floor{\lfloor}{\rfloor}

% Macros
% properly typeset ε-δ (epsilon en dash delta)
\newcommand{\epsdel}[1][\delta]{\ensuremath{\epsilon\mathit{\textnormal{--}}#1}}
\newcommand{\by}[1]{& \text{by #1}}
\newcommand{\IH}{\by{inductive hypothesis}}
% multiple choice (remove spacing between items)
\newenvironment{choices}
{\begin{enumerate}[(a)]
    \setlength{\parskip}{0ex}
    }{
  \end{enumerate}}

% Typesetting
\usepackage{array}   % for \newcolumntype macro
\newcolumntype{C}{>{$}c<{$}} % math version of "C" column type
\newcommand{\dlim}[2]{\displaystyle\lim_{#1\to#2}} % totally not \dfrac ripoff
\newcommand{\dilim}[1]{\dlim{#1}{\infty}} % infinite limits
\newcommand{\ilim}[1]{\lim_{#1\to\infty}}
\usepackage{cancel}

% Auto-number questions
\newcommand{\QType}{Q}
\renewcommand{\theparagraph}{\QType\ifnum\value{paragraph}<10 0\fi\arabic{paragraph}}
\setcounter{secnumdepth}{6}
\newcommand{\question}{\par\refstepcounter{paragraph}\textbf{\theparagraph}.\space}

% Question sections
\titleformat{\section}{\normalsize\bfseries}{\thesection}{1em}{}
\newcommand{\qsection}[2]{%
  \renewcommand{\QType}{#2}
  \section*{#1}
  \refstepcounter{section}
}

\usepackage[titles]{tocloft}
\title{MATH 146 Fall 1989: Hoffman--Kunze \emph{Linear Algebra}, 2nd ed.}
\renewcommand*\contentsname{\@title} % janky way to deal with chapter title styling

\begin{document}
\thispagestyle{firstpage}
\tableofcontents

\section{Linear Equations}
\subsection{Fields}\label{sec:fields}
\subsection{Systems of Linear Equations}

\xca{Verify that the set of complex numbers described in Example 1.2.4 is a subfield of $C$.}
\begin{prf}
  Consider the set of complex numbers $S$ of the form $x + y\sqrt2$ for rational $x$ and $y$,
  along with complex addition and multiplication.

  Both 0 and 1 can be expressed as $0_S = 0+0\sqrt2$ and $1_S = 1 + 0\sqrt2$.

  Let $a = x + y\sqrt2$ and $b = w + z\sqrt2$ be elements of $S$.
  Then, it follows that $a+b = (x+w) + (y+z)\sqrt2$, $-a = (-x) + (-y)\sqrt2$
  and $ab = (xw + 2yz) + (wy + xz)\sqrt2$ are also in $S$ by properties of the field of rationals.
  Finally, if $a \neq 0$, then
  \[ a^{-1} = \frac{1}{x+y\sqrt2} = \frac{x-y\sqrt2}{x^2-2y^2} = \frac{x}{x^2-2y^2} + \frac{-y}{x^2-2y^2}\sqrt2 \]
  which is in $S$.
  Therefore, $S$ is a subfield of $C$.
\end{prf}

\begin{xca}\label{xca:sysequiv}
  Let $F$ be the field of complex numbers.
  Are the following two systems of linear equations equivalent?
  If so, express each equation in each system as a linear combination of the equations in the other system.
  \[ \systeme{x_1-x_2=0, 2x_1+x_2=0} \qquad \systeme{3x_1+x_2=0, x_1+x_2=0} \]
\end{xca}
\begin{prf}
  They are equivalent. The first system is
  \begin{align*}
    x_1 - x_2  & = (3x_1 + x_2) - 2(x_1 + x_2)                \\
    2x_1 + x_2 & = \tfrac12(3x_1 + x_2) + \tfrac12(x_1 + x_2)
  \end{align*}
  and the second system is
  \begin{align*}
    3x_1 + x_2 & = \tfrac14(x_1 - x_2) + \tfrac34(2x_1 + x_2)  \\
    x_1 + x_2  & = -\tfrac13(x_1 - x_2) + \tfrac23(2x_1 + x_2)
  \end{align*}
  Since they are linear combinations of each other, they are equivalent.
\end{prf}

\begin{xca}
  Test the following systems of equations as in \Cref{xca:sysequiv}.
  \[
    \systeme{-x_1+x_2+4x_3=0, x_1+3x_2+8x_3=0, \frac12x_1+x_2+\frac52x_3=0}
    \qquad
    \systeme{x_1-x_3=0, x_2+3x_3=0}
  \]
\end{xca}
\begin{prf}
  They are equivalent and we can show this by creating linear combinations:
  \begin{align*}
    -x_1 + x_2 + 4x_3               & = -(x_1 - x_3) + (x_2 + 3x_3)        \\
    x_1 + 3x_2 + 8x_3               & = (x_1 - x_3) + 3(x_2 + 3x_3)        \\
    \tfrac12x_1 + x_2 + \tfrac52x_3 & = \tfrac12(x_1 - x_3) + (x_2 + 3x_3)
  \end{align*}
  and, eliminating using the above,
  \begin{align*}
    x_1 - x_3  & = -\tfrac34(-x_1 + x_2 + 4x_3) + \tfrac14(\tfrac12x_1 + x_2 + \tfrac52x_3) \\
    x_2 - 3x_3 & = \tfrac14(-x_1 + x_2 + 4x_3) + \tfrac14(\tfrac12x_1 + x_2 + \tfrac52x_3)
  \end{align*}
  Since they are linear combinations of each other, they are equivalent.
\end{prf}

\begin{xca}
  Test the following systems as in \Cref{xca:sysequiv}.
  \[
    \systeme[x_1x_2x_3x_4]{2x_1 + {(-1+i)}x_2 + x_4 = 0, 3x_2 - 2ix_3 + 5x_4 = 0}
    \qquad
    \systeme[x_1x_2x_3x_4]{{(1+\frac{i}{2})}x_1 + 8x_2 - ix_3 - x_4 = 0, \frac{2}{3}x_1 - \frac{1}{2}x_2 + x_3 + 7x_4 = 0}
  \]
\end{xca}
\begin{prf}
  They are not equivalent.
  Suppose for a contradiction that we may write the first equation of the second system as
  a linear combination $a$ and $b$ of the two equations from the first system:
  \[ {(1+\tfrac{i}{2})}x_1 + 8x_2 - ix_3 - x_4 = (2a)x_1 + (2(-1+i)a + 3b)x_2 - (2ib)x_3 + (a + 5b)x_4 \]
  Equating like terms, we have $2a = 1+\frac{i}{2}$ so $a = \frac{1}{2}+\frac{i}{4}$.
  Likewise, $2ib = i$ so $b = \frac{1}{2}$.
  However, $a + 5b = 3 + \frac{i}{4}$ which is not equal to $-1$.
  Therefore, no such $a$ and $b$ can exist.
\end{prf}

\begin{xca}
  Let $F$ be a set which contains exactly two elements, 0 and 1.
  Define an addition and multiplication by the tables:
  \begin{center}
    \begin{tabular}{C|CC}
      + & 0 & 1 \\ \hline 0 & 0 & 1 \\ 1 & 1 & 0
    \end{tabular}
    \qquad
    \begin{tabular}{C|CC}
      \cdot & 0 & 1 \\ \hline 0 & 0 & 0 \\ 1 & 0 & 1
    \end{tabular}
  \end{center}
  Verify that the set $F$, together with these two operations, is a field.
\end{xca}
\begin{prf}
  We prove the nine properties from \cref{sec:fields}:
  \begin{enumerate}
    \item Consider the only case of distinct $x$ and $y$ in $F$, $0 + 1 = 1 + 0 = 1$.
    \item Exhaust the 8 cases:
          \begin{align*}
            0 + (0 + 0) = (0 + 0) + 0 = 0 \\
            0 + (0 + 1) = (0 + 0) + 1 = 1 \\
            0 + (1 + 0) = (0 + 1) + 0 = 1 \\
            0 + (1 + 1) = (0 + 1) + 1 = 0 \\
            1 + (0 + 0) = (1 + 0) + 0 = 1 \\
            1 + (0 + 1) = (1 + 0) + 1 = 0 \\
            1 + (1 + 0) = (1 + 1) + 0 = 0 \\
            1 + (1 + 1) = (1 + 1) + 1 = 1
          \end{align*}
    \item Let $0_F=0$. Then, $0 + 0 = 0$ and $0 + 1 = 1$.
    \item We define $-0 = 0$ and $-1 = 1$ so that $0 + (-0) = 0$ and $1 + (-1) = 1$.
    \item Again, for the only case of distinct values, $0 \cdot 1 = 1 \cdot 0 = 0$.
    \item Exhaust again the 8 cases:
          \begin{align*}
            0 \cdot (0 \cdot 0) = (0 \cdot 0) \cdot 0 = 0 \\
            0 \cdot (0 \cdot 1) = (0 \cdot 0) \cdot 1 = 0 \\
            0 \cdot (1 \cdot 0) = (0 \cdot 1) \cdot 0 = 0 \\
            0 \cdot (1 \cdot 1) = (0 \cdot 1) \cdot 1 = 0 \\
            1 \cdot (0 \cdot 0) = (1 \cdot 0) \cdot 0 = 0 \\
            1 \cdot (0 \cdot 1) = (1 \cdot 0) \cdot 1 = 0 \\
            1 \cdot (1 \cdot 0) = (1 \cdot 1) \cdot 0 = 0 \\
            1 \cdot (1 \cdot 1) = (1 \cdot 1) \cdot 1 = 1
          \end{align*}
    \item Let $1_F=1$ so $0 \cdot 1 = 0$ and $1 \cdot 1 = 1$.
    \item The only non-zero element is 1, and $1 \cdot 1 = 1$, so define $1^{-1} = 1$.
    \item Exhaustion! If $x=0$, then the result is 0.
          If $y = z$, then we have $1(1 + 1) = 1(0 + 0) = 1(0) = 0$.
          Otherwise, WLOG since we have property 1, we have $1(1 + 0) = 1(1) = 1$.
  \end{enumerate}
  Therefore, $(F,+,\cdot)$ is a field.
\end{prf}

\begin{xca}\label{xca:homeq}
  Prove that if two homogenous systems of linear equations in two unknowns have the same solutions,
  then they are equivalent.
\end{xca}
\begin{prf}
  Represent two homogenous linear systems $a$ and $b$ in two unknowns, $x$ and $y$, as
  \begin{alignat*}{6}
    a_{11}x & +      & a_{12}y = 0 & \qquad & b_{11}x & +      & b_{12}y = 0 \\
    a_{21}x & +      & a_{22}y = 0 &        & b_{21}x & +      & b_{22}y = 0 \\
            & \vdots &             &        &         & \vdots &             \\
    a_{k1}x & +      & a_{k2}y = 0 &        & b_{k1}x & +      & b_{k2}y = 0
  \end{alignat*}
  Now, suppose that these systems share the same solutions.
  As systems of lines in $R^2$, the solution set is either a point or a line.
  If it is a line, then all equations are scalar multiples of that line.
  It immediately follows that the systems are equivalent.

  If the solution set is a point, the point must be the origin.
  In fact, the first system is uniquely determined by the first two equations which are not multiples of each other.
  Suppose WLOG that those are the first two equations.
  Let $i$ be arbitrary and solve for $m$ and $n$ such that
  $b_{i1}x + b_{i2}y = (ma_{11} + na_{21})x + (ma_{12} + na_{22})y$.
  Equating coefficients, $b_{i1} = ma_{11} + na_{21}$ and $b_{i2} = ma_{12} + na_{22}$.
  Then, solving and simplifying,
  \[
    m = \frac{a_{22}b_{i1} - a_{12}b_{i2}}{a_{11}a_{22}-a_{12}a_{21}}
    \qq{and}
    n = \frac{a_{11}b_{i2} - a_{21}b_{i1}}{a_{11}a_{22}-a_{12}a_{21}}.
  \]
  These are well-defined if and only if $a_{11}a_{22} \neq a_{12}a_{21}$, or equivalently,
  if $\frac{a_{11}}{a_{21}} \neq \frac{a_{12}}{a_{22}}$.
  However, this is identical to saying that the first two equations are linearly independent.
  Therefore, $m$ and $n$ exist for every $i$, and the systems are equivalent.
\end{prf}

\begin{xca}\label{xca:ratsf}
  Prove that each subfield of the field of complex numbers contains every rational number.
\end{xca}
\begin{prf}
  Let $F$ be a subfield of $C$ and $r$ be a rational number.
  By definition, we may write $r = \frac{p}{q}$ for integers $p$ and $q$.

  As a subfield of $C$, $0$ and $1$ are in $F$.
  Then, $F$ also contains $1+1 = 2$.
  It follows inductively that $F$ contains all positive integers.
  Likewise, $F$ contains the additive inverse of all its elements.
  Therefore, all negative integers are included.
  It follows that $p$ and $q$ are in $F$.

  We also know that $F$ contains multiplicative inverses, so $q^{-1}$ is in $F$.
  Finally, $F$ is closed under multiplication, so $pq^{-1} = r$ is in $F$, completing the proof.
\end{prf}

\begin{xca}
  Prove that each field of characteristic zero contains a copy of the rational number field.
\end{xca}
\begin{prf}
  As in \cref{xca:ratsf}, it suffices to embed the integers.
  Let $(F, +_F, \cdot_F)$ be a field of characteristic zero and $n$ be a non-negative integer.
  Then, there is an additive identity $1_F$.
  Let $n_F = 1_F + 1_F + \dotsb + 1_F$ with $|n|$ copies of $1_F$.
  If $n = 0$, let $0_F$ be the multiplicative identity.
  Since $F$ has characteristic zero, this is a distinct value for any $n$.
  Now, let $-n_F$ be the additive inverse of $n_F$.
  It follows that $n \mapsto n_F$ is a bijection that embeds $Z$ in $F$.

  We now apply multiplicative inverses and closure to find that for any $r = \frac{p}{q}$,
  we may write $r \mapsto r_F$ where $r_F = p_F q_F^{-1}$, embedding $Q$ in $F$.
\end{prf}

\subsection{Matrices and Row Operations}

\begin{xca}
  Find all solutions to the system of equations
  \[ \systeme[x_1x_2]{{(1-i)}x_1 - ix_2 = 0, 2x_1 + {(1-i)}x_2 = 0} \]
\end{xca}
\begin{sol}
  We express the solutions to $AX = 0$ by row-reducing $A = \m{1-i&-i\\2&i-1}$:
  \[
    \m{1-i&-i\\2&i-1}
    \xto{R_1 - \frac{1-i}{2}R_2} \m{0&0\\2&i-1}
    \xto{R_1 \harr R_2} \m{2&i-1\\0&0}
    \xto{\frac12 R_1} \m{1&\frac{i-1}{2}\\0&0}
  \]
  Therefore, the solutions are $(c, \frac{i-1}{2}c)$ for any $c$.
\end{sol}

\begin{xca}
  If \[ A = \m{3&-1&2 \\ 2&1&1 \\ 1&-3&0} \] find all solutions of $AX = 0$ by row-reducing $A$.
\end{xca}
\begin{sol}
  Row-reduce $A$:
  \begin{equation*}
    \begin{split}
      \m{3&-1&2 \\ 2&1&1 \\ 1&-3&0}
      & \xto{R_1 - 3R_3} \m{0&8&2 \\ 2&1&1 \\ 1&-3&0}
      \xto{R_2 - 2R_3} \m{0&8&2 \\ 0&7&1 \\ 1&-3&0}
      \xto{R_1 \harr R_3} \m{1&-3&0 \\ 0&7&1 \\ 0&8&2} \\
      & \xto{R_2 - \frac12 R_3} \m{1&-3&0 \\ 0&3&0 \\ 0&8&2}
      \xto{R_1 + R_3} \m{1&0&0 \\ 0&3&0 \\ 0&8&2}
      \xto{R_3 - \frac83 R_2} \m{1&0&0 \\ 0&3&0 \\ 0&0&2}
    \end{split}
  \end{equation*}
  which after a few scalar multiplications is the identity matrix.
  Therefore, $(0,0,0)$ is the only solution.
\end{sol}

\begin{xca}
  If \[ A = \m{6&-4&0 \\ 4&-2&0 \\ -1&0&3} \]
  find all solutions of $AX = 2X$ and all solutions of $AX = 3X$.
  (The symbol $cX$ denotes the matrix each entry of which is $c$ times the corresponding entry of $X$.)
\end{xca}
\begin{sol}
  We can represent the matrix equation $AX = 2X$ with the system
  \[ \systeme{6x - 4y = 2x, 4x - 2y = 2y, -x + 3z = 2z} \]
  which can be simplified to the homogenous system
  \[ \systeme{4x - 4y = 0, 4x - 4y = 0, -x + z = 0} \]
  or, equivalently, $\m{4&-4&0 \\ 4&-4&0 \\ -1&0&1}X = 0$. We now row-reduce:
  \begin{equation*}
    \begin{split}
      \m{4&-4&0 \\ 4&-4&0 \\ -1&0&1}
      & \xto{R_2 - R_1} \m{4&-4&0 \\ 0&0&0 \\ -1&0&1}
      \xto{R_1 + 4R_3} \m{0&-4&4 \\ 0&0&0 \\ -1&0&1}
      \xto{\frac14 R_1} \m{0&-1&1 \\ 0&0&0 \\ -1&0&1} \\
      & \xto{-R_1 \harr -R_3} \m{1&0&-1 \\ 0&0&0 \\ 0&1&-1}
      \xto{R_2 \harr R_3} \m{1&0&-1 \\ 0&1&-1 \\ 0&0&0}
    \end{split}
  \end{equation*}
  This is equivalent to the equations $x - y = 0$ and $y - z = 0$, so $x = y = z$, that is,
  the solutions are of the form $(c,c,c)$ for some $c$.

  We now do the same with $AX = 3X$, giving the system
  \[
    \systeme{6x - 4y = 3x, 4x - 2y = 3y, -x + 3z = 3z}
    \iff
    \systeme{3x - 4y = 0, 4x - 5y = 0, -x = 0}
  \]
  By inspection, $x = 0$, which implies $y = 0$, so the solution is $(0,0,c)$ for some $c$.
\end{sol}

\begin{xca}
  Find a row-reduced matrix which is row-equivalent to
  \[ A = \m{i&-(1+i)&0 \\ 1&-2&1 \\ 1&2i&-1} \]
\end{xca}
\begin{sol}
  We row-reduce by elementary row operations:
  \begin{equation*}
    \begin{split}
      \m{i&-(1+i)&0 \\ 1&-2&1 \\ 1&2i&-1}
      & \xto{R_2 + iR_1} \m{i&-1-i&0 \\ 0&-1-i&1 \\ 1&2i&-1}
      \xto{R_3 + iR_1} \m{i&-1-i&0 \\ 0&-1-i&1 \\ 0&1+i&-1} \\
      & \xto{R_1 + R_3} \m{i&0&-1 \\ 0&-1-i&1 \\ 0&1+i&-1}
      \xto{R_3 + R_2} \m{i&0&-1 \\ 0&-1-i&1 \\ 0&0&0} \\
      & \xto{-iR_1} \m{1&0&i \\ 0&-1-i&1 \\ 0&0&0}
      \xto{-\frac{1-i}{2}R_2} \m{1&0&i \\ 0&1&-\frac{1-i}{2} \\ 0&0&0}
    \end{split}
  \end{equation*}
  which is in row-reduced form.
\end{sol}

\begin{xca}
  Prove that the following two matrices are not row-equivalent:
  \[ A = \m{2&0&0 \\ a&-1&0 \\ b&c&3} \qq{and} B = \m{1&1&2 \\ -2&0&1 \\ 1&3&5} \]
\end{xca}
\begin{sol}
  We row-reduce the two matrices and compare them.
  Consider matrix $A$:
  \begin{equation*}
    \begin{split}
      A & \xto{\frac12 R_1} \m{1&0&0 \\ a&-1&0 \\ b&c&3}
      \xto{R_2 - aR_1} \m{1&0&0 \\ 0&-1&0 \\ b&c&3}
      \xto{-R_2} \m{1&0&0 \\ 0&1&0 \\ b&c&3} \\
      & \xto{R_3 - bR_1} \m{1&0&0 \\ 0&1&0 \\ 0&c&3}
      \xto{R_3 - cR_2} \m{1&0&0 \\ 0&1&0 \\ 0&0&3}
      \xto{\frac13 R_3} \m{1&0&0 \\ 0&1&0 \\ 0&0&1}
    \end{split}
  \end{equation*}
  Therefore, the equation $AX = 0$ has one solution $(0,0,0)$.

  Now, consider matrix $B$:
  \begin{equation*}
    \begin{split}
      B & \xto{R_2 + 2R_1} \m{1&1&2 \\ 0&2&3 \\ 1&3&5}
      \xto{R_3 - R_1} \m{1&1&2 \\ 0&2&3 \\ 0&2&3}
      \xto{R_3 - R_2} \m{1&1&2 \\ 0&2&3 \\ 0&0&0} \\
      & \xto{\frac12 R_2} \m{1&1&2 \\ 0&1&\frac32 \\ 0&0&0}
      \xto{R_1 - R_2} \m{1&0&\frac12 \\ 0&1&\frac32 \\ 0&0&0}
    \end{split}
  \end{equation*}
  Therefore, the equation $BX = 0$ has the solutions $(-\frac12c, -\frac32c, c)$ for any $c$.

  Since the solution sets are not equivalent, the matrices are not row-equivalent.
\end{sol}

\begin{xca}
  Let \[ A = \m{a&b\\c&d} \] be a $2 \times 2$ matrix with complex entries.
  Suppose that $A$ is row-reduced and also that $a+b+c+d=0$.
  Prove that there are exactly three such matrices.
\end{xca}
\begin{prf}
  For $A$ to be row-reduced, the first row is all zero, begins with a 1, or is $\m{0&1}$.

  If the first row is zero ($a = 0$ and $b = 0$), then $c+d = 0$.
  Likewise, the second row must either be zero (in which case $c = d = 0$),
  begin with a 1 (in which case $c = 1$ and $d = -1$),
  or be $\m{0&1}$ (but $c+d = 0$).
  This gives two options: $A = \m{0&0\\0&0}$ and $\m{0&0\\1&-1}$.

  If the first row begins with 1, then $c = 0$ and $b+d = -1$.
  Again, the second row is either zero (so $b = -1$ and $d = 0$),
  begin with a 1 (but $c = 0$),
  or be $\m{0&1}$ (but then $a$ cannot be 1).
  This also gives two options: $A = \m{0&0\\0&0}$ and $\m{1&-1\\0&0}$.

  Finally, if the first row is $\m{0&1}$, the second row must be $\m{0&0}$,
  but then $a+b+c+d \neq 0$.

  Therefore, there are only three matrices $A$, namely,
  $\m{0&0\\0&0}$, $\m{1&-1\\0&0}$, and $\m{0&0\\1&-1}$.
\end{prf}

\begin{xca}
  Prove that the interchange of two rows of a matrix can be accomplished by
  a finite sequence of elementary row operations of the other two types.
\end{xca}
\begin{prf}
  Let $A$ be a matrix with rows $R_1, R_2, \dotsc, R_n$.
  Without loss of generality, we interchange $R_1 \harr R_2$.

  We represent the matrix as $A = \m{R_1 \\ R_2 \\ \vdots \\ R_n}$ and row-reduce:
  \begin{equation*}
    \m{R_1 \\ R_2 \\ \vdots \\ R_n}
    \xto{R_1 + R_2} \m{R_1 + R_2 \\ R_2 \\ \vdots \\ R_n}
    \xto{R_2 - R_1} \m{R_1 + R_2 \\ -R_1 \\ \vdots \\ R_n}
    \xto{R_1 + R_2} \m{R_2 \\ -R_1 \\ \vdots \\ R_n}
    \xto{-R_2} \m{R_2 \\ R_1 \\ \vdots \\ R_n}
  \end{equation*}
  completing the proof.
\end{prf}

\begin{xca}
  Consider the system of equations $AX = 0$ where \[ A = \m{a&b \\ c&d} \]
  is a $2 \times 2$ matrix over the field $F$. Prove the following:
  \setlength\parskip{0pt}
  \begin{enumerate}[(a)]
    \item If every entry of $A$ is 0, then every pair $(x_1, x_2)$ is a solution of $AX = 0$.
    \item If $ad-bc \neq 0$, the system $AX = 0$ has only the trivial solution $x_1 = x_2 = 0$.
    \item If $ad-bc = 0$ and some entry of $A$ is different from 0,
          then there is a solution $(x_1^0, x_2^0)$ such that
          $(x_1, x_2)$ is a solution if and only if there is a scalar $y$ such that
          $x_1 = yx_1^0$ and $x_2 = yx_2^0$.
  \end{enumerate}
\end{xca}
\begin{prf}
  Suppose that every entry of $A$ is zero.
  Then, $AX = 0$ is equivalent to the system $0x_1 + 0x_2 = 0$ and $0x_1 + 0x_2 = 0$.
  Both equations simplify to $0 = 0$.
  Therefore, the solution set is $(x_1, x_2)$ for any such pair, as desired in (a).

  Otherwise, suppose that $ad - bc \neq 0$.
  We can then apply the generic solution for a linear system in two equations with two unknowns
  (see \cref{xca:homeq}):
  \[ x_1 = \frac{d(0) - b(0)}{ad - bc} = 0 \qq{and} x_2 = \frac{a(0) - c(0)}{ad - bc} = 0 \]
  as desired in (b).

  Finally, suppose that $ad - bc = 0$ and some entry of $A$ is non-zero.
  We take without loss of generality (through row reduction and variable symmetry) that $a \neq 0$.
  Then, we can say $d = \frac{bc}{a}$.
  It follows that $\frac{c}{a}(ax_1 + bx_2) = cx_1 + dx_2$.
  Therefore, the first equation implies the second.

  Now, the first equation is equivalently expressed $x_1 = -\frac{b}{a}x_2$.
  Thus, the solutions to both equations are $(c, -\frac{b}{a}c)$ for any $c$.
  Equivalently, with the solution $x_1^0 = 1$ and $x_2^0 = -\frac{b}{a}$,
  they are all of the form $(yx_1^0, yx_2^0)$ for some scalar $y$, as desired in (c).
\end{prf}

\subsection{Row-Reduced Echelon Matrices}

\begin{xca}
  Find all solutions to the following system of equations by row-reducing the coefficient matrix:
  \[ \systeme{\frac13x_1 + 2x_2 - 6x_3 = 0, -4x_1 + 5x_3 = 0, -3x_1 + 6x_2 - 13x_3 = 0, -\frac73x_1 + 2x_2 - \frac83x_3 = 0} \]
\end{xca}
\begin{sol}
  Do what the problem says:
  \begin{equation*}
    \begin{split}
      \m{\frac13&2&-6 \\ -4&0&5 \\ -3&6&-13 \\ -\frac73&2&-\frac83}
      & \xto{R_1 - R_4} \m{\frac83&0&-\frac{10}3 \\ -4&0&5 \\ -3&6&-13 \\ -\frac73&2&-\frac83}
      \xto{R_2 + \frac32 R_1} \m{\frac83&0&-\frac{10}3 \\ 0&0&0 \\ -3&6&-13 \\ -\frac73&2&-\frac83} \\
      & \xto{R_3 - 3R_4} \m{\frac83&0&-\frac{10}3 \\ 0&0&0 \\ 4&0&-5 \\ -\frac73&2&-\frac83}
      \xto{R_1 - \frac23 R_3} \m{0&0&0 \\ 0&0&0 \\ 4&0&-5 \\ -\frac73&2&-\frac83}
      \xto{R_4 + \frac7{12} R_3} \m{0&0&0 \\ 0&0&0 \\ 4&0&-5 \\ 0&2&-\frac{67}{12}} \\
      & \xto{\frac14 R_3}
      \xto{\frac12 R_4} \m{0&0&0 \\ 0&0&0 \\ 1&0&-\frac52 \\ 0&1&-\frac{67}{24}}
      \xto{R_1 \harr R_3}
      \xto{R_2 \harr R_4} \m{1&0&-\frac52 \\ 0&1&-\frac{67}{24} \\ 0&0&0 \\ 0&0&0}
    \end{split}
  \end{equation*}
  Therefore, $x_1 = \frac52 x_3$ and $x_2 = \frac{67}{24} x_3$,
  so for any scalar $c$, there is a solution $(\frac52c, \frac{67}{24}c, c)$.
\end{sol}

\begin{xca}
  Find a row-reduced echelon matrix which is row-equivalent to \[ A = \m{1&-i \\ 2&2 \\ i&1+i}. \]
  What are the solutions of $AX = 0$?
\end{xca}
\begin{sol}
  We have:
  \begin{equation*}
    \begin{split}
      A & \xto{R_2 - 2R_1} \m{1&-i \\ 0&2-2i \\ i&1+i}
      \xto{R_3 - \frac{i}{2}R_2} \m{1&-i \\ 0&2-2i \\ i&0}
      \xto{\frac{1+i}{4}R_2}
      \xto{-iR_3} \m{1&-i \\ 0&1 \\ 1&0} \\
      & \xto{R_1 - R_3}
      \xto{R_1 + iR_2} \m{0&0 \\ 0&1 \\ 1&0}
      \xto{R_1 \harr R_3} \m{1&0 \\ 0&1 \\ 0&0}
    \end{split}
  \end{equation*}
  Therefore, the only solution is the trivial one where $X$ is the zero vector.
\end{sol}

\end{document}