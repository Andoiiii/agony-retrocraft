\RequirePackage{chngcntr}
\counterwithout{section}{chapter}
\titleformat{\section}{\normalfont\large\bfseries}{Lecture~\thesection}{1em}{}
\settowidth{\cftsecnumwidth}{Lecture~0000~}
\renewcommand\cftsecpresnum{Lecture~}

\setlength\topsep{6pt}
\declaretheorem[style=definition,name=Definition,
  thmbox={M,leftmargin=1.5em,rightmargin=1.5em,nocut,headstyle=\textbf{Definition},bodystyle=\noindent},
  refname={definition,definitions},Refname={Definition,Definitions}]{defn}
\declaretheorem[name=Theorem,
  thmbox={L,leftmargin=1.5em,rightmargin=1.5em,nocut,headstyle=\textbf{Theorem},bodystyle=\noindent},
  refname={theorem,theorems},Refname={Theorem,Theorems}]{theorem}
\declaretheorem[name=Lemma,
  thmbox={M,leftmargin=1.5em,rightmargin=1.5em,nocut,headstyle=\textbf{Lemma},bodystyle=\noindent},
  refname={lemma,lemmas},Refname={Lemma,Lemmas}]{lemma}
\declaretheorem[name=Proposition,
  thmbox={S,leftmargin=1.5em,rightmargin=1.5em,nocut,headstyle=\textbf{Proposition},bodystyle=\noindent},
  refname={proposition,propositions},Refname={Proposition,Propositions}]{prop}
\declaretheorem[style=remark,unnumbered]{remark}
\declaretheorem[style=remark,unnumbered]{corollary}
\declaretheorem[style=definition,name=Example,numberwithin=section,
  refname={example,examples},Refname={Example,Examples}]{example}

\RequirePackage[hybrid,underscores=false]{markdown}

% Use inside of definitions: "a \term{thing} is a whatever"
\newcommand{\term}{\underline}

% For thmbox (see tex.se/41100)
\RequirePackage{tablefootnote}
\newcommand{\spewnotes}{\tfn@tablefootnoteprintout\global\let\tfn@tablefootnoteprintout\relax\gdef\tfn@fnt{0}}
